\documentclass[11pt]{article}
\usepackage[utf8]{inputenc}
\usepackage[T1]{fontenc}
\usepackage{amsmath}
\usepackage{amsfonts}
\usepackage{amssymb}
\usepackage[version=4]{mhchem}
\usepackage{stmaryrd}

\begin{document}
Popular Structured Product Types

A popular class of structured products offers payouts based on absolute returns. An absolute return structured product offers payouts over some or all underlying asset returns that are equal to the absolute value of the underlying asset's returns. Thus, whether the underlier rises $2 \%$ or declines $2 \%$, the structured product pays $+2 \%$.

The core concept of an absolute return structured product is easily replicated in the options market with an at-the-money straddle (see the session, Derivatives and Risk-Neutral Valuation). In the case of a long option straddle, the option buyer pays a price or premium to establish the straddle, makes money if the underlying asset makes a large directional move, and loses money if the underlying asset does not move substantially. In the case of a structured product based on absolute returns, the benefit to the investor of gaining from large moves in either direction must be offset by features that benefit the issuer.

A principal protected absolute return barrier note offers to pay absolute returns to the investor if the underlying asset stays within both an upper barrier and a lower barrier over the life of the product. If the underlying asset reaches either barrier, the payout is equal to the principal of the product. Note that as a path-dependent option, the underlying asset may lie inside the barriers at the termination of the structured product but fail to produce absolute returns if its path reached a barrier.

If the barriers are placed $5 \%$ from the initial value of the underlier, the principal protected absolute return barrier note would pay the absolute return of the underlier if the barrier was not reached, or 0\% if the barrier was reached. This structured product can be replicated as a long straddle position in exotic options (knock-out options):

$$
\begin{aligned}
\text { Product }= & + \text { At-the-Money Up-and-Out Call } \\
& + \text { At-the-Money Down-and-Out Put }
\end{aligned}
$$

By placing this product into an individual savings account, many UK investors can enjoy tax-free distributions of any profits.

Many structured products are listed and are therefore liquid alternatives. For example, in the United States, there are numerous structured products issued by major institutions that trade on the New York Stock Exchange.

A disadvantage of a liquid structured product is that it must be standardized in terms of maturity, participation rates, principal protection, and so forth, in order to attract numerous investors; however, some investors may prefer a structured product that is tailored to their individual preferences. The advantage of a liquid structured product to an investor is not only that the product can be sold through the listing market but also that its price and its price volatility can be observed through time.

Interestingly, although many structured products in the United States continue to be registered with the SEC, the proportion of these products actually being listed has diminished in recent years. Apparently the benefits of listing are not perceived as being worth the costs, yet the products are standardized and registered so that they can be marketed to a wider audience.


\end{document}