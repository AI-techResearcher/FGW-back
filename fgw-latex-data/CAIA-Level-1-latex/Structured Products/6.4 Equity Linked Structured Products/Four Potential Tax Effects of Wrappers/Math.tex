\documentclass[11pt]{article}
\usepackage[utf8]{inputenc}
\usepackage[T1]{fontenc}
\usepackage{amsmath}
\usepackage{amsfonts}
\usepackage{amssymb}
\usepackage[version=4]{mhchem}
\usepackage{stmaryrd}

\begin{document}
\section*{APPLICATION A}
Question : An investor in a $40 \%$ tax bracket earns an after-tax return of $9 \%$. What must be the investor's pre-tax return?

\section*{Answer and explanation}
Let's use Equation 1 to solve for the investors pre-tax return. First we need to rearrange the formula, since we know the after-tax return, but not the pre-tax return.

$$
\begin{gathered}
r^{*}=r(1-T) \\
\frac{r^{*}}{1-T}=r
\end{gathered}
$$

Now, we need to plug in the variables: $T=0.40$, and $r^{\star}=0.09$ :

$$
\frac{0.09}{1-0.40}=r
$$

$$
\frac{0.09}{0.60}=r=0.15
$$

The pre-tax return for the investor is $15 \%$.

\section*{APPLICATION B}
Question : An investor in a $40 \%$ tax bracket on ordinary income invests in a product that earns a pre-tax return of $10 \%$. Sixty percent of the income is distributed as a capital gain that is taxed at $40 \%$ of the ordinary income tax rate. What is the investor's total after-tax return?

\section*{Answer and explanation}
The investor's total after-tax return is the weighted average of the after-tax returns of the return components. Sixty percent of the total return (i.e, $6 \%$ ) is taxed at a capital gains rate of $16 \%$ (found as $40 \% \times 40 \%$ ), leaving an after-tax capital gain return of $5.04 \%$. Forty percent of the total return (i.e., $4 \%$ ) is taxed at the ordinary rate of $40 \%$, leaving an after-tax ordinary income return of $2.40 \%$. The total weighted average is $7.44 \%$, found as the sum of the two components $(5.04 \%+2.40 \%)$. This can also be found as the pre-tax return of $10 \%$ reduced by the weighted average tax rate of $25.6 \%$. The average tax rate of $25.6 \%$ reflects the weighted average of $60 \%$ of the income being taxed as capital gains at $16 \%$, and $40 \%$ of the income being taxed at the ordinary rate of $40 \%$.

\section*{APPLICATION C}
Question : Consider an investor with a current and anticipated tax rate of $30 \%$ who anticipates withdrawing funds in 20 years. If the investor places money into a wrapper that offers tax deferment, how much will the after-tax annual rate of return improve through the use of the wrapper if the pre-tax rate is $8 \%$ and the time horizon is 20 years?

\section*{Answer and explanation}
There are two components we need to calculate in order to determine the rate of return improvement achieved by using the tax deferment wrapper. First, let's calculate the simple after-tax rate of return without using the tax deferment wrapper, which uses Equation 1.

$$
\begin{gathered}
r^{*}=r(1-T) \\
r^{*}=0.08(1-0.30) \\
r^{*}=0.08(0.70) \\
r^{*}=0.056
\end{gathered}
$$

The after-tax rate of return without using the tax deferment wrapper is 5.6\%. To calculate the after-tax rate of return with the tax wrapper, we need to turn to Equation 2.

$$
\begin{aligned}
& r^{*}=\left\{1+\left[(1+r)^{N}-1\right](1-T)\right\}^{\frac{1}{N}}-1 \\
& r^{*}=\left\{1+\left[(1+0.08)^{20}-1\right](1-0.30)\right\}^{\frac{1}{20}}-1 \\
& r^{*}=\left\{1+\left[(1.08)^{20}-1\right](1-0.70)\right\}^{\frac{1}{20}}-1 \\
& r^{*}=\left\{1+[(4.661-1](0.70)\}^{\frac{1}{20}}-1\right. \\
& r^{*}=\left\{1+[(3.661](0.70)\}^{\frac{1}{20}}-1\right. \\
& r^{*}=\{1+2.563\}^{\frac{1}{20}}-1 \\
& r^{*}=\{3.563\}^{\frac{1}{20}}-1
\end{aligned}
$$

The after-tax return with the tax wrapper is $6.56 \%$. The improvement is $6.56 \%$ minus $5.60 \%$ which is a difference of $0.96 \%$.

\section*{APPLICATION D}
Question : Consider an investor in a current tax rate of $35 \%$ who anticipates a reduced tax rate of $20 \%$ in 10 years (after retirement). If the investor places money into a wrapper that offers tax deduction and tax deferment, what will the investor's after-tax rate of annual return be if the pre-tax rate is $6 \%$ and the time horizon is 10\\
years?

\section*{Answer and explanation}
We need to solve Equation 3 to solve for the after-tax rate of annual return with the tax deferment wrapper.

$$
\begin{aligned}
& r^{*}=\left\{(1+r)^{N}\left[\frac{1-T_{N}}{1-T_{0}}\right]\right\}^{\frac{1}{N}}-1 \\
& r^{*}=\left\{(1+0.06)^{10}\left[\frac{1-0.20}{1-0.35}\right]\right\}^{\frac{1}{10}}-1 \\
& r^{*}=\left\{(1+0.06)^{10}\left[\frac{0.80}{0.65}\right]\right\}^{\frac{1}{10}}-1 \\
& r^{*}=\{(1.791)[1.231]\}^{\frac{1}{10}}-1 \\
& r^{*}=\{2.2041\}^{\frac{1}{10}}-1 \\
& r^{*}=\{1.0822\}-1=0.0822
\end{aligned}
$$

The after-tax rate of return with the tax deferment wrapper is $8.22 \%$


\end{document}