\documentclass[11pt]{article}
\usepackage[utf8]{inputenc}
\usepackage[T1]{fontenc}
\usepackage{amsmath}
\usepackage{amsfonts}
\usepackage{amssymb}
\usepackage[version=4]{mhchem}
\usepackage{stmaryrd}

\begin{document}
Four Potential Tax Effects of Wrappers

Different wrappers can offer different taxability of cash flows from investment products. The pre-tax internal rate of return, $r$, of an investment prior to consideration of taxes is found as the rate that discounts the anticipated inflows to being equal to the cost of acquiring the asset. The after-tax rate of return, $r^{*}$, is the analogous rate computed on after-tax cash flows.

This lesson examines the relationship between pre-tax and after-tax returns for four tax scenarios. In some cases, marginal income tax rates are assumed constant through time for an investor and are denoted as $T$. In other cases, two tax rates are considered, $T_{0}$ as the initial tax rate and $T_{N}$ as the terminal tax rate. All returns are expressed as annualized and annually compounded rates.

\section*{Tax Effects of Tax-Free Wrappers}
A tax-free wrapper takes an investment that would ordinarily be subject to income tax and allows tax-free accrual and distribution of income and capital gains. Roth individual retirement accounts (IRAs) in the United States and individual savings accounts (ISAs) in the United Kingdom are examples of these wrappers. The mathematics of these accounts is simplified because the after-tax return, $r^{*}$, equals the pre-tax return, $r$. Thus, investors using either a Roth IRA or an ISA in a fund that generates a pre-tax return of $10 \%$ can expect the after-tax value of their investment to grow at $10 \%$.

Tax-free wrappers do not generally offer tax deductibility of investment contributions. Although a tax-free return can be very attractive, as noted in the following text, the benefits of tax-deductible contributions to investors in high-income tax brackets may exceed the advantages of tax-free wrappers.

\section*{Tax Effects of Fully Taxed Wrappers}
Fully taxed investments refer to products for which income and gains are taxable in the year in which they accrue or are distributed. The after-tax return on a fully taxed investment is shown in Equation 1:


\begin{equation*}
r^{*}=r(1-T) \tag{1}
\end{equation*}


Thus, an investor in a $40 \%$ tax bracket earning a pre-tax return of $10 \%$ experiences an after-tax return of only $6 \%$.

It should be noted that in most jurisdictions, some components of investment returns are tax-free or partially taxed. For example, capital gains are often taxed at a proportion (e.g., 50\%) of the rate of fully taxed income items, especially in the taxation of long-term investments. When components of investment income are taxed differently, the after-tax return of the investment can be estimated as a weighted average of the after-tax return of each component by applying Equation 1 to each component.

\section*{Tax Effects of Tax-Deferral Wrappers}
Tax deferral refers to the delay between when income or gains on an investment occur and when they are taxed. Without wrappers, income is usually taxed when distributed, and gains are usually taxed when recognized (e.g., when a position is closed). Wrappers often defer taxation until funds are distributed from the wrapped product to the investor.

Consider the case of a product that defers all income and gains until the funds are fully distributed at a termination date $N$ years later. The after-tax return on this investment is a function of $r, T$, and $N$ :


\begin{equation*}
r^{*}=\left\{1+\left[(1+r)^{N}-1\right](1-T)\right\}^{1 / N}-1 \tag{2}
\end{equation*}


\section*{Tax Effects of Tax-Deferral and Tax-Deduction Wrappers}
An especially powerful wrapper for tax purposes is one that allows both immediate tax deduction of contributions and full deferral of taxes on income and gains until funds are withdrawn. Tax deduction of an item is the ability of a taxpayer to reduce taxable income by the value of the item. Retirement investment wrappers often offer these tax benefits, as do some insurance products (when contributions are classified as deductible premiums). The benefits can be astounding when the tax rate at withdrawal, $T_{N}$, is substantially less than the tax rate at contribution, $T_{0}$ :


\begin{equation*}
r^{*}=\left\{(1+r)^{N}\left[\left(1-T_{N}\right) /\left(1-T_{0}\right)\right]\right\}^{1 / N}-1 \tag{3}
\end{equation*}


Brackets have been placed around the terms involving the two tax rates, $T_{N}$ and $T_{0}$. Note that when tax rates do not change, a tax-deductible and tax-deferred investment wrapper enables investors to receive an after-tax rate of return equal to the pre-tax rate of return. When tax rates decline between contribution and withdrawal $\left(T_{N}<T_{0}\right.$ ), the after-tax rate of return exceeds the pre-tax rate of return. The intuition of this fascinating result is that the tax savings from the deductibility of contributions serve as an interest-free loan.


\end{document}