\documentclass[11pt]{article}
\usepackage[utf8]{inputenc}
\usepackage[T1]{fontenc}
\usepackage{amsmath}
\usepackage{amsfonts}
\usepackage{amssymb}
\usepackage[version=4]{mhchem}
\usepackage{stmaryrd}

\begin{document}
\section*{APPLICATION A}
Question : Consider a five-year zero-coupon cash-and-call position on the S\&P 500 Index that has an initial cost of $\$ 1,000$ and offers $\$ 1,000$ principal protection (ignoring counterparty risk). The product's payout will be the greater of $\$ 1,000$ and $\$ 1,000 \times(1+r)$, where $r$ is the total return (non-annualized) of the underlying index over the five-year life of the product. If the riskless market interest is 5\% (compounded annually), what is the value of the call option and the cash that replicates this product as a cash-and-call strategy (ignoring dividends)?

\section*{Answer and explanation}
Assuming that the position is efficiently priced and that the riskless market interest rate is 5\% (compounded annually), the present value of the minimum $\$ 1,000$ payout is $\$ 783.53$. Thus, the cash position at the start of the investment is $\$ 783.53$. The remaining value of the structured product (\$216.47) is attributable to the call option with a strike price of $\$ 1,000$.

\section*{APPLICATION B}
Question : An asset sells for $\$ 100$. A European knock-in call option on that asset has a strike price of $\$ 110$ and a barrier of $\$ 90$. Describe the option using the terms in the exhibit, Barrier Calls and Puts and describe that payoff under each of the following scenarios: (a) the asset moves monotonically to $\$ 120$; (b) the asset declines monotonically to $\$ 89$ before rising monotonically to $\$ 110$ at expiration.

\section*{Answer and explanation}
Answer: The option is a down-and-in call option. It pays nothing under scenario (a) because the option never knocks in; it pays nothing under scenario (b) because although the option becomes active, it does not finish in-the-money.

\section*{APPLICATION C}
Question : consider two indices: a gold index and a copper index. Consider a European option that pays $0 \%$ if the gold index has performance equal to or better than $-2 \%$ relative to the copper index. For each percentage point that the gold index return is worse than $2 \%$ below the copper index, the option pays $1 \%$ of its notional value. Describe the type of option and its strike price in terms of both calls and puts.

\section*{Answer and explanation}
Answer: The option is a spread option. In the case of a spread put, the strike price of the put is $-2 \%$, and the spread is defined as the performance of the gold index less the performance of the copper index. In the case of a spread call, the strike price of the call is $+2 \%$, and the spread is defined as the performance of the copper index less the performance of the gold index.


\end{document}