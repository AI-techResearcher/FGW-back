\documentclass[11pt]{article}
\usepackage[utf8]{inputenc}
\usepackage[T1]{fontenc}
\usepackage{amsmath}
\usepackage{amsfonts}
\usepackage{amssymb}
\usepackage[version=4]{mhchem}
\usepackage{stmaryrd}
\usepackage{hyperref}
\hypersetup{colorlinks=true, linkcolor=blue, filecolor=magenta, urlcolor=cyan,}
\urlstyle{same}

\begin{document}
Structured Product Valuation

This lesson begins by describing three methods of valuing structured products. The approaches to the estimation of structured product values follow the approaches used to value many complex derivatives.

\section*{Valuing Structured Products Using Dynamic Hedging}
Structured products are often valued using the dynamic hedging approach. In the dynamic hedging approach, a portfolio consisting of the structured product and its underlying asset is created. Dynamic hedging alters the portfolio weights through time to maintain a desired risk exposure, such as zero risk. The dynamics of this risk-free portfolio are often represented by partial differential equations; therefore, this approach is also referred to as the partial differential equation approach. The partial differential equation approach (PDE approach) finds the value to a financial derivative based on the assumption that the underlying asset follows a specified stochastic process and that a hedged portfolio can be constructed using a combination of the derivative and its underlying asset(s).

An example of a dynamically hedged portfolio is a long position in a stock that is initially hedged by a short position in four units of the call options on that stock when the delta of the call option is 0.25 . As the delta of the call option continuously changes through time, the number of short calls must be continuously changed to maintain the hedge. Thus, if the delta fell to 0.20 or rose to 0.50 , the option hedge would be adjusted to being short five calls or short two calls, respectively.

The PDE approach can be illustrated through a simple example. Consider a riskless security in a world of fixed and certain interest rates. The riskless security is a zero-coupon bond that pays $\$ F$ at time $T$. The first step is to express the change in the value of the riskless security. Since it is riskless, the change in price would be equal to the value of the security $(P)$ times the periodic interest rate, which can be factored to produce the following ordinary differential equation:

$$
d P / d t=r P
$$

The fact that the value of the bond must be $\$ F$ at time $T$ is a boundary condition. A boundary condition of a derivative is a known relationship regarding the value of that derivative at some future point in time that can be used to generate a solution to the derivative's current value. The boundary condition combined with the mathematics of ordinary differential equations generates the following solution to the price of the bond, $P$, at time $t$.

$$
P=F e^{-r(T-t)}
$$

In a similar fashion, the PDE approach uses one or more boundary values and a differential equation to generate a price model. There are two major differences between the actual PDE approach and the previous simple example: (1) The PDE approach uses partial differential equations that are based on changes in two factors, time and the price of the underlying asset; and (2) the PDE approach requires construction of a riskless portfolio. Note that in the simplified example, the bond itself was riskless, and hence there was no need to construct a riskless hedge.

Partial differential equations are based on continuous-time mathematics. By specifying the relationship between the changes in two or more variables through time, one can derive a functional relationship between their levels. The PDE approach (1) relates the stochastic process followed by an option to the process followed by its underlying asset, (2) constructs a riskless portfolio by combining the derivative and its underlying asset, and (3) solves for the price of the derivative by setting the return of the riskless portfolio to $r$ and imposing boundary conditions. ${ }^{1}$

Candidates wishing to explore the PDE approach further may be interested in the following material: To value a derivative, $V$, using the PDE approach, the underlying asset, $S$, is often assumed to be a stochastic process with instantaneous returns subject to a normally distributed random process. Ito's formula is used to derive a stochastic process for $V$, knowing that $V$ is a function of $S$. A risk-free portfolio containing the derivative and the underlying asset is formed, and its return can be set equal to the riskless rate, $r$.


\begin{equation*}
d(v+\Delta s) / d t=r \times(V+\Delta S) \tag{1}
\end{equation*}


This equation indicates that a portfolio consisting of one unit of the derivative and delta $(\Delta)$ units of the underlying asset earns the riskless return, $r$, on the investment required $(V+\Delta S)$. Relating the evolution of $V$ to the process followed by $S$ creates the following PDE, which can lead to a solution for the value of the derivative, $V$, once appropriate boundary conditions are imposed.


\begin{equation*}
\frac{\partial V}{\partial t}+r S \frac{\partial V}{\partial S}+\frac{\sigma^{2}}{2} \frac{\partial^{2} V}{\partial S^{2}}-r V=0 \tag{2}
\end{equation*}


In the case of a simple European option, Black and Scholes derived an analytic solution in the form of the well-known Black-Scholes option pricing model, in which the option price is a simple function of five underlying variables. The boundary conditions are that the call price is zero when $S$ (the price of the underlying stock) is zero; the call price approaches infinity as $S$ approaches infinity; and the value of the call option at expiration is max $\{S-K, 0\}$, where $K$ is the strike price. The solution is analytical because the model can be exactly solved using a finite set of common mathematical operations. In the case of the Black-Scholes option pricing model, the solution is analytical because the option's price is a relatively simple function of five underlying variables.

Complex options and complex structured products often lack an analytical solution. Cases involving complex underlying stochastic processes or numerous boundary conditions often require solutions through numerical methods. Numerical methods for derivative pricing are potentially complex sets of procedures to approximate derivative values when analytical solutions are unavailable. Numerical methods can be difficult. Solutions to derivative values are often estimated using the methods discussed in the following two sections: simulation and building blocks.

\section*{Valuing Structured Products with Simulation}
A powerful and popular approach to valuing complex financial positions is Monte Carlo simulation, introduced and discussed in the session, Measures of Risk and Performance. Consider a complex structured product with possible payouts that depend on the values of two or more underlying assets at various points in time through the product's life.

A solution to the value of such a complex product using the PDE approach may be intractable. However, it is relatively easy to estimate the product's value if the potential paths of the underlying assets can be reasonably estimated.

As a simplified example, a very large number of projected paths for the value of an asset could be formed under the assumption that its price followed a particular stochastic process. The assumed process is simulated under the assumption that investors are risk neutral. For example, while simulating the behavior of a common stock, the mean return of the process is set equal to the risk-free rate. The payoffs of a derivative on that asset could then be projected for each path. The discounted values of the derivative payoffs for each path could then be averaged to form an estimate of the current value for the derivative. Since the underlying asset is simulated under risk neutrality, the structured product's average payoff is discounted using the risk-free rate. The simulation approach can be a conceptually simple method of estimating the value of complex derivatives and complex structured products when analytical solutions are unavailable and numerical methods are complex.

\section*{Valuing Structured Products with Building Blocks}
The building blocks approach (i.e., portfolio approach) models a structured product or other derivatives by replicating the investment as the sum of two or more simplified assets, such as underlying cash-market securities and simple options. The value of the structured product is simply the sum of the values of its building blocks. The value of each building block is in turn estimated through observation of market prices or well-known derivative pricing equations (e.g., option pricing models).

The primary distinction between the building blocks approach and the dynamic hedging approach (PDE approach) is that the portfolio weights are regularly and dynamically adjusted to maintain the desired risk exposure in PDE.

In the building blocks approach, portfolios are formed using a static hedge. A static hedge is when the positions in the portfolio do not need to be adjusted through time in response to stochastic price changes to maintain a hedge. For example, a static hedge approach can be used to value a European put option using a portfolio of three assets: the underlying asset, a European call option with the same maturity and strike price as the put that is being valued, and a riskless bond. As indicated in previous sessions, put-call parity establishes that a short position in the stock, a long position in the call option, and a long position in a riskless bond will replicate the return from holding the put option. Note that the key to the building block approach is that the value of the portfolio at some horizon point (e.g., expiration of the options) will be equal to the value of the derivative that is being analyzed regardless of what happens to the values of the securities used to create the static hedge.

In practice, the building block positions necessary to replicate a complex structured product perfectly may not be available or may not be trading at informationally efficient values.

\section*{Two Principles from Payoff Diagram Shapes and Levels}
The exhibit, Equivalence of Two Strategies in the Structured Products with Exotic Option Features session, illustrates a few of the many different payoff shapes that structured products offer. The payoff diagram shape indicates the risk exposure of a product relative to an underlier. The shape of the payoff diagram can be analyzed by investors to ascertain the extent to which the product's payoffs align with the investor's risk preferences or the investor's market view of the return distribution of the underlying asset.

This exhibit, Equivalence of Two Strategies, does not indicate the level of the payoff diagram relative to the cost of the product. ${ }^{2}$ The diagrams do not indicate the option price (cost) relative to the strike price. The payoff diagram level determines the amount of money or the percentage return that an investor can anticipate in exchange for paying the price of the product. Thus, the investor can use the level of the payoff diagram relative to the cost of the product to estimate whether the product is attractively or unattractively priced.

Principle 1 is that any payoff diagram shape can be constructed given a sufficient availability of options. In other words, any relationship between a portfolio of options and a related asset can be engineered if there are sufficient derivatives with which to manage the exposure. Slopes can be mimicked using calls and puts; discontinuous jumps can be mimicked using binary options.

Principle 2 is that it is the level of the payoff diagram that dictates whether the product is overpriced, underpriced, or appropriately priced. In other words, the vertical level of the payoff diagram drives the relative magnitudes of the profits and losses; therefore, it is the level of the payoffs that determines the attractiveness of an exposure in terms of prospective returns.

The enormous spectrum of structured products available enables investors to locate products that best meet their preferences regarding risk (the shape of the payoff diagram). If an investor's market view turns out to be correct, then the variety of structured products serves the purpose of enabling the investor to better achieve attractive returns or other financial goals.

However, the enormous spectrum of structured products available can also play into the investor's behavioral biases. In other words, an investor analyzing a very large number of diverse structured products may substantially overestimate the value of some products and underestimate the value of other products. The spectrum of available products may lead an investor with behavioral biases into taking otherwise undesirable risks if the investor falsely believes that a particular product is underpriced. For example, investors subject to the behavioral trait known as overconfidence bias will tend to overweight structured products that appear underpriced based on the investor's market view even when those products are overpriced due to high fees. An overconfidence bias is a tendency to overestimate the true accuracy of one's beliefs and predictions.

\section*{Evidence on Structured Product Prices}
A key issue in complex structured products is whether the prices at which the investments are issued are fair. In other words, how do the actual prices of the products compare with the estimated prices of the products using market-based valuation methods? The high degree of complexity in some structured products makes valuation challenging and subject to discretion.

Deng and others examine the issue price of principal protected absolute return barrier notes (ARBNs) and find that the fair price of ARBNs "is approximately 4.5\% below the actual issue price on average."3 Geng Deng, Ilan Guedj, Craig J. McCann, and Joshua Mallett, "The Anatomy of Principal Protected Absolute Return Notes," Journal of Derivatives 19, no. 2 (2011): 61-70.

A white paper by McCann and Luo estimates that "between $15 \%$ and $20 \%$ of the premium paid by investors in equity-linked annuities is a transfer of wealth from unsophisticated investors to insurance companies and their sales forces." 4 Craig J. McCann and Dengpan Luo, "An Overview of Equity-Indexed Annuities," Securities Litigation \& Consulting Group, June 2006.

Some industry sources point to lower fees for some products than those indicated by the previously cited empirical analyses of particular products. For example, in the Bank of Scotland's A Guide to Structured Products, the "total fees \& expenses" component (i.e., building block) of its structured products is listed as representing $2 \%$ to $3 \%$ of the product's price. ${ }^{5}$ Bank of Scotland, "A Guide to Structured Products," January 2012, \href{https://www.bankofscotland.co.uk/sharedealing/filestore/}{https://www.bankofscotland.co.uk/sharedealing/filestore/} BoS Guide to Structured Products.pdf.


\end{document}