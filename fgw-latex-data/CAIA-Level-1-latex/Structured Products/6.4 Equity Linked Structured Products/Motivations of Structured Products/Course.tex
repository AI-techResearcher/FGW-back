\documentclass[11pt]{article}
\usepackage[utf8]{inputenc}
\usepackage[T1]{fontenc}
\usepackage{amsmath}
\usepackage{amsfonts}
\usepackage{amssymb}
\usepackage[version=4]{mhchem}
\usepackage{stmaryrd}

\begin{document}
Motivations of Structured Products

The session, Introduction to Structuring, and the session, CDO Structuring of Credit Risk, listed a total of six investor motivations for structured products. Those six motivations are repeated in Investor Motivations for Structured Products followed by two additional motivations.

The seventh motivation in Investor Motivations for Structured Products involves the investor's income taxes. As discussed earlier in this session, investment wrappers can have an effect on after-tax returns. In many jurisdictions, capital gain investment income is taxed at a lower rate than are other forms of income and in some jurisdictions long-term capital gains are not taxed at all for individual investors. Structured products can reduce the effective tax rates (i.e., increase the tax efficiency of the investment) in many circumstances by structuring cash flows such that the investor's income is directed toward preferred classifications and away from undesirable classifications.

Some jurisdictions impose taxes on transactions. For example, in the United Kingdom, there is a Stamp Duty Reserve Tax (SDRT) imposed on share transactions at a rate of $0.5 \%$, which is paid by both residents and nonresidents. Structured products can be designed to mitigate some transaction taxes, the eighth motivation in Investor Motivations for Structured Products.

\begin{enumerate}
  \item Risk management: Investors may be better able to manage risk through structured products.

  \item Return enhancement: Investors may be better able to establish positions that will enhance returns if the investor's market view is superior.

  \item Diversification: Investors may be better able to achieve diversification through structured products.

  \item Relaxing regulatory constraints: Investors may be able to use CDO structures to circumvent restrictions from regulations.

  \item Access to superior management: Investors may obtain efficient access to any superior investment skills of the manager of the CDO.

  \item Liquidity enhancement: Tranches of CDOs can be more liquid than the underlying collateral pool.

  \item Income tax efficiency.

  \item Transaction tax efficiency.

\end{enumerate}

\section*{Investor Motivations for Structured Products}
A primary investor motivation of the structured products discussed in this session is the ability of structuring to make additional investment opportunities available to an investor. The session, Introduction to Structuring details the ability of structured products to complete the market, or more precisely, to reduce the level of market incompleteness. Equity-linked structured products enable investors to achieve otherwise unavailable combinations of risk and return.

This investor motivation to structured products enables efficient access of investors to otherwise unavailable exposures. For example, structured products can be engineered to help investors tailor their exposures to match their market views.

Up to this point, the discussion of motivations has generally focused on the motivations of investors in structured products. The motivations of the issuers of structured products tend to focus on fee revenue and profitability. However, other motivations exist. Some issuers can issue uncollateralized structured products as a source of financing. To some issuers, structured products may offer lower financing costs, preferable risk exposures, or preferable maturities. For example, the session, CDO Structuring of Credit Risk on CDOs details the benefits of balance sheet CDOs, through which institutions can divest assets to free regulatory capital. ${ }^{1}$ The CAIA Association deeply appreciates comments and suggestions from Paolo Piccitto that were used in the revision of this session.


\end{document}