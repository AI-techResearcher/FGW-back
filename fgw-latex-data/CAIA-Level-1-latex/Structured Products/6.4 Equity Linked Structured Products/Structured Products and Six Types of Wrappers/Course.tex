\documentclass[11pt]{article}
\usepackage[utf8]{inputenc}
\usepackage[T1]{fontenc}
\usepackage{amsmath}
\usepackage{amsfonts}
\usepackage{amssymb}
\usepackage[version=4]{mhchem}
\usepackage{stmaryrd}

\begin{document}
Structured Products and Six Types of Wrappers

Financial institutions throughout the world are offering investors innovative structured products with complex payouts based on one or more market values, such as the returns of an equity index. One example might be an insurance-related product that guarantees to protect the investor against losses while offering upside returns based on the returns of the FTSE 100 index up to a certain limit. Large institutions offer these structured products using trademarked names along with descriptions of the potential attractiveness of each product in various market environments. This session refers to these products as equity-linked structured products, even though some of them have returns driven by market values other than equity values, such as interest rates or commodity prices. The session introduces and provides an overview of this large and growing sector of alternative investment opportunities.

Most of the structured products discussed in the sessions from Introduction to Structuring through CDO Structuring of Credit Risk emphasize the goal of transferring relatively simple risk exposures related to an asset or a portfolio from one party to another. Often this transfer serves the dual purpose of meeting the risk preferences of both the issuer and the investor.

Equity-linked structured products, as defined in this session, are distinguished from the structured products in the session, Introduction to Structuring to the session, CDO Structuring of Credit Risk by one or more of the following three aspects: (1) They are tailored to meet the preferences of the investors and to generate fee revenue for the issuer; (2) they are not usually collateralized with risky assets; and (3) they rarely serve as a pass-through or simple tranching of the risks of a longonly exposure to an asset, such as a risky bond or a loan portfolio.

The primary distinction of these equity-linked structured products is that while the issuers of the products may hedge their exposures by issuing the products, the main purpose for the transactions from the perspective of the issuer is fee generation, not risk management.

The structured products in this session represent a large and growing sector of investments. Estimates of the global market for structured products range from just over one trillion dollars to several trillion dollars, with annual issuances exceeding $\$ 100$ billion.

Structured products are often placed inside wrappers. A wrapper is the legal vehicle or construct within which an investment product is offered. As an example, for more than 30 years U.S. banks have issued insured certificates of deposit (CDs) that offer a low guaranteed minimum interest rate with the potential for higher interest based on the growth of a pre-specified index, such as the S\&P 500 Equity Index. These CDs are commonly referred to as market-linked, equity-linked, or indexed CDs. The wrapper in this example is a bank deposit. By using a bank deposit wrapper, U.S. investors can enjoy government protection against the counterparty risk of a bank default on the principal and any guaranteed interest.

The wrapper that is used to offer an investment typically has regulatory and tax consequences. BNP Paribas provides the following six examples of structured product wrappers in its Equities and Derivatives Handbook: ${ }^{1}$ BNP Paribas, BNP Paribas Equities \& Derivatives Structured Products Handbook, London 2010.

\begin{enumerate}
  \item Over-the-counter (OTC) contracts: Private contracts negotiated between the investor and the issuing institution. Like credit default swaps (CDSs), they are usually formed under an International Swaps and Derivatives Association (ISDA) framework (as discussed in the session, Credit Risk and Credit Derivatives).

  \item Medium-term notes/certificates/warrants: Low-cost securities that can be public or private. Many such securities are traded on major stock exchanges.

  \item Funds: A pooled investment vehicle with an objective of replicating a structured product. Funds may be public and may offer tax advantages.

  \item Life insurance policies: Life insurance policies embedded within structured products. The products are subject to investment restrictions but may offer tax advantages.

  \item Structured deposits: Offered through deposits at a financial institution, as illustrated in the previous CD example.

  \item Islamic wrappers: Legal envelopes that are Shari'a compliant. Common interpretations of this compliance include the avoidance of interest and speculation (or excessive interest and speculation), and the avoidance of investing in prohibited underlying activities.

\end{enumerate}

A key aspect of wrappers can be to give investors access to underlying investment opportunities that would otherwise not be available or would be less costeffectively accessed through other means. For example, an investor may be able to invest in a portfolio of hedge funds through an insurance wrapper, thereby circumventing minimum subscription requirements. Or a mutual fund might invest in commodities through a gold-linked note, thereby circumventing regulatory restrictions on direct holdings of illiquid assets.


\end{document}