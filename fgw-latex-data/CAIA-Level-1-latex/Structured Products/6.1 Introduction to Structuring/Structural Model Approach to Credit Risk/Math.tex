\documentclass[11pt]{article}
\usepackage[utf8]{inputenc}
\usepackage[T1]{fontenc}
\usepackage{amsmath}
\usepackage{amsfonts}
\usepackage{amssymb}
\usepackage[version=4]{mhchem}
\usepackage{stmaryrd}

\begin{document}
\section*{APPLICATION A}
Question : Consider a firm with $\$ 50$ million in assets and $\$ 25$ million in equity value. The firm has one debt issue: a zero-coupon bond maturing in one year with a face of $\$ 30$ million. A riskless zero-coupon bond of the same maturity sells for $90 \%$ of its face value. What is the value of the firm's debt? What is the value of a one-year put option on the firm's assets with a strike price of $\$ 30$ million?

\section*{Answer and Explanation}
The easiest and most important step is recognizing that the value of the firm's debt is simply the difference between $\$ 50$ million (the firm's asset) and $\$ 25$ million (the firm's equity value or the value of the call option) for a difference of $\$ 25$ million. Next, we break the value of the firm's debt into its two components: the riskless portion and the short put position. So we rearrange Equation 3 to solve for the value of the put.

$$
\begin{gathered}
\text { Assets }=[\text { Call }]+[\text { Riskless Bond }- \text { Put }] \\
\text { Assets }- \text { Call }=\text { Debt }=\text { Riskless Bond }- \text { Put }
\end{gathered}
$$

After rearranging the equation to solve for the value of the put, we can solve:

$$
\begin{gathered}
\$ 25 \text { million }=\text { Riskless Bond }- \text { Put } \\
\$ 25 \text { million }=\$ 27 \text { million }- \text { Put } \\
\text { Put }=\$ 2 \text { million }
\end{gathered}
$$

The value of the put option is $\$ 2$ million.

\section*{APPLICATION B}
Question : Consider a firm with $\$ 100$ million in assets and $\$ 60$ million in equity value. The firm's debt has a face value of $\$ 50$ million and a maturity of one year. The volitity of the firm's equity is estimated at $40 \%$. How would an analyst estimate the value of the firm's equity if the volatility of the firm's assets doubled?

\section*{Answer and Explanation}
Step 2 is based on Equation 4 and unlevers the current equity volatility from $40 \%$ to an asset volatility of $24 \%$ through multiplying the equity volatility ( $40 \%$ ) by the ratio of the value of the equity to the value of the assets ( $\$ 60$ million $/ \$ 100$ million, or 0.60 ). A doubling in the asset volatility increases the asset volatility to $48 \%$. The value of the firm's equity can be found using an option pricing model for a call option, with an underlying asset value of $\$ 100$ million, an underlying asset volatility of $48 \%$, a strike price of $\$ 50$ million, a time to expiration of one year, and the prevailing riskless rate.


\end{document}