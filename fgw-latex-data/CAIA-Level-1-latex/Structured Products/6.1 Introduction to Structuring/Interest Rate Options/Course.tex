\documentclass[11pt]{article}
\usepackage[utf8]{inputenc}
\usepackage[T1]{fontenc}
\usepackage{amsmath}
\usepackage{amsfonts}
\usepackage{amssymb}
\usepackage[version=4]{mhchem}
\usepackage{stmaryrd}

\begin{document}
Interest Rate Options

In this lesson, we describe and value interest rate caps and floors-two important types of interest rate derivatives. The interest rate derivatives market is the largest derivative market in the world.

\section*{Interest Rate Caps}
In an interest rate cap, one party agrees to pay the other when a specified reference rate is above a predetermined rate (known as the cap rate, which is similar to the strike price of a European call option). A caplet is an interest rate cap guaranteed for only one specific date. A cap is a series of caplets, and its price is equal to the sum of the prices of the caplets, which, in turn, can be valued using various term-structure models and a procedure similar to the Black-Scholes option pricing model. Issuers of floating-rate debt can buy these options contracts to hedge against the possibility of increases in short-term interest rates (i.e., against variable or floating interest rate risk). Caps, also known as ceilings, work as insurance, a service for which purchasers of these contracts pay sellers a premium. Equation 1 denotes the periodic payment for a cap based on $m$ periods per year:


\begin{equation*}
\text { Cap Payment }=\operatorname{Max}[(\text { Reference Rate }- \text { Strike Rate }), 0] \times \text { Notional Value } / m \tag{1}
\end{equation*}


To illustrate, consider a three-year interest rate cap. Party A buys the interest rate cap from party B with the following terms: The contract is for three years, the strike rate is $3 \%$, the reference rate is three-month LIBOR, settlement is every three months, and the notional value is $\$ 10$ million. Thus, every three months for the next three years, B will pay A if three-month LIBOR exceeds the strike rate of $3 \%$ at settlement. For example, suppose that the three-month LIBOR (i.e., the reference rate) is $4 \%$ on a settlement date. In this case, B will pay A $\$ 25,000$, which is given by $(4 \%-3 \%) \times \$ 10,000,000 / 4=\$ 25,000$. Suppose instead that three-month LIBOR is $2 \%$ on a settlement date. In this case, B will not make any payment to A. The maximum amount that A can lose from entering into this options contract is the up-front premium that A paid for the option.

\section*{Interest Rate Floors}
In an interest rate floor, one party agrees to pay the other when a specified reference rate is below a predetermined rate (known as the floor rate, which is analogous to the strike price of a European put option). A floorlet is an interest rate floor guaranteed for only one specific date. A floor is a series of floorlets, and its price is equal to the sum of the prices of the floorlets. Similar to caps, floors can be valued using derivative pricing models like the Black-Scholes option pricing model. These options contracts can be purchased by lenders in floating-rate debt to hedge against the possibility of declining short-term interest rates. A seller in an interest rate floor is compensated for guaranteeing the interest rate. Equation 2 denotes the periodic payment for a cap based on $m$ periods per year:


\begin{equation*}
\text { Floor Payment }=\operatorname{Max}[(\text { Strike Rate }- \text { Reference Rate }), 0] \times \text { Notional Value } / m \tag{2}
\end{equation*}


To illustrate, consider a four-year interest rate floor. Party A buys the interest rate floor from party B with the following terms: The contract is for four years, the strike rate is 3\%, the reference rate is three-month LIBOR, settlement is every three months, and the notional value is $\$ 20$ million. Thus, every three months for the next four years B will pay A if three-month LIBOR is less than the strike rate of $3 \%$ at settlement. For example, suppose that the three-month LIBOR (i.e., the reference rate) is $1 \%$ on a settlement date. In this case, B will pay A $\$ 100,000$, which is given by $(3 \%-1 \%) \times \$ 20,000,000 / 4=\$ 100,000$. Suppose instead that three-month LIBOR is $4 \%$ on a settlement date. In this case, B will not make any payment to A. The maximum amount that A can lose from entering into this options contract is the up-front premium that A paid for the option.

\section*{Interest Rate Options and Counterparty Risk}
The parties to caps or floors are exposed to interest rate risk, which can be used to hedge the risk present in an existing position. However, since caps and floors are not typically exchange-traded instruments, there will be some exposure to counterparty risk. With either instrument, the buyers are exposed to counterparty risk because the potential payoffs are paid by long options writers. However, because the options writers are paid up front, the writers have no exposure to the credit risk of the buyer.


\end{document}