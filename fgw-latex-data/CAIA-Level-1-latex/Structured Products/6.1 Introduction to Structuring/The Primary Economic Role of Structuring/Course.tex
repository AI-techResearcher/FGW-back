\documentclass[11pt]{article}
\usepackage[utf8]{inputenc}
\usepackage[T1]{fontenc}
\usepackage{amsmath}
\usepackage{amsfonts}
\usepackage{amssymb}
\usepackage[version=4]{mhchem}
\usepackage{stmaryrd}

\begin{document}
The Primary Economic Role of Structuring

What economic roles do structured products serve? A structured product exists because both the issuer of the structured product and the investor in the structured product were driven by one or more economic motivations. The primary direct motivation of the issuer is usually to earn fees-either explicit fees or implicit fees. However, other motivations of the issuer and the investor exist and are discussed throughout these four sessions on structured products. The motivation to the buyer could be risk management, tax minimization, liquidity enhancement, or some other goal. From the perspective of a financial economist, the primary economic role of structured products is usually market completion.

\section*{Completing Markets as an Economic Role}
One of the most central motivations to structured products is market completion. A complete market is a financial market in which enough different types of distinct securities exist to meet the needs and preferences of all participants.

For example, consider a world without any risk, uncertainty, taxes, or transaction costs. In such a world, the only difference between securities would be the timing of their cash flows. A complete market in this idealized example would exist when investors could assemble a portfolio that offered exactly the cash flows they desired on every possible date. Thus, a pension fund obligated to disperse cash on the first day of every month would be able to establish long and short positions in existing securities that generated cash on exactly those days that the cash was needed (i.e., the first day of every month).

In the United States, investors seeking riskless investments (in terms of U.S. dollars) tend to invest in U.S. Treasury bills, notes, and bonds. The market for Treasury securities contains many securities across a wide spectrum of maturity dates. But even ignoring the risk of a U.S. Treasury default, the Treasury market could not be described as being perfectly complete. The longest ordinary Treasury security is the 30-year Treasury bond, with an initial maturity of 30 years. What should an investor such as a pension fund do with liabilities requiring cash flows in perhaps 40 or 50 years? And there was a four-year period (2002 to 2006) when even the 30year Treasury bond was no longer being issued. The U.S. Treasury began issuing the 30 -year bond again based in part on the very function being discussed here-the benefits of completing a market by creating investment products that meet the needs of investors (in this case, mostly financial institutions with long-term time horizons).

In the idealized world of a complete market, individual investors could manage their wealth optimally because sufficient distinct securities would exist to allow any desired investment exposure. It should be noted that the financial market will never be fully completed. The term completing the market simply means that the market is being brought one step closer to completion by offering investors unique opportunities with which to manage their finances.

\section*{States of the World within Structured Products}
In the real world of uncertainty and asymmetric information, markets are highly incomplete. Incomplete markets are understood in the context of "states of the world." A state of the world, or state of nature (or state), is a precisely defined and comprehensive description of an outcome of the economy that specifies the realized values of all economically important variables. For example, a particular state of the world might be briefly summarized as being when an equity market index closes at $\$ \mathrm{X}$, a bond market index closes at \$Y, the gross domestic product (GDP) of a particular nation reaches $\$ \mathrm{Z}$, and so on. The concept is theoretical since it is impossible to fully describe the entire world or all states that could occur. However, the concept provides valuable insight into why many structured products exist.

To demonstrate, let's examine a highly simplified example in which an investor defines the states of the world on only three outcomes: her job, the level of the equity market, and the level of the debt market. One of the many states in this example might be an outcome in which global stock markets rise, interest rates fall, but the investor gets fired from her job. How can this investor prepare for this potential state of the world? One answer would be to purchase unemployment insurancealthough it might be very expensive or impossible to get large amounts of insurance against the economic consequences of being fired. The reason, of course, is that the insurance company would be concerned about moral hazard: the possibility that the insured would intentionally perform poorly at work in order to collect insurance. The point is that markets will always be in a condition of having substantial and important incompleteness.

\section*{Structured Products as Market Completers}
Although markets can never be complete, the primary role of structured products is to move them toward being more complete. For example, most investors would define the states of the world as including the condition of their physical properties. How can investors prepare for the potential that fire might destroy their real estate? The answer, of course, is to purchase fire insurance. Centuries ago, in a world with very incomplete markets, investors might not have been able to purchase fire insurance and so would have had to bear the very undesirable and highly diversifiable risk of losing substantial wealth due to fire. But in a complete market, investors could purchase fire insurance, a "security" that pays a substantial payoff in states in which the real estate burns and pays nothing in other states. This example illustrates that the primary economic role of insurance companies is to make the market more complete.

To summarize, people and organizations can be viewed as analyzing future scenarios of the world (i.e., states of the world) and estimating their probabilities. For risk management purposes, investors typically seek products that offer high payoffs in those states in which the investor's wealth would otherwise be low. For return enhancement purposes, investors might seek products that offer high payouts in states that the investor believes are unusually likely to occur. In both cases, the structuring of products serves the economic role of meeting the needs and preferences of these investors by completing the market. In other words, the structured products offer an otherwise unavailable combination of payoffs in various states that enables the investor to better manage risk and return.

In the context of alternative investments, financial institutions strive to meet the preferences of various investors by creating securities or products that move the market toward being more complete. As shown, insurance companies are an excellent example of a type of financial institution that addresses the deficiencies of incomplete markets. Major banks, insurance companies, and other financial institutions offer structured products that are tailored to the needs of individuals and institutions for risk management or risk enhancement purposes.

It should be noted that many simple financial derivatives, such as call options and put options, trade in the financial markets and can be used by market participants to manage basic risks of traditional assets, such as indices and individual securities. But when a market participant desires a product that is peculiar to individualized circumstances or preferences, structured products may be the solution that can be engineered to tailor a solution.


\end{document}