\documentclass[11pt]{article}
\usepackage[utf8]{inputenc}
\usepackage[T1]{fontenc}
\usepackage{amsmath}
\usepackage{amsfonts}
\usepackage{amssymb}
\usepackage[version=4]{mhchem}
\usepackage{stmaryrd}

\begin{document}
\section*{APPLICATION A}
Question : Suppose that the CDO depicted in the exhibit Simplified CDO Structure alters its portfolio such that the average coupon on the assets is $6 \%$. Ignoring defaults, fees, and expenses, how much annual income should be available to the equity tranche?

\section*{Answer and Explanation}
The answer is that $\$ 3.3$ million would go to the senior and mezzanine tranches, and $\$ 2.7$ million would be available for the equity tranche.\\
more explanation :\\
The equity tranche receives all the income from the assets that is left over after paying funds due to all the other tranches. Therefore, we need to determine overall CDO income, income to the senior tranche, and income to the mezzanine tranche. To determine the overall income we need to multiple $6 \%$ (average yield on assets) by $\$ 100$ million (total value of assets) for a product of $\$ 6$ million (net income from the assets of the CDO). Now, the cash flow to the mezzanine and senior tranche is based on their coupons and principal amounts (because there is enough income for both of the tranches. The income to the senior tranche is $3 \%$ (yield on senior tranche) multiplied by $\$ 70$ million for a product of $\$ 2.1$ million (income to the senior tranche). Now there is $\$ 3.9$ million of income left for the mezzanine and equity tranches. The income to the mezzanine tranche is $6 \%$ (yield on mezzanine tranche) multiplied by $\$ 20$ million for a product of $\$ 1.2$ million (income to the mezzanine tranche). That leaves $\$ 2.7$ million of income for the equity tranche, which is calculated by subtracting $\$ 6$ million (income to the CDO) by the sum of $\$ 2.1$ million (income to the senior tranche) and $\$ 1.2$ million (income to the mezzanine tranche).

\section*{APPLICATION B}
Question :Suppose that the CDO depicted in the exhibit, Simplified CDO Structure experiences defaults in $\$ 50$ million of the assets with $30 \%$ recovery. What will happen to the tranches?

\section*{Answer and Explanation}
First, note that the $30 \%$ recovery reduces the losses to $70 \%$ of $\$ 50$ million ( $\$ 35$ million). After the equity tranche is eliminated due to the first $\$ 10$ million in defaults, the mezzanine tranche is eliminated due to the next $\$ 20$ million in defaults. The remaining $\$ 5$ million of defaults will bring down the notional value of the senior\\
tranche from $\$ 70$ million to $\$ 65$ million. The senior tranche has first priority to the recovered value of the bonds (\$15 million), which may be distributed to the senior tranche, further reducing its notional value to $\$ 50$ million.

More explanation :

To calculate the recovered assets multiply $\$ 50$ million (the combined principal values of the bonds experiencing default) by 1 minus $30 \%$ (the $30 \%$ is the loss given default rate and the $70 \%$ is the recovery rate) for a product of $\$ 35$ million (the recovery on the defaulted assets). Now the equity tranche is the least protected from defaults (first to bear losses) and is currently valued at $\$ 10$ million. Therefore, the entire equity tranche is eliminated because $\$ 35$ million (defaulted assets) is greater than $\$ 10$ million (value of equity tranche). The next subordinated tranche is the mezzanine tranche, which is valued at $\$ 20$ million. There is $\$ 25$ million left of default losses to be covered, as we have allocated $\$ 10$ million to the equity tranche, which eliminated that tranche. Therefore, the mezzanine tranche is also eliminated because $\$ 25$ million (remaining defaulted assets after the equity tranche) is greater than $\$ 20$ million (value of mezzanine tranche). There is now $\$ 5$ million left of the defaults to be covered, which will impact the senior tranche. The senior tranche is valued at $\$ 70$ million. Thus, we will subtract the $\$ 5$ million remaining of the defaulted assets by the $\$ 70$ million value of the senior tranche reducing the value of the senior tranche to $\$ 65$ million.


\end{document}