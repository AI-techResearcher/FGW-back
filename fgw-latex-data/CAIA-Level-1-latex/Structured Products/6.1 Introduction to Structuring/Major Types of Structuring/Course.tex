\documentclass[11pt]{article}
\usepackage[utf8]{inputenc}
\usepackage[T1]{fontenc}

\begin{document}
Major Types of Structuring

As noted in the Overview of Financial Structuring lesson and in the session, What Is an Alternative Investment?, structured products are instruments created to exhibit particular return, risk, taxation, or other attributes. The key element of a structured product is that it offers an investor an altered exposure to one or more underlying assets. A collateralized debt obligation (CDO), detailed in the session, CDO Structuring of Credit Risk, is a good example of a structured product because the tranches of a typical CDO provide substantially altered risk exposures to the pool of assets underlying the CDO. However, a forward contract on an equity index would not be commonly described as a structured product because most forward contracts do not provide a substantially altered exposure to the fundamental characteristics of the underlying asset.

The sessions Credit Risk and Credit Derivatives through Equity-Linked Structured Products sessions cover three topics related to structuring: credit derivatives, CDOs, and equity-linked structured products. The next three sections briefly overview these topics.

\section*{Hedging with Credit Derivatives}
The Credit Risk and Credit Derivatives session discusses simple credit derivatives. Although simple credit derivatives, such as credit default swaps (CDS), are not usually referred to as structured products, they often serve similar roles. Credit default swaps allow for the cost-effective transfer of default risk.

Consider an investor who holds the bonds of XYZ Corporation but wishes to hedge the risk that the bonds of XYZ might default. The investor enters a CDS on the debt of XYZ Corporation with a major bank. In effect, the bank sells credit protection to the investor that functions much like an insurance contract. The CDS transfers the financial risk of XYZ's default from the credit protection buyer (the investor) to the credit protection seller (the bank). Now the investor is hedged. If the XYZ bonds that the investor holds experience default, the investor is made whole through the CDS. Of course, the bank receives compensation from the investor for providing the credit protection. CDSs help organizations manage their credit risk.

\section*{Structuring with Tranches}
CDOs are structures that partition the risk of a portfolio into ownership claims called tranches which differ in seniority. More senior tranches tend to be the first to receive cash flows and the last to bear losses. The key point of a CDO, therefore, is to engineer the risk of a portfolio into a spectrum of risks tailored to meet the needs, preferences, or market views of various investors. The tranching of CDOs performs a function quite similar to the capital structure of an operating corporation.

For example, the sponsor of a highly simplified CDO structure might buy bonds of XYZ Corporation and place them into the portfolio of a CDO structure (typically along with other corporate bonds). The CDO structure has various tranches with claims to receiving the coupons and principal payments from those bonds. Investors can select a tranche that best meets their preferences for risk and return.

\section*{Creating Structured Products}
The term structured products can be used as an umbrella term to describe a spectrum of innovative financial instruments, or it can be used more specifically to refer to specially tailored securities that are financially engineered to provide specific attributes, such as risk, that meet the preferences of one or more investors. An example of a structured product based on the equity of XYZ Corporation would be a security that paid an investor greater amounts of money if the value of XYZ equity performed poorly and lower amounts of money if XYZ performed well, but had a minimum value to the payout. The structured product might be ideal for an investor with a very large position in XYZ stock who is trying to avoid selling that position due to the potential tax liabilities from a sale. The investor desires downside protection while retaining some upside potential, so a major bank structures a product tailored to meet the investor's precise preferences with regard to size, timing, and payoff profiles.


\end{document}