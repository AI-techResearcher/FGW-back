\documentclass[11pt]{article}
\usepackage[utf8]{inputenc}
\usepackage[T1]{fontenc}
\usepackage{amsmath}
\usepackage{amsfonts}
\usepackage{amssymb}
\usepackage[version=4]{mhchem}
\usepackage{stmaryrd}

\begin{document}
Balance Sheet CDOs and Arbitrage CDOs

The distinction between balance sheet and arbitrage CDOs focuses on the purposes for the creation of the structure. Balance sheet CDOs are created to assist a financial institution in divesting assets from its balance sheet. Arbitrage CDOs are created to attempt to exploit perceived opportunities to earn superior profits through money management.

\section*{Three Goals for Issuing Balance Sheet CDOs}
Banks and insurance companies are the primary sources of balance sheet CDOs. Issuers have the economic motivation to use balance sheet CDOs to manage the assets on their balance sheets. In a balance sheet CDO, the seller of the assets, a financial institution, seeks to remove a portion of its loan portfolio or other assets from its balance sheet. The bank constructs an SPV to dispose of some of its balance sheet assets into the CDO structure. The CDO's asset manager is often the selling bank, which is hired under a separate agreement to manage the portfolio of loans that it sold to the CDO trust. In addition, the CDO trust will have a trustee whose job it is to protect the interests of the CDO tranche investors. This is usually not the bank or an affiliate due to conflict-of-interest provisions. The financial institution using a balance sheet CDO to divest assets may be looking to achieve one or more of three goals: (1) to reduce its credit exposure to a particular client or industry by transferring those risks to the CDO, (2) to get a much-needed capital infusion, or (3) to reduce its regulatory capital charges. By selling a portion of its loan or bond portfolio to a CDO, the institution can free up regulatory capital required to support those credit-risky assets.

\section*{The Balance Sheet CDO Structure}
Many balance sheet CDOs are self-liquidating. All interest and principal payments from the commercial loans are passed through to the CDO investors rather than reinvested in new assets. Other balance sheet CDOs provide for the reinvestment of loan payments into additional commercial loans to be purchased by the CDO trust. After any reinvestment period, the CDO trust enters into an amortization period, during which the loan proceeds are used to pay down the principal of the outstanding CDO tranches. A Balance Sheet CDO exhibit shows schematically the transactions between CDO investors (who, in this example, put up $\$ 100$ million in cash), the CDO issuer, and the lending institution.

\section*{A Balance Sheet CDO}
\begin{center}
\begin{tabular}{|c|c|c|c|}
\hline
\begin{tabular}{c}
Pension funds, \\
endowments, \\
hedge funds, \\
high-net-worth \\
individuals, \\
foundations \\
\end{tabular} & \begin{tabular}{c}
CDO Issuer \\
A trust or special \\
purpose vehicle. Holds \\
a portfolio of bank \\
loans purchased from \\
the lending institution \\
\end{tabular} & \begin{tabular}{c}
Lending Institution \\
The originator and holder \\
of leveraged loans as \\
well as the sponsor of the \\
CDO Trust. The bank \\
sells its loan portfolio to \\
reduce balance sheet risk \\
\end{tabular} \\
\hline
\end{tabular}
\end{center}

\section*{Arbitrage CDO Structures}
Whereas balance sheet CDOs are motivated by the desire of an institution such as a bank to divest assets, arbitrage CDOs are primarily motivated by a goal of successful selection and management of the CDO's collateral pool. A sponsor, such as a money management firm, establishes a CDO and takes an equity stake to earn a direct profit from the CDO. Arbitrage CDOs are designed to make a profit by capturing a spread for the equity investors in the CDO and by earning fees for money management services. The spread is captured as the excess of the higher-yielding securities that the CDO contains in its collateral portfolio and the yield that it must pay out on its fixed-income tranches issued to CDO investors. Put differently, an arbitrage profit is earned if the CDO trust can issue its tranches at a yield substantially lower than the yield earned on the bond collateral contained in the trust, such that the equity tranche of the trust receives expected residual income disproportionate to its risk. Further, money management firms earn fees on the amount of assets under management. By creating an arbitrage CDO, an investment management firm can increase both its assets under management and its income.

Another way to view the profit motive of an arbitrage CDO is in terms of market values rather than spreads and yields. The profit is earned by selling (issuing) securities (tranches) to outside investors at an aggregated price that is higher than that paid for all of the assets placed into the CLO/CBO structure as collateral. Thus, the value of the equity tranche to the issuer could be greater than the money the sponsor invested in the equity tranche.


\end{document}