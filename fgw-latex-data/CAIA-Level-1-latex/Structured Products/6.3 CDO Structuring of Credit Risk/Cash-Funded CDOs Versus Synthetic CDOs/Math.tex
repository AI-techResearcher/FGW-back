\documentclass[11pt]{article}
\usepackage[utf8]{inputenc}
\usepackage[T1]{fontenc}
\usepackage{amsmath}
\usepackage{amsfonts}
\usepackage{amssymb}
\usepackage[version=4]{mhchem}
\usepackage{stmaryrd}

\begin{document}
\section*{APPLICATION A}
Consider a bank with a $\$ 500$ million loan portfolio that it wishes to sell. It must hold risk-based capital equal to $8 \%$ to support these loans. If the bank sponsors a CDO trust in which the trust purchases the $\$ 500$ million loan portfolio from the bank for cash, how much reduction in risk-based capital will the bank receive if it finds outside investors to purchase all of the CDO securities?

\section*{Answer and Explanation :}
Since the bank no longer has any exposure to the basket of commercial loans, it has now freed $\$ 40$ million of regulatory capital ( $8 \% \times \$ 500$ million $=\$ 40$ million) from needing to be held to support these loans.

\section*{APPLICATION B}
Consider a bank with a $\$ 400$ million loan portfolio that it wishes to sell. It must hold risk-based capital equal to $8 \%$ to support these loans. If the sponsoring bank has to retain a $\$ 10$ million equity piece in the CDO trust to attract other investors, how much reduction in regulatory capital will result?

\section*{Answer and Explanation :}
Since the bank must take a one-for-one regulatory capital charge (\$10 million) for this first-loss position, only $\$ 22$ million (\$32 million - \$10 million) of regulatory capital is freed by the CDO trust.


\end{document}