\documentclass[11pt]{article}
\usepackage[utf8]{inputenc}
\usepackage[T1]{fontenc}
\usepackage{amsmath}
\usepackage{amsfonts}
\usepackage{amssymb}
\usepackage[version=4]{mhchem}
\usepackage{stmaryrd}

\begin{document}
Overview of CDO Variations

The basic collateralized debt obligation (CDO) structure was introduced in the session, Introduction to Structuring. This session takes a close look at CDO structures with a focus on CDOs that structure credit risk.

The CDO structure can be used to partition or distribute cash flows from the structure's assets and other positions to various tranches. The CDO structure has several variations, including the balance sheet $C D O$, the arbitrage $C D O$, and the market value CDO. All of these CDO structures share the feature that the entire risk of the portfolio is gathered within a special purpose vehicle (SPV) and then distributed to investors through various CDO securities or tranches. The session, Introduction to Structuring illustrated the CDO structure with stylized CDOs. In those simplified illustrations, there were only three tranches and very limited discussion of details and terms. This session explores CDOs in greater detail.

\section*{Credit-Related Motivations for CDOs}
The CDO structure was born in the late 1980s. One of the first major uses of the structure was to place a portfolio of high-yield (i.e., speculative or non-investmentgrade) bonds into a CDO structure to serve as its collateral and to issue securities (tranches) against that collateral. The portfolio of non-investment-grade bonds inside the CDO offers diversification benefits, and the remaining risks can be partially reduced through credit enhancements, which are discussed later. The risks of the portfolio are then distributed to various tranches. The tranches vary in the degree to which they bear credit risk, from junior tranches that bear the brunt of the risk to senior tranches that bear risk only from the most extreme levels of losses.

The key to the use of the CDO structure in the case of credit risk is that a large portion of the financing of the CDO (i.e., the security tranches) can be in the form of senior tranches, which contain relatively little credit risk compared to the CDO's underlying collateral portfolio. Thus, a large portion of a capital structure financing high-yield debt (or other credit-risky assets) can be rated as investment grade by the rating agencies. Many institutions, such as insurance companies and banks, are restricted from directly holding non-investment-grade debt. The use of CDO structuring can transform undesirable securities (high-yield debt) into desirable securities (highly rated senior tranches).

The high credit ratings given to senior tranches when the underlying collateral pool consists of non-investment-grade bonds are based on three primary justifications: (1) the senior position; (2) the diversification inherent in the collateral portfolio; and (3) credit enhancements that were structured into the deal, such as a major bank providing additional safety features.

\begin{enumerate}
  \item Risk management: Investors may be better able to manage risk through structured products.

  \item Return enhancement: Investors may be better able to establish positions that will enhance returns if the investor's market view is superior.

  \item Diversification: Investors may be better able to achieve diversification through structured products.

  \item Relaxing regulatory constraints: Investors may be able to use CDO structures to circumvent restrictions from regulations.

  \item Access to superior management: Investors may obtain efficient access to any superior investment skills of the manager of the CDO.

  \item Liquidity enhancement: Tranches of CDOs can be more liquid than the underlying collateral pool.

\end{enumerate}

\section*{Investor Motivations for Structured Products}
The exhibit above adds to the list of two economic motivations for structured products begun in the exhibit, Investor Motivations for Structured Products in the lesson, Collateralized Mortgage Obligations. The first two additional economic motivations in Investor Motivations for Structured Products, exhibit \#3 and \#4, relate to the CDO structuring of non-investment-grade debt just discussed. CDOs provide diversified investment opportunities to investors by assembling highly diversified collateral pools. Further, the CDOs allow financial institutions restricted from substantial investments in high-yield debt to obtain indirect exposure without violating regulations. Strong arguments can be made that using CDOs to circumvent regulations on high-yield debt offerings does not interfere with the goals of the regulations. It is reasonable to believe that financial institutions investing in a senior position of a CDO holding a highly diversified and credit-enhanced portfolio of high-yield debt are taking less risk than are financial institutions that concentrate their portfolios in the investment-grade bond market. In other words, in this situation, the regulations interfere with diversification, and CDO structuring enables institutions to achieve the benefits of better diversification.

Originally, these deals focused on bonds and were called collateralized bond obligations (CBOs). Following on the heels of CBOs, banks began to realize that they had assets on their balance sheets (e.g., leveraged loans) that could be repackaged into a collateral pool and sold to investors. Hence, collateralized loan obligations (CLOs) were born in the early 1990s. From these two streams of asset-backed securities, CDOs were born. A CDO can be a security that is backed by a portfolio of bonds and loans together. However, the term CDO is often used broadly to refer to any CLO or CBO structure. CDOs are usually designed to repackage and transfer risk, typically credit risk. But CDOs can also be used to transfer the uncertainty of insured mortgages with regard to the timing and size of prepayments. Often, CDOs of mortgages are called collateralized mortgage obligations (CMOs).

\section*{General Structure of CDOs}
In most CDOs, there is a three-period life cycle. First, there is the ramp-up period, during which the CDO trust issues securities (tranches) and uses the proceeds from the CDO note sale to acquire the initial collateral pool (the assets). The CDO's trust documents govern what type of assets may be purchased. The second phase is normally called the revolving period, during which the manager of the CDO trust may actively manage the collateral pool for the CDO, potentially buying and selling securities and reinvesting the excess cash flows received from the CDO collateral pool. The last phase is the amortization period. During the amortization period, the\\
manager of the CDO stops reinvesting excess cash flows and begins to wind down the CDO by repaying the CDO's debt securities. As the CDO collateral matures, the manager uses these proceeds to redeem the CDO's outstanding notes.

A major bank usually serves as the sponsor for the trust. The sponsor of the trust establishes the trust and bears the associated administrative and legal costs. At the center of every CDO structure is a special purpose vehicle. A special purpose vehicle (SPV) is a legal entity at the heart of a CDO structure that is established to accomplish a specific transaction, such as holding the collateral portfolio. In the United States, an SPV is usually set up as either a Delaware or a Massachusetts business trust or as a special purpose corporation (SPC), typically Delaware based. The SPV owns the collateral placed in the trust, and issues notes and equity (tranches) against the collateral it owns.

SPVs are often referred to as being bankruptcy remote. Bankruptcy remote means that if the sponsoring bank or money manager goes bankrupt, the CDO trust is not affected. In other words, the trust assets remain secure from any financial difficulties suffered by the sponsoring entity so that investors in the CDO tranches have a direct claim on the collateral. In structured products and elsewhere, investments that are bankruptcy remote provide enhanced liquidity by lowering the probability that an investment will become tied up in a bankruptcy process.

Each tranche of a CDO structure may have its own credit rating. Typically, most of the tranches of notes issued by the CDO receive an investment-grade rating by a nationally recognized statistical rating organization (NRSRO), with the exception of highly subordinated fixed-income tranches or the equity tranche. The equity tranche is the first-loss tranche. It is the last tranche to receive any cash flows from the CDO collateral and the first tranche on the hook for any defaults or lost value of the CDO collateral. Often, the issuer of the trust holds the equity tranche.

\section*{Terms and Details of CDOs}
The underlying portfolio or pool of assets (and/or derivatives) held in the SPV within the CDO structure is also known as the collateral or reference portfolio. Every CDO active manager must balance risk and return. The risk and return of credit-risky collateral assets are often described using three major terms: weighted average rating factor, weighted average spread, and diversity score.

Risk is typically measured with the weighted average rating factor of the underlying collateral pool and its diversity score. The weighted average rating factor measures the average credit rating of the underlying collateral contained in the CDO trust. Return is typically measured as the weighted average return spread over LIBOR.

The weighted average rating factor (WARF), as described by Moody's Investors Service, is a numerical scale ranging from 1 (for AAA-rated credit risks) to 10,000 (for the worst credit risks) that reflects the estimated probability of default. The rating factor increases nonlinearly, with small numerical differences between the higher ratings and large numerical differences between the lower ratings. The WARF of a portfolio is an average of those numbers across the securities weighted by market values. The CDO indentures contain covenants as to the average rating factor of the collateral pool.

A diversity score is a numerical estimation of the extent to which a portfolio is diversified. Portfolios of 100 securities can have substantially different levels of diversification, depending on the extent to which the securities are correlated. The diversity score is designed to indicate the number of uncorrelated securities in a hypothetical portfolio that would have the same probabilities of losses as the portfolio for which the diversity score is being computed. For example, if all 100 of the securities in a portfolio were perfectly correlated, the portfolio would behave as if it contained only one large position in one security and would have a diversity score of 1 . If all 100 of the securities were uncorrelated, the diversity score would be 100 . Values between these two extremes are computed using estimates of correlations.

The CDO indentures often have a weighted average spread over LIBOR that they are required to maintain. The weighted average spread (WAS) of a portfolio is a weighted average of the return spreads of the portfolio's securities in which the weights are based on market values. The spread of each security is computed as the excess of the security's yield over a specified reference rate, such as LIBOR, with a specified maturity. Historically, there is a very strong positive relation between rating factors and credit spreads. An active manager of a CDO can increase the WARF to get more yield (WAS). Conversely, the manager may increase the creditworthiness of the CDO collateral pool (lower the level of WARF), but only at the expense of yield (a lower WAS).

The tranche width is the percentage of the CDO's capital structure that is attributable to a particular tranche. The session, Introduction to Structuring discussed attachment and detachment points. The tranche width is a positive percentage that is computed as the distance between those two points. Thus, a $10 \% / 25 \%$ tranche would have a tranche width of $15 \%$ (i.e., $25 \%$ - 10\%). The process of structuring a CDO typically involves altering the risk of the structure's assets and the widths of various tranches in an attempt to earn credit ratings for the more senior tranches that allow those tranches to be sold to investors at attractive financing rates.


\end{document}