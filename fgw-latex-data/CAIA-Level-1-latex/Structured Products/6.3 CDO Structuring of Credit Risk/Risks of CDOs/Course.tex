\documentclass[11pt]{article}
\usepackage[utf8]{inputenc}
\usepackage[T1]{fontenc}
\usepackage{amsmath}
\usepackage{amsfonts}
\usepackage{amssymb}
\usepackage[version=4]{mhchem}
\usepackage{stmaryrd}

\begin{document}
Risks of CDOs

The risks associated with CDO trusts are considerable. The meltdown in the subprime mortgage market that began in 2007 and spilled over into the CDO marketplace with a vengeance illustrated these risks. By the end of 2008, large financial institutions such as Citigroup, UBS, and Merrill Lynch had written down more than $\$ 160$ billion of CDOs linked to the mortgage market. These are complicated instruments, and the risks are not always apparent. This lesson reviews the major risks associated with CDOs.

\section*{Risk from the Underlying Collateral}
The risk of the underlying collateral is the single greatest driver of risk associated with an investment in a CDO structure. This session and the previous two sessions on structured products have focused on credit risk. But the CDO structure can also be used to engineer commodity price risk, private equity risk, hedge fund risk, and interest rate risk, such as risks inherent with unscheduled principal payments.

Note that a CDO structure does not change the risk of the assets in the underlying portfolio. Instead, the structure merely distributes the risks of the collateral pool to the various tranche holders of the CDO. The risks of the collateral portfolio can change due to either changes in market conditions or changes in the composition of the portfolio itself, and CDO investors bear the risk that the true nature of the collateral will differ from the previously understood nature. In other words, the nature of the actual portfolio may stray from the intended nature of the portfolio. Further, in times of stress, CDO managers may be slow or reluctant to write down or write off the poorly performing investments contained in the CDO trust. The investor may need to perform an independent analysis to determine accurate values and risks of the actual portfolio.

Default rates are a key driver of returns to collateral portfolios exposed to credit risk. Further, collateral portfolio value is driven by the level of losses given default (i.e., the proportion of the underlying credit risk that is not recovered in the event of default). Low recovery rates can combine with high default rates to generate high credit losses.

\section*{Financial Engineering Risk}
The massive losses beginning in 2008 on CDO investments that had the highest possible credit rating (AAA) illustrate just how wrong financially engineered products can go. Financial engineering involves powerful tools that can generate enormous benefits. For example, the securitization and structuring of residential mortgages have been estimated to have substantially reduced the costs of financing homes for more than three decades. However, financial engineering can also be used, intentionally or unintentionally, to allocate risks in highly complex manners that are not well understood. Financial engineering risk is potential loss attributable to securitization, structuring of cash flows, option exposures, and other applications of innovative financing devices.

The financial engineering of insured residential mortgages in the 1990s facilitated a cost-effective supply of mortgage financing. CMOs played an important role in facilitating efficient mortgage financing, developing more and more sophisticated and complex structuring of tranches. By 1994, the complexities of CMO structures had soared to the point that many CMO tranches contained enormous interest-rate-related risks, even though the underlying collateral assets were virtually free of default risk. In 1994, a CMO crisis was triggered by rising interest rates; several large entities failed, and many others suffered enormous unanticipated losses. Interest rates reversed their course by the end of 1994, and further damage was averted.

Despite the grave lessons that should have been learned from the 1994 CMO crisis, a larger and more serious crisis emerged in 2007, primarily due to the default risk of subprime mortgages. At the heart of the subprime debacle were mortgage loan borrowers with substantially greater default risk than prime-grade borrowers. Small banks and mortgage lenders made these loans and then sold them into pools that were eventually financed by mortgage-backed securities (MBSs). Large investment banks purchased these MBSs and repackaged them yet again into a second pool, a CDO trust. The structures were used to slice and dice the risks of the subprime MBSs. When the underlying subprime mortgages began to default at much faster rates than previously experienced, the whole financial structure collapsed, bringing down Fannie Mae, Freddie Mac, and several major investment banks.

The lesson that was apparently not fully learned in 1994, and that was again taught in 2008, is that financial engineering is powerful and complicated. All market participants are directly or indirectly exposed to risks from the use of financially engineered products. Therefore, market participants should be aware of financial engineering risk and participate directly in engineered products with care and concern.

\section*{Correlation Risk}
CDOs are often called correlation products because the collateral pool of a CDO can reference numerous assets and because the correlations of the returns of those assets drive the aggregate risks of the portfolio. Higher correlation increases aggregate risk. Investors in a CDO are therefore exposed to correlation risk. The major risk of large losses comes from numerous defaults occurring at or near the same time. Thus, large losses occur when defaults are correlated. If defaults are uncorrelated, then the risk is diversified, and default rates tend to be steady. The safety of more senior tranches is maximized when correlation risk is minimized. That is, the senior tranche holders do not want numerous defaults to occur at the same time such that all subordinated tranches are wiped out. Rather, senior tranche holders want default risk diversified such that default losses do not reach the magnitude necessary to wipe out mezzanine tranches and more.

\section*{Risk Shifting}
Risk shifting is the process of altering the risk of an asset or a portfolio in a manner that differentially affects the risks and values of related securities and the investors who own those securities. A potential conflict of interest exists between the issuer of the CDO and the investors in the CDO tranches. The issuer may have an incentive to divest or otherwise place assets into the collateral pool that contain worse credit quality than is recognized by the investors. Also, the managers of the assets of a CDO may take on increasing risk or greater risk than initially indicated. Or, the manager may fail to take risk-reducing actions when the risks of the portfolio change due to market conditions. To reduce moral hazard, sometimes the equity tranche is held by the issuer. The idea is that equity tranche holders are then first in line to bear losses from asset defaults and have an incentive to lessen the default risks. However, as shown in the next section, ownership of junior tranches can actually encourage risk taking.

\section*{The Effects of Risk Shifting and Correlation on Tranches}
At first glance, it may appear that if higher-risk assets are placed into the collateral asset pool and/or if those assets are poorly diversified due to high return correlations, the higher risk will make all tranches less desirable. However, risk shifting in CDOs can have very different effects on different tranches. As discussed earlier, an equity tranche position in a CDO may be viewed as a call option. As a call option, equity tranches, and to a lesser extent other highly subordinated tranches, can actually benefit from increases in the risk of the collateral pool. The potential for equity holders to benefit from upward shifts in asset risks is detailed in the structural model approach in the session, Introduction to Structuring.

The relations between the level of risk of the collateral assets of a CDO and the values of the CDO's various tranches is interesting. Let's assume that the risks of a CDO's assets can be altered substantially without having an immediate impact on the value of the assets. Generally, the sum of the values of all of the tranches, including the equity tranche of a CDO, should tend to equal the value of the collateral pool. However, a large change in the risks of the assets (e.g., an increase in the WARF) can have immediate effects on the relative values of the tranches. Specifically, increases in the risks of the CDO's assets tend to transfer wealth from the holders of more senior tranches to the holders of less senior tranches.

It is intuitively obvious that senior tranches become less valuable as the volatility of the CDO's assets rise (with asset values held constant). The senior tranches have less probability of being fully paid while the coupons remain fixed. It is less obvious why the junior tranches might gain in value. However, if the value of the assets remains constant and if the value of the senior tranches declines, then the value of the junior tranches should rise. This effect is also consistent with the structural model's view of equity as a call option and the well-known result of option theory that call option values increase when the volatility of the underlying assets increases. The most junior tranche may be viewed as a long call option on the collateral assets. The most senior tranche may be viewed as a long riskless bond and short an out-of-the-money put option on the collateral assets. Higher volatility of the collateral pool helps the tranches that are long options (i.e., long vega) at the expense of the tranches that are short options.

Finally, note that higher risk in the collateral asset pool can occur both from higher-risk assets and from higher return correlations among the assets (i.e., reduced diversification). Thus, a lower diversity score can shift wealth from senior tranches to junior tranches even when the WARF is held constant. Note that a very well diversified portfolio will generate a low but constant default rate. A low but constant default rate will spare senior tranches from losses, as all of the losses will be absorbed by the junior tranches. A very poorly diversified portfolio gives senior tranche holders an increased chance of losses (when the assets experience very large losses) and junior tranche holders an increased chance of bearing few or no losses (when assets experience minimal losses).

\section*{Other CDO Risks}
The successful risk management of a CDO's portfolio requires understanding numerous potential risks. This section briefly surveys these risks.

A risk due to the difference in payment dates arises from a mismatch between the dates on which payments are received on the underlying trust collateral and the dates on which the trust securities must be paid. This risk can be compounded when payments on different assets are received with different frequencies, known as periodicity. This problem is often solved through the use of a swap agreement with an outside party, in which the trust swaps the payments on the underlying collateral in return for interest payments that are synchronized with those of the trust securities.

A type of basis risk occurs when the index used for the determination of interest earned on the CDO trust collateral is different from the index used to calculate the interest to be paid on the CDO trust securities. For instance, the interest paid on most bank loans is calculated on LIBOR plus a spread, but other assets may be based on certificate of deposit rates in the United States. The risk in this case is when a mismatch occurs and the indices underlying cash income from the collateral assets differ from the indices underlying payments to the tranche holders.

CDO tranches suffer when the collateral pool performs poorly. Collateral assets may perform poorly for several reasons. The market prices of collateral assets respond immediately to shifts in the levels, slope, or curvature of the yield curve of riskless rates. Yield curve shifts can cause the value of the collateral assets to change and can affect the cash flows available from reinvestment of cash flows from existing assets. Spread compression, when credit spreads decline or compress over time, reduces interest rate receipts from the CDO's collateral and may cause the CDO to face cash shortfalls even in the absence of defaults. A steeply upwardsloping yield curve can magnify the negative carry between the interest earned on the CDO's cash reserve accounts and the coupon rates of the CDO's tranches.

\section*{Modeling Credit Risk in CDOs}
Initially, it may appear that modeling credit risk should not be that different from modeling other risks, such as equity, interest rate, currency, and commodity risks. For instance, in theory (such as in the CAPM), it could be argued that one should be able to calculate the beta of the CDO collateral portfolio and use that beta to estimate the beta of the various tranches. However, credit risk displays a number of properties that are not shared by these other sources of risk; thus, a different type of model is required. First, default is a relatively rare event. Most corporations currently in existence have never defaulted. Therefore, there are limited observations available with which to estimate various statistics through historical analysis. Second, many defaults occur due to systematic factors, such as changes in macroeconomic conditions, rather than idiosyncratic factors, such as mismanagement at the firm level. Third, many of the financial institutions that invest in credit products are not able to hold diversified portfolios of credit products to eliminate the idiosyncratic risks of these securities. Fourth, in some cases (e.g., sovereign debt), credit risk may arise not just because of the inability of the counterparty to pay but also because of its unwillingness to do so.

The drivers of losses to a CDO of underlying credit risks are the default rate and loss rate given default. The default rate refers to the percentage of the collateral assets experiencing default. The loss rate given default, as discussed in the session, Credit Risk and Credit Derivatives, is the percentage of the defaulted security values that cannot be ultimately recovered. The primary method for ascertaining the risks of tranches due to default risk in the CDO portfolio uses a copula approach.

A copula approach to analyzing the credit risk of a CDO may be viewed like a simulation analysis of the effects of possible default rates on the cash flows to the CDO's tranches and the values of the CDO's tranches. The idea behind the copula model of CDO default risk is that defaults are generated by two normally distributed factors: an idiosyncratic factor and a market factor. The idiosyncratic factor takes on a different value for each credit risk (i.e., bond) and generates hypothetical defaults whenever the factor's value for that particular bond is sufficiently high. The market factor is common to all credit risks in the CDO portfolio and reflects the tendency of defaults to occur in unison.

A parameter set by the user of the copula model determines the relative weights of the two factors (i.e., idiosyncratic versus market). Taken together, along with a user-specified expected default rate, the model allows simulation of the probabilities of various default levels for the collateral pool. The estimated probabilities of various default levels are then combined with a user-supplied loss rate given default (i.e., 1 - recovery rate) to estimate the probabilities of losses to each of the\\
tranches in the CDO structure. Rating agencies have used the copula model to estimate return distributions for CDO tranches involving credit risk-both corporate bonds and uninsured mortgages. The copula model has been maligned as an important cause of the credit crisis that began in 2007. Specifically, the model was criticized for underestimating the risk of the most senior mortgage tranches from mortgage defaults. However, there is debate as to whether the difficulties, including apparently erroneous credit ratings, were caused by misunderstandings of the model, misspecification of the model, or misestimation of the model's parameters.

CDOs and other structured products are very powerful tools for engineering risk and other attributes. Those tools have been at the center of the 1994 CMO crisis as well as the financial crisis of 2007 to 2009 . Whenever the next financial crisis occurs, highly engineered products will undoubtedly be involved, as a transmitter of risk or even as a contributor to risk. Accordingly, these powerful tools need to be well understood by their users.


\end{document}