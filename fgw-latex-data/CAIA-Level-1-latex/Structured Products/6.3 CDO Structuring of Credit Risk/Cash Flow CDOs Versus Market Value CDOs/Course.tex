\documentclass[11pt]{article}
\usepackage[utf8]{inputenc}
\usepackage[T1]{fontenc}

\begin{document}
Cash Flow CDOs Versus Market Value CDOs

Under the arbitrage CDO structure, there can be a further subdivision between cash flow CDOs and market value CDOs. The primary distinctions relate to the extent to which the assets are selected to match the maturities of the liabilities or the extent to which assets are selected in an attempt to earn superior rates of return. Under a balance sheet CDO, the assets are selected according to the preferences of the financial institution wishing to divest the assets.

In a cash flow CDO, the proceeds of the issuance and sale of securities (tranches) are used to purchase a portfolio of underlying credit-risky assets, with attention paid to matching the maturities of the assets and liabilities. Typically, there is a fixed tenor (maturity) for a cash flow CDO's liabilities that coincides with the maturity of the underlying CDO portfolio assets. Cash inflows are anticipated to be received in time to meet the cash outflows required by the tranche holders. Thus, the CDO portfolio is managed to wind down and pay off the CDO's liabilities through the collection of interest and principal on the underlying CDO portfolio. The CDO manager should focus on maintaining sufficient credit quality for the underlying portfolio such that the portfolio can redeem the liabilities issued by the CDO.

In some cases, the cash flow arbitrage CDO is static. This means that the collateral held by the CDO trust does not change, remaining static throughout the life of the trust. There is no active buying or selling of securities once the CDO trust is established. For static CDOs, the key is minimizing the default risk of the underlying assets, because it is the return of principal from the underlying CDO portfolio securities that is used to pay back the CDO investors. However, most arbitrage CDOs are actively managed. This means that after the initial CDO portfolio is constructed, the manager of the CDO trust can buy and sell bonds that meet the CDO trust's criteria to enhance the yield to the CDO investors and reduce the risk of loss through default.

In a market value CDO, the underlying portfolio is actively traded without a focus on cash flow matching of assets and liabilities. The liabilities of the CDO are paid off through the trading and sale of the underlying portfolio. In a market value CDO, the portfolio manager is most concerned with the market value of the assets and the volatility of those market values, because precipitous declines in the CDO's portfolio reduce the CDO's ability to redeem its liabilities. In market value CDO structures, the return earned by investors is linked to the market value of the underlying collateral contained in the CDO trust.

Consider the example of a CDO trust that buys high-yield bonds. It is unlikely that the trust will be able to issue tranches that perfectly match the maturity of the highyield bonds held as collateral. The cash flows associated with a market value arbitrage CDO come not only from the interest payments received on the collateral bonds but also from the potential sale of these bonds to make the principal payments on the CDO securities. Therefore, the performance of the CDO securities is dependent on the market value of the high-yield bonds at the time of resale. Given this dependency on market prices, market value arbitrage CDOs use the total rate of return as a measure of performance. The total rate of return takes into account the interest received from the high-yield bonds as well as their appreciation or depreciation in value.


\end{document}