\documentclass[11pt]{article}
\usepackage[utf8]{inputenc}
\usepackage[T1]{fontenc}

\begin{document}
Other Types of CDOs

CDO structures can be used with various underlying assets including distressed debt, hedge funds, commodity exposures, and private equity, as well as being structured with single tranches.

\section*{Distressed Debt CDOs}
Default rates on debt increased in the United States during 2000 and 2001 and again beginning in 2008. This increase in default rates led to an increased availability of and interest in distressed debt, which in turn led to the development of distressed debt CDOs. The emergence of distressed debt CDOs followed the pattern of using the CDO structure to facilitate investments in diversified portfolios of credit-risky assets.

As its name implies, a distressed debt CDO uses the CDO structure to securitize and structure the risks and returns of a portfolio of distressed debt securities, in which the primary collateral component is distressed debt. Distressed debt CDOs usually have a combination of defaulted securities, distressed but unimpaired securities, and nondistressed securities. The appeal of the CDO structure is the ability to provide a series of tranches of collateralized securities that can have an investmentgrade credit rating, even though the underlying collateral in the CDO is mostly distressed debt. The CDO securities can receive a higher investment rating than the underlying distressed collateral through diversification, subordination, and one or several of the other credit enhancements described previously in this session. Investors are then able to diversify into the distressed debt market and to do so more effectively by choosing a distressed debt CDO tranche that matches their level of risk aversion.

Historically, the main suppliers of assets for distressed debt CDOs have been banks, which use the CDOs to manage the credit exposure on their balance sheets. Assets for a CDO are purchased at market value. When a bank sells a distressed loan or bond to a distressed debt CDO, it usually takes a loss because it issued the loan or purchased the bond at par value. It was after the issuance of the loan or bond purchase that the asset became distressed, resulting in a decline in market value. Banks are willing to provide the collateral to distressed debt CDOs for several reasons. First, it improves the bank's balance sheet by removing distressed loans and reducing its nonperforming assets. The divestiture of distressed debt also allows the bank to obtain regulatory capital relief by reducing the amount of regulatory capital it is required to maintain. Finally, the divestiture provides cash, or liquidity, to the bank.

\section*{Hedge Fund CDOs}
Another new application of the CDO structure has been the extension of CDOs to hedge funds. A collateralized fund obligation (CFO) applies the CDO structure concept to the ownership of hedge funds as the collateral pool. This innovation came as a result of the tremendous amount of capital pouring into the hedge fund market prior to the financial crisis that began in 2007. The CDOs of hedge funds facilitate diversification and allow investors to have professional management and reduced difficulties due to minimum investment sizes. Because CFOs are structured, they can offer access to hedge funds with a spectrum of risks and returns.

\section*{Single-Tranche CDOs}
Single-tranche CDOs provide a highly targeted structure of credit risk exposure. In a single-tranche CDO, the CDO may have multiple tranches, but the sponsor issues (sells) only one tranche from the capital structure to an outside investor. In a single tranche CDO, the sponsor could sell just one of these tranches and potentially keep the rest for its balance sheet. A single-tranche CDO uses a CDS, just like a regular synthetic CDO. The main difference is that in a single-tranche CDO, only a specific slice of the portfolio risk is transferred to the investors, rather than the entire portfolio risk.

Single-tranche CDOs allow even more customization for an investor, such as collateral composition, maturity of the single-tranche note, and weighted average credit rating. As a result, single-tranche CDOs are the most fine-tuned of any structure. For this reason, single-tranche CDOs are sometimes referred to as bespoke CDOs, or CDOs on demand.


\end{document}