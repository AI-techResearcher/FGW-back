\documentclass[11pt]{article}
\usepackage[utf8]{inputenc}
\usepackage[T1]{fontenc}
\usepackage{amsmath}
\usepackage{amsfonts}
\usepackage{amssymb}
\usepackage[version=4]{mhchem}
\usepackage{stmaryrd}

\begin{document}
Credit Enhancements

The measurement and analysis of credit risk are central aspects in the study of CDOs involving credit risk. Understanding the credit risk of the CDO's collateral portfolio is essential to understanding the risks of the tranches. This lesson discusses the measurement of that risk and the potential effects of risk changes on the values of the tranches.

One widely used method of modifying the risk of the various CDO tranches is to alter the securities in the collateral portfolio. However, other methods fall under the category of credit enhancements. Most CDO structures contain some form of credit enhancement to ensure that the majority of the securities issued to investors will receive an investment-grade credit rating. These enhancements can be internal or external. An internal credit enhancement is a mechanism that protects tranche investors and is made or exists within the CDO structure, such as a large cash position. Generally, credit enhancements are made at the expense of lower coupon rates paid on the CDO securities.

\section*{Subordination}
Subordination is the most common form of credit enhancement in a CDO transaction, and it flows from the structure of the CDO trust. It is an internal credit enhancement. Subordination is the process of protecting a given security (i.e., tranche) by issuing other securities that have a lower seniority to cash flows.

For instance, CDO trusts typically issue several classes or tranches of securities. The lower-rated, or subordinated, tranches provide credit support for the higher rated tranches. The equity tranche in a CDO trust is the first-loss position and therefore provides credit enhancement for every class of CDO securities above it. Junior tranches of a CDO are rated lower than senior tranches; however, they receive a higher coupon rate commensurate with their subordinated status and therefore greater credit risk.

CDO structures can also be used for collateral assets with little or no credit risk, such as insured mortgages. In these cases, subordination affects the timing of payments to the various tranches rather than the credit risk of those payments. In a traditional sequential-pay CDO, the principal of the senior tranches must be paid in full before any principal is paid to the junior tranches. This sequential payment structure is often referred to as a waterfall. As interest and principal payments are received from the underlying collateral, they flow down the waterfall: first to the senior tranches of the CDO trust and then to the lower-rated tranches. Subordinated tranches must wait for sufficient interest and principal payments to flow down the tranche structure before they can receive a payment.

\section*{Overcollateralization}
Overcollateralization refers to the excess of assets over a given liability or group of liabilities. Overcollateralization of a senior tranche occurs when there are subordinated tranches in a CDO. For example, consider a CDO trust with a market value of collateral trust assets of $\$ 100$ million. The CDO trust issues three tranches: Tranche A is the senior tranche and consists of $\$ 70$ million of securities; Tranche B consists of $\$ 20$ million of subordinated fixed-income securities and is paid after the senior tranche is paid in full; finally, there is a $\$ 10$ million equity tranche with the lowest seniority.

The level of overcollateralization is the ratio of the assets available to meet an obligation to the size of the obligation and all other obligations senior to that obligation. The overcollateralization rate for the senior tranche in this example is $\$ 100 / \$ 70=143 \%$. The numerator is the millions of dollars of assets. The denominator is the millions of dollars of value that would be necessary to pay off that obligation, as well as any other obligation of equal or greater seniority.

The funds used to purchase the excess collateral come from both of the subordinated tranches, Tranche B plus the equity tranche. The level of overcollateralization of Tranche B is $\$ 100 / \$ 90=111 \%$. The equity tranche provides the overcollateralization to Tranche B. Overcollateralization is an internal credit enhancement.

\section*{Spread Enhancement}
Another internal enhancement can be excess spread of the loans contained in the CDO collateral portfolio compared to the interest, or coupons, promised on the CDO tranche securities. In other words, the average coupon on the assets may exceed the average coupon on the tranches such that in the absence of default, the CDO should be able to receive more cash than it is required to distribute. This excess interest may be retained and serve to enhance the credit-worthiness of the outstanding tranches. The excess spread may arise because the assets of the CDO trust earn a premium for illiquidity or because the assets are of lower credit quality than the CDO securities and therefore yield a higher interest rate than the rate paid on the CDO securities. A higher yield on the trust assets may also result from a sloped term structure and mismatched assets and liabilities. This excess spread may be used to cover losses associated with the CDO portfolio. If there are no losses on the loan portfolio, the excess spread accrues to the equity tranche of the CLO trust.

\section*{Cash Collateral or Reserve Account}
A reserve account holds excess cash in highly rated instruments, such as U.S. Treasury securities or high-grade commercial paper, to provide security to the debt holders of the CDO trust. Cash reserves are often used in the initial phase of a cash flow transaction. During this phase, cash proceeds received by the trust from the sale of its securities are used to purchase the underlying collateral and fund the reserve account. It is sometimes argued that cash reserves are not the most efficient form of internal credit enhancement because they generally earn a lower rate of return than that required to fund the CDO securities.

\section*{External Credit Enhancement}
An external credit enhancement is a protection to tranche investors that is provided by an outside third party, such as a form of insurance against defaults in the loan portfolio. This insurance may be a straightforward insurance contract, the purchase of a put option by the CDO, or the negotiation of a CDS to protect the downside from any loan losses. The effect is to transfer the credit risks associated with the CDO trust collateral from the holders of the CDO trust securities to an outside company. These external credit enhancements from a third party guarantee timely payment of interest and principal on the CDO securities up to a specified amount and thereby enhance the credit ratings of the tranches.


\end{document}