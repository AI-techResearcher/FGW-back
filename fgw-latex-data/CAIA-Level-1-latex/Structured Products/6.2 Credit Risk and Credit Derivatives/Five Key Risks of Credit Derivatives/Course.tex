\documentclass[11pt]{article}
\usepackage[utf8]{inputenc}
\usepackage[T1]{fontenc}

\begin{document}
Five Key Risks of Credit Derivatives

Although credit derivatives offer investors alternative strategies to access credit-risky assets, they come with specialized risks. These risks apply both to credit options and to credit swaps.

\begin{enumerate}
  \item Excessive Risk Taking: First, there is the risk that traders or portfolio managers may use CDSs to obtain excessive and imprudent leverage, either by design or by chance. Since these are off-balance-sheet contractual agreements, excessive credit exposures can be achieved without appearing on an investor's balance sheet (although it should be discernible elsewhere in the accounts, such as in footnotes). As with all investments, proper accounting systems and other back office operations should be utilized.

  \item Pricing Risk: OTC credit derivatives can involve pricing risk, including risk from valuation subjectivity. As the derivative markets have matured, the mathematical models used to price derivative contracts have become increasingly complex. These models are dependent on assumptions regarding underlying economic parameters. Consequently, the pricing of credit derivatives is sensitive to the assumptions of the models and the specification of the parameters. Accounting and control procedures can be hampered by the lack of market prices.

  \item Liquidity Risk: Another source of risk is liquidity risk. Credit derivatives that are OTC contractual agreements between two parties can be illiquid. A party to a custom-tailored credit derivative contract may not be able to obtain the fair value of the contract in exiting the position. Further, the legal documentation associated with a CDS usually prevents one party from selling its share of the CDS without the other party's consent. For a standardized CDS, there are likely to be market makers providing liquidity.

  \item Counterparty Risk: Most OTC credit derivatives contain counterparty risk. Exchange-traded derivatives are backed not only by the parties on the other side of the contracts but also by institutions, such as brokerage firms and clearinghouses.

\end{enumerate}

In the case of OTC options, there is only one side of a transaction that can be at counterparty risk: the long position. The reason that the long side faces counterparty risk is that if the option writer defaults, the option becomes worthless. Note that the credit protection buyer only suffers a counterparty loss when all three of the following conditions occur: the referenced entity experiences a credit event, the counterparty to the derivative defaults, and there is insufficient collateral posted to cover the loss.

The reason that the short side does not face counterparty risk is that once the option has been purchased, there is no loss to the option writer from the buyer's insolvency. However, in the case of a swap, both sides of the derivative can face counterparty risk. After a swap is initiated, it is possible for market prices to move such that one side of the contract has a positive market value and the other side has a negative market value. The side with the positive market value clearly has counterparty risk. The side with the negative market value has counterparty risk to the extent that it is possible that the market value may become positive.

The primary counterparty risk created by a CDS is to the credit protection buyer. Losses to the credit protection buyer due to counterparty risk may be manifested in two ways. First, if there is a credit event on the underlying credit risky asset that triggers the CDS and the credit protection seller defaults on its obligations to the credit protection buyer, then the credit protection buyer can lose the entire amount due under the CDS. However, even if a trigger event has not occurred, the true value of the CDS to the credit protection buyer varies directly with the financial health of the credit protection seller. This is because reduction in the credit-worthiness of the credit protection seller decreases the probability that the seller will be able to fulfill its commitments to the buyer that are contained in the CDS.

Note that the probability that the credit protection seller will default at the same time that the referenced asset of the CDS experiences a trigger event can be relatively high if both events are driven by the same macroeconomic factors. In other words, a major credit crisis can cause CDS trigger events at a time when both the seller experiences distress and the buyer most needs the protection. It is ironic that a credit protection buyer with a goal of reducing credit risk can introduce a new form of credit risk, known as counterparty risk, into a portfolio from the purchase of a CDS.

Prior to the financial crisis that began in 2007, counterparty risk was considered a relatively small risk in credit derivative documentation. However, counterparty risk wreaked havoc on firms when Lehman Brothers, a huge financial institution, declared bankruptcy in September 2008. Even though many participants in the market had agreements with Lehman Brothers that required Lehman to post collateral, the bankruptcy of Lehman froze much of that collateral; years later, many counterparties to Lehman were still waiting for their collateral to be released through the bankruptcy process.

\begin{enumerate}
  \setcounter{enumi}{4}
  \item Basis Risk: Finally, credit derivatives may be viewed as having basis risk. In this context, basis risk is risk due to imperfect correlation between the values of the CDS and the asset being hedged by the protection buyer. The protection buyer takes on basis risk to the extent that the reference entity specified in the CDS does not precisely match the asset being hedged. A bank hedging a loan, for example, might buy protection on a bond issued by the borrower instead of negotiating a more customized, and potentially less liquid, CDS linked directly to the loan. If the value of the loan and the value of the bond are not perfectly correlated, there is basis risk. Another example is a bank using a CDS with a five-year maturity to hedge a loan with four years to maturity. The reason for doing so is potentially higher liquidity in CDSs with five years to maturity. However, the protection buyer takes on basis risk to the extent that the four- and five-year loan values experience different price movements.
\end{enumerate}

\end{document}