\documentclass[11pt]{article}
\usepackage[utf8]{inputenc}
\usepackage[T1]{fontenc}
\usepackage{amsmath}
\usepackage{amsfonts}
\usepackage{amssymb}
\usepackage[version=4]{mhchem}
\usepackage{stmaryrd}

\begin{document}
Credit Derivatives Markets

Derivatives are cost-effective vehicles for the transfer of risk, with values driven by an underlying asset. Credit derivatives transfer credit risk from one party to another such that both parties view themselves as having an improved position as a result of the derivative. Roughly, most credit derivative transactions transfer the risk of default from a buyer of credit protection to a seller of credit protection.

\section*{Three Economic Roles of Credit Derivatives}
The primary way that credit derivatives contribute to the economy and its participants is by facilitating risk management in general and diversification in particular. Consider the challenge faced by a major bank that has established a long-term relationship with a traditional operating firm. The bank provides many services to its clients, including payment services and credit. If the client is very large, the credit risk exposure of the bank to the firm through its loans to the firm may become substantial relative to the size of the bank. However, the bank may wish to be the sole direct creditor of the firm for several reasons. Perhaps the bank may view meeting all of the client's loan needs as increasing the chances that the bank will remain the firm's sole supplier of other services. Alternatively, the bank may wish to avoid the potential conflicts of interest and legal complexities of making loans to a firm alongside other creditors. As the sole creditor, the bank may be better able to pursue its self-interest. Credit derivatives can provide the bank with a cost-effective solution: The bank can make large loans to the firm and transfer as much risk as the bank desires to other market participants through credit derivatives. At the same time, other banks can transfer the credit risk of their portfolios to other market participants through credit derivatives. Through this process, banks and other institutions may be able to hold relatively well-diversified portfolios of credit risks while maintaining efficient and effective relationships with key clients.

Second, credit derivatives can provide liquidity to the market in times of credit stress. The availability and use of credit derivatives has soared in recent decades, with the result that credit risk has gradually changed from an illiquid risk that was not considered suitable for trading to a risk that can be traded like other sources of risk (e.g., equity, interest rates, and currencies).

Third, highly liquid markets for credit derivatives provide ongoing and reliable price revelation. Price revelation, or price discovery, is the process of observing prices being used or offered by informed buyers and sellers. Prices are the mechanism through which values of resources are communicated in a large economy. Ongoing and reliable price revelation regarding the credit risk of major firms serves as a highly valuable tool for decision making and enhances overall economic efficiency.

\section*{Three Groupings of Credit Derivatives}
Credit derivatives can differ in many ways. Following are three major methods for grouping credit derivatives.

Single-name versus multi-name instruments: Single-name credit derivatives transfer the credit risk associated with a single entity. This is the most common type of credit derivative and can be used to build more complex credit derivatives. Most single-name credit derivatives are credit default swaps (CDSs), which are the most popular way to allow one party to buy credit protection from another party.

Multi-name instruments, in contrast to single-name instruments, make payoffs that are contingent on one or more credit events (e.g., defaults) affecting two or more reference entities. Credit indices are examples of multiname credit instruments. CDSs on baskets of credit risk offer specified payouts based on specified numbers of defaults in the underlying credit risks. In the most common form of a basket CDS, a first-to-default CDS, the protection seller compensates the buyer for losses associated with the first entity in the basket to default, after which the swap terminates and provides no further protection.

Unfunded versus funded instruments: Unfunded credit derivatives involve exchanges of payments that are tied to a notional amount, but the notional amount does not change hands until a default occurs. An unfunded credit derivative is similar to an interest rate swap in which there is no initial cash purchase of a promise to receive principal but rather an agreement to exchange future cash flows. The most common unfunded credit derivative is the CDS. As discussed later in this session, unfunded instruments expose at least one party to counterparty risk. Unfunded instruments can be for a single name or for multiple names.

Funded credit derivatives require cash outlays and create exposures similar to those gained from traditional investing in corporate bonds through the cash market. Credit-linked notes, discussed later in this session, are a common type of funded instrument. They can be thought of as a riskless debt instrument with an embedded credit derivative.

Sovereign versus nonsovereign entities: The reference entities of credit derivatives can be sovereign nations or corporate entities. Credit derivatives on sovereign nations tend to be more complex because their analysis has to consider not only the possible inability of the entity to meet its obligations but also the potential unwillingness of the nation to meet its obligations. The modeling of the credit risk associated with sovereign risk involves political and macroeconomic risks that are normally not present in modeling corporate credit risk. Finally, the market for credit derivatives on sovereign nations is smaller than the market for other credit derivatives.

\section*{Stages of Credit Derivative Activity}
Both Smithson and Mengle have observed four stages in the evolution of credit derivatives activity. ${ }^{1}$ Charles Smithson, Credit Portfolio Management (Hoboken, NJ: John Wiley \& Sons, 2003); David Mengle, "Credit Derivatives: An Overview," Federal Reserve Bank of Atlanta, Economic Review 92, no. 4 (2007): 1-24. The first, or defensive, stage, which started in the late 1980s and ended in the early 1990s, was characterized by ad hoc attempts by banks to lay off some of their credit exposures.

The second stage, which began about 1991 and lasted through the mid- to late 1990s, was the emergence of an intermediated market in which dealers applied derivatives technology to the transfer of credit risk, and investors entered the market to seek exposure to credit risk. ${ }^{2}$ Karen Spinner, "Building the Credit Derivatives Infrastructure," Derivatives Strategy (Credit Derivatives Supplement), June 1997. An example of dealer applications of derivatives technology is the total return swap, which is detailed later in this session. Another innovation during this phase was the synthetic securitization structure. Synthetic securitization represents the\\
extension of credit derivatives to structured finance products, such as CDOs, in which the CDOs take credit risks through selling CDSs rather than through purchasing bonds.

The third stage was maturing from a new product into one resembling other forms of derivatives. Major financial regulators issued guidance for the regulatory capital treatment of credit derivatives, and this guidance served to clarify the constraints under which the emerging market would operate. Further, in 1999, the International Swaps and Derivatives Association (ISDA) issued a set of standard definitions for credit derivatives to be used in connection with the ISDA master agreement, as discussed in more detail later in the session. Finally, dealers began warehousing risks and running hedged and diversified portfolios of credit derivatives. During this stage, the market encountered a series of challenges, ranging from credit events associated with restructuring to renegotiation of emerging market debts.

The fourth stage centered on the development of a liquid market. With new ISDA credit derivative definitions in place in 2003, dealers began to trade according to standardized practices (e.g., standard settlement dates) that went beyond those adopted for other over-the-counter (OTC) derivatives. Further, substantial index trading began in 2004 and grew rapidly, and hedge funds entered the market on a large scale as both buyers and sellers.

The development of all these activities served to increase liquidity, price discovery, and efficiency in the market. And now, in the United States and elsewhere, legislation may require some credit derivatives to be exchange traded and backed by a clearinghouse; similar changes are likely to emanate from the European Union. This could take credit derivative activity into a fifth stage, from its OTC origins to the domain of the futures and derivatives exchanges.


\end{document}