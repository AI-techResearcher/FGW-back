\documentclass[11pt]{article}
\usepackage[utf8]{inputenc}
\usepackage[T1]{fontenc}
\usepackage{amsmath}
\usepackage{amsfonts}
\usepackage{amssymb}
\usepackage[version=4]{mhchem}
\usepackage{stmaryrd}

\begin{document}
Other Credit Derivatives

Generally, CDSs are not viewed as options, because in many ways they do not fit the classic view of options: They do not tend to require a single up-front premium, and they do not offer the buyer a right to initiate a transaction. However, in some ways CDSs are option-like. They tend to offer an asymmetric payout stream, much like an option: If no default or other trigger event occurs, then there is no related payment; and if there is an event, then there is a potentially large payment from the protection seller to the protection buyer. However, another key distinction between CDSs and classic options is that in most cases the decision to exercise a classic option and receive a potentially large payment is initiated at the discretion of the option buyer. In CDSs, payments are automatically triggered by specified events; there is no discretion on the part of the credit protection buyer as to whether the protection is provided or when it is provided. In summary, in credit derivatives, there can be a fine line between options and other derivatives.

The next three sections focus on credit options: credit derivatives that more closely resemble classic options. Like CDSs, credit options may be used for transferring or accumulating credit exposure. Whereas CDSs involve a series of payments from the protection buyer to the protection seller, credit options involve a single payment from the credit protection buyer to the credit protection seller that leads to an asymmetric payout (i.e., a potentially large payment from the credit protection seller to the credit protection buyer). The decision to exercise the option may be governed by the discretion of the option buyer, or it may be automatically generated by the terms of the contract and the specification of a trigger event. Thus, not all credit options give an option buyer the right but not an obligation to exercise the option.

\section*{Term of Credit Options}
A credit call option allows the holder to "buy" a credit-risky price or rate, whereas a credit put option allows the holder to "sell" a credit-risky price or rate. "Buy" and "sell" are in quotation marks here to reflect that the option may be on a rate, rather than a price, and that rates are generally not viewed as being bought or sold. Typically, the underlying asset is a credit-risky bond, and so a credit put option is the right to sell a credit-risky bond at a prespecified price. However, the underlying asset can also be a credit spread. For example, a credit call option can be the right to buy a credit spread at a prespecified level.

Since prices and spreads move inversely, a call option on a price is the opposite directional bet as a call option on a rate. Thus, while either a call or a put can reference a rate or a price, an entity wishing to purchase credit protection can establish a long position in a put option on a bond price or a call option on a credit spread. The two positions both purchase credit protection because prices and credit spreads move inversely. Credit options may trade on a stand-alone basis or may be a component of a security or a contract.

Binary options (sometimes termed digital options) offer only two possible payouts, usually zero and some other fixed value. Thus, binary options do not offer the payout structure of a classic option: limited downside risk with large upside potential. Accordingly, binary credit options offer a fixed payout if exercised or triggered; traditional options offer a payout based on prevailing market conditions, such as the difference between the market price of a credit-risky asset and the strike price of the option. In a binary option, there is little or no discretion regarding exercise of the option; the binary option's contract specifies the basis on which the final payout will or will not be made. As with other options, European credit options are credit options exercisable only at expiration, and American credit options are credit options that can be exercised prior to or at expiration.

\section*{Credit Put Option on a Bond Price}
Consider an American credit put option on a bond that pays the holder of the option the excess, if any, of the strike price of the option over the market value of the bond. The option is typically exercised if the bond experiences a credit event, such as a default. In OTC options, the contract specifies whether the exercise of the option is triggered by specified events or by the discretion of the option buyer. This option may be described as paying:


\begin{equation*}
\operatorname{Max}[0, X-B(t)] \text { in default, and } 0 \text { otherwise } \tag{1}
\end{equation*}


where $X$ is the strike price of the put option and $B(t)$ is the market value of the bond at default.

This option may be combined with the underlying credit risk to provide a hedged position. The combination of the underlying bond and the credit put option offers full repayment of the bond's principal if no credit event occurs, and payment of the option's strike price if a credit event does occur. Note that the option is not a binary option, which pays a fixed amount when a credit event occurs.

\section*{Call Option on a CDS}
Consider an American call option on a CDS. A long position in the option is established by paying a premium. The call option allows the holder of the call option to enter a CDS at the rate (strike) specified in the option contract. Suppose that a bank holds a credit-risky asset and seeks credit protection using a call option on a CDS on the risky asset. If the credit-worthiness of the bond issuer deteriorates or if overall credit market conditions deteriorate, the credit-risky asset's price falls and its credit spread widens. After the credit spread widens, the call option holder may choose to enter a CDS at the prespecified spread by exercising the option.

The combination of a call option on a CDS and the underlying bond offers a different payout than the combination of a CDS and the underlying bond. With the call option, the bondholder can benefit from improvements in credit; the bond price rises, and the option goes out-of-the-money. If credit conditions deteriorate, the call option can be exercised to purchase credit protection using a CDS at a prespecified rate. The combination of a credit-risky bond and a CDS is hedged such that the value is protected from loss but also prevented from benefiting if credit conditions improve. Of course, the option buyer pays a premium for this ability to benefit from bond price increases while being protected from bond price declines.

\section*{Credit-Linked Notes}
Credit-linked notes (CLNs) are bonds issued by one entity with an embedded credit option on one or more other entities. Typically, these notes can be issued with reference to the credit risk of a single corporation or to a basket of credit risks. A CLN with an embedded credit option on Firm XYZ is not issued by Firm XYZ. The CLN\\
is like a CDS in that it is engineered to have payoffs related to the credit risk of Firm XYZ while being legally distinct from Firm XYZ.

The holder of the CLN is paid a periodic coupon and then the par value of the note at maturity if there is no default on the underlying referenced corporation or basket of credits. However, if there is some default, downgrade, or other adverse credit event, the holder of the CLN receives a lower coupon payment or only a partial redemption of the CLN principal value. Note that the cash flows received by the holder of the CLN are not delivered by the underlying referenced corporation.

Thus, the long position in a CLN bears credit risk of the referenced entity or entities without being a direct part of any bankruptcy. By agreeing to bear some of the credit risk associated with a corporation or basket of other credits, the holder of the CLN receives a higher yield on the CLN than would be received on a riskless note. In effect, the holder of the CLN has sold some credit insurance (i.e., served as a credit protection seller) to the issuer of the note (i.e., the credit protection buyer). If a credit event occurs, the CLN holder must forgo some of the coupon or principal value to make the seller of the note whole. If there is no credit event, the holder of the CLN collects an insurance premium in the form of a higher yield.

CLNs appeal to investors who wish to take on more credit risk but are either wary of stand-alone credit derivatives such as swaps and options or limited in their ability to access credit derivatives directly. A CLN is a coupon-paying note. Unlike traditional derivatives, they are on-balance-sheet debt instruments that virtually any investor can purchase. Furthermore, they can be tailored to achieve the specific credit risk profile that the CLN holder wishes to target.


\end{document}