\documentclass[11pt]{article}
\usepackage[utf8]{inputenc}
\usepackage[T1]{fontenc}

\begin{document}
An Overview of Credit Risk

Credit risk is dispersion in financial outcomes associated with the failure or potential failure of a counterparty to fulfill its financial obligations. In contrast to equityrelated risk, which tends to have somewhat symmetrical payoff distributions, credit risk generally leads to payoff distributions that are substantially skewed to the left. In other words, the upside performance of a traditional position exposed to credit risk is limited to the recovery of the original investment plus the promised yield, whereas the downside performance could lead to the loss of the entire investment.

Default risk is the risk that the issuer of a bond or the debtor on a loan will not repay the interest and principal payments of the outstanding debt in full. A debtor is deemed to be in default when it fails to make a scheduled payment on its outstanding obligations. Default risk can be complete, in that no amount of the bond or loan will be repaid, or it can be partial, in that some portion of the original debt will be recovered.

Credit risk is influenced by both macroeconomic events and company-specific events. For instance, credit risk typically increases during recessions or slowdowns in the economy. In an economic contraction, revenues and earnings decline across a broad swath of industries, reducing the interest coverage with respect to loans and outstanding bonds for many companies caught in the slowdown. Additionally, credit risk can be affected by a liquidity crisis when investors seek the haven of liquid U.S. government securities. This was demonstrated clearly in the global financial crisis of 2007 to 2009.

Idiosyncratic or company-specific events are unrelated to the business cycle and affect a single company at a time. These events could be due to a deteriorating client base, an obsolete business plan, noncompetitive products, outstanding litigation, fraud, or any other reason that shrinks the revenues, assets, and earnings of a particular company.

As a company's credit quality deteriorates, a larger credit risk premium is demanded to compensate investors for the risk of default. In fact, the non-U.S. Treasury fixed-income market is often referred to as the spread product market. This is because all other U.S.-dollar-denominated fixed-income products (e.g., bank loans, high-yield bonds, investment-grade corporate bonds, and emerging markets debt) trade at a credit spread relative to U.S. Treasury securities. Similarly, risky debt denominated in other currencies trades at a credit spread over the bonds of the dominant sovereign issuer in that currency.


\end{document}