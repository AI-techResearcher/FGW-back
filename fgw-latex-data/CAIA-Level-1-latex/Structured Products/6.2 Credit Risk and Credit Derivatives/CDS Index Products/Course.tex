\documentclass[11pt]{article}
\usepackage[utf8]{inputenc}
\usepackage[T1]{fontenc}
\usepackage{amsmath}
\usepackage{amsfonts}
\usepackage{amssymb}
\usepackage[version=4]{mhchem}
\usepackage{stmaryrd}

\begin{document}
CDS Index Products

CDS indices are indices or portfolios of single-name CDSs. They are tradable products that allow investors to create long or short positions in baskets of credits and have now been developed globally under the CDX (North America and emerging markets) and iTraxx (Europe and Asia) banners. The CDX and iTraxx indices now encompass all the major corporate bond markets in the world.

CDS indices reflect the performance of a basket of assets-namely, a basket of single-name CDSs. For instance, CDX and iTraxx indices consist of 125 credit names. CDS indices have a fixed composition and fixed maturities. Equal weight is given to each underlying credit in the CDX and iTraxx portfolios. If there is a credit event in an underlying CDS, the credit is effectively removed from the indices.

As time passes, the maturity term of an index decreases, making it substantially shorter than the benchmark's term. A new series of indices is established periodically, with a new underlying portfolio and maturity date to reflect changes in the credit market and to help investors maintain a relatively constant duration, if they so choose. The latest series of the index represents the current on-the-run index. Markets have continued to trade previous series of indices, albeit with somewhat less liquidity.

The indices roll every six months. Investors who were holding an existing (i.e., on-the-run) index may decide to roll into the new index by selling the old index contract and buying the new one. The new index has a longer maturity and therefore a higher market value because the credit spread curve tends to be upward sloping. The composition of the new index is likely to be different from that of the old one. For example, some of the old credit names may have been downgraded since the first index was created.

The market for CDS indices is highly liquid, meaning that the spread on a CDS index is likely to contain a smaller liquidity premium than the premium embedded in a single-name CDS. In a rapidly changing market, the index tends to move more quickly than the underlying credits, because in buying and selling, index investors can express positive and negative views about the broader credit market in a single trade. This creates greater liquidity in the indices than with the individual credits. As a result, the basis to theoretical valuations for the indices tends to increase in magnitude in volatile markets. In addition, CDX and iTraxx products are increasingly used to hedge and manage structured credit products.

Just as in the case of a single-name CDS, the credit protection buyer of a CDS index pays a fixed premium (such as $4 \%$ per year of notional value), typically on a quarterly basis. But in the case of a CDS index, the notional value of the index is based on the combined notional values of 100 or more credit risks rather than on a single credit risk. Since the referenced asset is a portfolio of credit risks, the credit protection seller must make settlement payments for credit events on each and every credit risk in the index. Each credit event in a CDS index causes a payment and then lowers the notional value of the index.

For example, consider a CDS index on 125 investment-grade U.S. corporate bonds. Suppose that an institution with a $\$ 1$ billion portfolio of such bonds wishes to temporarily hedge part (\$100 million) of the portfolio's risk. The institution enters a position with $\$ 100$ million of notional value in the CDS index as a credit protection buyer. The credit protection buyer pays a fixed coupon on a quarterly basis to the protection seller. Suppose that during the first year, one of the 125 bonds underlying the index defaults and there is no recovery; that is, there are no proceeds to bondholders from the liquidation of the firm. The credit protection buyer would receive $\$ 800,000$ from the credit protection seller, and the notional value of the CDS index would drop by $\$ 800,000$. Note that the credit protection buyer and seller do not directly gain or lose when the notional value of the CDS index falls; the size of the notional value simply serves to scale the size of future payments. The CDS index functions much like a portfolio of 125 separate single-name CDSs.


\end{document}