\documentclass[11pt]{article}
\usepackage[utf8]{inputenc}
\usepackage[T1]{fontenc}
\usepackage{amsmath}
\usepackage{amsfonts}
\usepackage{amssymb}
\usepackage[version=4]{mhchem}
\usepackage{stmaryrd}

\begin{document}
\section*{APPLICATION A}
Question : In this example, a hypothetical transaction takes place between a hedge fund (the Fund) as a credit protection seller and a commercial bank (the Bank) as a credit protection buyer. The reference entity is an airline company (the Firm). The referenced asset is $\$ 20$ million of face value debt. The term of the transaction is seven years. In exchange for the protection provided over the next seven years, the Fund receives $2 \%$ of the notional amount per year, payable quarterly. The contract will be settled physically. This means that if a credit event takes place, the Bank will deliver $\$ 20$ million in face value of any qualifying senior unsecured paper issued by the Firm in return for a $\$ 20$ million payment by the Fund. Further, the contract will be terminated, and no further payments will be made by the Bank. Let's assume that default takes place after exactly three years. What cash flows and exchange take place?

\section*{Answer and Explanation}
Each quarter for 12 quarters, the Bank pays the Fund $\$ 100,000$. This value is found by multiplying the notional amount ( $\$ 20$ million) by the quarterly rate of $0.5 \%$ (i.e., 2\%/4). When the default occurs, the Bank delivers $\$ 20$ million in face value of the referenced bond to the Fund in exchange for $\$ 20$ million in cash. The CDS terminates immediately after these exchanges.


\end{document}