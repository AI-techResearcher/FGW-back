\documentclass[11pt]{article}
\usepackage[utf8]{inputenc}
\usepackage[T1]{fontenc}
\usepackage{amsmath}
\usepackage{amsfonts}
\usepackage{amssymb}
\usepackage[version=4]{mhchem}
\usepackage{stmaryrd}

\begin{document}
Multialternatives and Other Hedge Fund Liquid Alternatives

Liquid exposure to hedge fund strategies is a large and relatively new category of the liquid alternatives introduced in the The Environment of Alternative Investments session.

\section*{Emergence of Liquid Alternatives}
As the quality and the number of liquid strategies increase, the business model of funds of funds could be negatively affected. When individual investors can easily access transparent, liquid hedge funds with a low minimum investment, the advantages of a fund of funds to diversify investments and offer a low minimum subscription become much less compelling. Not only do liquid alternatives provide access to hedge fund strategies at lower fees and small minimum investments, but search costs are also reduced, as exchange-traded strategies have regulatory-mandated disclosures that allow both large and small investors to quickly access information regarding all exchange-traded hedge fund strategies.

Historically, most hedge funds have been offered mainly as illiquid and less-than-transparent private placements, sold to high-net-worth and institutional investors. Liquid alternative investments are innovative products that democratize alternative investments by allowing all investors to easily access these strategies in an exchange-traded and transparent format.

A hedge fund is an investment pool or investment vehicle that is privately organized in most jurisdictions and usually offers performance-based fees to its managers. Hedge funds can usually apply leverage, invest in private securities, invest in real assets, actively trade derivative instruments, establish short positions, invest in structured products, and hold relatively concentrated positions.

Investment managers in private placement vehicles have the ultimate flexibility, in that they can take as much or as little risk as investors or counterparties allow. In private placement formats, long lockup periods can encourage holding illiquid or complex assets, which may earn higher long-term returns. If there is a liquidity premium, in which less liquid assets tend to earn higher returns, then some fund managers may choose to hold most or all of their assets in less liquid holdings.

Whereas hedge funds are relatively unregulated, exchange-traded or liquid alternative investments must comply with local regulations, such as the Investment Company Act of 1940 (commonly referred to as "the '40 Act") in the United States, National Instrument 81-102 in Canada, and Undertakings for Collective Investment in Transferable Securities (UCITS) in Europe. These regulations specifically legislate minimum levels of liquidity and transparency, and maximum levels of leverage, derivatives, shorting, and investment concentration.

On average, private placement funds will have higher returns and higher risks due to the extra freedom allowed in the portfolio management process. Exchangetraded liquid alternatives will generally have lower returns and lower risks than private placements trading similar strategies, as the regulatory restrictions reduce investment manager flexibility. The investors attracted to liquid alternative products are those who may value the lower fees, greater transparency and liquidity, as well as the reduced risk of these products over the potentially higher returns from private placement products. There is also evidence of retail investors, or those not legally allowed to access private placement products due to low net worth levels, increasingly investing in liquid alternative products to diversify their portfolios in ways that were previously not possible.

\section*{UCITS Framework for Liquid Alternatives}
UCITS funds were introduced in the session, The Environment of Alternative Investments. UCITS-compliant funds were generally managed as long-only stock and bond funds for the first 15 years of the regulatory regime. When UCITS III was enacted in 2001, the regulations allowed the use of options, futures, and other strategies for the first time, which opened the door for managers to offer hedge-fund-like strategies in a UCITS-compliant vehicle. UCITS IV, enacted in 2011, allows fund mergers and master-feeder structures, which gives even greater flexibility to hedge fund managers. UCITS V, enacted in 2016, tightened requirements for custody of assets, requiring the use of a single depository to be responsible for explicit safekeeping requirements.

Although private placements offer the investment manager a great deal of flexibility when implementing an investment strategy, UCITS regulations have strict requirements for transparency, risk, and liquidity of compliant funds. UCITS regulations require reporting of holdings at least every two weeks to enable investors to view the composition of their funds on a regular basis.

In some aspects, UCITS regulations are less flexible than those of the U.S. Investment Company Act of 1940. For example, investments in property, private equity, and commodities are generally not permitted in UCITS funds. Leverage and concentration risks are also tightly controlled in UCITS funds, with leverage and risk typically limited to $200 \%$ of the NAV or risk of the underlying index. UCITS-compliant funds are required to be highly diversified, meaning that there are limits on the size of specific holdings within each fund. For example, UCITS regulations limit the holdings of a single European Union sovereign debt issuer to $35 \%$ of fund assets, the holdings of a single investment fund to $20 \%$, the holdings of illiquid investments to $10 \%$, and the amount of assets deposited within a single institution to $20 \%$. Finally, there is a $10 \%$ limit on holdings of a single corporate issuer, or $20 \%$ when derivatives are included.

The liquid alts regulations in Canada under National Instrument 81-102 are largely similar to those of UCITS, which allows liquid alternative funds to charge asymmetric incentive fees similar to those found in private hedge funds. Canadian funds are limited to borrowing at 50\% of NAV, leverage with an aggregate gross exposure of three times NAV, a concentration of $20 \%$ in a single issuer, and a hard cap on illiquid holdings of $15 \%$.

\section*{Funds Registered under the ' 40 Act}
Unlike the less liquid regime of private placements, funds compliant with the ' 40 Act regulations must offer regular liquidity, with redemptions being paid within seven days. Fund holdings must also be disclosed on a regular basis. Perhaps the most interesting aspect of the ' 40 Act regulations is that performance fees for funds that trade securities must be symmetric. That is, the sharing of investment profits by investment managers must be matched by the sharing of investment losses. In the private placement world, hedge fund or private equity managers frequently earn asymmetric incentive fees. Hedge fund managers typically have incentive fee structures that allow managers to receive $20 \%$ of all fund profits without the requirement of compensating investors for $20 \%$ of their losses. This asymmetric fee structure, which is very attractive to managers of private placement products, is not compliant with the regulations of the ' 40 Act. Although most managers of ' 40 Act funds do not charge performance fees, there are a few managers who charge symmetric performance fees. These performance-based fees are unlikely to ever be as popular with liquid alternative managers as the asymmetric performance fees that can be earned in the private placement world.

The '40 Act also places limits on each fund's leverage. There is a 300\% asset coverage rule, which requires a fund to have assets totaling at least three times the total borrowings of the fund, thus limiting borrowing to $33 \%$ of assets. A fund with $150 \%$ long positions and $50 \%$ short positions would comply with the regulations, whereas a fund with $200 \%$ long positions and $100 \%$ short positions would not. The concentration limits of the ' 40 Act are less complex than those of UCITS. Under the ' 40 Act, diversification regulations apply to $75 \%$ of the fund's portfolio, while the remaining $25 \%$ of the fund has no concentration limits. In the diversified portion of a fund, investment concentrations cannot exceed $5 \%$ of assets invested in one issuer, $25 \%$ in one industry, or more than $10 \%$ of the shares outstanding of a single company. Finally, there are limits on liquidity risk, specifying that no more than $15 \%$ of the fund can be invested in illiquid assets.

\section*{Availability of Liquid Alternative Strategies}
Let's review our four key hedge fund strategies in the context of the regulatory framework for liquid alternative investments.

Macro and managed futures funds are broadly available as liquid alternative products, given that the underlying holdings of futures and forward contracts can be extremely liquid. Funds using a ' 40 Act fund structure can access managed futures returns by holding funds as collateral and entering into swap agreements that transport the returns of managed futures into the liquid alternative structure. Liquid alternative funds, though, may be managed with less concentration and leverage than typical macro funds. Investors need to perform due diligence carefully on funds in this sector, as leverage and asymmetric incentive fees may be buried in the swap products that are often used inside the liquid alternative vehicles.

Event-driven hedge funds are a broad variety of strategies that focus on corporate events. Strategies such as merger arbitrage or activism are likely to be compliant with liquid alternative regulations but aren't yet broadly available in a liquid alternative format. Other strategies, such as distressed investing, are likely to be too illiquid to be offered in an exchange-traded format. Similarly, traditional private equity strategies are typically not available as liquid alternatives, as the time to exit may be years away, far longer than the daily or weekly liquidity expected by investors in liquid alternative funds.

Relative value hedge funds focus mainly on convertible bond arbitrage and fixed-income arbitrage strategies. These funds generally hold long positions in underpriced bonds and short positions in bonds or stocks meant to hedge the long positions. Because the divergences between the long positions and fair value are often small in these strategies, relative value funds are often managed at levels of leverage far in excess of those allowed under the UCITS or ' 40 Act regulations. As such, relative value funds are not generally available in a liquid alternative format.

Compared to other hedge fund styles, equity hedge fund strategies have attracted the largest AUM in the liquid alternative sector. Long/short equity funds are the most popular strategy, with equity market-neutral funds following closely behind. Perhaps the availability of equity funds in a liquid alternative format is so prevalent due to the similarity of the strategy in the private and public formats. That is, a large number of equity hedge funds are likely to be compliant with the regulatory requirements for alternative investments, even when managed in a private placement structure.

The liquid alternative space is broader than hedge funds, including funds with exposure to commodities, currencies, and nontraditional bonds. Nontraditional or unconstrained bond funds do not simply take long positions in investment-grade sovereign and credit securities, but may also invest in high-yield or emerging markets debt, often including leverage and short positions. These funds may increase exposure to credit risk while reducing the risk to changes in the level of investment-grade interest rates.

\section*{Engineering Illiquid and Leveraged Strategies into Multialternatives}
Multialternative funds are liquid alternative funds that offer a strategy similar to that of funds of funds, in that they diversify across fund managers and strategies. Long/short equity and multialternative funds comprise more than half of assets under management within the category of liquid alternatives focused on hedge-fundlike strategies.

Both investors and hedge fund managers may find multialternative funds attractive. Investors can buy a single multialternative fund as a diversified offering, similar to that of a fund of funds. Hedge fund managers may wish to serve as a sub-adviser to a multialternative fund, especially when their stand-alone strategy does not comply with the liquid alternative regulations.

For example, some hedge fund strategies, such as highly levered fixed-income arbitrage or event-driven, are difficult to manage within the leverage constraints of the ' 40 Act. However, these fund managers are finding success within the multialternative or multimanager structure, as the provisions of the ' 40 Act apply to the full fund, not to the individual strategies. That is, highly levered fixed-income strategies can be mixed with strategies that tend to use less leverage, such as equity long/short strategies. By combining strategies with varying levels of target leverage within a multialternative fund, the total fund may comply with the leverage provisions of the ' 40 Act without materially changing the strategy or positions preferred by more highly levered managers.

Managers may also prefer to be a sub-adviser to a multialternative fund rather than offering their own liquid alternative funds, as that role more clearly delineates between the manager's private placement and liquid alternative investments. Managers who offer both a private placement fund and a liquid alternative fund following similar strategies must be careful to demonstrate that the private placement fund adds value relative to the greater liquidity and lower fees offered in the exchange-traded market.

\section*{Performance of Liquid Alternative Vehicles}
Next, let's look closely at the empirical evidence: the studies of actual historical returns of alternative investments. We will look at return performance using various approaches: matched-sample tests, comparison of U.S. indices to one another, and comparison of U.S. indices to European Union indices.

We begin with matched-sample tests. Perhaps the best way to determine the true risk and return difference between liquid alternative funds and private placements is to find a subset of funds in which each manager offers both a hedge fund and a mutual fund running similar strategies. Of course, this matched-sample performance analysis technique is not perfect, as only a small number of funds will be included in each study.

A study by Cliffwater LLC does just that, comparing two investment vehicles offering the same strategy. ${ }^{1}$ Cliffwater LLC, "Performance of Private versus Liquid Alternatives: How Big a Difference?," 2013. Its finding is that, on average, liquid alternative funds have lower risks than limited partnership (LP) funds that employ the same strategy. This makes sense to us, as the regulatory restrictions constrain the investment flexibility of managers in the mutual fund vehicle. The good news for mutual fund investors is that net of fees, returns for liquid alternative funds trail returns of the LP fund offered by the same manager by less than $1 \%$ per year. Some strategies, such as equity long/short, credit, market-neutral, and macro and managed futures funds, had return differences between $0.42 \%$ and $0.94 \%$ per year. Other\\
strategies, namely event-driven and multistrategy funds, had return differences as large as $2.18 \%$. As previously stated, the higher leverage employed in these strategies exacerbates the difference between the more highly levered LP vehicle and the much less levered mutual fund.

A study by David McCarthy looks at a sample of LP and registered funds and finds that equity long/short funds have had similar returns and market exposures across the two fund types. ${ }^{2}$ David McCarthy, "Hedge Funds versus Hedged Mutual Funds: An Examination of Equity Long/Short Funds," Journal of Alternative Investments 16, no. 3 (2014): 6-24. This is good news for retail investors, as it means that retail investors in hedged equity mutual funds are getting a qualitatively similar experience to investors in equity hedge funds.

Finally, a study by Barclays segregates ' 40 Act fund offerings by whether the manager has had experience managing hedge funds or long-only mutual funds. ${ }^{3}$ Barclays, "Going Mainstream: Developments and Opportunities for Hedge Fund Managers in the ' 40 Act Space," April 2014. This study finds that all ' 40 Act funds with net long positions have earned a return of $0.9 \%$ per year, the HFRI Fund Weighted Composite Index have earned a return of $2.3 \%$ per year, and mutual funds run by hedge fund managers have earned $1.6 \%$ per year. This study also shows that during the crisis year of 2008 , liquid alternative funds experienced lower drawdowns than the average long-only mutual fund.


\end{document}