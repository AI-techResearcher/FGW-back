\documentclass[11pt]{article}
\usepackage[utf8]{inputenc}
\usepackage[T1]{fontenc}
\usepackage{amsmath}
\usepackage{amsfonts}
\usepackage{amssymb}
\usepackage[version=4]{mhchem}
\usepackage{stmaryrd}

\begin{document}
Investing in Funds of Hedge Funds

The primary purposes of funds of funds are to reduce the idiosyncratic risk of an investment with any one hedge fund manager and to tap into the potential skill of the fund of funds manager in selecting and monitoring hedge fund investments. Also, some funds of funds have continued access to investing with managers whose funds are closed to new investors. Fund access is an investor's ability to place new or increased money in a particular fund. The access to otherwise closed funds is a potential advantage of a fund of funds for an investor relative to the investor forming his own portfolio. Additionally, some funds of funds arrange to have a liquidity facility that can bridge the fund's mismatches between subscriptions and redemptions. A liquidity facility is a standby agreement with a major bank to provide temporary cash for specified needs with prespecified conditions.

\section*{Funds of Hedge Funds as Diversified Pools}
There is safety in numbers. An analogy is that as mutual funds are to single stocks, funds of funds are to single managers. Funds of funds offer diversification and professional management, just like mutual funds. Just as mutual funds invest in a large number of stocks across industries to diversify risk, funds of funds invest in multiple hedge fund managers and strategies to control risk. Investing in a single stock has some commonalities to investing in a single hedge fund manager in that there is a substantial amount of idiosyncratic risk. The company's industry or the hedge fund manager's style may be out of favor, or the CEO of the company or the fund manager may make some substantial mistakes. If a fund of funds invested in a single manager or strategy that experienced dramatic losses, the investor's losses would be reduced by the other investments that maintained or grew their value. Whereas concentrated investments in single stocks or hedge funds can lead to riches or ruin, diversified investments in mutual funds and funds of funds earn returns in a much narrower range, due to the reduction in idiosyncratic risk inherent in portfolios that contain multiple investments.

A fund of funds may seek to reduce operational risk and improve transparency for the fund of funds manager by placing the fund's money in managed accounts or separate accounts. Rather than investing as a limited partner and allowing the individual hedge fund managers as general partners to take custody of the assets of the fund of funds, the manager of the fund of funds can invest using a managed account or separate account that allows the hedge fund managers to trade the assets while the fund of funds controls the custody of the assets. This arrangement nearly eliminates the ability of the hedge fund managers to steal the funds or misrepresent performance. Because the assets are controlled by the fund of funds, the manager has perfect transparency, allowing the fund of funds manager to see all performance and positions in real time, which improves the manager's ability to manage risk and oversee investors. The liquidity of the fund of funds portfolio also increases, as the underlying hedge funds typically can't enforce lockup and gating provisions in a managed account framework.

Empirical evidence indicates that the returns to funds of funds have underperformed the returns of a broad hedge fund index. However, it may be inappropriate to directly make this comparison. Fung and Hsieh use hedge fund databases to document findings that funds of funds suffer less from survivor bias and selection bias than do individual hedge funds. ${ }^{1}$ William Fung and David A. Hsieh, "Performance Characteristics of Hedge Funds: Natural versus Spurious Biases," Journal of Financial and Quantitative Analysis 35, no. 3 (2000): 291-307. Hedge fund survivor bias was found to be $3 \%$ annually, whereas the survivor bias of funds of funds was $1.4 \%$ annually. Instant history bias was also less for funds of funds, $0.7 \%$, than for hedge funds, $1.4 \%$. In fact, Fung and Hsieh suggest that analyzing the returns to funds of funds may give a more realistic view of the performance of the hedge fund universe. There are several reasons that funds of funds would give a less biased view of hedge fund performance, including the following:

\begin{itemize}
  \item Survivor bias arises when returns from dead funds are removed from, or never included in, a database. Funds of funds that invested in funds that eventually liquidated, however, retain the returns of those funds in their track records.
  \item Similarly, instant history bias is reduced, as funds of funds count the returns to their investments in single hedge funds from the date of investment.
  \item Funds of funds use actual investment weights, which may better reflect the weights used by typical investors.
\end{itemize}

Because of the second layer of fees, the after-fee returns of funds of funds are, on average, lower than hedge fund returns. However, it would be a mistake to conclude that funds of funds do not add value. In addition to reducing the due diligence cost of building a diversified portfolio of single-manager hedge funds, funds of funds may have skill in evaluating the hedge fund managers. In one study, Ang, Rhodes-Kropf, and Zhao argue that funds of funds should not be evaluated relative to hedge fund returns from reported databases. ${ }^{2}$ Andrew Ang, Matthew Rhodes-Kropf, and Rui Zhao, "Do Funds-of-Funds Deserve Their Fees-on-Fees?" NBER Working Paper, 2007. Instead, the correct fund of funds benchmark is the return an investor would achieve from direct hedge fund investments individually, without recourse to funds of funds. Once fund of funds performance is compared to the correct benchmark, Ang and colleagues conclude that on average, funds of funds add value on an after-fee basis.

Ammann and Moerth find that larger funds of funds have statistically significant levels of higher returns and alpha than do smaller funds. ${ }^{3}$ Manuel Ammann and Patrick Moerth, "Impact of Fund Size on Hedge Fund Performance," Journal of Asset Management 6, no. 3 (2007): 219-38. In addition, the larger funds also have significantly lower standard deviations, which lead to higher Sharpe ratios. The authors surmise that the larger funds of funds have greater operational resources, which can be used to invest in stronger risk management, portfolio construction, and manager due diligence capabilities. The larger funds may also cater to a more institutionally focused clientele. If the large institutional investors demand lower fees from their fund of funds managers, this fee difference may explain a portion of the return advantage experienced by the larger funds of funds. Brown, Fraser, and Liang argue that the difference in returns between smaller and larger funds of funds represents economies of scale from the fixed cost of performing operational due diligence. ${ }^{4}$ Brown, Fraser, and Liang, "Hedge Fund Due Diligence."

\section*{Funds of Hedge Funds Have Varying Investment Objectives}
Funds of funds, like any other investor, can choose to build a portfolio with a wide range of investment objectives. HFR maintains indices that measure the performance of funds of funds, including composite, conservative, diversified, market-defensive, and strategic indices. The composite and diversified indices look most like the hedge fund universe, investing across the macro, equity, event-driven, and relative value strategies, and can be most closely compared to the HFRI Fund Weighted Composite Index of single-strategy hedge funds. Funds of funds included in the conservative index focus on strategies with lower standard deviations, such as equity market-neutral, relative value, and event-driven. Investors in funds included in the strategic index seek to maximize total returns, which is quite different from the risk-reduction goal espoused by many funds of funds. To earn these higher returns, strategic funds tend to make larger allocations to directional strategies, such as equity hedge or emerging markets funds.

Managers of funds of funds included in the market-defensive index seek returns that are uncorrelated to stock and bond markets and have lower downside risk. Defensive funds are likely to have minimal investments in event-driven and relative value strategies, as these managers prefer to overweight investments in macro, systematic diversified, and short selling funds.

Although the vast majority of funds of funds are diversified across a number of strategies, some funds of funds eschew this diversification to focus on a single sector. The most popular of these focused funds invest only in equity strategies, only in managed futures, only in smaller and emerging managers, or only in funds within a specific geographic region. These single strategy or sector-focused funds of funds may be attractive to investors who seek the specific return profile of one strategy, such as managed futures, but believe that it is important to invest in a number of managers to reduce the fund-specific risk.

\section*{Funds of Funds as Venture Capitalists}
In some cases, it can be difficult to tell the difference between private equity funds and hedge funds. Within the specific strategies of distressed investments or equity activists, the line between private equity funds and hedge funds becomes increasingly blurred, especially when hedge funds invest in private securities or private equity funds invest in public securities.

Some funds of funds also blur the line between hedge fund and private equity investments. Seeding funds, or seeders, are funds of funds that invest in newly created individual hedge funds, often taking an equity stake in the management companies of the newly minted hedge funds. One reason that a seeding fund may create new funds is to obtain transparency and capacity in its underlying hedge fund managers, which can be difficult to obtain with existing hedge funds. Perhaps the best way for a fund of funds to guarantee transparency and capacity over the long run with specific hedge fund managers is for the fund of funds to own a stake in the hedge fund management company.

Further, although hedge fund managers are experts at trading strategies, not all hedge fund managers have the time, connections, or skill to raise funds, and some may not have the resources or knowledge to build the infrastructure of a new hedge fund. Funds of funds are experts at raising capital from investors and structuring new investment vehicles. These complementary needs and skills can form the basis for a seeding relationship, or an incubating relationship, between a fund of funds and a start-up hedge fund manager. In a seeding relationship, the fund of funds may provide the fledgling hedge fund manager with $\$ 20$ million or so in capital, in addition to the legal and accounting documents, infrastructure, and relationships needed to start the hedge fund. The fund of funds manager may also serve as a third-party marketer, soliciting investors for the new hedge fund. In return, the hedge fund manager guarantees capacity to the fund of funds, even when the hedge fund has closed its doors to other investors. The fund of funds also has an equity stake in the hedge fund manager, which may earn the fund of funds $20 \%$ of the hedge fund's total fees and/or the value of the firm upon the sale of the hedge fund management company to an external investor.

The seeding activity of a fund of funds may eventually reach 10 managers across a number of strategies. At $\$ 20$ million per manager, the fund of funds has $\$ 200$ million of investor capital placed with the underlying managers, quite similar to a traditional fund of funds without the seeding activity. The seeding fund of funds earns the return to the underlying hedge fund portfolio, perhaps at preferential fees. In addition to the return on the hedge fund portfolio, the fund of funds also receives an equity kicker. To the extent that any of the underlying managers becomes extremely successful, perhaps raising $\$ 500$ million in investor capital, the value of the fee and equity sharing agreement with the fund of funds can become quite valuable, possibly exceeding the return on the investment in the underlying hedge fund strategy.


\end{document}