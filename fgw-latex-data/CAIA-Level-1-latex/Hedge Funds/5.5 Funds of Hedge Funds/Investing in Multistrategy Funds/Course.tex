\documentclass[11pt]{article}
\usepackage[utf8]{inputenc}
\usepackage[T1]{fontenc}
\usepackage{amsmath}
\usepackage{amsfonts}
\usepackage{amssymb}
\usepackage[version=4]{mhchem}
\usepackage{stmaryrd}

\begin{document}
Investing in Multistrategy Funds

A growing number of hedge fund managers are adopting a multistrategy approach to investing, in which a single hedge fund diversifies its trading and its positions across the macro and managed futures, event-driven, relative value, and equity hedge fund strategies. In many cases, the multistrategy fund designates one portfolio manager to allocate funds across strategies to various sub-managers, moving assets across teams trading each of the underlying strategies.

\section*{Incentive Fees as a Potential Advantage of Multistrategy Funds}
The key advantage of a multistrategy fund over a fund of funds is the lack of an explicit second level of fees. Many multistrategy funds charge fees similar to those of a single-strategy hedge fund manager, such as 1.5 and 17.5. Although funds of funds pay each of their underlying managers similar fees, the fund of funds manager also earns an additional fee, perhaps 0.5 and 7.5. The second layer of fees can cause a fund of funds to have total fees of 2 and 25.

Reddy, Brady, and Patel discuss the importance of fee netting when evaluating multistrategy funds versus funds of funds. ${ }^{1}$ Girish Reddy, Peter Brady, and Kartik Patel, “Are Funds of Funds Simply Multi-Strategy Managers with Extra Fees?" Journal of Alternative Investments 10, no. 3 (Winter 2007): 49-61. Most multistrategy funds charge the incentive fee on the aggregated returns of the combined portfolio of underlying strategies. Fee netting, in the case of multistrategy fund, is when the investor pays incentive fees based only on net profits of the combined strategies, rather than on all profitable strategies. This is a distinct advantage over a fund of funds. With a fund of funds arrangement, each underlying fund can charge 1.5 and 17.5, irrespective of the performance of other funds; there is no netting of profits and losses across funds in determining incentive fees charged by the underlying funds in an FoF.

For example, consider an otherwise identical multistrategy fund and a fund of funds, each having a $0 \%$ aggregated return after management fees but before considering incentive fees. The multistrategy fund manager clearly earns no incentive fee, as there are no aggregated profits to share between the manager and the investors. Suppose, however, that half of the funds underlying the fund of funds posted $10 \%$ returns before incentive fees, while the other half posted $10 \%$ losses before incentive fees. The fund of funds, like any other limited partner in the funds posting a profit, has to pay $20 \%$ incentive fees to those funds that have earned a profit. Thus, the fund of funds pays incentive fees in the amount of $1 \%$ of AUM to the half of the underlying managers earning $10 \%$ returns (assuming no hurdle rates). However, ignoring possible clawbacks, the fund of funds does not receive an offset on incentive fees from the funds posting $10 \%$ losses. Thus, even though the fund of funds does not pay incentive fees to the funds posting losses, the incentive fees on the funds showing profits place the fund of funds at a disadvantage to the multistrategy manager.

Incentive fees were discussed in the session, Structure of the Hedge Fund Industry as call options on the NAVs of the fund. Limited partners in funds with incentive fees can be viewed as having written call options to the managers. An investor in a fund of funds can be viewed as having written a portfolio of call options, one on each fund. An investor in a multistrategy fund may be viewed as having written a call option on the portfolio of the aggregated strategies. Because the volatility of a portfolio is generally much lower than the volatilities of the constituent assets, the call options written by the multistrategy fund's investors are less expensive (i.e., the incentive fees owed through the multistrategy fund are generally lower than the incentive fees owed through a fund of funds). Lomtev, Woods, and Zdorovtsov estimate that the mean savings from fee netting gives multistrategy managers a $0.23 \%$ annual return advantage relative to funds of funds. ${ }^{2}$ Igor Lomtev, Chris Woods, and Vladimir Zdorovtsov, "Fund of Hedge Fund vs. Multi-Strategy Providers: Implications for Cost-Effectiveness and Portfolio Risk," Journal of Investment Strategy 2, no. 1 (2007): 73-82.

\section*{Flexibility and Transparency}
Multistrategy funds also have a greater ability to make tactical strategy allocation and risk management decisions than do funds of funds. When a fund of funds manager invests with 20 underlying managers, each investment is subject to possible liquidity terms and limited transparency. The fund of funds manager may agree to an initial lockup period of one year, with quarterly redemption periods thereafter. Whereas some managers may provide monthly portfolio snapshots, other managers jealously guard the details of their holdings. Thus, the fund of funds manager may not be able to obtain financial information or act on information in a timely manner.

Using a multistrategy fund approach, the portfolio manager has real-time access to all positions, making it easy to identify the exact positions, performance, and risks at all times. The multistrategy manager has the ability to direct trading teams to reduce or expand positions. For example, if the portfolio manager tactically believes that macro funds will underperform other funds over the coming quarter, capital can be quickly reallocated across traders from macro funds to other funds within the multistrategy fund without the complications of lockups and redemption periods experienced by fund of funds managers. Also, the transparency allows the portfolio manager to determine and implement portfolio-level hedges to manage the total risk of the multistrategy fund, since the manager has timely and complete information on the composition of the portfolio.

A recent development in the hedge fund world is the emergence of hedge fund companies that build their own internal funds of funds. These hedge fund companies offer several different hedge fund strategies to their investors, housing such funds as equity hedge, event-driven, relative value, merger arbitrage, and global macro all under one roof. These companies then create another hedge fund that optimally rebalances across the underlying hedge funds, effectively creating an internal fund of funds from their existing hedge fund offerings. Although this approach can be used to address transparency issues and liquidity constraints, it focuses the fund of funds investment opportunities on the products of a single company.

Reddy, Brady, and Patel discuss the potential returns to tactical reallocation across hedge fund strategies as adding less value than tactical reallocation between traditional stock and bond investments. ${ }^{3}$ Girish Reddy, Peter Brady, and Kartik Patel, “Are Funds of Funds Simply Multi-Strategy Managers with Extra Fees?" Journal of Alternative Investments 10, no. 3 (Winter 2007): 49-61. Whereas the best- and worst-performing hedge fund reallocation strategies had returns differing by $3.8 \%$ per year, the value of switching between stock and bond investments was $8.6 \%$ per year. Although it is difficult to measure the style timing skill of multistrategy managers, several papers, including Beckers, Curds, and Weinberger, and Gregoriou, have concluded that funds of funds have not convincingly demonstrated positive market timing skill. ${ }^{4}$ See, for example, Stan E. Beckers, Ross Curds, and Simon Weinberger, "Funds of Hedge Funds Take the Wrong Risks," Journal of Portfolio Management33, no. 3 (Spring 2007): 108-21; and Gregoriou, “Are Managers of Funds of Hedge Funds Good Market Timers?”

\section*{Managerial Selection and Operational Risks}
Whereas multistrategy managers have potential advantages in fees, risk management, and tactical allocation, funds of funds may have a greater ability to add value through manager selection. At a multistrategy fund, the portfolio manager hires a number of trading teams, each of which executes a specific strategy and agrees to have its capital allocation regularly increased or decreased at the discretion of the portfolio manager. The number of traders employed in a multistrategy approach may range from one trader in each of four strategies to possibly five traders in each of 10 strategies. Thus, the multistrategy manager has hired anywhere from four to 50 traders, among whom the manager can manage risks and make capital allocation decisions. In contrast, the fund of funds manager may have the ability to allocate to any of the more than 8,000 single-strategy fund managers, clearly a wider selection than the multistrategy fund has to choose from once the multistrategy team has been formed.

Although asset allocation is much more important than manager selection in traditional investments, the opposite is probably true in the hedge fund universe. Reddy, Brady, and Patel estimate a $7 \%$ annual difference in returns between top quartile and bottom-quartile hedge fund managers within hedge fund styles, with only a 3.8\% spread across strategies. ${ }^{5}$ Reddy, Brady, and Patel, "Are Funds of Funds Simply Multi-Strategy Managers with Extra Fees?"

Some investors may be concerned with the operational risks of investing in a multistrategy fund. Whereas funds of funds diversify operational risk across 10 to 20 independent managers and organizations, a multistrategy fund has a single operational infrastructure. Market risk may also be a concern, as a catastrophic loss in even one of the multistrategy fund's underlying strategies may sink the entire fund. Conversely, the failure of one of a fund of funds' 20 managers may subject investors to only a $5 \%$ loss and not affect the fund's other investments.

Empirical evidence indicates that multistrategy funds have historically outperformed funds of funds on a risk-adjusted basis, predominantly due to the extra layer of fees charged by fund of funds managers. Agarwal and Kale estimate that multistrategy funds outperform funds of funds by a net-of-fees alpha of $3.0 \%$ to $3.6 \%$ per year after accounting for exposure to market risks. ${ }^{6}$ Vikas Agarwal and Jayant R. Kale, "On the Relative Performance of Multi-Strategy and Funds of Hedge Funds," Journal of Investment Management 5, no. 3 (2007): 41-63. Agarwal and Kale attribute the superior performance of multistrategy managers to a self-selection effect. The self-selection effect in this case is when only the most successful and confident single-strategy hedge fund managers choose to become multistrategy managers by hiring a team of experts and expanding into the world of multistrategy funds. However, it can be argued that the best and brightest among the available hedge fund managers do not remain satisfied in the role of multistrategy fund manager, preferring to manage their own single-strategy fund in order to link their compensation to their own money management skill rather than to the performance of the managers they oversee.


\end{document}