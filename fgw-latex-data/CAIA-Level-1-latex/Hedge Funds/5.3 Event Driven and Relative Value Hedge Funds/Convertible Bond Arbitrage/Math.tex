\documentclass[11pt]{article}
\usepackage[utf8]{inputenc}
\usepackage[T1]{fontenc}
\usepackage{amsmath}
\usepackage{amsfonts}
\usepackage{amssymb}
\usepackage[version=4]{mhchem}
\usepackage{stmaryrd}

\begin{document}
\section*{APPLICATION A}
Question : Consider a firm with a borrowing cost of $8 \%$ on unsecured, subordinated straight debt and a current stock price of $\$ 40$. The firm may be able to issue three-year\\
convertible bonds at an annual coupon rate of perhaps $4 \%$ by offering a conversion ratio such as 20 . What is the bond's strike price, and what does the conversion\\
option allow the bond investors to do?

\section*{Answer and Explanation}
The conversion ratio of 20 is equivalent to a $\$ 50$ strike price assuming that the bond's face value is $\$ 1,000$. On or before maturity, bond investors can opt to convert\\
each $\$ 1,000$ face value bond into 20 shares of the firm's equity rather than receive the remaining principal and coupon payments.

The bond's strike price is $\$ 50$ found by dividing the $\$ 1,000.00$ (convertible bond face value) by 20 (the conversion ratio). The conversion option allows the bond investor to convert each $\$ 1,000$ face value convertible bond into 20 shares of the firm's equity rather than receive the remaining principal and coupon.

\section*{APPLICATION B}
Question : Returning to the previous example of an $8 \%$ unsecured bond rate, a $\$ 40$ stock price, and a conversion ratio of 20 , and assuming that a three-year European-style call option - given a current stock price of $\$ 40$, a strike price of $\$ 50$, and other parameters, such as volatility and dividends-is valued at $\$ 5.14$ per share according to the Black-Scholes option pricing model, what are the value of the convertible bond, the conversion value, and the conversion premium?

\section*{Answer and Explanation}
Let's first find the value of the convertible bond. This is the same as finding the present value in Application A by using the following parameters: $n=3, I=8, P M T=$ 40 , and $F V=1,000$ for a present value of $\$ 896.92$. Now, we need to add the value of 20 options (conversion ratio of the three-year convertible bonds) to the present value. The value of 20 options is computed by multiplying 20 by $\$ 5.14$ (price of a three-year European-style call option) for a product of $\$ 102.80$. The sum of $\$ 896.92$ and $\$ 102.80$ of $\$ 999.72$ is the value of the convertible bond. The conversion value is $\$ 800.00$ found by multiplying 20 (conversion ratio of the three-year convertible bonds) and $\$ 40$ (current stock price). The conversion premium is 24.96 and is found by subtracting $\$ 800$ (the conversion value) from $\$ 999.72$ (value of the convertible bond) and dividing the difference by $\$ 800$ (the conversion value).


\end{document}