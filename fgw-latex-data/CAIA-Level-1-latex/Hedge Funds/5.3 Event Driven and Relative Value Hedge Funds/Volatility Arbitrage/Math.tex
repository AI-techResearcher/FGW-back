\documentclass[11pt]{article}
\usepackage[utf8]{inputenc}
\usepackage[T1]{fontenc}
\usepackage{amsmath}
\usepackage{amsfonts}
\usepackage{amssymb}
\usepackage[version=4]{mhchem}
\usepackage{stmaryrd}

\begin{document}
\section*{APPLICATION A}
Question : Consider a 30-day variance swap with a notional value of $\$ 250,000$. The strike variance is 9.00 . The realized variance of the index is 7.00 . What would be the payment or payoff of the swap?

\section*{Answer and Explanation}
This problem can be solved by using Equation 1a:

$$
\begin{aligned}
\text { Variance Swap Payoff } & =\text { Variance National Value } \times(\text { Realized Variance }- \text { Strike Variance }) \\
& =\$ 250,000 \times(7.00-9.00) \\
& =-\$ 500,000
\end{aligned}
$$

The swap buyer received the realized variance and pays the strike variance, so in this example the swap buyer pays $\$ 500,000$ to the variance swap payer.

\section*{APPLICATION B}
Question : Consider a 30 -day volatility swap with a notional value of $\$ 500,000$. The strike volatility is 17.50 . The realized volatility of the refence asset is 18.25 . What would be the payment of the swap?

\section*{Answer and Explanation}
The volatility swap payoff:

$$
\$ 500,000 \times(18.25-17.50)=\$ 375,000
$$

The swap buyer receives the realized volatility and pays the strike volatility, so in this example the swap payer pays $\$ 375,000$ to the swap buyer.

\section*{APPLICATION C}
Question : The payoff of a variance swap is $\$ 120,000$. The strike variance is 9.00 and the realized variance is 10.00 . What are the vega notional value and the variance notional value?

\section*{Answer and Explanation}
From the formula for the variance swap payoff (Equation 2):

$$
\begin{aligned}
\text { Variance Swap Payoff } & =\frac{\text { Vega Notional Value } \times(\text { Realized Variance }- \text { Strike Variance })}{2 \times \sqrt{\text { Strike Variance }}} \\
\$ 120,000 & =\frac{\text { Vega National Value } \times(10.00-9.00)}{2 \times \sqrt{9.00}} \\
\text { Vega National Value } & =\$ 720,000
\end{aligned}
$$

From the definition of variance notional value:

$$
\begin{aligned}
& \text { Variance National Value }=\frac{\text { Vega National Value }}{2 \sqrt{\text { Strike Variance }}} \\
& \text { Variance National Value }=\frac{\$ 720,000}{2 \sqrt{9.00}}=\$ 120,000
\end{aligned}
$$

\section*{Application D}
Question : Suppose that the realized volatility of an asset has exactly five equally likely outcomes: $1 \%, 3 \%, 4 \%, 5 \%$, or $7 \%$. Calculate: (1) The expected value of the realized volatility, (2) the five equally likely realized variances corresponding to the five given outcomes (volatilities), (3) the expected value of the five variances, and (4) the value of volatility that corresponds to the value found in Step 3 (i.e., the volatility that corresponds with the expected variance).

\section*{Answer and Explanation}
The purpose of this exercise is to compare the difference between variance-based and volatility-based payoffs. To solve for these four problems, we must do the following:

To find the expected value of the outcomes (1), we will take a simple average of the five expected volatilities (Note: it is a simple average because they are all equally likely. If there was a higher probability associated with them, a simple average would not be appropriate).

$$
\text { Average }=\frac{1 \%+3 \%+4 \%+5 \%+7 \%}{5}=4 \%
$$

To determine the variances (2), we simply have to square each volatility (i.e., standard deviation)

$$
\begin{aligned}
\text { Variance } & =0.01^{2}=0.0001 \\
\text { Variance } & =0.03^{2}=0.0009 \\
\text { Variance } & =0.04^{2}=0.0016 \\
\text { Variance } & =0.05^{2}=0.0025 \\
\text { Variance } & =0.07^{2}=0.0049
\end{aligned}
$$

Next, we must find the expected value of the variances, which is a simple average of the five variances found above (Note: it is a simple average because they are all equally likely. If there was a higher probability associated with them, a simple average would not be appropriate).

$$
\text { Average }=\frac{0.0001+0.0009+0.0016+0.0025+0.0049}{5}=0.0020
$$

Finally, to find the value of the volatility (4) using the expected value of the variances from the previous step, we simply take the square root of the variance:

$$
\text { Standard Deviation }=\sqrt{0.002}=0.0447=4.47 \%
$$


\end{document}