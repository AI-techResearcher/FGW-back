\documentclass[11pt]{article}
\usepackage[utf8]{inputenc}
\usepackage[T1]{fontenc}
\usepackage{amsmath}
\usepackage{amsfonts}
\usepackage{amssymb}
\usepackage[version=4]{mhchem}
\usepackage{stmaryrd}

\begin{document}
Overview of Relative Value Strategies

Relative value strategies attempt to capture alpha through predicting changes in relationships between prices or between rates. For example, rather than trying to predict the price of oil, a relative value strategy might predict that there will be a narrowing of the margin between the price of oil and the price of gasoline.

Relative value fund managers take long and short positions that are relatively equal in size, volatility, and other risk exposures. Ideally, the combined positions have little net market risk but can profit from short positions in relatively overvalued securities and long positions in relatively undervalued securities. Relative value funds tend to profit during normal market conditions when valuations converge to their equilibrium values. Convergence is the return of prices or rates to relative values that are deemed normal. Since returns to these convergence strategies are normally very small, managers have to employ substantial leverage to generate acceptable returns for these strategies. Therefore, relative value funds can experience substantial losses during times of market crisis, as leveraged funds may be forced to liquidate positions and wind down leverage at times when relative values appear dramatically abnormal.

Within the relative value class of hedge funds, four styles will be discussed: convertible bond arbitrage, volatility arbitrage, fixed-income arbitrage, and relative value multistrategy funds. Hedge Fund Research (HFR) estimates that relative value hedge funds hold more than a quarter of hedge fund industry assets, totaling over $\$ 1,030$ billion at the end of 2021 . This includes nearly $\$ 70$ billion in convertible arbitrage funds, $\$ 21$ billion in volatility arbitrage, and $\$ 394$ billion in fixed-income arbitrage. Within fixed-income arbitrage, nearly $\$ 190$ billion is invested in corporate bond strategies, $\$ 31$ billion in sovereign bonds, and over $\$ 100$ billion in assetbacked securities. Many relative value hedge funds mix these styles, as evidenced by the $\$ 600$ billion in relative value multistrategy fund assets under management.

The classic relative value strategy trade is based on the premise that a particular relationship or spread between two prices or rates has reached an abnormal level and will, therefore, tend to return to its normal level. This classic trade involves taking a long position in the security that is perceived to be relatively underpriced and a short position in the security that is perceived to be relatively overpriced. The normal level to which the price or rate relationship is anticipated to return is usually a level deemed by the fund manager to represent a long-term tendency as observed empirically or derived theoretically.

Relative value strategies tend to perform well during periods of decreasing volatility and increasing market calm when positions with diverse values converge and credit spreads narrow. However, relative value strategies can experience large losses in crisis markets when there is a flight-to-quality response to risk, with increased volatility and widening credit spreads, resulting in returns that have large exposures to kurtosis and negative skewness.


\end{document}