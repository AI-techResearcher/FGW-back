\documentclass[11pt]{article}
\usepackage[utf8]{inputenc}
\usepackage[T1]{fontenc}
\usepackage{amsmath}
\usepackage{amsfonts}
\usepackage{amssymb}
\usepackage[version=4]{mhchem}
\usepackage{stmaryrd}

\begin{document}
The Sources of Most Event Strategy Returns

The event-driven category of hedge funds includes activist hedge funds, merger arbitrage funds, and distressed securities funds, as well as special situation funds and multistrategy funds that combine a variety of event-driven strategies. Event-driven hedge funds search for mispriced securities during both the anticipation of and the realization of events. Events include mergers and acquisitions, spinoffs, tracking stocks, accounting write-offs, reorganizations, bankruptcies, share buybacks and secondary offerings, special dividends, and any other corporate events that are generally associated with substantial market price reactions in the securities related to the transactions.

The most common transaction for an event-driven strategy fund is to enter positions in one or more corporate securities during a period of event risk. For example, an event-driven strategy fund may purchase the equity of a target firm and short sell the equity in the acquiring firm in a proposed merger and hold those positions until the merger is completed or the deal falls through. Event-driven funds profit when events unfold as predicted and suffer losses when events unfold with the opposite consequences. In the case of a traditional merger arbitrage transaction, the fund benefits if the specified event (such as a merger) takes place and suffers a loss if the event fails.

Within the event-driven class of hedge funds, four styles will be discussed: activist funds, merger arbitrage funds, distressed securities funds, and multistrategy eventdriven funds. Hedge Fund Research (HFR) estimates that event-driven hedge funds totaled nearly $\$ 1,100$ billion at the end of 2021 . This includes nearly $\$ 190$ billion in activist funds, $\$ 29$ billion in merger arbitrage funds, and nearly $\$ 250$ billion in distressed securities funds. Multistrategy and special situation funds add another $\$ 615$ billion in assets under management.

By their very nature, special events are nonrecurring and usually contain unique or unusual circumstances. Therefore, market prices may not fully adjust to the information associated with these transactions in a timely manner. This provides an opportunity for event-driven managers to act quickly and capture a return premium-perhaps a risk premium-associated with these transactions. Event-driven hedge funds are generally exposed to substantial event risk. Corporate event risk is dispersion in economic outcomes due to uncertainty regarding corporate events. A central issue in the analysis of event-driven strategies is the extent, if any, to which returns are driven by beta (systematic risks) or alpha (superior risk-adjusted returns).

\section*{Insurance-Selling View of Event Strategy Returns}
Consider the case of a proposed merger that, if completed, will result in a $\$ 100$ per share payment to the shareholders of the target firm in exchange for their shares. In this scenario, the shares of the target firm jumped from $\$ 70$ per share to $\$ 90$ per share when the proposed merger was announced. Although the news of the proposed merger is public knowledge, it will be several months before it will be known whether the necessary approvals can be obtained. Thus, there is a period of event risk during which share prices will be expected to react to news on whether the proposed merger will be consummated. It is common for existing shareholders of a target firm to sell shares soon after the share prices jump as a result of the proposed merger announcement. It is also common for event-driven hedge funds to purchase shares during the period between the proposed merger announcement and the resolution of uncertainty regarding the event.

After the merger announcement, existing shareholders of the target firm need to decide whether to continue to hold their shares, in hopes that the merger will be approved and share prices will rise from $\$ 90$ to $\$ 100$ per share, or to sell their shares at $\$ 90$, avoiding the risk that the merger will fail and that the share prill fall back to $\$ 70$ or perhaps lower. Some shareholders wish to avoid the event risk and choose to sell their shares to hedge funds at a discount to this $\$ 100$ offer, such as the $\$ 90$ price described in the example. Presumably, they reinvest their sales proceeds in firms that are not subject to substantial event risk and, in so doing, reduce the total event risk of their portfolio. These shareholders are often viewed as having purchased insurance against the failure of the firms to complete the anticipated merger.

Event-driven hedge funds may be viewed as seeking to earn risk premiums for selling insurance against failed deals. Selling insurance in this context refers to the economic process of earning relatively small returns for providing protection against risks, not the literal process of offering traditional insurance policies. Further, a merger arbitrage hedge fund's portfolio typically consists of several potential mergers, and therefore its exposure to each deal might be relatively small, similar to an insurance company with relatively small exposure to each contract or set of contracts. Finally, a merger arbitrage manager is typically able to use derivative securities to manage its exposure to large deals, a relatively complicated alternative that other investors may not be able to employ because of legal restrictions on the use of derivative securities.

Other types of event-driven funds may have other return drivers. Distressed funds might earn returns for providing liquidity, purchasing securities from investors who need to sell when there are few bidders. Activist fund managers benefit from event risk, often in the case when the actions of the activist help to make that event possible.

\section*{Binary Option View of Event Strategy Returns}
Continuing with this example, the hedge fund purchases the shares at $\$ 90$ and holds the shares until either the merger succeeds, in which case the fund receives $\$ 100$ per share, or the merger fails, in which case the fund receives perhaps $\$ 70$ per share. In this simplistic example, a long position in the merger target may be viewed as a long position in a riskless bond with a $\$ 70$ face value and a long position in a binary call option that pays $\$ 30$ if the deal is consummated and $\$ 0$ if the deal fails. A long binary call option makes one payout when the referenced price exceeds the strike price at expiration and a lower payout or no payout in all other cases. A long binary call option would be priced at $\$ 20$ in this example if it were assumed for simplicity that riskless interest rates were $0 \%$. A long binary put option makes one payout when the referenced price is lower than the strike price at expiration and a lower payout or no payout in all other cases. A long binary put option pays higher cash when the referenced price falls and would be priced at $\$ 10$ in this example, assuming a $0 \%$ riskless rate. Thus, using put options rather than call options, the hedge fund's long position in the merger target may be viewed as a long position in a riskless bond with a $\$ 100$ face value and a short position in a binary put option that pays $\$ 30$ if the deal fails. Note that whether the fund's position is described with long calls or short puts, the initial cost of $\$ 90$ and the final payout of either $\$ 100$ or $\$ 70$ are the same.

The binary put option view of the hedge fund's position illustrates that the hedge fund has, among other things, written a put option on the event such that the hedge fund will bear a loss if the merger does not occur. In this view, event-driven hedge funds are writing put options and will tend to have payouts consistent with writing out-of-the-money options, meaning modest upside potential with large downside potential. In many cases, this asymmetric payout is an accurate description of the event risks that a hedge fund takes. It should be noted that in the previous merger examples, the downside risk of owning the target firm's stock was that the target firm's stock price could fall to a prespecified price. However, in practice, the downside risk could be substantial, and a long position in the target firm's stock could include losses due to a general market decline in addition to a failed deal.

\section*{Binary Call Option View of Events}
Not all positions of hedge funds in event-driven strategies offer the potential for large losses and small profits. Thus, not all positions can be well approximated as being short an out-of-the-money put option. For example, event-driven hedge funds may play either side of an event such as a merger: the side that benefits if the event is consummated or the side that loses. The fund tries to use superior information or analysis to ascertain whether market prices overestimate, underestimate, or properly reflect the outcomes of various events. This may be possible because these are unique and rare events, and long-only investors may not typically have the skills and the data to make accurate predictions about the eventual outcome. The hedge fund seeks to earn higher returns from formulating better predictions of the event outcomes than are reflected in market prices.

The binary call option view of the hedge fund's position illustrates that the hedge fund has, among other things, purchased a call option on the event such that the hedge fund will gain if the merger is consummated and lose if the event deal does not occur. This view of the transaction as a long position in a call option illustrates the expected risk premium that the hedge fund's position should earn. A long position in a call option on equity is a very bullish bet. In competitive markets, long positions in equities tend to have positive beta risk and should generally earn expected risk premiums. Long positions in call options in equities tend to have higher betas than their underlying stocks and in theory should offer even higher expected risk premiums. Thus, using this long binary call option view, it may be argued that typical event-driven strategies contain substantial systematic risk and that higher returns for this strategy may reflect bearing systematic risk, or beta, rather than alpha.

By their nature, events are primarily firm specific and cause idiosyncratic risks. However, some events, such as mergers, have probabilities of consummation that are positively correlated with the performance of the overall market. In bull markets and good economic times, mergers and other deals are more likely to be proposed and consummated. Thus, some of the event risk may be systematic. Event-driven fund managers may take both long and short positions in equity or debt securities. However, they will find it difficult to fully hedge against both the event risk and the overall movement in the market and thus achieve complete market neutrality. The result is a tendency to bear systematic risks in addition to the obvious idiosyncratic risks of the events. Therefore, event-driven funds typically profit during normal and healthy market conditions, when deals are consummated on a timely basis. However, substantial losses can result during times of market crisis and failed deals, when market participants back away from deals and, typically, sell off securities that exhibit illiquidity and high risk.

This call option binary view of event-driven investing is most applicable to events with binary outcomes, such as mergers and spin-offs. Distressed investing and activist investing are complex areas of investments which may have longer investment time horizons and different paths of potential outcomes.

The following sections examine each of the major subcategories of the event strategy more closely, starting with activist investing.


\end{document}