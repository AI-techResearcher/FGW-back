\documentclass[11pt]{article}
\usepackage[utf8]{inputenc}
\usepackage[T1]{fontenc}
\usepackage{amsmath}
\usepackage{amsfonts}
\usepackage{amssymb}
\usepackage[version=4]{mhchem}
\usepackage{stmaryrd}

\begin{document}
\section*{APPLICATION A}
Question : A bond is purchased at $40 \%$ of face value. After bankruptcy, $30 \%$ of the bond's face value is ultimately recovered. Express the rate of return as a non-annualized rate, as an annualized rate based on a four-month holding period, and as an annualized rate based on a four-year holding period, ignoring compounding and assuming no coupon income.

\section*{Answer and ExpLanation}
In order to calculate the bond's rate of return as a non-annualized rate we need to subtract the $40 \%$ (percentage of face value bond is purchased at) from $30 \%$ (percentage of the bond's face value recovered after bankruptcy) and divide by $40 \%$ (percentage of the bond's face value recovered after bankruptcy) for a quotient of $-25 \%$. The annual rate based on a 4 -year holding period is found by dividing $-25 \%$ by 4 for a quotient of $-6.25 \%$. To find the annualized rate based on a four-month holding period multiply $-25 \%$ by $12 / 4$ for a product of $-75 \%$.


\end{document}