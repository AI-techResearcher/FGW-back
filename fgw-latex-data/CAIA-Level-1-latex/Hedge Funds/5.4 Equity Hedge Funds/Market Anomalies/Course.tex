\documentclass[11pt]{article}
\usepackage[utf8]{inputenc}
\usepackage[T1]{fontenc}
\usepackage{amsmath}
\usepackage{amsfonts}
\usepackage{amssymb}
\usepackage[version=4]{mhchem}
\usepackage{stmaryrd}

\begin{document}
Market Anomalies

Investment strategies that can be identified based on available information and that offer higher expected returns after adjustment for risk are known as market anomalies, which are violations of informational market efficiency. In the equity markets, these anomalies focus on such attributes as value, market capitalization, accounting accruals, price momentum or reversal, earnings surprise, net stock issuance, and insider trading. Good overviews of all anomalies, including the segregation of performance across the micro-cap, small-cap, and large-cap stock sectors, are presented by Fama and French, and by Stambaugh, Yu, and

Yuan. ${ }^{1}$ Eugene Fama and Kenneth French, "Dissecting Anomalies," Journal of Finance 63 (August 2008): 1653-78; and Robert F. Stambaugh, Jianfeng Yu, and Yu Yuan, "The Short of It: Investor Sentiment and Anomalies," Journal of Financial Economics 104, no. 2 (May 2012): 288-302. Interestingly, Stambaugh et al. argue that the timing of anomaly profits can be influenced by sentiment. Although anomaly profits accruing to holders of long stock positions are relatively constant over time, short sellers have earned their greatest profits after periods of above-average sentiment.

Several major anomalies that equity hedge fund managers have used to generate ex ante alpha are reviewed in the following sections. First, however, is a discussion of key issues in the identification and verification of anomalies.

\section*{Market Efficiency Tests as Joint Hypotheses}
Practitioners and academics have debated the existence of a variety of market anomalies using empirical tests based primarily on asset pricing models, especially the CAPM. An empirical test of market efficiency is a test of joint hypotheses because the test assumes the validity of a model of the risk-return relationship to test whether a given trading strategy earns consistent risk-adjusted profits.

Thus, any finding of consistent superior risk-adjusted returns may be caused by model misspecification for adjusting returns for risk differentials rather than by market inefficiency. A return model is misspecified when the model omits explanatory variables or incorrectly describes the relationships between variables. Empirical indications of market inefficiency should be viewed as reliable only to the extent that the risk-adjustment procedures are viewed as well specified.

Extensive empirical analysis through the 1970 s and 1980 s provided indications of numerous anomalies based on single-factor models, such as the CAPM. But the multifactor models discussed in Level II of the CAIA curriculum indicate that returns can be substantially better explained by a multiple-factor model than by a singlefactor market model. Empirical studies based on the CAPM may have falsely indicated that a strategy offers superior returns because the model for risk adjustment was misspecified, resulting in a failure in the research to adjust fully for risk. For example, if a strategy of investing in young firms is shown to have earned abnormal returns, it might be due to the fact that young firms tend to be relatively small. Thus, the strategy might be earning extra returns because of exposure to the size factor rather than because the firms are young. Consequently, to be seen as a valid market anomaly, a perceived anomaly needs to earn excess returns using a wellregarded model of returns.

\section*{Predicting Persistence of Market Anomalies}
There are several critical issues regarding the application of investment strategies based on evidence of the past performance of anomaly-based strategies. Is the statistical result due to spurious correlations or to true underlying correlations? Even if the statistical results are reliable, is there a basis for believing that the anomaly will continue? How long should a manager continue implementing a strategy based on a perceived anomaly when the strategy begins suffering losses?

Anomalies based entirely on empirical observation should be viewed with more skepticism than anomalies that also appear to be consistent with reasoning. Accordingly, the search for reasoned explanations of empirical findings should be used in tandem with the empirical search for return patterns. In other words, both the decision of when to implement a strategy based on a perceived anomaly and the decision of when to abandon the strategy should be based on the extent to which the explanations for the anomaly can be reasoned. When the success of a strategy has a reasonable explanation, the empirical results are more trustworthy, and the decision of when to abandon a strategy can be based at least in part on whether the explanation for the anomaly remains valid. A reasonable explanation for the anomaly should include (1) from whom the excess returns are being earned, and (2) why the entity on the other side of the trade is willing to transact at prices that the fund manager perceives as beneficial to the fund and harmful to the other trader.

The rest of this section discusses major anomalies, with a focus on their potential behavioral explanations.

\section*{Accounting Accruals and Market Anomalies}
Sloan discusses the role of accounting accruals in equity valuation. ${ }^{2}$ Richard G. Sloan, "Do Stock Prices Fully Reflect Information in Accruals and Cash Flows about Future Earnings?" Accounting Review 71, no. 3 (1996): 289-315. An accounting accrual is the recognition of a value based on anticipation of a transaction. Sloan contrasts the cash flow of a firm with its net income. Net income includes the effects of accounting accruals. For example, sales of products on credit enter into the calculation of net income but have very little or no impact on free cash flow. According to this anomaly, investors seem to focus too much on net income, even though free cash flow appears to be the main driver of long-term returns. Since managers can manipulate accruals to generate positive net income and to meet the market's expectations of quarterly earnings, it is argued that investors should ignore higher net incomes that are mostly caused by large accruals (i.e., noncash items). The reason is that when current net income is largely due to accounting accruals rather than cash flows, the inflated short-term profits evolve into reduced subsequent profits when the cash flows associated with those accruals are received, since profits have already been recognized. Subsequent profits may also not be received and therefore must have their associated profits written off. For instance, a firm that is trying to meet particular earnings targets may be tempted to sell too many products on credit, which will create higher net income and larger accounts receivable. If some of these receivables cannot be collected later, then future net income will be adversely affected by the current urge to increase sales. In behavioral finance terms, investors are overreacting to the temporary accounting profitability of the accruals while underreacting to the more reliable indications provided by cash flows.

Accruals are reflected by changes in noncash items. Equation 1 provides an accounting definition of total accruals (ignoring accrued taxes):

where $\triangle \mathrm{CA}$ is the change in current assets, $\triangle \mathrm{CL}$ is the change in current liabilities, $\triangle \mathrm{C}$ ash is the change in cash, $\triangle \mathrm{STDEBT}$ is the change in short-term debt, D\&A is depreciation and amortization expenses.

According to the anomaly, an increase in noncash current assets ( $\triangle \mathrm{CA}-\Delta \mathrm{Cash}$ ), such as accounts receivable and inventory, can indicate lower future earnings if customers do not eventually pay for the goods and services provided, if the inventory becomes obsolete, or if the inventory is sold at a discounted price. Similarly, current liabilities and short-term debt ( $-\triangle \mathrm{CL}+\triangle$ STDEBT) may include deferred accounts payable or tax liabilities that increase earnings but defer current-period expenses until a future date. Finally, a decline in depreciation and amortization expenses (D\&A) increases current-year income but actually reduces free cash flow, as the higher reported net income incurs a larger tax expense.

Bradshaw, Richardson, and Sloan conduct an empirical analysis that indicates that firms with especially large accruals, in which net income is significantly higher than operating cash flow, tend to have negative future earnings surprises that lead to stock price underperformance. ${ }^{3}$ Mark T. Bradshaw, Scott A. Richardson, and Richard G. Sloan, "Do Analysts and Auditors Use Information in Accruals?" Journal of Accounting Research 39, no. 1 (1996): 45-74. The implication is that equity hedge fund managers can buy stocks with negative accruals (higher ratio of free cash flow to net income) and sell short stocks with positive accruals (lower ratio of free cash flow to net income). To the extent that this is a true anomaly and that the anomaly continues, the strategy would generate superior risk-adjusted returns.

\section*{Price Momentum and Market Anomalies}
Although many anomalies focus on the fundamental analysis of items on corporate financial statements, there is also evidence that technical factors, such as price and volume, may be used to predict superior returns. Price momentum is trending in prices such that an upward price movement indicates a higher expected price and a downward price movement indicates a lower expected price. A strategy based on price momentum is a trend-following strategy in which stock prices are believed to have positive serial correlation (i.e., positive autocorrelation).

Chan, Jegadeesh, and Lakonishok demonstrate the price momentum effect by measuring the performance of stocks ranked by their return over the prior six months. ${ }^{4}$ Louis K. C. Chan, Narasimhan Jegadeesh, and Josef Lakonishok, "The Profitability of Momentum Strategies," Financial Analysts Journal 55, no. 6 (1999): $80-$ 90. For the subsequent six to 12 months, they provide evidence that superior risk-adjusted profits can be earned by buying stocks that performed well over the previous six months (winners) and selling stocks that performed poorly over the previous six months (losers). Thus, price momentum appears to prevail using sixmonth intervals. However, a reversal effect is seen at very short- and long-term horizons, as stocks with the strongest price performance at one-month or five-year time frames underperform over a similar time horizon, whereas losers over the same time period outperform. Thus, both consistent price momentum and price reversals have been observed, each based on different time intervals.

Many reasons have been put forth to explain the presence of momentum. One potential explanation is that well-informed investors cannot take large positions in stocks because their superior information is likely to be leaked to the market. Thus, these investors have to build positions in equities gradually. For instance, if a hedge fund manager through hard work and detailed analysis discovers that a particular firm is undervalued, rather than buying a large number of shares immediately, the optimal strategy would be to build a position in the stock a little at a time so that the price impact of its purchase is minimized. The stock of the firm is still likely to increase gradually through time, especially if it is a small firm, as the hedge fund builds its position. Eventually, the rest of the market learns about the firm and the stock price increases further, leading to the presence of momentum. This line of reasoning is consistent with available empirical evidence showing that momentum is strongest in small-cap stocks.

\section*{Earnings Momentum and Market Anomalies}
Earnings are primary drivers of idiosyncratic stock returns. Unlike patterns in share prices, patterns in corporate earnings may exist in an efficient market, since speculators cannot trade directly on earnings. In an efficient market, share prices respond quickly to changes in a firm's prospects, whereas earnings may tend to respond on a delayed basis. For example, firms that experience rapid growth in earnings in one year due to a successful new product are likely to continue to experience earnings growth in the subsequent year. In other words, accounting numbers are often conservative in the speed with which they recognize increases in underlying value. There is no doubt that earnings show patterns; the key question is whether patterns in earnings can be used to find patterns in share prices.

Brown summarizes the research on earnings momentum and earnings surprise. ${ }^{5}$ Lawrence D. Brown, "Earnings Surprise Research: Synthesis and Perspectives," Financial Analysts Journal 53, no. 2 (1997): 13-19. Earnings momentum is the tendency of earnings changes to be positively correlated. Earnings surprise is the concept and measure of the unexpectedness of an earnings announcement.

Earnings surprise may be estimated using mechanical rules applied to historical earnings or may be calculated by comparing actual earnings to the forecasts of analysts. Equity analysts working for large banks and brokerage firms routinely publish estimates of quarterly earnings per share (EPS) for thousands of corporations worldwide. When a corporation announces a new EPS, the market compares the actual result to the average, or consensus, of the estimates produced by analysts. A positive earnings surprise results when actual profits exceed estimates, and a negative earnings surprise occurs when earnings fall below estimates. Standardized unexpected earnings (SUE) is a measure of earnings surprise. Although exact definitions of the SUE vary, a representative example of the SUE for a firm based on analysts' expectations in the most recent quarter is defined as follows:


\begin{equation*}
\text { SUE }=\frac{\text { EPS }- \text { Analyst Consensus EPS Estimate }}{\text { Standard Deviation of Earnings Surprises }} \tag{2}
\end{equation*}


The denominator of Equation 2 is a measure of the dispersion in previous earnings or in the amount by which analysts' estimates missed actual earnings. For example, one popular computation of the denominator in Equation 2 is the standard deviation of (EPS - Analyst Consensus EPS Estimate) over the previous eight quarters.

On average, stock prices have been shown to continue to drift in the same direction of the SUE even after the announcement of quarterly profit figures, meaning that stocks with positive earnings surprises outperform the market, and stocks with negative earnings surprises underperform the market. In an efficient market, prices should immediately and fully react to the earnings announcement and return to a random walk immediately thereafter. However, a post-earnings-announcement\\
drift anomaly has been documented, in which investors can profit from positive surprises by buying immediately after the earnings announcement or selling short immediately after a negative earnings surprise.

If markets are perfectly efficient, there should be no post-earnings-announcement drift in risk-adjusted share prices. The post-earnings-announcement drift anomaly indicates a tendency of market participants to underestimate serial correlation in quarterly earnings, as earnings surprises tend to repeat in the same direction. In other words, traders underestimate earnings momentum. Once companies report a positive earnings surprise, earnings in future quarters tend to exceed analyst estimates.

Trading strategies can be developed to predict earnings surprises rather than merely react to them. Brown, Han, Keon, and Quinn build a model to predict earnings surprises. ${ }^{6}$ Lawrence D. Brown, Jerry C. Y. Han, Edward F. Keon Jr., and William Quinn, “Predicting Analysts' Earnings Surprise,” Journal of Investing 5, no. 1 (Spring 1996): 17-23. The most relevant factors for predicting an earnings surprise in the next quarter are the prior quarter's SUE and the market capitalization of the stock. Small-capitalization stocks and stocks for which analysts have been increasing their earnings estimates are factors found to correspond to higher earnings surprises.

\section*{Net Stock Issuance and Market Anomalies}
Corporations can expand and contract the number of their shares outstanding, and research has indicated that there may be anomalies associated with these activities. When a company chooses to reduce its shares outstanding, a share buyback program is initiated, and the company purchases its own shares from investors in the open market or through a tender offer. These shares are retired from the share count, thereby increasing the proportional ownership of all other shareholders. Reduced shares outstanding can immediately increase earnings per share, reduce dividends payable, and even generate earnings-per-share growth. Share repurchase activity directly increases the demand for shares and reduces the supply of shares, both of which may exert upward pressure on the stock price in the absence of other effects, such as signals that the firm lacks superior investment opportunities.

Whereas share repurchases reduce the number of shares outstanding, issuance of new stock increases the number of shares outstanding. Issuance of new stock is a firm's creation of new shares of common stock in that firm and may occur as a result of a stock-for-stock merger transaction, through a secondary offering, or the issuance of employee and executive stock options. Issuance of new stock causes positive net stock issuance. Net stock issuance is issuance of new stock minus share repurchases. Companies that issue large amounts of new shares, such as more than $20 \%$ of the shares currently outstanding, frequently see their stock price substantially underperform the market. Singal documents the subsequent five-year underperformance of acquiring firms that pay for their acquisition in shares, noting that the decision to issue shares in a merger transaction indicates management's view that the price of its own firm's stock is overvalued. ${ }^{7}$ Vijay Singal, Beyond the Random Walk: A Guide to Stock Market Anomalies and Low Risk Investment (New York: Oxford University Press, 2003). Conversely, acquirers who pay for their acquisition in cash outperform the market over a five-year period, as management's decision not to issue new shares signals management's confidence that its firm's shares are undervalued.

There is evidence that positive or negative net stock issuance has been one of the most profitable anomalies. Continued informational market inefficiency with respect to net stock issuance offers equity hedge fund managers another fundamental-analysis based strategy for identifying sources of alpha.

\section*{Insider Trading and Market Anomalies}
Singal also documents an insider-trading anomaly. ${ }^{8}$ Ibid. Illegal insider trading varies by jurisdiction but may involve using material nonpublic information, such as an impending merger, for trading without required disclosure. In many countries, senior executives of a corporation are required to report their trading in company shares to regulatory authorities, who disseminate that information to market participants. These company insiders may also be subject to trading windows, which restrict trades to only those times when they are assumed to not have material information regarding sensitive topics, such as profitability or an upcoming merger. Such windows may occur immediately after an earnings announcement, because known accounting results have recently been disclosed and it is too early in the quarter to have strong knowledge of future results. Trading by insiders can be legal insider trading when it is performed in accordance with legal requirements.

Legal insider trading by senior executives can signal potentially valuable information. Evidence indicates that even during the restricted trading windows, corporate insiders tend to execute especially prescient trades, as should be expected given that their knowledge of the firm's operations is better than that of nearly all other market participants.

The key to implementing information from announced legal insider trading is differentiating insider trading driven by personal financial circumstances from trading driven by perceived mispricing of shares. Many of these executives acquire shares as compensation in the form of options, grants, or bonuses. Selling shares is frequently driven by idiosyncratic factors, such as the insider's desire for liquidity, for diversification, or to meet large personal expenses. Insider selling, then, especially by a single executive in small amounts, is not likely to indicate that the insider believes that the stock is overvalued. However, if multiple insiders at a single firm sell large portions of their holdings in a short time period, that signal is likely to be more negative.

The rationale for insider buying is much clearer than that for insider selling, as the predominant motive for buying is to earn a profit from investing in undervalued shares. Generally, insiders would have little motivation to increase their holdings in the firm if they viewed the shares as fairly valued or overvalued. Insider buying can also be used to forecast the direction of the stock market as a whole, as insider buying across firms tends to be more prevalent near a bottom in the stock market.


\end{document}