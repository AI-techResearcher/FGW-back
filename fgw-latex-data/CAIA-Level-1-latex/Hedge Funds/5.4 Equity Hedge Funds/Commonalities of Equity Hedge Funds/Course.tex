\documentclass[11pt]{article}
\usepackage[utf8]{inputenc}
\usepackage[T1]{fontenc}
\usepackage{amsmath}
\usepackage{amsfonts}
\usepackage{amssymb}
\usepackage[version=4]{mhchem}
\usepackage{stmaryrd}

\begin{document}
Commonalities of Equity Hedge Funds

Equity hedge funds follow the most popular hedge fund strategy, whether measured in terms of assets under management (AUM) or in terms of the number of funds. As of the end of 2021, Hedge Fund Research estimated that equity hedge funds of all styles constituted $30 \%$ of hedge fund industry AUM and $50 \%$ of the number of hedge funds. Although this sector is the largest by AUM, equity hedge funds have a smaller average asset size than funds in the event-driven, macro, or relative value categories.

At their heart, equity hedge funds of all styles share a common strategy focused on taking long positions in undervalued stocks and short positions in overvalued stocks. A major difference among equity hedge fund strategies is the typical net market exposure maintained by managers. Positive systematic risk levels are typically maintained by equity long/short hedge funds. Equity long/short funds tend to have net positive systematic risk exposure from taking a net long position, with the long positions being larger than the short positions. Equity market-neutral funds attempt to balance short and long positions, ideally matching the beta exposure of the long and short positions and leaving the fund relatively insensitive to changes in the underlying stock market index. Finally, short-bias funds have larger short positions than long positions, leaving a persistent net short position relative to the market index that allows these funds to profit during times of declining equity prices.

The success of funds within each of these strategies is primarily related to the extent to which a manager is successful in establishing long positions in stocks that outperform the market and short positions in stocks that underperform the market. It is not necessary for the long positions to increase in value and the short sales to decline in price for the equity manager to profit.

Consider a market-neutral manager with an $8 \%$ return target attributable to a $3 \%$ alpha on long positions, a $5 \%$ alpha on short positions, and equal long and short positions that are equivalent to the fund's net asset value. When the market is rising, it is unrealistic to expect the short positions to earn absolute profits. If the market index rises by $20 \%$, the goal would be for the long positions to rise by $23 \%$ and for the short positions to lose only $15 \%$. The key is for the returns of the stocks underlying the long positions to exceed the returns on the stocks underlying the short positions by $8 \%$.

This session focuses on equity long/short funds, equity market-neutral funds, and short-bias funds, which control $\$ 850$ billion, $\$ 74$ billion, and $\$ 5.2$ billion, respectively. Hedge Fund Research also covers a number of other styles within the equity category, most notably sector managers who focus exclusively on a particular sector, such as energy or technology stocks.


\end{document}