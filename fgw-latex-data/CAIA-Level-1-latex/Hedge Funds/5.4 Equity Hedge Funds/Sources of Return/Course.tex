\documentclass[11pt]{article}
\usepackage[utf8]{inputenc}
\usepackage[T1]{fontenc}
\usepackage{amsmath}
\usepackage{amsfonts}
\usepackage{amssymb}
\usepackage[version=4]{mhchem}
\usepackage{stmaryrd}

\begin{document}
Sources of Return

What are the potential sources of ex-ante alpha to equity hedge funds? In markets with relatively low transaction costs, such as major equity markets, any return in excess of the equity market portfolio received by one market participant must be offset by deficient returns to another market participant. This section provides three explanations of why some equity hedge funds might be able to generate consistently superior returns.

\section*{Providing Liquidity}
Some equity hedge funds provide liquidity to the market by buying securities at relatively small discounts from large anxious sellers and selling securities at small premiums to large anxious buyers. Liquidity, in this context, is the extent to which transactions can be executed with minimal disruption to prices. The term anxious refers to market participants placing orders, especially large orders, with more concern about getting the full order executed on a timely basis and less concern about getting the most favorable possible price based on short-term movements.

For example, consider a major financial institution that decides to substantially alter its portfolio composition by liquidating one holding to establish another holding. Suppose that the current price of the stock the institution wishes to buy is $\$ 50.00$ bid and $\$ 50.01$ offered. A short-term equilibrium for the stock currently holds, wherein there are no traders currently preferring to buy at $\$ 50.01$ and no traders preferring to sell at $\$ 50.00$. For the institution to substantially increase its holding in the shares, other market participants need to be induced to decrease their holdings. If the institution is anxious to increase its holding, it will begin purchasing the shares currently available at the offer price of $\$ 50.01$ and continue buying shares at higher and higher prices to find more and more willing sellers. Depending on the size and urgency of the institution's trades, the buying pressure may drive the price of the stock up perhaps 5 cents, 10 cents, or even more to find a sufficient number of willing sellers. A hedge fund or other market participant may intervene to offer the shares at the increased price in hopes that the price will decline once the buying pressure of the institution dissipates. If the hedge fund does not hold any position in the stock, it may short the stock to satisfy this temporarily increased demand for the stock.

The institution modifying its portfolio may be described as taking liquidity, since the institution's trading activities reduce the current supply of available sellers. More generally, taking liquidity refers to the execution of market orders by a market participant to meet portfolio preferences that cause a decrease in the supply of limit orders immediately near the current best bid and offer prices. The institution is trading to attain its preferred long-term positions.

Market participants who list their bid orders to purchase and offer orders to sell, or who stand by willing to enter the market to take positions offsetting the price pressure, may be described as providing liquidity. Providing liquidity refers to the placement of limit orders or other actions that increase the number of shares available to be bought or sold near the current best bid and offer prices. These providers of liquidity are trading with the primary purpose of making short-term trading profits, not to adjust their positions toward long-term preferences.

A market maker is a market participant that offers liquidity, typically both on the buy side by placing bid orders and on the sell side by placing offer orders. A market maker meets imbalances in supply and demand for shares caused by idiosyncratic trade orders. Typically, the market maker's purpose for providing liquidity is to earn the spread between the bid and offer prices by buying at the bid price and selling at the offer price.

Most hedge funds do not explicitly make markets by bidding to buy and offering to sell the same security at the same time. But many hedge fund managers provide liquidity by searching markets to detect price movements that appear to be driven by orders that are large relative to existing liquidity. For example, when one or more large sellers of a stock cause a drop in a stock price, it entices these hedge fund managers to intervene by buying the stock at the depressed price. In so doing, the hedge fund manager is providing liquidity. The goal of the fund manager in buying at a depressed price is to subsequently liquidate the position when the price recovers from the selling pressure. Conversely, large urgent buy orders can cause price increases that lead hedge funds and other providers of liquidity to short sell shares at the increased price levels.

In the case of an imbalance between buy and sell orders, providers of liquidity should be concerned that they might be taking positions in a firm whose share price is rising or falling due to factors other than liquidity, such as news regarding an unexpected change in anticipated earnings. A quick price movement in a stock may reflect idiosyncratic and temporary trade imbalances, or it may be the first leg of a large unidirectional move due to important fundamental information regarding the stock that is not widely known. If a hedge fund or other provider of liquidity notices a quick price movement and provides liquidity, the provider is taking the risk that the price movement will trend rather than revert toward its previous level. A provider of liquidity succeeds or fails based on the ability to distinguish between liquidity-driven price movements that will reverse and fundamentally driven price movements that will continue to trend.

It should be noted that the institution taking liquidity is very happy that there are hedge funds or other arbitrageurs providing liquidity. Every time an arbitrageur that is providing liquidity executes a trade, the taker of liquidity on the other side of that trade is receiving a price that is better than the price offered by any other market participant. Thus, provision of liquidity can be a long-term source of higher returns to market participants who are skilled at detecting illiquidity and executing appropriate trades. The situation is similar to antique dealers, ticket scalpers, and other traders in used goods who provide liquidity to their markets by being available to buy goods from anxious sellers and sell goods to anxious buyers. Providers of liquidity make money only when they execute trades, and they can execute trades only when they provide the highest available bid price or the lowest available offer price.

Provision of liquidity as a source of long-term superior returns is further discussed later in this session in the section on pairs trading.

\section*{Providing Informational Efficiency}
Another explanation of consistently superior returns with an equity hedge fund is that the profitability results from exploiting the inefficiencies caused by poorly informed traders or traders making decisions based on behavioral biases rather than evidence. Although many academics believe in efficient markets, most hedge fund managers believe that markets are not always efficient and that they can take advantage of temporary inefficiencies in prices. Markets are said to be informationally efficient when security prices reflect available information. Stated another way, when markets are efficient, there is no reliable, consistent way to outperform the market at a risk level that is similar to the market.

Abnormal profit opportunities tend to come from market inefficiencies, and market inefficiencies tend to come from reduced competition. Theoretically, the competition for finding overvalued securities is less than that experienced in the search for undervalued securities, as fewer market participants can or do engage in short selling. The reduced competition for short selling is evidenced in the volume of short interest. Short interest is the percentage of outstanding shares that are\\
currently held short. Choie and Hwang demonstrate that stocks with high short interest tend to underperform the market, with the implication that short sellers are skilled at selecting overpriced securities. ${ }^{1}$ Kenneth S. Choie and S. J. Hwang, "Profitability of Short-Selling and Exploitability of Short Information," Journal of Portfolio Management 20, no. 2 (1994): 33-38. Thus, a potential source of return to equity hedge funds is short selling overpriced securities. By doing so, hedge fund managers provide increased confidence to all market participants that there is a mechanism tending to keep security prices from remaining grossly overpriced for long periods of time.

Asynchronous trading is an example of market inefficiency in which news affecting more than one stock may be assimilated into the price of the stocks at different speeds. A hedge fund manager may observe the release of information or may observe that a particular stock has experienced an abnormally large price change, presumably due to news that affected that stock. The hedge fund manager then establishes a position in another firm that the manager has fundamentally or empirically identified as being expected to experience similar price movements but on a delayed basis.

Another potential source of abnormal profits for hedge funds is overreacting/underreacting, in which short-term price changes are too large or too small, respectively, relative to the value changes that should occur in a market with perfect informational efficiency. For example, analysis of past market prices may indicate tendencies of the stocks of some firms to consistently overreact in the short term to some types of bad news regarding the firm and to eventually correct for the overreaction. If patterns of overreacting and underreacting exist, hedge fund managers can generate consistently superior returns by identifying the patterns and establishing positions that would benefit from repetition of the pattern.

Consistently superior returns from market inefficiencies are a transfer of wealth to the market participant recognizing the inefficiency from the market participant on the other side of each trade. Efforts by market participants to exploit market inefficiencies by purchasing underpriced assets and selling overpriced assets drive prices toward their efficient levels. In a society in which resources are allocated by prices, informationally efficient pricing provides substantially improved resource allocation.

Trading profits are the market-based incentive for participants to perform the analysis that ensures that prices are more efficient, resulting in better-allocated resources. In a market-based economy, the best producers of any goods (based on market values) tend to earn superior profits. Secondary security markets provide liquidity and reveal prices that convey valuable information. Therefore, the most talented and best-informed market participants should be able to earn superior rates of return in excess of their costs of analysis. Otherwise, no market participants would have an incentive to analyze information. The added return may be viewed as a complexity premium. A complexity premium is a higher expected return offered through the consistently lower prices of securities that are difficult to value with precision and therefore must be priced to offer an incentive to market participants to perform the requisite analysis.

\section*{Identifying Factors That Can Create Profit Opportunities}
Some equity hedge fund managers analyze the factors that drive the equity returns of each company in search of those that offer the ability to predict the equities that offer ex ante alpha. These quantitative hedge fund managers use factor models to find those financial variables that explain stock price changes and that might be used in predicting price changes. These are bottom-up models that concentrate on firm-specific financial information as opposed to macroeconomic or industrial data.

As discussed in Level II of the CAIA curriculum, Fama and French show that exposure to value stocks, measured as firms having relatively high ratios of book value to stock price, and small-capitalization stocks explained returns and added average returns, even for portfolios with similar beta exposure, as defined using the capital asset pricing model (CAPM). ${ }^{2}$ Eugene Fama and Kenneth French, "The Cross Section of Expected Stock Returns," Journal of Finance 47, no. 2 (1992): 427-65. Simply put, small-capitalization stocks and stocks viewed as value stocks have demonstrated consistently higher returns than large-capitalization stocks and growth stocks. Debate exists as to why these return differentials have existed in the past and whether they will be exhibited in the future. Since Fama and French's seminal work, numerous factor models have been proposed to explain past returns and, potentially, to predict future returns.

The key issues are twofold: (1) Are expected returns for equities predictable based on past return factors? (2) Would any return factor offering a high expected return be attributable to alpha or beta? As discussed in the Financial Economics Foundations session, if the CAPM holds, then expected returns are determined entirely by one beta: the beta of each asset with the market portfolio. But the CAPM clearly does not hold exactly. It is possible that equity returns are driven by more than one beta and that some betas offer higher rewards for risk than do other betas. Identifying those systematic risk factors (i.e., betas) that offer disproportionate returns is a potential source of consistently superior returns. For example, suppose that equity markets continue to offer higher returns to small-cap value stocks, as identified using historical returns by Fama and French. Would a portfolio manager earning superior future returns with a portfolio of small cap value stocks be better described as having earned those returns from alpha or beta?

The provision of liquidity and attempts to find inefficiently priced assets or lucrative return factors all involve some level of speculation. In this context, speculation is defined as bearing abnormal risk in anticipation of abnormally high expected returns. Abnormal risk and abnormal expected returns are defined here as including any risk and return other than those consistent with and commensurate with a market equilibrium, as described by the CAPM. Speculators have been defended as providing a valuable role in a market-based economy by moving market prices toward more informationally efficient levels. Extensive empirical analysis in various markets has investigated whether increased speculative activity stabilizes or destabilizes market prices. Generally, the results either are inconclusive or tend to find that speculators contribute to reduced price volatility. Note that speculators make profits by buying at low prices and selling at high prices, two actions that tend to stabilize prices.


\end{document}