\documentclass[11pt]{article}
\usepackage[utf8]{inputenc}
\usepackage[T1]{fontenc}
\usepackage{amsmath}
\usepackage{amsfonts}
\usepackage{amssymb}
\usepackage[version=4]{mhchem}
\usepackage{stmaryrd}

\begin{document}
\section*{Reading}
Hedge Fund Fees

What attracts talented managers to the hedge fund industry? Hedge funds allow managers the flexibility to implement sophisticated strategies and to reap financial gains from high returns through fees and through their own investment in the fund. A typical hedge fund fee arrangement has two components: a management fee and an incentive (or performance) fee. The management fee is a constant percentage applied to the net asset value (NAV) of the fund. The NAV is the value of the fund's assets minus its liabilities. An incentive fee is a form of profit sharing wherein managers receive a stated percentage of profits. The incentive fee is applied to the profits of the firm after the management fee has been deducted. Incentive fees are received only if the hedge fund manager earns a profit for investors, and may be subject to other conditions and limitations. The management and incentive fees are typically expressed as a pair of numbers-such as 2 and 20 , which would represent annual management fees of $2 \%$ and incentive fees of $20 \%$ of profit.

Fund managers usually have their own capital invested in the fund. The primary reason for fund managers to devote a substantial portion of their capital to their own fund is to align their financial interests with the financial interests of the limited partners and to communicate their faith in the fund and their alignment of interests to prospective investors. Fund managers therefore can benefit from high fund returns both through the incentive fees and through their investment in the fund.

The attraction of high potential returns and the associated fees generated as a result of high returns, along with the satisfaction of the freedom to use more flexible investment tools and trading strategies, fueled the exodus of talent from traditional investment vehicles to the hedge fund industry, and this pool of talent continues to drive the growth of assets to this brand of investing. In some cases, especially since the financial crisis of 2007 to 2009, investors can negotiate with the manager to reduce the total amount of fees paid. A hedge fund fee arrangement often includes other terms, such as hurdle rates, clawback provisions, and details regarding exact computations and payment dates.

\section*{Computation of Hedge Fund Fees}
Although management fees vary, $1.5 \%$ annual management fees are common in the hedge fund industry and can substantially exceed the management fees of most traditional investment pools, such as mutual funds. Further, a $17.5 \%$ incentive fee is common in hedge funds in addition to the management fee. Incentive fees are generally not found in traditional investment vehicles. Typical hedge fund fees range from $1 \%$ to $3 \%$ for management fees and up to $30 \%$ for incentive fees. The management fee may be collected on a quarterly, semiannual, or annual basis, and the incentive fee is usually collected annually. Incentive fees can be very large when returns are high. The total annual fee as a percentage of fund assets is expressed in Equation 1:


\begin{align*}
& \text { Annual Fee }=\text { Management Fee }+\{\operatorname{Max}[0, \text { Incentive Fee }  \tag{1}\\
& \quad \times(\text { Gross Return above HWM }- \text { Management Fee }- \text { Hurdle Rate })]\}
\end{align*}


In Equation 1, management fee, incentive fee, and hurdle rate are expressed as percentages. HWM denotes the high-water mark of the fund, which is discussed in detail in the next section. The total annual fee in currency can be found by multiplying the percentage fees by the fund's NAV.

Note that in the previous application, the fund's NAV before fees grew by $\$ 103$ million, whereas the NAV after fees grew by only $\$ 70$ million. The key figure of $\$ 100$ million growth (after the management fee) is found by dividing $\$ 70$ million by $(100 \%-30 \%)$, where $30 \%$ is the incentive fee.

\section*{Hedge Fund Fees through Time}
The concept of a high-water mark is crucial to understanding incentive fees through time. The high-water mark (HWM) is the highest NAV of the fund on which an incentive fee has been paid. Thus, the HWM is the highest NAV recorded on incentive fee computation dates but not necessarily the highest overall NAV. If incentive fees are calculated at the end of each calendar year, the HWM would be the maximum of the NAVs corresponding to the last day of each year. In practice, it would be unlikely that the overall highest NAV would happen to occur on the incentive fee computation date.

The idea of paying fees relative to an HWM ensures that fees subsequent to the date on which a fund reached its HWM will not be paid on recouped losses. To illustrate, if a fund's annual year-ending NAV fluctuated between $\$ 100$ million and $\$ 110$ million, the HWM would be $\$ 110$ million. The managers could receive an incentive fee on the profits that were generated to reach the $\$ 110$ million value the first time. However, when the fund's value declined back to $\$ 100$ million and then returned to $\$ 110$ million, the manager would not receive additional incentive fees on the recouped losses. Of course, if managers had returned earlier incentive fees due to a clawback arrangement (discussed in the session, Quantitative Foundations), then the incentive fees would apply to recouped losses.

Consider the following example, which is detailed in the spreadsheet shown in the next exhibit Fee Calculations with and without Hurdle Rate. A hedge fund charges an annual management fee of $2 \%$, and an incentive fee is paid in the amount of $20 \%$ of profits net of the management fee. Fees are paid annually subject to an HWM provision. Each time the annual NAV makes a new high at the end of a period, incentive fees are paid, and a new HWM is set. No incentive fee is paid during a drawdown, which is when losses in NAV have pushed the fund value below its HWM. The idea of an HWM is that incentive fees should be paid only once on each dollar of cumulative net profit to the fund.

The manager of the fund shown in the top panel in the next exhibit is not subject to a hurdle rate. A hurdle rate is a specified minimum return that must be earned by the investor before the incentive fee is applied to profits. Everything else being equal, hurdle rates tend to lower the total fees paid by the investor. In year 1 and year 2 , the fund's investors pay an annual incentive fee of $20 \%$ of the $3 \%$ return net of management fees (incentive fee in this period is $0.6 \%$ of assets) in addition to the $2 \%$ management fee. Notice that the managers do not receive an incentive fee in either year 4 or year 5 . It is clear why no incentive fee applies in year 4, as the fund posted a negative return, and there are no gains to share between the manager and the investor. The reason that no incentive fee is paid in year 5 is the HWM provision, as the $15.6 \%$ gain in year 5 is offset by the $10 \%$ loss in year 4 , as well as the $2 \%$ management fee paid in each year. Because the gains in year 5 are simply offsetting prior losses, no incentive fee is paid in year 5.

The arithmetic average of the gross returns is $7.27 \%$, while the manager in the first example earned average annual fees of $3.00 \%$ over the six-year period. Of course, the investor may not be pleased with this level of fees, as the manager's fee income was nearly as great as the investor's average net return of $4.27 \%$ over the same period. The idea of offering 2 and 20 fee arrangements is to attract managerial talent and effort to design and implement investment strategies that generate high enough returns to provide generous fees to managers and returns net of fees to investors.

One way to reduce the total fees paid is to have a hurdle rate provision included in the investor's subscription agreement. The bottom panel of the next exhibit illustrates a hard hurdle rate, as discussed in the session, Quantitative Foundations. With a $2 \%$ annual management fee and a $3 \%$ annual hurdle rate, the manager earns an incentive fee only when the NAV of the fund before fees exceeds that of the HWM by at least $5 \%$ annually. Notice that the insertion of the $3 \%$ hurdle rate leaves the investor $\$ 2.9$ million richer over the six-year period, as the manager's average annual fee has fallen to $2.6 \%$. The hurdle rate provision saved the investor $0.6 \%$ (that is, the incentive fee of $20 \%$ multiplied by the hurdle rate of $3 \%$ ) in each year the incentive fee was paid. In the example of the lower panel of the next exhibit, with a positive hurdle rate, the investor paid incentive fees only in years 3 and 6 , whereas the investor without the hurdle rate provision also paid incentive fees in years 1 and 2 .

Fee Calculations with and without Hurdle Rate

\begin{center}
\begin{tabular}{|c|c|c|c|c|c|c|c|c|c|c|}
\hline
Year & \begin{tabular}{l}
Gross \\
Return \\
\end{tabular} & \begin{tabular}{l}
Hurdle \\
Rate \\
\end{tabular} & \begin{tabular}{l}
Management Fee \\
$(\%)$ \\
\end{tabular} & \begin{tabular}{c}
Incentive Fee (\% of \\
Profits) \\
\end{tabular} & \begin{tabular}{c}
Incentive Fee (\% of \\
Assets) \\
\end{tabular} & \begin{tabular}{l}
Total \\
Fee \\
\end{tabular} & \begin{tabular}{l}
Net \\
Return \\
\end{tabular} & \begin{tabular}{l}
Beginning \\
NAV \\
\end{tabular} & \begin{tabular}{l}
Ending \\
NAV \\
\end{tabular} & \begin{tabular}{l}
Ending \\
HWM \\
\end{tabular} \\
\hline
Year 1 & $5.00 \%$ & $0.00 \%$ & 2.00 & 20.00 & 0.60 & $2.60 \%$ & $2.40 \%$ & $\$ 100.0$ & $\$ 102.4$ & $\$ 102.4$ \\
\hline
Year 2 & $5.00 \%$ & $0.00 \%$ & 2.00 & 20.00 & 0.60 & $2.60 \%$ & $2.40 \%$ & $\$ 102.4$ & $\$ 104.9$ & $\$ 104.9$ \\
\hline
Year 3 & $20.00 \%$ & $0.00 \%$ & 2.00 & 20.00 & 3.60 & $5.60 \%$ & $14.40 \%$ & $\$ 104.9$ & $\$ 120.0$ & $\$ 120.0$ \\
\hline
Year 4 & $-10.00 \%$ & $0.00 \%$ & 2.00 & 20.00 & 0.00 & $2.00 \%$ & $-12.00 \%$ & $\$ 120.0$ & $\$ 105.6$ & $\$ 120.0$ \\
\hline
Year 5 & $15.60 \%$ & $0.00 \%$ & 2.00 & 20.00 & 0.00 & $2.00 \%$ & $13.60 \%$ & $\$ 105.6$ & $\$ 119.9$ & $\$ 120.0$ \\
\hline
Year 6 & $8.00 \%$ & $0.00 \%$ & 2.00 & 20.00 & 1.20 & $3.20 \%$ & $4.80 \%$ & $\$ 119.9$ & $\$ 125.7$ & $\$ 125.7$ \\
\hline
Year & \begin{tabular}{l}
Gross \\
Return \\
\end{tabular} & \begin{tabular}{c}
Hurdle \\
Rate \\
\end{tabular} & \begin{tabular}{c}
Management Fee \\
$(\%)$ \\
\end{tabular} & \begin{tabular}{c}
Incentive Fee (\% of \\
Profits) \\
\end{tabular} & \begin{tabular}{c}
Incentive Fee (\% of \\
Assets) \\
\end{tabular} & \begin{tabular}{l}
Total \\
Fee \\
\end{tabular} & \begin{tabular}{l}
Net \\
Return \\
\end{tabular} & \begin{tabular}{l}
Beginning \\
NAV \\
\end{tabular} & \begin{tabular}{l}
Ending \\
NAV \\
\end{tabular} & \begin{tabular}{l}
Ending \\
HWM \\
\end{tabular} \\
\hline
Year 1 & $5.00 \%$ & $3.00 \%$ & 2.00 & 20.00 & 0.00 & $2.00 \%$ & $3.00 \%$ & $\$ 100.0$ & $\$ 103.0$ & $\$ 103.0$ \\
\hline
Year 2 & $5.00 \%$ & $3.00 \%$ & 2.00 & 20.00 & 0.00 & $2.00 \%$ & $3.00 \%$ & $\$ 103.0$ & $\$ 106.1$ & $\$ 106.1$ \\
\hline
Year 3 & $20.00 \%$ & $3.00 \%$ & 2.00 & 20.00 & 3.00 & $5.00 \%$ & $15.00 \%$ & $\$ 106.1$ & $\$ 122.0$ & $\$ 122.0$ \\
\hline
Year 4 & $-10.00 \%$ & $3.00 \%$ & 2.00 & 20.00 & 0.00 & $2.00 \%$ & $-12.00 \%$ & $\$ 122.0$ & $\$ 107.4$ & $\$ 122.0$ \\
\hline
Year 5 & $15.60 \%$ & $3.00 \%$ & 2.00 & 20.00 & 0.00 & $2.00 \%$ & $13.60 \%$ & $\$ 107.4$ & $\$ 122.0$ & $\$ 122.0$ \\
\hline
Year 6 & $8.00 \%$ & $3.00 \%$ & 2.00 & 20.00 & 0.60 & $2.60 \%$ & $5.40 \%$ & $\$ 122.0$ & $\$ 128.6$ & $\$ 128.6$ \\
\hline
\end{tabular}
\end{center}

\section*{Incentive Fees and Manager Behavior}
Whereas management fees are widely accepted throughout the money management industry, it is the incentive fee that draws the most scrutiny and publicity to the hedge fund community. Unlike hedge fund managers, most traditional investment managers do not receive fees based on performance. In some cases, managers are prohibited by law from earning fees directly related to investment performance. Incentive fees, like management fees, are designed to compensate managers for their time, effort, and expertise. Further, incentive fees are designed to align manager and investor interests by encouraging fund managers to generate superior returns. The alignment of manager and investor interests can reduce agency costs. However, incentive fees can encourage managers to be aggressive in risk taking, and hence regulators often discourage them, especially for investments open to the public and to smaller, possibly unsophisticated investors. For example,\\
asymmetric incentive fees, in which managers earn a portion of investment gains without compensating investors for investment losses, are generally prohibited for stock and bond funds offered as ' 40 Act mutual funds in the United States.

Perfect alignment of manager and investor interest is not possible, because contracting is not costless and because the parties differ with regard to risk tolerance, diversification, and other factors. Optimal contracting between investors and hedge fund managers attempts to align the interests of both parties to the extent that the interests can be aligned cost-effectively, with marginal benefits that exceed marginal costs.

Managerial coinvesting is an agreement between fund managers and fund investors that the managers will invest their own money in the fund. The idea is that by having their own money in the fund, managers will work hard to generate high returns and control risk. However, the downside to managerial coinvesting can be excessive conservatism by the hedge fund manager. Excessive conservatism is inappropriately high risk aversion by the manager, since the manager's total income and total wealth may be highly sensitive to fund performance. Note that investors tend to be better diversified than managers, meaning they are less exposed to the idiosyncratic risks of the fund in relation to their total wealth.

The idea of incentive fees is to provide managers with a share of upside profits without promoting excessive conservatism through high exposure to losses, as is possible in the case of substantive coinvesting. But incentive fees can generate higher agency costs through perverse incentives. A perverse incentive is an incentive that motivates the receiver of the incentive to work in opposition to the interests of the provider of the incentive. Specifically, the behavior of fund managers may become especially contrary to the interests of the investors depending on the relative values of the fund's NAV and HWM. Kouwenberg and Ziemba suggest that this perverse incentive is substantially reduced when the fund manager invests in the fund along with investors, especially when the investment exceeds $30 \%$ of the manager's personal net worth, as the upside from additional incentive fees earned on risky investments is offset by the potential losses of the manager's personal investment in the fund. ${ }^{1}$ Roy Kouwenberg and William T. Ziemba, "Incentives and Risk Taking in Hedge Funds," Journal of Banking and Finance 31, no. 11 (2007): 3291-310.

\section*{The Present Value of a Hedge Fund Fee Annuity}
The purpose of this section is to demonstrate the potential value of an annuity of fees available to hedge fund managers to illustrate the manager's incentive to maintain profitability. The annuity view of hedge fund fees represents the prospective stream of cash flows from fees available to a hedge fund manager. The example assumes a standard $2 \%$ management fee and a $20 \%$ incentive fee that is distributed to the fund manager each year. The example allows for a hurdle rate (or preferred return), even though hedge funds are less likely to have these provisions than are private equity funds. For simplicity, it is assumed that the fund earns a constant rate of return and uses that same rate as a discount rate in present value computations. No additional capital is invested in the fund, no partners withdraw their funds, and no distributions are made to the limited partners until the fund liquidates.

For example, consider Fund A with an initial NAV of $\$ 100$ million. Fund A earns $17 \%$ each year, and there is no preferred return. Thus, at the end of year 1 , Fund A has gross earnings of $\$ 17$ million, from which it distributes $\$ 2$ million as a management fee ( $2 \%$ of the starting-year NAV of $\$ 100$ million) and $\$ 3$ million in incentive fees ( $20 \%$ of the net profit of $\$ 15$ million). Fund A begins year 2 with an NAV of $\$ 112$ million ( $\$ 117-\$ 2-\$ 3$ ), on which it again earns $17 \%$ before fees. In the second year, Fund A earns $\$ 19.04$ million ( $17 \%$ of $\$ 112$ million), pays a management fee of $\$ 2.24$ million ( $2 \%$ of the starting-year NAV of $\$ 112$ million) and an incentive fee of $\$ 3.36$ million (20\% of the net profit of $\$ 16.8$ million), and ends year 2 with an NAV of $\$ 125.44$ million.

Now suppose that Fund A liquidates its $\$ 125.44$ million in assets and distributes the proceeds to its limited partners at the end of year 2 . The value of this distribution to the limited partners, discounted at $17 \%$, is $\$ 91.636$ million, as shown in the following equation:

$$
\mathrm{PV}=\$ 125.44 \text { million } /(1.17)^{2}=\$ 91.636 \text { million }
$$

Since Fund A began with $\$ 100$ million, the remaining $\$ 8.364$ million represents the present value (PV) of the fees distributed to the managers in years 1 and 2 :

$$
\mathrm{PV}=(\$ 5.00 \text { million } / 1.17)+\left[\$ 5.60 \text { million } /(1.17)^{2}\right]=\$ 8.364 \text { million }
$$

Thus, 8.36\% of Fund A's present value was distributed to managers in the form of management and performance fees. Note that Fund A's investors received a 12\% annual return, which is the IRR (internal rate of return) from investing $\$ 100$ million and receiving proceeds of $\$ 125.44$ million after two years. This $12 \%$ return is a result of earning $17 \%$ before fees each year, while paying $2 \%$ in management fees and $3 \%(20 \% \times 15 \%)$ in incentive fees each year. It is reasonable to believe that investors will be happy with earning $12 \%$ from providing capital and taking risk, and that managers will be happy with dedicating their talent and time in generating superior fund profits in return for receiving management and incentive fees.

The same approach is used to examine the effects of changing rates of return, hurdle rates, and fund longevity on the percentage of a fund's value that is received by fund managers. The next exhibit summarizes the results under a variety of scenarios.

The second-to-last row of the next exhibit contains the values for a fund that earns $17 \%$ per year before fees and has no hurdle rate. For two-year investments, the present value of fees equals approximately $8.4 \%$ of the NAV, as previously demonstrated in the example of Fund A. Note that if the time horizon is extended to 10 years, the percentage rises to $35.4 \%$. It may initially appear surprising that the number is so large when fees are being discounted at $17 \%$. However, the key factor that drives the magnitude of the percentages associated with longer-term horizons is that fees are distributed annually from the fund to the managers while the profits of the investors are reinvested. Thus, long-term time horizons enable managers to collect fees on retained earnings in addition to fees on the initial investments. When the fees on reinvested earnings are included and are expressed as a percentage of the initial investment, the percentage becomes large.

Percentage of NAV Earned by the Hedge Fund before Fees and Distributed to Managers in the Form of $2 \% / 20 \%$ Fees

\begin{center}
\begin{tabular}{|cccccccccc|}
\hline
 &  &  &  &  & \multicolumn{2}{c|}{Longevity (years)} &  &  \\
Returns & Hurdle Rate & $\mathbf{1}$ & $\mathbf{2}$ & $\mathbf{5}$ & $\mathbf{1 0}$ & $\mathbf{2 5}$ & $\mathbf{1 0 0}$ & $\mathbf{5 0 0}$ \\
\hline
$7 \%$ & $0 \%$ & $2.8 \%$ & $5.5 \%$ & $13.3 \%$ & $24.8 \%$ & $50.9 \%$ & $94.2 \%$ & $100.0 \%$ \\
$7 \%$ & $5 \%$ & $1.9 \%$ & $3.7 \%$ & $9.0 \%$ & $17.2 \%$ & $37.6 \%$ & $84.8 \%$ & $100.0 \%$ \\
\hline
\end{tabular}
\end{center}

\begin{center}
\begin{tabular}{|cccccccccc|}
\hline
 &  & \multicolumn{8}{c}{Longevity (years)} \\
\cline { 9 - 10 }
Returns & Hurdle Rate & $\mathbf{1}$ & $\mathbf{2}$ & $\mathbf{5}$ & $\mathbf{1 0}$ & $\mathbf{2 5}$ & $\mathbf{1 0 0}$ & $\mathbf{5 0 0}$ &  \\
\hline
$12 \%$ & $0 \%$ & $3.6 \%$ & $7.0 \%$ & $16.6 \%$ & $30.5 \%$ & $59.7 \%$ & $97.4 \%$ & $100.0 \%$ &  \\
$12 \%$ & $5 \%$ & $2.7 \%$ & $5.3 \%$ & $12.7 \%$ & $23.8 \%$ & $49.3 \%$ & $93.4 \%$ & $100.0 \%$ &  \\
$17 \%$ & $0 \%$ & $4.3 \%$ & $8.4 \%$ & $19.6 \%$ & $35.4 \%$ & $66.4 \%$ & $98.7 \%$ & $100.0 \%$ &  \\
$17 \%$ & $5 \%$ & $3.4 \%$ & $6.7 \%$ & $16.0 \%$ & $29.4 \%$ & $58.1 \%$ & $96.9 \%$ & $100.0 \%$ &  \\
\hline
\end{tabular}
\end{center}

It should be noted that the above exhibit does not indicate that investors suffer poor returns. As indicated earlier, investors earned $12 \%$ per year after fees when the assets generated $17 \%$ before fees. To the extent that the returns generated on the fund's assets substantially exceed the returns available on other investments, the investors can enjoy superior rates of return under a 2 and 20 fee arrangement. As long as the gross returns are high, the investors will do well.

The above exhibit indicates two primary implications of a traditional hedge fund compensation scheme, such as a 2 and 20 arrangement. First, there is an enormous incentive for fund managers to generate high returns. The significant benefits of doing so attract the best managers to the hedge fund space. Note that in the exhibit above, the value of the potential fees that managers can collect does not include the fees that they can ultimately receive on additional investments attracted by the high returns. With growth opportunities from new investors included, managers can reap even higher financial benefits from superior performance. The exhibit above also does not reflect the managers' profit on their own investment in the fund.

Second, the exhibit above illustrates the importance to fund managers of being able to remain in operation. It is the ability to earn fees on reinvested money and on newly attracted capital that offers managers the highest long-run benefits. Thus, managers have a very strong incentive to avoid poor returns, retain existing investors, and attract new investors.

The numerical analysis of this section assumed a constant rate of asset returns, essentially ignoring uncertainty. In practice, returns are likely to experience volatility and vary between high returns and low returns. If a fund experiences negative returns within a reporting period, the fund's manager may view the fund as likely to close, in which case the manager may have a strong incentive to take excessive risks in an attempt to recoup losses and stay in business. Even if the manager does not fear that the fund will close, if the fund's NAV falls substantially below its HWM, the manager may foresee no realistic chance of earning incentive fees in the near term unless the fund's risk is increased. Thus, an incentive fee structure may encourage enormous risk taking by managers.

Further, one of the benefits to fund managers of incentive fees is the ability to earn fees on high returns while not being liable for losses, other than possibly returning incentive fees due to clawback provisions. This asymmetric arrangement is another factor that may encourage enormous risk taking by managers. Thus, uncertainty in fund returns can have enormous implications on the behavior of managers with regard to risk taking. The next section explores the implications of uncertainty using option theory.

\section*{Hedge Fund Fees and Option Theory}
The previous section illustrated the importance to a manager of maintaining high performance through an annuity view of hedge fund fees. The analysis assumed that the fund generated a constant profit. Of course, hedge fund returns contain volatility. This section illustrates the effect of return volatility on the value of managerial incentive fees using option theory. The option view of incentive fees uses option theory to demonstrate the ability of managers to increase the present value of their fees by increasing the volatility of the fund's assets.

This section focuses on a one-period model of incentive fees and assumes that the fund's hurdle rate is zero. In a more realistic framework, the manager not only has to be concerned with the value of the current period's incentive fees but also has to examine the effects of decisions on the future value of incentive and asset management fees.

Hedge fund incentive fees can be considered a call option on a portion of the profits that the hedge fund manager earns for investors. If the fund earns a profit, the manager collects an incentive fee. The hedge fund manager who does not generate a profit collects no incentive fee. The call option is on the fund's NAV, with a strike price equal to the HWM and an expiration date equal to the end of the period to which the incentive fee applies. This payoff is described using Equation 2 :

Payout on Incentive Fee Option $=\operatorname{Max}[i($ ENAV $-\mathrm{BNAV}), 0]$

where $i$ is the incentive fee rate (e.g., 20\%), ENAV is the ending NAV of the hedge fund, and BNAV is the strike price of the call option, which is equal to the NAV of the hedge fund at the start of the period or the fund's HWM if it is larger.

The maturity, or expiration, of the incentive fee option is one year. The manager pays for the option by providing time, effort, and talent. If the option is out-of-themoney at maturity (the end of the year), the hedge fund manager receives no incentive fee. Alternatively, if the option is in-the-money at the end of the year, the hedge fund manager can be considered as exercising the option and collecting the incentive fee. At the beginning of every year, the hedge fund manager receives a new call option. The new call option is at-the-money whenever an incentive fee has just been paid and there is no hurdle rate. The new call option is out-of-themoney to the extent that either the fund's NAV is below its HWM or the fund's hurdle rate exceeds zero.

The incentive fee option value is the risk-adjusted present value of the incentive fees to a manager that have been adjusted for its optionality. The incentive fee option value is often expressed using the Black-Scholes option pricing model, which generates the price of an option using five inputs. These five inputs, with their corresponding values in parentheses, are the current value of the underlying assets (the fund's NAV), the strike price (the higher of the beginning-of-period NAV or the HWM), the time until maturity of the option (in this discussion, one year), the risk-free rate (a one-year risk-free bond yield), and the volatility of the underlying asset's returns (the standard deviation of the returns of the fund's NAV). Note that in the case of a hurdle rate, the strike price is the future value of the NAV using the hurdle rate.

The Black-Scholes option pricing model can be easily solved on a spreadsheet given these five values. However, a much easier approximation for at-the-money option prices can be used for discussion purposes. The at-the-money incentive fee approximation expresses the value of a managerial incentive fee as the product of $40 \%$, the fund's NAV, the incentive fee percentage, and the volatility of the assets $\left(\sigma_{1}\right)$ over the option's life. This approximation, which assumes that the incentive fee option is at-the-money and interest rates are very low, is shown in Equation 3 and provides a reasonably accurate approximation of the value of a manager's incentive fee over one incentive fee computation period:

Consider a $\$ 100$ million hedge fund with a $20 \%$ incentive fee for its managers. At the beginning of a one-year period, when the incentive fee option has been reset to being at-the-money, the approximated value of the incentive fee call option is as shown in Equation 3:

Incentive Fee Call Option Value $\approx 8 \% \times \$ 100$ million $\times \sigma_{1}$

Inserting annual volatilities of $5 \%, 10 \%$, and $20 \%$ would generate values of the one-year incentive fee of $\$ 0.4$ million, $\$ 0.8$ million, and $\$ 1.6$ million, respectively. Equation 3 demonstrates the tremendous influence of the volatility of the fund's assets on the value of the incentive fee held by managers. Simply put, the fund manager can manipulate the value of the single-period incentive fee call option into any value desired by simply changing the volatility of the fund's assets through the fund's investment strategy. Note that the relationship is not intended to be applied when the incentive fee call option is in-the-money or out-of-the-money.

Option analysis reveals several important implications of how the incentive fee can affect hedge fund manager behavior. First, hedge fund managers can increase the value of their incentive fee call options by increasing the volatility of a hedge fund's NAV. The holder of a call option will always prefer more volatility in the value of the underlying asset, because the greater the volatility, the greater the upside profits, whereas downside losses are limited.

Unlike managers, investors have more symmetric payoffs, since they must bear all of the losses when a fund's NAV falls below its HWM. This establishes a key conflict of interest between investors in the hedge fund and the hedge fund manager. Investors in the hedge fund own the underlying partnership units and receive payoffs offered by the entire distribution of return outcomes associated with the hedge fund NAV.

The costs associated with perverse incentives, meaning a potential increase in strategic risk taking, must be compared to the potential benefits that the performance fee provides, such as the alignment of other interests. In the absence of an incentive fee, managers may become pure asset gatherers, driven by the annuity view of fees. A pure asset gatherer is a manager focused primarily on increasing the AUM of the fund. A pure asset gatherer is likely to take very little risk in a portfolio and, like mutual fund managers, become a closet indexer. A closet indexer is a manager who attempts to generate returns that mimic an index while claiming to be an active manager.

A second important implication of the option view of hedge fund fees is how hedge fund managers react when their incentive fee call option is far-out-of-the-money. This happens when the hedge fund NAV has declined substantially below the HWM or when the fund has an unrealistically high hurdle rate, causing the strike price for the incentive fee option to be substantially higher than the NAV of the hedge fund. When an option is out-of-the-money, the hedge fund manager has two choices for increasing the value of the option. The first is to increase the volatility of the underlying asset, as demonstrated with option theory. The second is to pursue repricing of the option. When the incentive fee call option is far-out-of-the-money, it is unlikely that the hedge fund manager's current investors will allow the manager to lower the HWM. Therefore, hedge fund managers with incentive fees that are far-out-of-the-money have an incentive to close the existing fund and start a new hedge fund.

\section*{Hedge Fund Fees and Managerial Behavior}
There is no doubt that hedge fund incentive fees motivate managers to try to generate higher expected returns. The primary issue that arises is the extent to which managerial decisions with regard to risk taking conflict with the preferences of the fund's investors. The annuity view of hedge fund fees indicates the enormous fees available to managers for being able to sustain long-term growth in assets. The option view of hedge fund fees indicates the enormous gains in single-period expected incentive fees that managers can generate by increasing the volatility of a fund's assets. The literature demonstrates some interesting facts regarding hedge fund fees and how they may influence the behavior of hedge fund managers.

\section*{1. Managers May Take Fewer Risks after a Period of High Returns and Take More Risks after a Period of Negative Returns}
Hodder and Jackwerth's "Incentive Contracts and Hedge Fund Management" has a number of interesting theoretical results regarding incentive fees. ${ }^{2}$ James E. Hodder and Jens Carsten Jackwerth, "Incentive Contracts and Hedge Fund Management," Journal of Financial and Quantitative Analysis, 42, no. 4 (2007), 811-26. They model the financial preferences for hedge fund managers being compensated with incentive fees and find that consistent with a single-period option view, fund managers have an incentive to take large risks, and that this preference for risk taking depends "dramatically" on fund value. When the incentive option is far-intothe-money, the wealth effects to managers from risk taking are nearly symmetrical, and therefore excessive risk taking is not encouraged. Rather, managers have an incentive to lower risk to preserve their fees, known as the lock-in effect. The lock-in effect in this context refers to the pressure exerted on managers to avoid further risks once high profitability and a high incentive fee have been achieved.

Hodder and Jackwerth further note that as fund values decline and the incentive option becomes far-out-of-the-money, the payoff to managers is skewed to the right, and risk taking is strongly encouraged. In a multiple-period framework, Hodder and Jackwerth find that risk-taking behavior is rapidly moderated, or brought into reasonable bounds, when the fund experiences acceptable levels of positive subsequent return performance. They also find that if the fund asset value continues to decline, there is a point at which it is optimal for the fund manager to close the fund to pursue other opportunities. However, as a fund's value approaches the point that will trigger a decision to close, the manager acquires an especially strong incentive to take even higher risks. These managerial behaviors\\
are not optimal from the perspective of the fund's investors. Common sense says that fund investors would prefer that risk-taking behavior be governed by analysis of market opportunities rather than by the effect of risk on the manager's compensation.

\section*{2. Managers May Modify the Time Series of Returns to Enhance Risk-Adjusted Performance or to Improve the Number of Profitable Months}
A possible role of incentive fee structures in influencing managerial decisions involves dynamic behavior during the period in which the incentive fee is being applied. Agarwal, Daniel, and Naik examine the hypothesis that hedge funds have incentives to manage or massage reported returns upward as the accounting period for computing incentive fees is ending. ${ }^{3}$ Vikas Agarwal, Noveen Daniel, and Narayan Naik, "Role of Managerial Incentives and Discretion in Hedge Fund Performance," Journal of Finance 64 (October 2009): 2221-56. The terms managing returns and massaging returns refer to efforts by managers to alter reported investment returns toward preferred targets through accounting decisions or investment changes. Consistent with this hypothesis, Agarwal and colleagues find that December returns for hedge funds were higher than other months by $1.5 \%$ and that, after controlling for risk, residual returns continued to be $0.4 \%$ higher. The authors conclude that hedge funds may be managing, or massaging, their reported returns. However, they cannot explain why returns were unusually low between June and October of each year.

Kazemi and Li show that hedge fund managers may manage the volatility of their return processes to balance several risks and incentives. ${ }^{4}$ Hossein Kazemi and Ying Li, "Managerial Incentives and Shift of Risk-Taking in Hedge Funds" (working paper, Isenberg School of Management, University of Massachusetts, Amherst, 2008). A fund manager has an incentive to increase the fund's volatility, especially if the option is about to expire out-of-the-money, meaning that the fund's NAV is below its HWM near the end of the period. However, higher volatility increases the probability that the fund may experience negative performance. This can negatively affect the manager's welfare in four ways: (1) a manager who has personal capital invested in the fund will have to share in the losses; (2) negative performance means that the fund's NAV will be further below the HWM, making it less likely that the manager can collect incentive fees in the future; (3) negative performance could lead investors to redeem their capital, reducing the asset management fees of the fund for current and future periods; and (4) negative performance, along with higher volatility, could damage the fund manager's reputation, reducing future income. Kazemi and Li empirically test the impact of all these incentives on the behavior of hedge fund managers. They show that a manager tends to increase the fund's return volatility if (1) the incentive option is at-the-money, (2) the fund's NAV has spent a significant amount of time under the HWM, and (3) the fund's assets are liquid enough to allow the manager to adjust the fund's volatility. Further, they show that small and young funds do not tend to adjust their volatility: At a time when fund managers are trying to establish their reputations, they are reluctant to risk losing their assets by adjusting the volatility of their funds.

Thus, both theoretical and empirical analyses indicate that managers respond to incentive fees in ways that include behavior other than attempts to serve fund investors. The implications of incentive fees are numerous and substantial. Hedge fund managers usually understand options and the implications of their management decisions on their wealth. However, managerial behavior should not be overgeneralized based on potentially perverse incentives. Most managers may be driven toward serving investors based on ethical considerations or in order to preserve their professional reputations and their ability to continue to work in their chosen field. Further, to the extent that incentive fees generate perverse incentives, hedge fund investors have an incentive to demand compensation arrangements that ameliorate the difficulties of excessive risk taking, such as requiring substantial co-investment by managers. The primary conclusions should be the importance of understanding the potential conflicts of interest caused by incentive fees and the need to perform careful due diligence and monitor managerial behavior.


\end{document}