\documentclass[11pt]{article}
\usepackage[utf8]{inputenc}
\usepackage[T1]{fontenc}
\usepackage{amsmath}
\usepackage{amsfonts}
\usepackage{amssymb}
\usepackage[version=4]{mhchem}
\usepackage{stmaryrd}

\begin{document}
\section*{APPLICATION A}
Question :TTMAR Hedge Fund has a 1.5 and 30 fee arrangement, with no hurdle rate and a NAV of $\$ 200$ million at the start of the year. At the end of the year, before fees, the NAV is $\$ 253$ million. Assuming that management fees are computed on start-of-year NAVs and are distributed annually, find the annual management fee, the incentive fee, and the ending NAV after fees, assuming no redemptions or subscriptions.

\section*{EXPLANATION and Answer}
The annual management fee is simply $1.5 \%$ of $\$ 200$ million, or $\$ 3$ million. After the management fee of $\$ 3$ million, the fund earned a profit of $\$ 50$ million $(\$ 253-\$ 3-\$ 200)$. The incentive fee on the profit is $\$ 15$ million $(\$ 50 \times 30 \%=\$ 15)$. Therefore, the ending NAV after distribution of fees to the fund manager is $\$ 235$ million $(\$ 253-\$ 3-\$ 15)$.

\section*{APPLICATION B}
Question : VVMAR Hedge Fund has a 1.5 and 30 fee arrangement, with no hurdle rate and a NAV of $\$ 200$ million at the start of the year. At the end of the year, after fees, the NAV is $\$ 270$ million. Assuming that management fees are computed on start-of-year NAVs and are distributed annually, find the annual management fee, the incentive fee, and the ending NAV before fees, assuming no redemptions or subscriptions.

\section*{Answer and Explanation}
The incentive fee represents $30 \%$ of the total profits and so represents the proportion $30 \% / 70 \%$ to the net profits to limited partners. Since the profit to the limited partners is $\$ 70$ million, the incentive fee to the manager must be $\$ 30$ million (i.e., $\$ 70$ million $\times 30 \% / 70 \%$ ). Thus, the NAV after management fees but before incentive fees must be $\$ 300$ million. The management fees are $1.5 \%$ of the starting NAV: $1.5 \% \times \$ 200$ million $=\$ 3$ million, inferring an ending NAV of $\$ 303$ million before fees. To recap: $\$ 303$ million is reduced to $\$ 300$ million by the $1.5 \%$ management fee on the starting value of $\$ 200$ million. The fund therefore earned a profit of $\$ 100$ million after management fees ( $\$ 300$ million $-\$ 200$ million). The incentive fee to the manager was $30 \%$ of $\$ 100$ million, or $\$ 30$ million. The profit after fees to the limited partners was $\$ 70$ million, leaving a NAV of $\$ 270$ million after all fees.

\section*{APPLICATION C}
Question : Consider a $\$ 1$ billion hedge fund with a $20 \%$ incentive fee at the start of a new incentive fee computation period. If the hedge fund computes incentive fees annually and begins the year very near its high-water mark, what would be the value of the incentive fee over the next year for annual asset volatilities of $10 \%$, $20 \%$, and $30 \%$ using the at-the-money incentive fee approximation formula?

Inserting $i=20 \%, \mathrm{NAV}=\$ 1$ billion, $T=1$, and the three given volatilities generates approximations of $\$ 8$ million, $\$ 16$ million, and $\$ 24$ million.

\section*{Answer and Explanation}
Utilizing Equation 3, we can solve for the annual asset volatilities of $10 \%, 20 \%$, and $30 \%$.

$$
\text { Incentive Fee Call Option Value }=i \times 40 \% \times N A V \times \sigma_{1}
$$

The parameters are $i=20 \%$ and NAV $=\$ 1$ billion.

Solving for the $10 \%$ annual asset volatility:

Incentive Fee Call Option Value $\approx 0.20 \times .40 \times \$ 1,000,000,000 \times 0.10$ Incentive Fee Call Option Value $\approx 0.08 \times \$ 1,000,000,000 \times 0.10$ Incentive Fee Call Option Value $\approx \$ 80,000,000 \times 0.10$ Incentive Fee Call Option Value $\approx \$ 8,000,000$

Solving for the $20 \%$ annual asset volatility:

Incentive Fee Call Option Value $\approx 0.20 \times .40 \times \$ 1,000,000,000 \times 0.20$

Incentive Fee Call Option Value $\approx 0.08 \times \$ 1,000,000,000 \times 0.20$

Incentive Fee Call Option Value $\approx \$ 80,000,000 \times 0.20$

Incentive Fee Call Option Value $\approx \$ 16,000,000$

Solving for the $30 \%$ annual asset volatility:

Incentive Fee Call Option Value $\approx 0.20 \times .40 \times \$ 1,000,000,000 \times 0.30$

Incentive Fee Call Option Value $\approx 0.08 \times \$ 1,000,000,000 \times 0.30$

Incentive Fee Call Option Value $\approx \$ 80,000,000 \times 0.30$

Incentive Fee Call Option Value $\approx \$ 24,000,000$


\end{document}