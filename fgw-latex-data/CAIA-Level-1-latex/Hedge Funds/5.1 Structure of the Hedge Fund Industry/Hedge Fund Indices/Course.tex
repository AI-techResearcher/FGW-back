\documentclass[11pt]{article}
\usepackage[utf8]{inputenc}
\usepackage[T1]{fontenc}
\usepackage{amsmath}
\usepackage{amsfonts}
\usepackage{amssymb}
\usepackage[version=4]{mhchem}
\usepackage{stmaryrd}

\begin{document}
Hedge Fund Indices

Most traditional investment funds can be reasonably benchmarked to an index of traditional investments. But Brown, Goetzmann, and Ibbotson contend that a hedge fund investment is almost a pure bet on the skill of a specific manager. ${ }^{1}$ Stephen Brown, William Goetzmann, and Roger Ibbotson, "Offshore Hedge Funds: Survival and Performance, 1989-1995," Journal of Business 72, no. 1 (1999): 91-117. Hedge fund managers tend to seek out arbitrage or mispricing opportunities in the financial markets, using a variety of cash and derivative instruments. They typically take small amounts of market exposure to exploit mispricing opportunities but employ large amounts of leverage to extract higher potential value. The key point is that hedge fund managers pursue investment strategies that usually cannot be clearly associated with a conventional financial market benchmark. The investment styles of hedge funds are alpha driven rather than beta driven. Capturing the risk and returns of skill-based investing in a benchmark index can be problematic. Still, hedge fund indices are constructed and published for two key reasons. First, they can serve as a proxy for a hedge fund asset class, which is important for asset allocation studies. Second, they can serve as performance benchmarks to judge the success or failure of hedge fund managers.

Hedge fund indices start with the collection of a hedge fund database. Unlike returns on traditional listed assets, the returns on hedge funds are not centrally reported. Hedge fund managers may choose to report their returns to one or more database collectors. The databases, in turn, may be used by hedge fund index providers to construct indices and to calculate and publish returns.

There are more than 15 hedge fund index providers, each with its own unique way of constructing databases and benchmarks. Each index is computed with a different number of constituent hedge funds, and there is relatively little overlap. Most of the indices use equally weighted returns across hedge funds, whereas the others use assets under management to weight the returns on individual hedge funds. Also, some index providers collect the underlying data themselves, whereas others allow the hedge fund managers to enter the data. Some hedge fund indices include managed futures, whereas some do not. In sum, there are many different construction techniques of hedge fund indices. We discuss the challenges of implementing these methodologies in the following sections.

\section*{Management and Incentive Fees}
According to Fung and Hsieh, more than 70\% of live hedge funds (in Trading Advisor Selection System [TASS], HFR, and Center for International Securities and Derivatives Markets [CASAM/CISDM] databases) charge a management fee between $1 \%$ and $2 \% .{ }^{2}$ William Fung and David Hsieh, "Hedge Funds: An Industry in Its Adolescence," Federal Reserve Bank of Atlanta Economic Review 91, no. 4 (May 2006): 1-33. The same authors also find that the majority of live hedge funds in the aforementioned three databases charge a $20 \%$ incentive fee. Fees have fallen since this study was published, with most incentive fees now between $15 \%$ and $20 \%$ with management fees of $1.5 \%$ as common as $2 \%$.

All hedge fund indices calculate hedge fund performance net of fees. However, two issues related to fees can result in different performance than portrayed by a hedge fund index.

First, incentive fees are normally calculated on an annual basis. However, all of these indices provide month-by-month performance. Therefore, on a monthly basis, incentive fees must be forecasted and subtracted from performance. Since the forecasted fees may be different from the actual fees collected at year-end, the estimated monthly returns may contain estimation errors.

Second, hedge funds are a form of private investing. Indeed, virtually all hedge funds are structured as private limited partnerships. As a consequence, often the terms of specific investments in hedge funds may not be negotiated in a consistent manner among different investors or across different time periods. The lack of consistency means that the net-of-fee returns earned by one investor may not be what another investor can negotiate. In fact, the more successful the hedge fund manager, the greater the likelihood that the manager will increase the fee structure to take advantage of that success. We call this fee bias. Fee bias is when index returns overstate what a new investor can obtain in the hedge fund marketplace because the fees used to estimate index returns are lower than the typical fees that a new investor would pay.

\section*{Inclusion of Managed Futures}
Managed futures funds, or commodity trading advisers (CTAs), are sometimes considered a subset of the hedge fund universe and are therefore included in index construction. These are investment managers who invest in the commodity futures markets using either fundamental economic analysis or technical analysis such as trend-following models. They may invest in financial futures, energy futures, agriculture futures, metals futures, livestock futures, or currency futures. Because their trading style (mostly trend-following models) and the markets in which they invest are different from those of other hedge fund managers, CTAs and managed futures accounts are sometimes segregated from the hedge fund universe. Thus, returns may vary across hedge fund indices due to the decision to include or exclude managed futures funds.

\section*{Asset Weighted versus Equally Weighted}
A hedge fund index return is constructed as an average of the returns on the underlying funds. Some databases report returns on an equally weighted basis, in which the returns to each fund have the same influence on the index return. Other databases report asset-weighted returns, in which the largest funds have the most significant impact on returns. HFR data indicate that the largest $19 \%$ of hedge fund managers now control nearly $90 \%$ of hedge fund industry assets, and the smallest half of hedge fund managers, those with AUM below $\$ 100$ million, control only $1.2 \%$ of industry assets.

Equal weighting has the advantage of not favoring large funds or hedge fund strategies that attract a lot of capital, like global macro or relative value. The downside to an equally weighted index is that the small funds together have an extremely large weight in the reported index returns, yet a relatively small role in determining the returns experienced by actual investors in hedge funds. An asset-weighted index is dominated by large funds and is therefore influenced by the flows of capital. Some of the largest funds choose not to report their data to public databases, so it may be difficult to interpret an asset-weighted index return that does not include some of the larger hedge funds. Most hedge fund index providers argue that a hedge fund index should be equally weighted to fully reflect all strategies.

There are further worthwhile arguments for and against an asset-weighted hedge fund index. First, smaller hedge funds can transact with a smaller market impact, which enables them to do so at more favorable prices. An asset-weighted index more accurately reflects the market impact experienced by the majority of the money invested in hedge funds. Second, many other asset classes are benchmarked against capitalization-weighted (cap-weighted) indices. The S\&P 500 and the Russell

1000, for example, are cap-weighted equity indices. Large institutional investors use these cap-weighted indices in their asset allocation decision models. Therefore, to compare on an apples-to-apples basis, hedge fund indices should also be asset weighted when used for asset allocation decisions. However, this argument might be moot. Empirically, equally weighted and asset-weighted hedge fund indices have similar correlations to equity and fixed-income indices.

\section*{The Size of the Hedge Fund Universe}
One of the problems with constructing a hedge fund index is that the size of the total universe of hedge funds is not known with certainty. This uncertainty regarding the true size of the hedge fund industry stems from its loosely regulated nature. Especially in the past, hedge funds enjoyed relative secrecy compared to their mutual fund counterparts. For example, in the United States, mutual funds are regulated investment companies that are required, along with investment advisers, to register with the SEC, since mutual funds are considered public investment companies that issue public securities on a continual basis. Further, they are required by law to report and publish their performance numbers to the SEC and to the public. Recent regulatory changes around the world are resulting in more hedge fund managers having a more public profile, as registration with local authorities is now more frequently mandated than in the past.

Although the hedge fund industry has become more transparent, Liang demonstrates a good example of the lack of knowledge about the exact size of the hedge fund universe. $^{3}$ Bing Liang, "Hedge Funds: The Living and the Dead," Journal of Financial and Quantitative Analysis 35, no. 3 (2000): 309-26. He studied the composition of indices constructed by two well-known providers: TASS and Hedge Fund Research, Inc. At the time of his study, there were 1,627 hedge funds in the TASS index and 1,162 hedge funds in the HFR Index. He found that only 465 hedge funds were common to both hedge fund indices. Further, of these 465 common hedge funds, only 154 had data covering the same time period.

Another problem with measuring the size of the hedge fund universe is that the attrition rate for hedge funds is quite high. Brown, Goetzmann, and Ibbotson and Park, Brown, and Goetzmann find that the average life of a hedge fund manager is 2.5 to 3 years, meaning that there will be considerable differences with respect to hedge fund index composition. ${ }^{4}$ Brown, Goetzmann, and Ibbotson, "Offshore Hedge Funds"; James Park, Stephen Brown, and William Goetzmann, "Performance Benchmarks and Survivorship Bias for Hedge Funds and Commodity Trading Advisors," Hedge Fund News, August 1999. In conclusion, there are large differences in the compositions of various hedge fund indices with relatively little overlap. As a result, many investors purchase access to several databases and combine the funds listed to get a more complete view of the universe. Hedge fund mortality may increase over time. Getmansky, Lee, and Lo show that the attrition rate from the Lipper TASS database was $6 \%$ to $10 \%$ each year from 1996 to 2006, but increased to between $15 \%$ and $22 \%$ each year from 2007 to $2012 .{ }^{5}$ Mila Getmansky, Peter A. Lee, and Andrew W. Lo, "Hedge Funds: A Dynamic Industry in Transition" (working paper, MIT Laboratory for Financial Engineering, 2014).

\section*{Representativeness and Data Biases}
Representativeness is a key aspect of hedge fund databases and indices. The representativeness of a sample is the extent to which the characteristics of that sample are similar to the characteristics of the universe. If the sample consistently favors inclusion of observations based on a particular characteristic, then the sample is biased in favor of that characteristic. There are several important data biases associated with hedge fund databases.

Survivorship bias arises when an index is constructed that disproportionately includes past returns of those investments that remain in operation, meaning they have survived, while excluding the return histories of those investments that have not survived. This means that the past performance of the index contains an upward bias in comparison to the true performance of all funds that were available in the past. The reason for the bias is that surviving funds are likely to have outperformed those funds that have left the industry. In other words, an investor in a diversified portfolio of funds several years ago would have earned a return lower than what is reported by an index that is constructed today using the past performance of surviving funds. The survivorship bias can be measured as the average return of surviving funds in excess of the average return of all funds, both surviving and defunct. Survivorship bias has been estimated as $2.6 \%$ to $5 \%$ per year. This bias is also common with mutual funds and other traditional investments.

A common misperception is that available published hedge fund indices have substantial survivorship bias. Survivorship is a problem that often affects databases but not usually return indices. The reason is that most published hedge fund indices use all available managers who report to a database to create the index at each period in time. Subsequently, some of these managers may stop reporting to the database for a variety of reasons. These managers' performance is not reflected in the future returns of the index. However, the historical performance of these managers continues to be reflected in the past returns and values of the index. In this sense, published hedge fund indices are similar to public equity indices. For example, the historical performance of defunct companies, such as Lehman Brothers and Enron, continues to be part of the historical performance of the Russell 1000 Index.

Survivorship bias occurs when the historical returns of a defunct fund are dropped from a database, are dropped from historical index return computations, or are not proportionately reflected in the construction of indices. Specifically, if one were to start a new index today based on the managers who report as of today, then the historical performance of this index prior to today would suffer from survivorship bias because it would not include the performance of all those managers who stopped reporting to the database during previous periods.

The lack of a regulatory environment for hedge funds creates the opportunity for other data biases that are unique to the hedge fund industry. In addition to survivorship bias, there are three other biases that may affect average performance figures estimated from databases.

First, there is selection bias, which occurs when an index disproportionately reflects the characteristics of managers who choose to report their returns. Essentially, it is voluntary for hedge fund managers to report their returns to a database provider. This managerial self-selection in reporting may cause a database to disproportionately represent those funds that have characteristics that make reporting more desirable. In particular, managers with lower returns and higher risks may disproportionately choose to conceal their track records relative to managers with higher returns and lower risks (those demonstrating excellent performance). However, it is also possible that fund managers with the most attractive performance may disproportionately fail to report their returns if their funds have reached capacity and are no longer seeking new investors (participation bias, discussed later). It is very difficult to quantify the magnitude of this important bias, as it affects past values of the indices as well as their future values. ${ }^{6}$ As we mentioned in the previous session, a contrary argument can be made for selection bias: that good hedge fund managers choose not to report their data to hedge fund index providers because they have no need to attract additional assets.

Closely related to selection bias is instant history bias, also referred to as backfill bias. Instant history bias or backfill bias occurs when an index contains histories of returns that predate the entry date of the corresponding funds into a database and thereby cause the index to disproportionately reflect the characteristics of funds that are added to a database. These biases therefore arise only when a hedge fund manager begins to report return performance to a database provider, and the provider includes or backfills the hedge fund manager's historical performance into the database.

Because it is more likely that a hedge fund manager will begin reporting performance history after a period of good performance, this bias pushes the historical performance of managers upward. For example, consider a manager who has compiled an excellent three-year track record. Based on this success, the manager chooses to begin reporting fund performance to the database. Backfill (instant history) bias pertains to the inclusion of fund returns that were generated prior to the fund's decision to report performance to the database. If successful funds are more likely than unsuccessful funds to begin reporting to a database, and if the database includes return histories, then the database will disproportionately reflect successful funds.

When a fund is added to a database, all future returns should be flagged by the database as live returns, and returns from the inception of the fund until the first reporting date should be excluded or flagged as backfilled returns. Again, similar to survivorship bias, this bias does not affect the historical performance of most published indices. The reason is that most index providers do not revise the history of an index once a new manager is added to the index. That is, only current and future performance of the manager affects the index on a forward-looking basis once the index has been established.

Estimates for backfill bias are highly dependent on the database that is being used. In general, the estimated average value of backfilled performance can be as low as $1 \%$ to as much as $5 \%$ per year higher than the performance of the manager after being listed in a database.

Last, there is liquidation bias, which occurs when an index disproportionately reflects the characteristics of funds that are not near liquidation. Frequently, hedge fund managers go out of business, especially to shut down an unsuccessful hedge fund. When the return histories of these funds are excluded from a database or an index, it causes survivorship bias. Liquidation bias is different in that it involves the partial reporting of the returns of defunct funds. When a hedge fund ceases operations, the fund manager typically stops reporting its performance in advance of the cessation of operations. Delayed reporting exacerbates the problem. If a fund's performance recovers, the manager is more likely to report returns. But if its performance does not recover, several months of poor performance are probably lost because the hedge fund manager is more concerned with winding down operations than with reporting final performance numbers to an index provider. To the degree that liquidating managers do not report large negative returns to databases, these figures do not get reflected in published databases. Therefore, this bias increases the reported performance of published indices. The flip side to liquidation bias is participation bias. Participation bias may occur for a successful hedge fund manager who closes a fund to new investors and stops reporting results because the fund no longer needs to attract new capital.

A related concept is that of the hazard rate, which is defined in this context as the proportion of hedge funds that drop out of a database at a given fund age. For example, Fung and Hsieh found that the highest dropout rate occurs when a hedge fund is 14 months old. This is a type of selection bias. The impact of this bias is that if an index is constructed that requires at least 24 months of performance history, a large number of funds may be excluded, introducing a bias relative to the overall universe of funds.

It is possible that in some applications, these biases can add up to $10 \%$ of annual enhancement to the average performance of the managers who report to a database. However, indices may or may not reflect these biases, because index computations may not be based on all of the data in the database. It is important to take note of these biases, because all indices suffer from one or more of them. The next exhibit summarizes the literature that estimates the size of these biases and their impact on hedge fund returns.

Biases Associated with Hedge Fund Data

\begin{center}
\begin{tabular}{|lllllll|}
\hline
 & \begin{tabular}{l}
Park, Brown, and \\
Goetzmann, \\
Bias \\
\end{tabular} & 1999 & \begin{tabular}{l}
Brown, Goetzmann, and \\
lbbotson, 1999 \\
\end{tabular} & \begin{tabular}{l}
Fung and Hsieh, \\
2000 \\
\end{tabular} & \begin{tabular}{l}
Ackermann, McEnally, and \\
Ravenscraft, 1999 \\
\end{tabular} & \begin{tabular}{l}
Barry, \\
$\mathbf{2 0 0 3}$ \\
\end{tabular} \\
\hline
Survivorship & $2.60 \%$ & $3.00 \%$ & $3.00 \%$ & $0.01 \%$ & $3.70 \%$ &  \\
Selection & $1.90 \%$ & Not estimated & Not estimated & No impact & Not estimated & Not estimated \\
Instant history & Not estimated & Not estimated & $1.40 \%$ & No impact & $0.40 \%$ & $1.97 \%$ \\
Liquidation & Not estimated & Not estimated & Not estimated & $0.70 \%$ & Not estimated & Not estimated \\
Total & $4.50 \%$ & $3.00 \%$ & $4.40 \%$ & $0.71 \%$ & $5.13 \%$ &  \\
\hline
\end{tabular}
\end{center}

Sources: James Park, Stephen Brown, and William Goetzmann, "Performance Benchmarks and Survivorship Bias for Hedge Funds and Commodity Trading Advisors," Hedge Fund News, August 1999; Stephen Brown, William Goetzmann, and Roger Ibbotson, “Offshore Hedge Funds: Survival and Performance, 1989-1995," Journal of

Business 72, no. 1 (1999): 91-117; William Fung and David Hsieh, "Performance Characteristics of Hedge Funds and Commodity Funds: Natural versus Spurious

Biases," Journal of Financial and Quantitative Analysis 25 (2000): 291-307; Carl Ackermann, Richard McEnally, and David Ravenscraft, "The Performance of Hedge

Funds: Risk, Return, and Incentives," Journal of Finance (June 1999): 833-74; Ross Barry, "Hedge Funds: A Walk through the Graveyard," Journal of Investment Consulting (2003); and Roger Ibbotson and Roger Chen, "The ABCs of Hedge Funds: Alphas, Betas, and Costs," Financial Analysts Journal 67, no. 1 (January/February 2011): $15-25$.

\section*{Strategy Definition and Style Drift}
Hedge fund databases and indices subdivide their funds based on strategies. Index providers determine their own hedge fund strategy classification system, and this varies from index to index. An index must have enough strategies to represent the broad market for hedge fund returns accurately and enough funds in each strategy to be representative. Strategy definitions, the method of grouping similar funds, raise two problems: (1) definitions of strategies can be very difficult for index providers to establish and specify, and (2) some funds can be difficult to classify in the process of applying the definition.

Consider a hedge fund manager who typically establishes a long position in the stock of a target company subject to a merger bid and a short position in the stock of the acquiring company. The strategy of this hedge fund manager may be classified alternatively as merger arbitrage by one index provider (e.g., HFR), relative value by another index provider (e.g., CASAM/CISDM), or event driven by yet another index provider (e.g., CSFB/Tremont). In summary, there is no consistent definition of hedge fund styles among index providers. Indeed, the dynamic trading nature of hedge funds makes them difficult to classify, which is part of their appeal to investors.

Further complicating the strategy definition is that most hedge fund managers are classified according to the disclosure language in their offering documents. However, consider the following language from a hedge fund private placement memorandum: "Consistent with the General Partner's opportunistic approach, there are no fixed limitations as to specific asset classes invested in by the Partnership. The Partnership is not limited with respect to the types of investment strategies it may employ or the markets or instruments in which it may invest."

How should this manager be classified? Relative value? Global macro? Market neutral? Unfortunately, with hedge funds, this type of strategy description is commonplace. The lack of specificity may lead to guesswork on the part of index providers with respect to the manager's strategy. Alternatively, some index providers may leave this manager out because of lack of clarity, but this adds another bias to the index by purposely excluding these types of hedge fund managers. In sum, there is no established format for classifying hedge funds. Each index provider develops its own scheme without concern for consistency with other hedge fund index providers, and this makes comparisons between hedge fund indices difficult.

Even if an index provider can successfully classify a hedge fund manager's current investment strategy, there is the additional problem of style drift. Style drift is a consistent movement through time in the primary style or strategy being implemented by a fund, especially a movement away from a previously identified style or strategy. Because of the mostly unregulated nature of hedge fund managers, there is no requirement for a hedge fund manager to notify an index provider when an investment style has changed.

Consider the potential for style drift among merger arbitrage managers. During the recession of 2001 and the financial crisis of 2008, the market for mergers and acquisitions declined substantially except for investment banks, brokerage firms, and traditional banks. There were simply too few deals to fuel all that merger arbitrage managers need for investment opportunities. Consequently, many of these managers changed their investment style to invest in the rising tide of distressed debt deals, which are countercyclical from mergers and acquisitions. In addition, many merger arbitrage managers expanded their investment portfolios to consider other corporate transactions, such as spin-offs and recapitalizations. However, once a hedge fund manager has been classified as merger arbitrage by a particular database manager, it will typically remain in that category despite substantial changes in its investment focus.

Finally, a growing recent trend in the industry has been for hedge funds to evolve from single-strategy specialists into multistrategy hedge funds. In addition, Fung and Hsieh comment on the growing trend of so-called synthetic hedge funds. Synthetic hedge funds attempt to mimic hedge fund returns using listed securities and mathematical models. These funds are designed to replicate the returns of successful hedge fund strategies but at a lower cost to investors as a result of lower fees. ${ }^{7}$ Fung and Hsieh, "Hedge Funds." Both of these trends further complicate the classification of hedge funds into strategy types.

\section*{Index Investability}
A key issue is whether a hedge fund index can be or should be investable. The investability of an index is the extent to which market participants can invest to actually achieve the returns of the index. This issue is usually more of a problem for hedge fund indices than it is for their traditional investment counterparts. Indices of listed securities are generally investable through holding the same portfolio described in the index.

There are numerous reasons that a market participant cannot simply hold a portfolio equivalent to the portfolio implied by a hedge fund index. First, hedge fund investments often have capacity limitations. Capacity is the limit on the quantity of capital that can be deployed without substantially diminished performance. Hedge funds generally have or develop capacity issues, since as a particular strategy performs well, there may be increasing competition to exploit the available opportunities. Limited capacity often leads hedge fund managers to refuse further investments of capital into the fund (i.e., to close the fund to new investment and new investors) when the managers have achieved a level of assets under management that makes it more difficult to generate strong returns with further capital. Market participants are inhibited from achieving the returns of an index that contains funds that are closed. Thus, to the extent that an index contains closed funds, there is less investability of the index.

A related issue is whether hedge fund indices should be investable. The argument is that an investable index excludes hedge fund managers that are closed to new investors and therefore excludes a large section of the hedge fund universe. Most index providers argue that the most representative index acts as a barometer for current hedge fund performance and that both open and closed funds should be included. The trade-off, therefore, is between having a very broad representation of current hedge fund performance and having a smaller pool of hedge fund managers that represent the performance that may be accessed through new investment. Billio, Getmansky, and Pelizzon compare the characteristics of investable and noninvestable databases and determine that investability affects the distributions of hedge fund returns. ${ }^{8}$ Minica Billio, Mila Getmansky, and Loriana Pelizzon, “Dynamic Risk Exposure in Hedge Funds” (Yale Working Paper 07-14, September 2007). Investable indices have generally underperformed noninvestable indices.

Research indicates that hedge fund investments can expand the investment opportunity set for investors. The returns to hedge funds have generally been positive, have had lower volatility than equity markets, and have had less-than-perfect correlation with traditional asset classes. Consequently, hedge funds have provided, and will probably continue to provide, a good opportunity to diversify a portfolio and an excellent risk management tool.


\end{document}