\documentclass[11pt]{article}
\usepackage[utf8]{inputenc}
\usepackage[T1]{fontenc}
\usepackage{amsmath}
\usepackage{amsfonts}
\usepackage{amssymb}
\usepackage[version=4]{mhchem}
\usepackage{stmaryrd}

\begin{document}
Three Research Studies on Whether Hedge Funds Adversely Affect the Financial Markets

Throughout history, speculators, speculation, and asset volatility are frequently observed together. Some commentators allege that hedge fund activity causes financial crises. But it is not clear whether speculation causes market volatility or market volatility attracts speculation. Theoretical work is mixed, but it should be noted that speculators profit only when they buy low and sell high-so successful speculation should generally stabilize prices, since buying low and selling high pushes price levels toward their mean. Empirical evidence throughout history is also mixed, but clearly there is little long-term evidence to suggest that markets that allow speculative activity are made substantially more volatile in the long run by allowing such activity.

More recently, hedge funds have often been accused of causing market volatility and exacerbating times of crisis in financial markets. The idea of headline risk deters some investors from allocating assets to hedge funds. Headline risk is dispersion in economic value from events so important, unexpected, or controversial that they are the center of major news stories. Some investors may be especially sensitive to negative publicity from investing with a manager who makes unfavorable headlines. For example, a charitable endowment fund may suffer reduced donations if its endowment is associated with a famous catastrophic loss or a financial scandal.

Hedge fund activity that can provoke controversy includes currency speculations, such as those attributed to George Soros. In 1992, Soros apparently bet against the British pound sterling and the Italian lira in correctly anticipating that the currencies would devalue, and generated a combined total profit of close to $\$ 3$ billion. In 1997, Soros was once again blamed for a currency crisis by Malaysian prime minister Mahathir bin Mohammad. The prime minister attributed the crash in the Malaysian ringgit to speculation in the currency markets by hedge fund managers, including Soros.

Brown, Goetzmann, and Park tested specifically whether hedge funds caused the crash of the Malaysian ringgit. ${ }^{1}$ Stephen Brown, William Goetzmann, and James Park, "Hedge Funds and the Asian Currency Crisis," Journal of Portfolio Management 26, no. 4 (Summer 2000): 95-101. They regressed the monthly percentage change in the exchange rate on the currency exposure held by hedge funds. Reviewing the currency exposures of 11 large global macro hedge funds, they concluded that there is no evidence that the Malaysian ringgit was affected by hedge fund manager currency exposures. Additionally, they tested the hypothesis that global hedge funds precipitated the slide of a basket of Asian currencies, known as the Asian contagion, in 1997, and found no evidence that hedge funds contributed to the decline of Asian currencies in the fall of that year.

Fung and Hsieh measured the market impact of hedge fund positions on several financial market events, from the October 1987 stock market crash to the Asian contagion of $1997 .{ }^{2}$ William Fung and David Hsieh, "Measuring the Market Impact of Hedge Funds," Journal of Empirical Finance 7, no. 1 (2000): 1-36. They found that there were certain instances in which hedge funds did have an impact on the market, most notably with the devaluation of the pound sterling in 1992. However, in no case was there evidence that hedge funds were able to manipulate the financial markets away from their natural paths driven by economic fundamentals. For instance, the sterling came under pressure in 1992 due to large capital outflows from the United Kingdom. The conclusion is that, for instance, George Soros bet correctly against the sterling and exacerbated its decline, but he did not trigger the devaluation.

Khandani and Lo analyzed the extraordinary stock market return patterns observed in August 2007, when losses to quantitative hedge funds in the second week of the month were presumably started by a short-term price impact that was the result of a rapid unwinding of large quantitative equity market-neutral hedge funds. ${ }^{3}$ Amir Khandani and Andrew Lo, "What Happened to the Quants in August 2007?" Journal of Investment Management 5 (2007): 29-78. These authors argue that the return patterns of that week were a sign of a liquidity trade that can be explained as the consequence of a major hedge fund strategy liquidation. Khandani and Lo also contend that, unlike banks, hedge funds can withdraw liquidity at any time and that a synchronized liquidity withdrawal among a large group of funds could have devastating effects on the basic functioning of the financial system. In spite of the potential harm brought about by hedge funds, Khandani and Lo argue that the hedge fund industry has facilitated economic growth and generated social benefits by providing liquidity, engaging in price discovery, discerning new sources of returns, and facilitating the transfer of risk. Additionally, hedge funds engaging in short-selling activity may actually be reducing market volatility, as they seek to sell assets as prices rise and buy assets as prices fall.


\end{document}