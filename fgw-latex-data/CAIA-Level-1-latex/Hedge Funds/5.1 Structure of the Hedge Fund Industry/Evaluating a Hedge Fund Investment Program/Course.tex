\documentclass[11pt]{article}
\usepackage[utf8]{inputenc}
\usepackage[T1]{fontenc}
\usepackage{amsmath}
\usepackage{amsfonts}
\usepackage{amssymb}
\usepackage[version=4]{mhchem}
\usepackage{stmaryrd}

\title{Reading }

\author{}
\date{}


\begin{document}
\maketitle
Evaluating a Hedge Fund Investment Program

This lesson discusses the evaluation of a hedge fund investment program.

\section*{Hedge Fund Investment Program Parameters}
Setting specific parameters will determine how the hedge fund program is constructed and operated and should include risk and return targets, as well as the type of hedge fund strategies that may be selected. Absolute return parameters should operate at two levels: that of the individual hedge fund manager and that of the overall hedge fund program. For example, the investor should set target return ranges for each hedge fund manager and a specific target return level for the entire absolute return program. Parameters for the individual managers may be different from those for the program.

The program parameters for the hedge fund managers may be based on such factors as volatility, expected return, types of instruments traded, leverage, and historical drawdown. Other factors may be included, such as length of track record, periodic liquidity, minimum investment, and assets under management. Liquidity is particularly important, because an investor needs to know the time frame for cashing out of an absolute return program if hedge fund returns appear unattractive or cash is needed.

Before considering how to incorporate hedge funds as part of a strategic investment program, the following question must be asked: Should hedge funds be included? Both the return potential of hedge funds and their role in diversifying or otherwise altering the aggregate risk of a portfolio that includes stocks and bonds should be considered.

Hedge funds can expand the investment opportunity set for investors, as many hedge fund strategies have offered risk-adjusted returns above those of stocks and bonds as well as provided risk management tools.

\section*{Three Research Findings Regarding Hedge Fund Performance}
Recent research on hedge funds indicates consistent positive performance and low correlation with traditional asset classes. Although these conclusions may present opportunity, there are several caveats to keep in mind with respect to the documented results for hedge funds.

First, research provides clear evidence that shocks to one segment of the hedge fund industry can be felt across many different hedge fund strategies. ${ }^{1}$ See Goldman, Sachs \& Co. and Financial Risk Management Ltd., "The Hedge Fund 'Industry' and Absolute Return Funds," Journal of Alternative Investments 1, no. 4 (Spring 1999): 11-27; Goldman, Sachs \& Co. and Financial Risk Management Ltd., "Hedge Funds Revisited," Pension and Endowment Forum (January 2000); Mark Anson, "Financial Market Dislocations and Hedge Fund Returns," Journal of Alternative Assets 5, no. 3 (Winter 2002): 78-88. Second, future results generally differ from past results. In the case of hedge fund returns, there are reasons to believe that past results may consistently overestimate future results. Most of the research to date on hedge funds has still not factored in the tremendous growth of this industry over the past 10 years. Thus, the impact on returns of this explosive growth has yet to be fully documented.

Third, some form of bias-either survivorship bias or selection bias-exists in the empirical studies. All of the cited studies make use of hedge fund databases that have biases embedded in the data. These biases, if not corrected, can unintentionally inflate the estimated returns to hedge funds. It has been estimated that these biases can add from 70 to 450 basis points to the estimated total annual returns of hedge funds.

Most of the prior studies of hedge funds have generally examined hedge funds within a mean-variance efficient frontier framework. Generally, Sharpe ratios are used to compare hedge fund performance to that of stock and bond indices. However, hedge funds may pursue investment strategies that have nonlinear payoffs or are exposed to significant event risk, both of which may not be apparent from a Sharpe ratio analysis because this type of analysis assumes that returns are symmetric and normally distributed, meaning that the mean and the variance fully explain returns. Bernardo and Ledoit demonstrate that Sharpe ratios are misleading when the distribution of returns is not normal, and Spurgin shows that fund managers can enhance their Sharpe ratios by selling off the potential return distribution's upper end-for example, by entering a swap to pay the year's highest monthly return and be compensated for the year's lowest monthly return. ${ }^{2}$ Antonio Bernardo and Oliver Ledoit, "Gain, Loss, and Asset Pricing," Journal of Political Economy 108, no. 1 (2001): 144-72; Richard Spurgin, "How to Game Your Sharpe Ratio," Journal of Alternative Investments 4, no. 3 (Winter 2001): 38-46.

\section*{Opportunistic Hedge Fund Investing}
Several hedge fund investment strategies can be referred to as opportunistic, which is when a major goal is to seek attractive returns through locating superior underlying investments. Opportunistic investing is driven by the identification of and potentially aggressive exposure to investments that appear to offer superior returns (ex ante alpha), typically on a temporary basis. Opportunistic investing can be contrasted to traditional portfolio management, which is dominated by longerterm positions and acceptance of risks and returns commensurate with broad market conditions. The opportunistic nature of hedge funds can provide an investor with new investment opportunities that cannot otherwise be obtained through traditional long-only investments.

\section*{The Approach}
There are several ways hedge funds can be opportunistic. First, many hedge fund managers can add value to an existing investment portfolio through specialization in a sector or in a market strategy. These managers seek to contribute above-market returns through application of superior skill or knowledge of a narrow market or strategy. In fact, this style of hedge fund investing describes most of the sector hedge funds in existence.

Consider a portfolio manager whose particular expertise is the biotechnology industry and who has followed this industry for years, and has developed a superior information set to identify winners and losers. In a traditional investing approach, the manager purchases those biotech stocks he believes will increase in value and avoids those biotech stocks he believes will decline in value. Often, the selections are made with an effort to be moderately diversified within the sector, and positions are adjusted slowly. However, this strategy may be criticized for not using the manager's superior information set to its fullest advantage. The ability to go both long and short biotech stocks in a hedge fund is the only way to maximize the value of the manager's information set. Furthermore rapid trading, more\\
concentrated positions, and use of leverage can maximize the benefits of the manager's superior information. Therefore, a biotech hedge fund provides a new opportunity: the ability to extract value on both the long side and the short side of the biotech market, and to do so aggressively in terms of concentration, turnover, and leverage. This is consistent with the fundamental law of active management, which is described in detail in the CAIA Level II curriculum. The long-only constraint is the most expensive constraint in terms of lost alpha generation that can be applied to active portfolio management.

\section*{Benchmarks}
Sector hedge funds tend to have well-defined benchmarks. For the previous example of the biotech long/short hedge fund, an appropriate benchmark would be the AMEX Biotech Index, which contains 17 biotechnology companies. The point is that opportunistic hedge funds are generally not absolute return vehicles; their performance should typically be measured relative to a benchmark.

Traditional long-only managers are benchmarked to passive indices. The nature of benchmarking is such that it forces managers to focus on their benchmark and their fund's tracking error associated with that benchmark. This focus on benchmarking leads traditional active managers to make portfolio allocation decisions at least partly based on keeping the tracking error low, ensuring that the fund's returns are correlated with the benchmark. Even if a manager has the skills to outperform the benchmark in the long run, the manager might not take full advantage of these skills because it could substantially increase the tracking error of the portfolio relative to the benchmark. If the correlation between the benchmark and the portfolio is low, the manager runs the risk that the portfolio could underperform the benchmark over a short period of time, leading to loss of assets. The necessity to consider the impact of every trade on the portfolio's tracking error relative to its assigned benchmark reduces the flexibility of the investment manager.

In addition, long-only active managers are constrained in their ability to short securities. Generally, they may underweight a security only up to its weight in the benchmark index. If the security is only a small part of the index, the manager's efforts to underweight the stock are further constrained. The long-only constraint is a well-known limitation on the ability of traditional active management to earn excess returns. ${ }^{3}$ See Richard Grinold and Ronald Kahn, Active Portfolio Management (New York: McGraw-Hill, 2000).

\section*{Summary}
Opportunistic hedge fund investing is used to expand the set of available investments rather than to hedge traditional investments. Constructing an opportunistic portfolio of hedge funds depends on the constraints under which such a program operates. For example, if an investor's hedge fund program is not limited in scope or style, then diversification across a broad range of hedge fund styles will be appropriate. If, however, the hedge fund program is limited in scope to, for instance, expanding the equity investment opportunity set, then the choices will be less diversified across strategies.

\section*{NOTES}
\begin{enumerate}
  \item See Goldman, Sachs \& Co. and Financial Risk Management Ltd., "The Hedge Fund 'Industry' and Absolute Return Funds," Journal of Alternative Investments 1, no. 4 (Spring 1999): 11-27; Goldman, Sachs \& Co. and Financial Risk Management Ltd., "Hedge Funds Revisited," Pension and Endowment Forum (January 2000); Mark Anson, "Financial Market Dislocations and Hedge Fund Returns," Journal of Alternative Assets 5, no. 3 (Winter 2002): 78-88.

  \item Antonio Bernardo and Oliver Ledoit, "Gain, Loss, and Asset Pricing," Journal of Political Economy 108, no. 1 (2001): 144-72; Richard Spurgin, "How to Game Your Sharpe Ratio," Journal of Alternative Investments 4, no. 3 (Winter 2001): 38-46.

  \item See Richard Grinold and Ronald Kahn, Active Portfolio Management (New York: McGraw-Hill, 2000).

\end{enumerate}

\end{document}