\documentclass[11pt]{article}
\usepackage[utf8]{inputenc}
\usepackage[T1]{fontenc}
\usepackage{amsmath}
\usepackage{amsfonts}
\usepackage{amssymb}
\usepackage[version=4]{mhchem}
\usepackage{stmaryrd}

\title{Reading }

\author{}
\date{}


\begin{document}
\maketitle
Hedge Fund Classification

A critical dimension in understanding hedge funds is the spectrum of trading strategies that underlie their performance. Hedge funds as a group are identified, at least in part, by their use of sophisticated trading strategies, and hedge funds are primarily differentiated from one another by their trading strategies. The diverse strategies that comprise the universe of hedge funds are often organized into a classification of hedge fund strategies. A classification of hedge fund strategies is an organized grouping and labeling of hedge fund strategies. Hedge funds are classified differently by different commentators, authors, and database managers. This course uses the classification of hedge fund strategies shown in the following list. The organization of the last five sessions of Topic 5 (sessions Macro and Managed Futures Funds through Funds of Hedge Funds) follow this organization closely, with categories I through V corresponding to sessions Macro and Managed Futures Funds through Funds of Hedge Funds.

\section*{CAIA Classification of Hedge Fund Strategies}
I. Macro and Managed Futures Funds

A. Macro

B. Managed Futures

II. Event-Driven Hedge Funds

A. Activists

B. Merger Arbitrage

C. Distressed

D. Event-Driven Multistrategy

III. Relative Value Hedge Funds

A. Convertible Arbitrage

B. Volatility Arbitrage

C. Fixed-Income Arbitrage

D. Relative Value Multistrategy

IV. Equity Hedge Funds

A. Long/Short

B. Market Neutral

C. Short Selling

V. Funds of Hedge Funds

The hedge fund industry is composed of single-manager funds (sessions Macro and Managed Futures Funds through Equity Hedge Funds) as well as funds of funds (the session, Funds of Hedge Funds). The distinction between single-manager hedge funds and funds of funds is important. A fund of funds in this context is a hedge fund with underlying investments that are predominantly investments in other hedge funds. A single-manager hedge fund, or single hedge fund, has underlying investments that are not allocations to other hedge funds.

A single hedge fund may be a multistrategy fund. A multistrategy fund deploys its underlying investments with a variety of strategies and sub-managers, much as a corporation would use its divisions. In a multistrategy fund, there is a single layer of fees, and the sub-managers are part of the same organization. The underlying components of a fund of funds are themselves hedge funds, with independently organized managers and a second layer of hedge fund fees to compensate the manager for activities relating to portfolio construction, monitoring, and oversight.

An analogy can be made between hedge funds and stocks in understanding the distinction between single-manager funds, multistrategy funds, and funds of funds. As single stocks are to mutual funds, single-manager hedge funds are to funds of funds. For example, investing in a single stock is like investing in a single-manager hedge fund in that in both cases there is a substantial amount of idiosyncratic risk. The company's industry or the hedge fund manager's style may go out of favor, or the CEO of the company or the fund manager may make some consequential mistakes. There is substantial dispersion in returns across single stocks or single hedge funds, so concentrating wealth in a single investment can lead to riches or ruin. Continuing with the analogy, investing in a multistrategy fund rather than a singlemanager hedge fund is akin to investing in a conglomerate stock rather than a stock focused on a single line of business.

Funds of funds can also be compared to mutual funds. Just as mutual funds invest in a large number of stocks across industries to diversify risk, funds of funds invest in multiple hedge fund managers and strategies to control risk. If a fund of funds includes a fund or strategy that experiences dramatic losses, investors' percentage losses are likely to be reduced by other managers or strategies in the fund of funds that maintained or grew their value. Whereas portfolio concentration in a few stocks or hedge funds can lead to success or failure, mutual funds and funds of funds offer returns in a much narrower range due to the diversification inherent in multiple investments.

Fund mortality, the liquidation or cessation of operations of funds, illustrates the risk of individual hedge funds and is an important issue in hedge fund analysis. The data underlying the exhibit, Estimated Number of Funds Launched/Liquidated from the lesson Distinguishing Hedge Funds shows that more than 15,000 hedge funds have liquidated since 1996, including 13,500 funds that have liquidated since 2006. These numbers reflect only a subset of total hedge fund liquidations, as HFR can only track fund liquidations among the funds that chose to report to its database. Of the hedge funds alive in 2021, HFR estimates that approximately $75 \%$ have operated more than five years, $18 \%$ are less than three years old, and $15 \%$ are between three and five years old. In a study of commodity trading advisers, Gregoriou and associates estimate that the average hedge fund life was 4.4 years. The study also notes that funds with larger AUM and lower volatility tended to exist longer. ${ }^{1}$ Greg N. Gregoriou, Georges Hubner, Nicolas Papageorgiou, and Fabrice Rouah, “Survival of Commodity Trading Advisors: 1990-2003," Journal of Futures Markets 25, no. 8 (2005): 795-815. Later, we will discuss survivor bias, which can be an important issue in the performance analysis of hedge funds.

\section*{NOTE}
\begin{enumerate}
  \item Greg N. Gregoriou, Georges Hubner, Nicolas Papageorgiou, and Fabrice Rouah, "Survival of Commodity Trading Advisors: 1990-2003," Journal of Futures Markets 25, no. 8 (2005): 795-815.
\end{enumerate}

\end{document}