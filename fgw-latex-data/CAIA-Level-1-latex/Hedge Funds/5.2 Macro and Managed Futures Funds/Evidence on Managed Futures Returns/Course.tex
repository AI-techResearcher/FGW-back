\documentclass[11pt]{article}
\usepackage[utf8]{inputenc}
\usepackage[T1]{fontenc}
\usepackage{amsmath}
\usepackage{amsfonts}
\usepackage{amssymb}
\usepackage[version=4]{mhchem}
\usepackage{stmaryrd}

\begin{document}
Evidence on Managed Futures Returns

There are a number of key questions with respect to managed futures: Can managed futures products produce consistent alpha? Can managed futures provide downside risk protection? What are the sources of returns, and what are the potential risks?

\section*{Evidence on Managed Futures Alpha}
There are two types of relevant empirical research on the issue of consistent alpha. The first type examines the actual returns of managed futures funds. The second type estimates returns to funds based on simulations of well-known trading strategies using historical prices.

The first empirical approach is addressed by Kazemi and Li, who use the direct examination of actual managers to show that systematic CTAs have demonstrated statistically significant, positive market-timing ability. ${ }^{1}$ Hossein Kazemi and Ying Li, "Market Timing of CTAs: An Examination of Systematic CTAs vs. Discretionary CTAs," Journal of Futures Markets 29, no. 11 (2009): 1067-99. Returns to trend-following systematic CTAs are achieved through long positions in rising markets and short positions in falling markets. Kazemi and Li conclude that CTAs have demonstrated skill in differentiating between upward- and downward-trending markets.

The simulation-based approach is studied by Miffre and Rallis, who simulate well-known momentum strategies, such as trend following. ${ }^{2}$ Joelle Miffre and Georgios Rallis, "Momentum Strategies in Commodity Futures Markets," Journal of Banking and Finance 31, no. 6 (2007): 1863-86. By examining historical returns of 31 U.S.based commodity futures contracts for evidence of shorter- and longer-term price momentum or reversal characteristics for the period January 31, 1979, through September 30, 2004, they find that 13 of the momentum strategies they studied were profitable for the period of their analysis.

In general, the empirical research supports the inclusion of managed futures in a diversified portfolio context. However, the potential benefits of managed futures may be neutralized if the investments take place through CPOs managing a pool of CTAs. The second layer of fees charged by these CPOs effectively eliminates most of the benefits associated with this asset class.

\section*{The Evidence on Downside Risk Protection}
The greatest concern for investors is typically downside risk. The ability to protect the value of an investment portfolio in hostile or turbulent markets is the key to the value of diversification. An asset class distinct from traditional financial asset classes has the potential to diversify and protect an investment portfolio from hostile markets. In 2008, the downside risk of the market crisis was severe. While some investors bemoaned that there was nowhere to hide from the market losses and risks, the last three lines of the next exhibit indicate that macro and managed futures funds emerged relatively unscathed from the turbulence of the financial crisis that began in 2007.

\begin{center}
\begin{tabular}{|c|c|c|c|c|c|}
\hline
 & \multicolumn{5}{|c|}{Asset Returns} \\
\hline
 & 2007 & 2008 & 2009 &  & 2007-2009 \\
\hline
GSCl Commodities &  & $32.7 \%$ & $-46.5 \%$ & $13.5 \%$ & $-19.4 \%$ \\
\hline
MSCI World Index &  & $9.0 \%$ & $-40.7 \%$ & $30.0 \%$ & $-16.0 \%$ \\
\hline
S\&P 500 &  & $5.5 \%$ & $-37.0 \%$ & $26.5 \%$ & $-15.9 \%$ \\
\hline
Convertible Arb &  & $5.2 \%$ & $-31.6 \%$ & $47.3 \%$ & $6.0 \%$ \\
\hline
Emerging Markets Hedge &  & $20.2 \%$ & $-30.4 \%$ & $30.0 \%$ & $8.8 \%$ \\
\hline
Fixed Income Arb &  & $3.8 \%$ & $-28.8 \%$ & $27.4 \%$ & $-5.9 \%$ \\
\hline
$60 \%$ MSCI World, 40\% Barclays Global &  & $9.4 \%$ & $-24.9 \%$ & $20.7 \%$ & $-0.8 \%$ \\
\hline
Equity Long/Short &  & $13.7 \%$ & $-19.7 \%$ & $19.5 \%$ & $9.0 \%$ \\
\hline
Hedge Fund Index &  & $12.6 \%$ & $-19.1 \%$ & $18.6 \%$ & $8.0 \%$ \\
\hline
Event-Driven Multistrategy &  & $16.8 \%$ & $-16.2 \%$ & $19.9 \%$ & $17.3 \%$ \\
\hline
Macro &  & $17.4 \%$ & $-4.6 \%$ & $11.5 \%$ & $24.9 \%$ \\
\hline
Barclays Global Aggregate &  & $9.5 \%$ & $4.8 \%$ & $6.9 \%$ & $22.7 \%$ \\
\hline
Managed Futures &  & $6.0 \%$ & $18.3 \%$ & $-6.5 \%$ & $17.2 \%$ \\
\hline
\end{tabular}
\end{center}

Source: Data from Bloomberg.

\section*{Mechanical Managed Futures Indices}
The Mount Lucas Management Index provides a useful comparison for evaluating trend-following futures strategies. The Mount Lucas Management (MLM) Index is a passive, transparent, and investable index designed to capture the returns to active futures investing. It provides a useful benchmark for evaluating trend following futures strategies. The MLM Index mechanically applies a simple price trend-following rule for buying and selling commodity, financial, and currency futures. Each of the three sub-baskets is weighted by its relative historical volatility, whereas markets within each sub-basket are equally weighted. The MLM Index can take long or short positions in any of its 22 constituent markets; there are no neutral positions. Because the MLM Index is investable, its performance is representative of what investors may actually obtain if they use the index's simple strategy in their portfolios. One of the biggest advantages of the MLM Index is the observed symmetry of its past returns. The distribution of returns has shown a somewhat bell-shaped curve, albeit with larger tails than those of a normal distribution. Also, there has been lower volatility in the MLM Index compared to all of the managed futures indices.

\section*{Why Might Managed Futures Provide Superior Returns?}
Whether managed futures funds can be a source of alpha can be addressed intuitively, not just empirically. Having an intuitive or theoretical explanation of the sources of superior returns can provide valuable information in differentiating between empirical results that help predict future performance and empirical results that do not indicate future performance because they are spurious or apply only to past specific market regimes. This section introduces a conceptual framework that could explain why managed futures funds may provide alpha to investors.

Managed futures funds tend to trade futures contracts in which the underlying assets are broad asset classes, such as equities, commodities, currencies, and fixedincome instruments. Further, managed futures funds trade in futures contracts that are highly liquid, with rather narrow bid-ask spreads. Finally, note that futures contracts represent zero-sum games: Any dollar received on one side of a futures contract is paid to managed futures funds by the other side of the futures contract. Therefore, capital gains earned must result from capital losses by other futures market participants. Thus, it appears that the typical arguments put forth to describe the economic sources of alpha for other investment strategies do not apply in this case. For instance, there is no illiquidity premium to be earned by managed futures funds. So, what is the potential source of alpha for typical managed futures funds? If a theoretical argument for the presence of ex ante alpha cannot be provided, then any empirically estimated ex post alpha must be looked at with an especially skeptical eye.

The starting point of this conceptual framework is the observation that most futures contracts are used as hedging instruments by some market participants and as speculating instruments by other market participants. A large group of futures market participants are natural hedgers. A natural hedger is a market participant who seeks to hedge a risk that springs from its fundamental business activities. Natural hedgers participate in futures markets to hedge their risks rather than to earn profits through speculation.

If there are more natural hedgers on one side of a market (e.g., long) than on the other side of the market, then speculators (e.g., managed futures funds) step in and fill the gap between supply and demand in futures contracts. In this case, managed futures funds, much like insurance agents, earn positive excess return for providing a valuable service to natural hedgers. In other words, managed futures funds earn a return by accepting risks that natural hedgers want to avoid. Managed futures funds are able and willing to accept this risk because, unlike natural hedgers, they tend to hold diversified portfolios of futures contracts.

The motive of managed futures funds is to make a profit. When demand by natural hedgers for short positions in a particular futures contract is strong, prices fall to the point that managed futures funds perceive a profit opportunity by taking an offsetting long position. When demand by natural hedgers for long positions in a particular futures contract is strong, prices rise to the point that managed futures funds perceive a profit opportunity by taking an offsetting short position. An example of a natural hedger seeking a long position in a futures contract is a manufacturer requiring a metal or other material for a production process.

The presence of natural hedgers with different time horizons, risk profiles, and break-even points could also explain the presence of trends in futures. As more producers come to market to hedge their positions, managed futures funds are willing to take larger long positions only if they expect ongoing increases in futures prices. In this context, managed futures funds may be viewed as providing protection from price risk for a group of natural hedgers who are willing to pay the cost for the protection that the futures contract provides.

Often there is approximately equal demand from potential natural hedgers on each side of a futures contract. If there are enough natural hedgers on each side of the market, the managed futures fund's potential source of alpha tends to disappear. However, even when long-term demand for long and short positions by natural hedgers is equal, there are occasional mismatches, provided the natural hedgers do not come to the market at the same time. These temporary mismatches between demand and supply of futures contracts provide a role for managed futures funds to play. They also explain why managed futures funds are not always long or short in a particular market. If there are more corn producers who are trying to hedge their income, then managed futures funds need to be long; and when more corn users come to the market to hedge their cost, managed futures funds need to take short positions.

One implication of this conceptual framework is that managed futures funds are likely to earn positive alphas in those markets in which there is a great need for hedging when natural hedgers come to market at different points in time. For example, futures markets for industrial metals, agricultural products, and currencies are more likely to be sources of alpha than are futures markets for equities and precious metals. There are fewer natural hedgers in the equity and precious metal markets and therefore less need for managed futures funds to provide a service to other market participants.

Additional arguments for a consistent source of alpha to managed futures funds are available. For example, central banks may be willing to manipulate exchange rates in the short run away from levels consistent with long-term market forces for the purpose of pursuing their domestic policy agenda. The massive level of governmental resources involved in these interventions raises the possibility that exchange rates are periodically, substantially, and temporarily dislocated, thereby generating profitable speculative opportunities to managed futures funds. In other words, ongoing intervention by central banks could cause persistent trends in exchange rates or sovereign debt yields until the intervention ceases, at which point the trend may reverse. Managed futures funds may profit from these patterns at the expense of the central banks.

\section*{Six Potentially Important Risks of Managed Futures Funds}
The risks of managed futures funds can be summarized as follows.

Many investors find it difficult to invest in black-box systems, in which trading algorithms are not disclosed, as it gives rise to transparency risk. Transparency is the ability to understand the detail within an investment strategy or portfolio. Transparency risk is dispersion in economic outcomes caused by the lack of detailed information regarding an investment portfolio or strategy. Trusting investment capital to an automated system may also bring fears of computer bugs, viruses, or connectivity issues, not unlike those that may cause flash crashes or rare instances of enormous market price changes for no apparent reason other than massive intentional or unintentional trades.

A second major risk is model risk. Model risk is economic dispersion caused by the failure of models to perform as intended. Systematic trend-following managers rely on algorithmic models to generate trade signals. Model risk arises if a model is not adequately tested before deployment and could therefore break down under particular market conditions. For example, the model may not have been tested for the situation in which the price of a futures contract rises substantially during one day and therefore hits a prespecified limit set by the exchange.

Capacity risk arises when a managed futures trader concentrates trades in a market that lacks sufficient depth (i.e., liquidity). The performance of a trader who has developed expertise in trading a thinly traded futures contract will suffer if investors decide to substantially increase their allocations to this fund.

A fourth risk, liquidity risk, is somewhat related to capacity risk in that it refers to how a large fund that is trading in a thinly traded market will affect the price should it decide to increase or decrease its allocation. However, liquidity risk can also arise in markets with high volume. If too many funds seek to trade the same markets at\\
the same price, competition for trades can lead to increased slippage and trading costs. If trading volume among other market participants declines, managed futures funds become a larger part of the market and find it difficult to execute in less liquid markets.

Given their association with speculation, futures exchanges are especially prone to change margin terms or to face actions by governmental entities that tax or restrict futures trading. This exposes managed futures to a fifth risk, regulatory risk, which is the risk of unanticipated changes in taxation or regulations.

Trend-following managed futures funds need trending markets to profit. Volatile, trendless markets can leave managed futures funds with substantial losses. This gives rise to the final risk, lack of trends risk, which comes into play when the trader continues allocating capital to trendless markets, leading to substantial losses. Therefore, the attrition rate among managed futures funds is relatively high.

\section*{Managed Accounts and Platforms}
Once investors decide to make an allocation to managed futures, they must tackle the problem of just how to structure the investment. The choice of vehicle employed to make an allocation is dependent on the size of the allocation as well as on the level of expertise and experience the investor possesses.

To tackle the problem of structuring the investments, investors will need to follow a decision-making process that proceeds along the following lines. First, the investor must determine how many managed futures funds he wants in the portfolio. Many family offices, and even some larger institutional investors, will decide to invest in a single managed futures fund. This decision has the virtue of simplicity and is possible to implement by choosing one of the large, diversified trendfollowing managed futures funds, whose performance correlates highly with a trend-following benchmark. If the investor decides to use this approach, the focus should be on examining differences between investing in a fund sponsored on behalf of the managed futures funds versus a managed account.

Generally, the single managed futures funds route exposes the investor to a greater amount of risk. In this scenario, the results depend on the performance of a single manager, concentrate risk to a single organization, and may be exposed to a limited number of trading models. To avoid these constraints, the investor may decide to form a diversified portfolio of managed futures funds.

Second, if the investors are large enough, they must decide how to create a diversified portfolio of managed futures funds. Initially, the most cost-effective approach to achieving diversification is to allocate to a multi-managed futures fund. Then, as the size of the allocation to managed futures funds increases, more options become available to the investor. Eventually, the investor must decide whether to assemble an in-house team to manage the portfolio and whether to use a managed account platform. There are cost issues associated with each choice.

Related to the size of allocation to managed futures funds, two issues arise as the investor decides on the best approach to creating a diversified portfolio of CTAs. The first issue is related to the level of allocation at which it becomes cost effective to move from working with a multi-managed futures fund's investment program to assembling an in-house team to create and manage a diversified portfolio of managed futures funds. While no exact figure exists for this cost threshold, it is primarily affected by four factors:

\begin{enumerate}
  \item The extra layer of fees that the investor will have to pay the multi-managed futures fund

  \item The cost of assembling a team of analysts who can construct and manage a managed futures fund portfolio

  \item The minimum size of the investment that managed futures funds are willing to accept

  \item The number of managed futures funds that should be included in the portfolio to achieve diversification

\end{enumerate}

For example, in order to achieve a reasonable degree of diversification, a portfolio may consist of about six managed futures funds. If the minimum investment size for large institutional-quality funds is assumed to be $\$ 5$ million, an investment of at least $\$ 30$ million is required to make it cost-effective to create a portfolio of managed futures funds. The management fee associated with a $\$ 30$ million portfolio is around $\$ 300,000$ or more. This amount may not be enough to create a team that can select and manage a portfolio of managed futures funds. As the size of the investment increases, the extra level of fees paid increases as well, and it will eventually become economical to create an in-house team to manage a portfolio of managed futures funds and forgo allocations to multi-managed futures funds.

The second issue is related to the next level of allocation, when it becomes viable for the investor to use a managed account platform. Managed accounts offer a number of very important advantages over managed futures funds, including transparency, security of collateral, and ease of opening and closing positions. However, they require the investor to have experienced people and reliable systems in place, which can be costly.

Some funds are structured with different share classes, which may differ in fee structure and withdrawal rights. For example, someone who wants daily liquidity might be willing to pay a higher management fee than someone willing to accept annual liquidity. A new investment may be held in a temporary share class until it reaches the same high-water mark as the rest of the fund. Some classes may be invested in additional assets that are not part of the main fund.

\section*{Multimanager Funds}
Multi-managed futures funds are known variously as managed futures funds of funds or commodity pools. From the investor's standpoint, both accomplish the same thing: They provide a single vehicle for investing in a diversified portfolio of managed futures funds. The differences are chiefly regulatory, relating to the way the funds are structured and where they are offered. Commodity pool operators (CPOs), for example, create vehicles that are distributed in the United States. They are common investment vehicles for retail investors, high-net-worth individuals, and even some small institutions. Some funds launch offshore funds and tend to attract larger institutional investors. The expression fund of funds derives from a time when the primary investment vehicle at the individual managed futures funds level was a fund. As the industry has evolved, multimanager funds have migrated to the use of managed accounts.

The primary benefits of a multi-managed futures fund structure are accessing the expertise that the fund manager has in choosing the managers, structuring the portfolio, performing both investment and operational due diligence, reporting performance, monitoring risk, and accounting. In addition, the fund manager performs less obvious tasks, such as collecting data, meeting managers, running background checks, analyzing performance and strategies, negotiating contracts and fees, monitoring performance, and rebalancing the portfolio as necessary. From an investor's perspective, the investment offering and services of such a structure consolidate much of the work into choosing and reviewing a single organization. Because individual multi-managed futures funds have different investment objectives, investors need to find a fund that is consistent with their needs for risk and return as well as for reporting and transparency.

With a multimanager futures fund, the manager assembles a portfolio of managed futures funds and then accepts investments in the entire portfolio. The multimanaged futures fund manager charges a fee for portfolio construction and oversight services. Each of the managers in the fund also charges a fee. Although investors negotiate contracts and fees with the multi-managed futures fund manager directly, and the investment is consolidated into a single organization, investors still have due diligence and monitoring obligations.

The fees charged by multi-managed futures fund managers raise important questions about how to structure the investment. If the level of allocation is relatively small, the investor would likely invest in managed futures funds. However, if investors intend to allocate a large amount to managed futures funds, they might be better advised to save the fees that would be paid to a multi-managed futures fund manager and simply hire the staff and consultants needed to select managed futures funds, perform the due diligence, construct the portfolios, and so on. Managed accounts, which are the vehicle of choice for most multi-managed futures fund managers, are a much bigger undertaking than are managed futures funds. To warrant the work involved in setting up brokerage accounts, negotiating agreements, monitoring the accounts, reconciling trades, complying with anti-money-laundering regulations, managing cash flows, and so forth, a reasonable break-even point is an investment in managed futures funds of around $\$ 500$ million.

\section*{Structuring Managed Futures Products with Managed Accounts}
A managed account is a brokerage account held by a brokerage firm that is also registered as a futures commission merchant, in which investment discretion has been assigned to the managed futures fund manager. The investor is responsible for opening and maintaining the account, reconciling brokerage statements, and maintaining cash controls, as well as negotiating contracts with managers, including investment management agreements and powers of attorney. The limited power of attorney gives the manager authority to trade on the investor's behalf, but the money has to remain in the investor's account. The investor controls the terms of the power of attorney, including the right to revoke trading privileges.

The key advantage to a managed account is complete control. By pulling trading privileges, the investor has the ability to manage the cash and liquidate the account at any time. Managed accounts, then, avoid the lockup and gating provisions frequently found in hedge fund investments. In theory, this gives the investor better than daily liquidity, as the account can be liquidated whenever the market is open. That alone is enough to make some investors demand managed accounts, especially investors with in-house staff to handle the paperwork.

Managed accounts have other advantages. The money is within the investor's control, not the fund manager's, at all times. The accounts offer complete transparency. The investor can see the positions, trades, and details at any time. Managed accounts, then, virtually eliminate the risk of fraud, as the transparency and security of these accounts prevent the manager from misstating leverage, manipulating returns, or stealing the investor's assets. The investor can choose the parameters for leverage based on the targeted volatility of returns. The choice of leverage makes it easier for the investor to manage the underlying cash. In fact, this type of managed futures fund account structure is often looked at as an overlay on the cash position in an investor's portfolio, rather than a separate asset class.

Of course, these advantages come at a cost. The first is the reduced pool of managers to choose from. Many large managers do not accept managed accounts, whereas those that do require a large minimum investment and other administrative stipulations. In addition, the previously mentioned transparency and control come with the responsibility for establishing and maintaining brokerage accounts that require legal, administrative, risk, and investment oversight in accordance with each organization's investment standards. Further, unless procured by the investor, there is no administrator or auditor.

Managed accounts can be set up in a variety of ways to meet different portfolio policy requirements. The limited partnership structure of hedge funds limits investor liability to the amount invested. For example, an investor who allocates $\$ 10$ million to a failed hedge fund cannot lose more than $\$ 10$ million, even if the hedge fund is highly leveraged and sustains losses greater than the amount of contributed client assets. Managed accounts, however, do not automatically have a limited liability structure. Especially in futures markets, where the required margin is much smaller than the notional value of contracts, investor losses in high margin-to-equity investments can be larger than the amount of contributed capital. Therefore, managed accounts must be carefully designed with a legal structure that ensures that limited liability is obtained. Structures offering limited liability vary by legal jurisdiction but may include limited liability companies, limited partnerships, special purpose vehicles (SPVs), or bankruptcy-remote entities. Each structure is designed to limit investor losses to the amount of cash invested, even if trading activity incurs greater losses.

In many managed account situations, the investor begins by setting up an SPV or another holding entity to fence off any trading liabilities from the rest of the money that the investor controls. It is not a necessary step, however, as there are other ways to manage the potential liability. In most cases, the investor uses the SPV to open an account at a brokerage firm where the managed futures funds manager has trading authority. The investor gives the manager the authority to trade in the account.

\section*{Structuring Managed Futures Products with Platforms}
An alternative way to structure a managed futures funds investment is through a platform. This is a relatively new product, offered by a handful of financial services firms. It operates almost like a multi-managed futures fund, except that investors can select their own leverage and create their own portfolios from the mix of funds offered through the platform.

Platform companies argue that a key advantage of their structure is having objective, independent boards of directors and vendors that are selected by the platform company, not a manager. The platform structure may also reduce custody concerns. Usually, these platforms pass on some of the advantages of managed accounts, such as transparency, liquidity, and customized leverage.

Investors can have a series of fund investments in the platform's participating money managers, receiving consolidated performance information as well as consolidated subscription and redemption paperwork from the platform. It is relatively easy to move money from one manager to another. Because of the transparency and liquidity, these are a hybrid of managed accounts and managed futures funds.

The next exhibit consolidates much of the previous discussion into a summary table, providing an overview of the primary characteristics of the four types of investment structures: managed futures funds, multi-managed futures funds, managed accounts, and platforms. While there are definitely exceptions to the assignment of characteristics in this table, it should serve as a good starting point when considering an investment structure.

Structural Characteristics of Managed Futures Funds, Multi-Managed Futures Funds, Managed Accounts, and Platforms

\begin{center}
\begin{tabular}{|c|c|c|c|c|c|c|c|c|}
\hline
 & Liability & Liquidity & \begin{tabular}{l}
Funding \\
and \\
Leverage \\
\end{tabular} & \begin{tabular}{l}
Oversight and \\
Control of \\
Assets \\
\end{tabular} & Maintenance & \begin{tabular}{l}
Position \\
and \\
Trade \\
Transparency \\
\end{tabular} & Availability & \begin{tabular}{l}
Due \\
Diligence \\
Burden \\
\end{tabular} \\
\hline
\begin{tabular}{l}
CTA \\
Fund \\
\end{tabular} & Limited & Monthly & \begin{tabular}{l}
Manager \\
determined \\
\end{tabular} & \begin{tabular}{l}
Directors \\
selected by \\
manager \\
\end{tabular} & Low & Usually not & \begin{tabular}{l}
Most managers offer flagship \\
fund. \\
\end{tabular} & Medium \\
\hline
\begin{tabular}{l}
Multi- \\
CTA \\
Fund \\
\end{tabular} & Limited & \begin{tabular}{l}
Weekly/ \\
Monthly \\
\end{tabular} & \begin{tabular}{l}
Manager \\
determined \\
\end{tabular} & \begin{tabular}{l}
Directors \\
selected by \\
manager \\
\end{tabular} & Low & Usually not & \begin{tabular}{l}
Most managers offer flagship \\
fund. \\
\end{tabular} & Low \\
\hline
\begin{tabular}{l}
Managed \\
Account \\
\end{tabular} & Unlimited & Daily & \begin{tabular}{l}
Customer \\
determined \\
\end{tabular} & Investor & High & Yes & \begin{tabular}{l}
Not all managers accept \\
managed accounts. \\
\end{tabular} & High \\
\hline
Platform & Limited & \begin{tabular}{l}
Weekly/ \\
Monthly \\
\end{tabular} & Hybrid & \begin{tabular}{l}
Directors \\
selected by \\
platform \\
\end{tabular} & Medium & \begin{tabular}{l}
Varies; \\
manager \\
determined \\
\end{tabular} & \begin{tabular}{l}
Not all managers have an \\
established relationship with a \\
platform. \\
\end{tabular} & Low \\
\hline
\end{tabular}
\end{center}


\end{document}