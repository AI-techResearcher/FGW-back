\documentclass[11pt]{article}
\usepackage[utf8]{inputenc}
\usepackage[T1]{fontenc}
\usepackage{amsmath}
\usepackage{amsfonts}
\usepackage{amssymb}
\usepackage[version=4]{mhchem}
\usepackage{stmaryrd}

\begin{document}
Global Macro

Most macro funds employ discretionary trading and are often concentrated in specific markets or themes. As their name implies, global macro hedge funds take a macroeconomic approach on a global basis in their investment strategy. These are top-down managers who invest opportunistically across national borders, financial markets, currencies, and commodities. They take large positions that are either long or short, depending on the hedge fund manager's forecast of changes in equity prices, interest rates, currencies, monetary policies, and macroeconomic variables such as inflation, unemployment, and trade balances.

Global macro funds have the broadest investment universe: They are not limited by market segment, industry sector, geographic region, financial market, or currency, and therefore tend to offer high diversification. Macro funds tend to have low correlation to stock and bond investments, as well as to other types of hedge funds. Given their broad mandate, the returns earned by macro managers may also have relatively low correlation to other macro funds.

The ability to invest widely across currencies, commodities, financial markets, geographic borders, and time zones is a double-edged sword. On the one hand, this mandate allows global macro funds the widest universe in which to implement their strategies. On the other hand, it lacks a predetermined focus. As more institutional investors have moved into the hedge fund marketplace, they are demanding fund managers who offer greater investment focus rather than investing with managers who have free rein.

Global macro funds tend to have large amounts of investor capital. In addition, they may apply leverage to increase the size of their macro bets. As a result, global macro hedge funds tend to receive the greatest attention and publicity in the financial markets. Although macro managers have broad latitude in their trades, examples of trading strategies that are common or classic across managers are discussed here, to illustrate the essence of global macro fund investing.

\section*{Global Macro Strategies: The Case of Exchange Rates}
Profit opportunities may exist when national governments impose fixed or managed exchange rates. Macro managers often seek to invest in markets that they perceive to be out of equilibrium or that exhibit a risk-reward trade-off skewed in the manager's favor. High levels of competition tend to drive market prices to approximate their informationally efficient values. However, actions by powerful national governments can, at least temporarily, cause market prices to diverge substantially from their expected long-run values in the absence of government actions. Perhaps the best example of these types of trades can be found in countries where the government has mandated fixed or managed currency rates between its currency and the currency of one or more other nations. Managers of macro funds monitor these currencies and estimate the likelihood of a currency revaluation or devaluation to a price other than the official rate.

Fund manager George Soros speculated famously in currency markets in the 1990s through the Quantum Fund. Soros made substantial gains in 1992 by successfully wagering that the British pound would devalue. In the days before the euro, the British pound (GBP) was a member of the European Exchange Rate Mechanism (ERM), which sought to keep currencies within a specified range of values relative to other European currencies. When the pound reached below the target rate, the British government would intervene to raise the value of the GBP relative to the DM. For the GBP, this scheme fell apart in September 1992. Hedge funds and other market participants were short selling the GBP, betting that the British government would stop purchasing the GBP in order to defend the ERM rate and system. Finally, the GBP moved to a floating rate and exited the ERM. The GBP suffered an overnight loss of $4 \%$ and fell $25 \%$ versus the U.S. dollar by the end of 1992 . Those who were short GBP against DM were able to book a large and swift profit as the market forced the GBP to trade at a rate more reflective of the fundamentals. At the time, Germany had stronger monetary and fiscal policy fundamentals.

Soros was accused of contributing to the Asian contagion in the fall of 1997, when Thailand devalued its currency, the baht, triggering a domino effect in currency movements throughout Southeast Asia. In this case, Thailand, Malaysia, and Indonesia had currency rates that were pegged relative to a basket of currencies, with a heavy weighting to the U.S. dollar. Each country had high interest rates, large external debt, and large current account deficits, in which the value of imports exceeded the value of exports. Soros and other market participants increasingly short sold these currencies at the government-supported fixed exchange rates, and the respective governments seemed to be the only buyers. Eventually, each government exhausted its official reserves and was forced to stop the defense of its currency. Once the governments stopped buying their currency at the official rate, each currency moved to a freely floating value. Within a short time, the Thai baht and other Asian currencies declined by $40 \%$ to $70 \%$.

\section*{Global Macro Strategies: The Case of Bonds}
Markets for sovereign bonds may also present global macro funds with potential trade and profit opportunities. In addition to attempting to control currency rates, national governments exert enormous influence on the interest rates of their bonds, which are known as sovereign bonds. Global macro hedge fund managers often speculate on sovereign bond prices. Between 1994 and 1998, a bullish bet was to take large positions on new entrants into the euro currency. The sovereign bonds of Portugal, Italy, Greece, and Spain-the countries that joined the euro currency in 2001-were extraordinarily profitable investments as the countries prepared to enter the economic union. As with all countries seeking to enter the union, these nations were required to meet the terms of the Maastricht Treaty, which required annual government budget deficits below $3 \%$ of GDP, total national debt below $60 \%$ of GDP, an inflation rate no higher than $1.5 \%$ above the strongest member countries, and long-term interest rates within $2 \%$ of the current members of the union.

As indications of the profits earned by funds establishing long positions in the debt and equity of countries entering the economic union, note that Greek sovereign bonds denominated in drachmas yielded 25\% in 1994 and declined to $11 \%$ by 1998, whereas the Greek stock market increased by $130 \%$ between 1998 and 1999 . More recently, some funds have profited from shorting sovereign debt.

\section*{Global Macro Strategies: The Case of Economic Policy}
Macro managers are expert at understanding the impact of central bank intervention in the markets. A recent example is the election of Shinzo Abe as the prime minister of Japan. His 2012 campaign focused on economic reform, seeking to restore inflation and economic growth after two decades of malaise. Abe's plan, now deemed "Abenomics," had three arrows: aggressive monetary easing, large public investments, and structural reforms. In just over one year, the monetary supply doubled, which led to a quick increase in the Nikkei index of over $50 \%$ and a decline in the yen against many world currencies of approximately $20 \%$. Macro managers with long stock and short yen positions made quick and substantial profits by buying into the short-term stimulus measures implemented by Abe shortly after his election. As the value of the yen declines, exports become more competitive and profitable. The goal is for these increased profits of large exporting firms to result in increased investment, productivity, employment, and wages. Ideally, these higher incomes would lead to increased domestic spending and consumption.

Fighting against Abenomics are demographics, a consumption tax increase, and the delay of a decline in corporate tax rates from $35 \%$ to a desired $29 \%$. Demographics are difficult, as the population of 127 million is expected to decline to less than 87 million by 2060, according to the National Institute of Population and Social Security Research. As the number of retirees increases and the number of births declines, old age benefits deplete government budgets faster than young entrants can increase the productive workforce and the resulting income tax payments. When sales taxes were increased from $3 \%$ to $5 \%$ in 1997 , a multiyear recession ensued. The consumption tax increased to $8 \%$ in 2014. These tax increases offset the optimism that higher stock prices and easing monetary policy are meant to provide. While exports have increased, domestic job growth and consumer demand remain weak, even in the face of import price inflation.

\section*{Global Macro Strategies: Thematic Investing}
Thematic investing is a trading strategy that is not based on a particular instrument or market; rather, it is based on secular and long-term changes in some fundamental economic variables or relationships-for example, trends in population, the need for alternative sources of energy, or changes in a particular region of the world economy. The last type is exemplified by the rise of China. Investors who believe that Chinese GDP growth will remain strong have a wide variety of trading ideas, many of them outside the Chinese markets. One such view might be the decline of the developed markets of the United States and Europe as they continue to deal with large trade and budget deficits. China's rise may have benefits for other Asian countries, including Japan, India, and South Korea. The strength of the Chinese economy may cause even the Chinese to outsource, which could lead to economic growth and additional wage income in less developed Asian countries, such as Vietnam.

The power of China is, perhaps, most clearly seen in the commodity markets. China's rise accounted for a substantial portion of the world's increased demand for a number of commodities. As China continued to urbanize and industrialize, building new roads, cities, workplaces, and consumer goods stoked the demand for commodities. As China's growth has slowed, commodity prices moved lower for a number of years. Savvy global macro managers who are able to better predict these major global economic themes may use bets in commodity markets and other markets to attempt to generate superior returns.

\section*{Global Macro Strategies: Macro and Micro}
These examples of common global macro investing illustrate the role of macro-economics in the implementation of the strategy. Many of the hedge fund strategies discussed in the remaining sessions of Topic 5 are implemented through an understanding of individual firms, individual securities, and market microstructure. Market microstructure is the study of how transactions take place, including the costs involved and the behavior of bid and ask prices. But each of the examples of global macro investing just discussed is more concerned with the economic workings of economy-wide or even global markets, institutions, and forces. The illustrations involved exchange rates, interest rates, inflation rates, country economic growth rates, regional growth rates, and so forth.

Success in global macro investing requires superior skills in forecasting changes at the macroeconomic level. The necessary macroeconomic analysis can be performed qualitatively or quantitatively. Quantitative macroeconomic models can be empirical models of how markets have behaved (i.e., positive models) or theoretical models of how they ought to behave (i.e., normative models). The models vary in size and sophistication. However, the importance of experience, intuition, and data gathering should not be underestimated.

\section*{Three Primary Risks of Macro Investing}
Macro funds often have higher risk exposures than most other strategies to market risk, event risk, and leverage risk.

Market risk refers to exposure to directional moves in general market price levels. Macro funds typically do not focus on equity markets, as equities can be highly influenced by microeconomic factors, such as company-specific events. However, macro funds can take substantial and concentrated risks in currency, commodity, and sovereign debt markets, especially when it is believed that changes in governmental policies will lead to large moves in the underlying markets.

Event risk refers to sudden and unexpected changes in market conditions resulting from a specific event (e.g., Lehman Brothers bankruptcy). Macro funds attempt to benefit from particular events. They seek profits from large market dislocations, especially those involving governmental financial policies. However, because macro funds seek out situations of event risk at the macroeconomic level, these funds can have substantial changes in value over short periods of time.

Leverage refers to the use of financing to acquire and maintain market positions larger than the assets under management (AUM) of the fund. Leverage is typically established through borrowing or derivatives positions and poses risks. Funds with leverage may be forced to deploy additional capital if they experience losses, and if they are unable to do so, they may be forced to liquidate positions at the least opportune time. Magnifying the risk of their concentrated positions in markets with substantial event risks, many macro funds use futures, swap, and forward markets to increase the leverage of the fund. While gains can be substantial, leverage can also lead to dramatic losses. Leverage in these derivatives markets, though, is less problematic than leverage in single securities sourced through prime brokers, as derivatives markets are less likely to require large changes in margin without notice.


\end{document}