\documentclass[11pt]{article}
\usepackage[utf8]{inputenc}
\usepackage[T1]{fontenc}
\usepackage{amsmath}
\usepackage{amsfonts}
\usepackage{amssymb}
\usepackage[version=4]{mhchem}
\usepackage{stmaryrd}

\begin{document}
\section*{APPLICATION A}
Question : Consider a CTA with $\$ 30$ million capital. The CTA has determined that $10 \%$ of this capital should be allocated to trading in the Brent Crude Oil market. The sizing function is estimated to be 0.8 , which means the trader's signal is strong and indicates a long position in this market. The size of each futures contract is 1,000 barrels, and assuming a current price of $\$ 50$ per barrel, the notional value of each contract will be $\$ 50,000$. Finally, assume that the annualized volatility target is $20 \%$ and that the annualized realized volatility using near-term futures prices of the past 30 days has been $30 \%$.

\section*{Answer and Explanation}
To solve this problem, we must use Equation 1:

$$
\text { Number of Futures Contracts }=\text { Sizing Function } \times \frac{\text { Risk Loading } \times \text { Equity }}{\text { Notional Value }} \times \frac{R V \text { ol } l_{T}}{R V \text { ol }_{R}}
$$

Number of Futures Contracts $=0.8 \times \frac{10 \% \times 30,000,000}{50,000} \times \frac{20 \%}{30 \%}=32$

\section*{APPLICATION B}
Question : Continuing with the previous example in which the CTA's capital is assumed to be $\$ 30$ million, suppose the sizing function is 0.8 and the risk loading is $0.2 \%$. The daily price volatility of oil is estimated to be $\$ 1.1$, and each contract is for 1,000 barrels.

\section*{Answer and Explanation}
Using Equation 2, we can solve for the number of contracts:

$$
\begin{aligned}
& \text { Number of Contracts }=\text { Sizing Function } \times \frac{\text { Risk Loading } \times \text { Capital }}{P V \text { ol }} R_{R} \times \text { Contract } \\
& \text { Number of Contracts }=0.8 \times \frac{0.2 \% \times 30,000,000}{1.1 \times 1,000} \approx 44
\end{aligned}
$$

The answer is 44


\end{document}