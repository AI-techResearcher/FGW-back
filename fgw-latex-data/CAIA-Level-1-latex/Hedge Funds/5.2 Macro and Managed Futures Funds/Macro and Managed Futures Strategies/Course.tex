\documentclass[11pt]{article}
\usepackage[utf8]{inputenc}
\usepackage[T1]{fontenc}
\usepackage{amsmath}
\usepackage{amsfonts}
\usepackage{amssymb}
\usepackage[version=4]{mhchem}
\usepackage{stmaryrd}

\begin{document}
Macro and Managed Futures Strategies

This first of five sessions on hedge fund strategies begins at, literally, the macro level. This session explores macro funds (i.e., global macro funds) and managed futures funds. Macro and managed futures strategies can differ substantially from other hedge fund strategies. Many investment strategies, especially in the arbitrage sector, focus on inefficiencies within markets at the security level. Macro and managed futures funds focus on the big picture, placing trades predominantly in futures, forward, and swap markets that attempt to benefit from anticipating price level movements in major sectors or to take advantage of potential inefficiencies at sector and country levels. At the end of 2021, Hedge Fund Research (HFR) estimated that macro and managed futures funds managed $\$ 637$ billion, which is $15 \%$ of the hedge fund universe.

Macro (i.e., global macro) and managed futures funds share many common features. They tend to have substantially greater liquidity and capacity and, when focused on exchange-traded futures markets, lower counterparty risks than hedge funds that follow other strategies. Capacity refers to the quantity of capital that a fund can deploy without substantial reduction in risk-adjusted performance. Counterparty risk is the uncertainty associated with the economic outcomes of one party to a contract due to potential failure of the other side of the contract to fulfill its obligations, presumably due to insolvency or illiquidity. This session focuses on the major distinctions within the category of macro and managed futures funds.

\section*{Discretionary versus Systematic Trading}
Discretionary fund trading occurs when the decisions of the investment process are made according to the judgment of human traders. The trader may rely on computers for calculations and other data analysis, but in discretionary trading, the trader must do more than simply mechanically implement the instructions of a computer program. Despite the fact that computer programs are obviously written with human judgment, in discretionary trading there must be an ongoing and substantial component of human judgment.

Systematic fund trading, often referred to as black-box model trading because the details are hidden in complex software, occurs when the ongoing trading decisions of the investment process are automatically generated by computer programs. Although these computer programs are designed with human judgment, the ongoing application of the program does not involve substantial human judgment. Traders make decisions about when to use the program and may even adjust various parameters, including those that control the size and risk of positions. The key is that individual trades are not regularly subjected to human judgment before being implemented.

The concept of discretionary versus systematic trading applies to all investment processes, not just macro and managed futures funds. However, the distinction is especially relevant in discussing macro and managed futures funds. Global macro funds tend to use discretionary trading, and managed futures funds tend to use systematic trading. However, there are many exceptions, and some funds use discretionary trading for some of their trading activity and systematic trading for their other trading activity.

\section*{Fundamental and Technical Analysis}
Trading strategies are based on analysis of information. A major distinction is whether the investment strategy analyzes information with fundamental analysis, technical analysis, or both. Briefly, technical analysis relies on data from trading activity, including past prices and volume data. Fundamental analysis uses underlying financial and economic information to ascertain intrinsic values based on economic modeling. Trading decisions can be based purely on technical or fundamental analysis or on a combination of the two. For example, some investment processes rely on fundamental analysis to determine potential long and short positions and on technical analysis to determine the timing of entering and exiting those positions.

Technical analysis focuses on price movements due to trading activity or other information revealed by trading activity to predict future price movements. Typically, technical analysis quantitatively analyzes the price and volume history of one or more securities with the goal of identifying and exploiting price patterns or tendencies.

One motivation for using technical analysis is based on the idea that prices already incorporate some economic information, but price patterns may be identified that could be exploited for profit opportunities. The underlying assumption is that prices may not instantaneously and completely reflect all available information (i.e., prices may be slow to react to new information). For instance, if there is asynchronous global economic growth, one might forecast exploitable price movements as some local markets (e.g., in country-specific equity indices) react on a delayed basis to information already reflected in larger and more efficient markets. A common strategy in macro and managed futures funds is to attempt to exploit currency exchange rate movements, such as trends resulting from announced government intervention in foreign exchange markets. The key to this technical trading is the assumption that although prices are predominantly based on underlying economic information, analysis of trading activity can reveal consistent patterns of how prices respond to new information.

A second motivation to pursuing strategies based on technical analysis is a belief that market prices are substantially determined by trading activity that is unrelated to a rational analysis of underlying economic information. These technical strategies attempt to identify price patterns generated by trading activity and to identify those patterns on a timely basis. An example would be a prediction that an index is not likely to pass through a particular level as it approaches that level (e.g., an index nearing a round number such as 1,000 ), but if that level is breached, then the price is likely to continue moving in the same direction.

Fundamental analysis attempts to determine the value of a security or some other important variable through an understanding of the underlying economic factors. Fundamental analysis can be performed at a macro level using economy-wide information, such as economic growth rates, inflation rates, unemployment rates, and data on commodity supply and demand. It can also be performed at the micro level using firm-specific data, such as revenues, expenses, earnings, and dividends, or security-specific information. Fundamental analysis often focuses on predicting price changes to securities based on current and anticipated changes in underlying economic factors. Underlying economic factors can include (1) market or economy-wide factors, such as changes to monetary or fiscal policies; (2) industry-wide factors, such as changes in relevant commodity prices or consumer preferences; and (3) firm-specific factors, such as product innovations, product failures, labor strikes, or accidents. Fundamental analysis and technical analysis are used throughout alternative investment strategies, but the distinction in macro and managed futures funds is especially interesting. As in the case of distinguishing between discretionary and systematic trading, some funds focus on strategies using fundamental analysis, some focus on strategies using technical analysis, and some have a mix. However, systematic trading strategies tend to be built around technical analysis.

\section*{Organization of the Session}
There are sufficient similarities between macro funds and managed futures funds to combine the two in this session. Global macro funds are more likely to be discretionary and emphasize fundamental analysis, whereas managed futures tend to be more systematic and emphasize technical analysis. So although exceptions are frequent, the remainder of this session begins with a section on global macro funds to discuss the use of discretionary trading and strategies based on fundamental analysis. Then, the session on managed futures funds discusses futures contracts, systematic trading, and technical analysis.


\end{document}