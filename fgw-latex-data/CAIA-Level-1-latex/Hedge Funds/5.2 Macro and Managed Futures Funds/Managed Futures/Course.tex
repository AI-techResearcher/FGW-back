\documentclass[11pt]{article}
\usepackage[utf8]{inputenc}
\usepackage[T1]{fontenc}
\usepackage{amsmath}
\usepackage{amsfonts}
\usepackage{amssymb}
\usepackage[version=4]{mhchem}
\usepackage{stmaryrd}

\begin{document}
Managed Futures

The term managed futures refers to the active trading of futures and forward contracts on physical commodities, financial assets, and exchange rates. Managed futures is a subclass of alternative investment strategies. For these strategies, professional money managers (also known as commodity trading advisers [CTAs]) manage client assets by actively taking positions primarily in futures markets, forward markets, options and other liquid derivatives, and structured products. Using highly liquid marked-to-market contracts, they typically provide their clients access to a wide range of asset classes, including fixed income, currencies, equity indices, soft commodities, energy, and metals. They apply leverage either directly via margin or indirectly via the use of derivative products, such as futures and options. Another key feature of this strategy group is the ability to go long or short with relative ease. The flexibility and liquid nature of managed futures strategies coupled with the wide array of markets they trade provide risk and return patterns not easily accessible through traditional asset classes (such as long-only stock and bond portfolios) or other alternative investments (such as hedge funds, real estate, private equity, or long-only commodities).

\section*{Background on Futures Contracts}
The "futures" in "managed futures" denotes the industry's primary focus on using futures contracts or similar instruments. These products are desirable for these strategies because of the transparency and reduced counterparty and credit risk associated with exchange-traded instruments. For example, futures contracts are transparent because they are standardized and highly specified and are traded on markets that pool the collateral of all participants. For these contracts, the clearinghouse takes the other side of the trade, and pooling funds reduces the counterparty and credit risks of bilateral transactions. The pooling of positions and the futures exchange structure allow for substantial reduction in the margin capital required for establishing positions in futures contracts. For example, according to the CME Group, the eurodollar contract, which is one of the most liquid, can have margins as low as $\$ 175$, or $0.017 \%$ of a $\$ 1$ million notional contract. By comparison, the required margin for a long position in an exchange-traded stock is $50 \%$.

Futures contracts emerged in the agricultural markets of the 1800 s as cost-effective vehicles for the transfer and management of the risk related to uncertain crop prices. The history of managed futures products goes back to the middle of the 1900s. The first public futures fund began trading in 1948 and was active until the 1960s. That fund was established before financial futures contracts were invented, and it consequently traded primarily in agricultural commodity futures contracts. The success of that fund spawned other managed futures vehicles, and a new industry was born. Financial futures contracts emerged in the 1970s and eventually offered opportunities for market participants to transfer and manage a variety of financial risks, including equity market risk, interest rate risk, exchange rate risk, and credit risk. With the advent and rapid growth of financial futures contracts, more and more managed futures trading funds and strategies were born.

Previous sessions provide detailed foundational material on the pricing of futures contracts and forward contracts. For the purposes of this session, it suffices to know that futures and forward contracts are similar and can be cost-effective means of establishing positions with risk exposures that very closely approximate those that can be established in the cash market. For example, a market participant may wish to speculate that a particular price, such as the price of corn, gold, a stock index, or a Japanese government bond, is likely to rise. The speculator could buy those assets in the cash market, store them, and then sell them to close the trade. However, cash positions can have numerous disadvantages, such as storage costs, financing costs, higher transaction costs, inconvenience, and restrictions on short selling. Market participants with short-term trading horizons often prefer futures and forward contracts. Futures and forward contracts usually offer lower transaction costs, higher liquidity, more observable pricing, and more flexibility to short sell.

\section*{The Structure of the Managed Futures Industry}
Investors can access the managed futures industry either by investing in a futures trading fund (via a managed account or a commingled fund) or through a commodity pool-a commingled investment vehicle that resembles a fund of funds and is managed by a commodity pool operator (CPO), who invests in a number of underlying CTAs. Investments from a number of investors are pooled together and then invested in futures contracts, either directly by the CPO or through one or more commodity trading advisers. CPOs may be either public or private. In the United States, the requirements for investing in public futures funds generally differ from state to state; globally, the requirements vary from country to country.

Globally, the futures trading industry has a relatively short history of regulation. In the United States, the Commodity Futures Trading Commission (CFTC) was initiated in 1974 as a federal regulatory agency for all futures and derivatives trading. This regulatory body was later supplemented with U.S. futures exchanges and the National Futures Association (NFA), an independent, industry-supported, self-regulatory body created in 1982. In Europe and Asia, managed futures funds are regulated under the same framework as hedge funds. For example, in Europe, managed futures managers are classified as alternative investment fund managers (AIFMs). AIFMs are regulated under the Alternative Investment Fund Managers Directive (AIFMD). In order to solicit business, a manager of such a fund must register as an AIFM and follow certain regulatory and reporting rules in order to qualify for a European passport to solicit business in the EU. Historically, foreign exchange has been one area of the managed futures industry that has remained largely unregulated. The vast majority of currency trading is conducted in the over-the-counter (OTC), interbank (spot), and forward markets, which are subject to only limited regulation. After the 2008 financial crisis, there has been increased regulatory tightening on all OTC markets.

Title VII of the Dodd-Frank Act in the United States and the European Market Infrastructure Regulation (EMIR) in Europe push for increased transparency and standardization of OTC products. This regulatory push has created an incentive to move many traditionally OTC contracts from bilateral contracts to the multilateral cleared contract structure of futures markets. Many in the industry term this movement from traditional OTC contracts to multilateral cleared contracts the futurization of OTC contracts. For the managed futures industry, this means that there is an incentive for growth in futures and a potential increase in the number of tradable futures contracts going forward. Managed futures programs and assets under management have grown substantially over the past several decades. Futures markets have grown in tandem.

\section*{The Purpose of the Managed Futures Industry}
The purpose of the managed futures industry is to enable investors to receive the risk and return of active management within the futures market while enhancing returns and diversification.

The managed futures industry provides a skill-based style of investing. Investment managers attempt to use their special knowledge and insight to establish and manage long and short positions in futures and forward contracts for the purpose of generating consistent, positive returns. These futures managers tend to argue\\
that their superior skill is the key ingredient in generating profitable returns from the futures markets.

Managed futures strategies tend to be based on systematic trading more than discretionary trading. Further, futures managers tend to use more technical analysis, as opposed to trading based on fundamental analysis. This section on managed futures takes a detailed look at systematic trading and technical analysis. The section begins, however, with an overview of futures contracts and futures markets.

\section*{Regulation and Organization of the Managed Futures Industry}
Until the early 1970s, the managed futures industry was largely unregulated. Anyone could advise an investor regarding commodity futures investing or form a fund for the purpose of investing in the futures markets. Recognizing the growth of this industry, the industry's potential impacts on an economy, and the lack of regulation associated with the industry, regulatory authorities have been established for managed futures funds, futures contracts, and, to a lesser extent, forward contracts. For example, in the United States in 1974, Congress enacted the Commodity Exchange Act (CEA) and created the Commodity Futures Trading Commission (CFTC). Under the CEA, Congress first defined the terms commodity pool operator (CPO) and commodity trading adviser (CTA). In addition, Congress established standards for financial reporting, offering memorandum disclosure, and bookkeeping. Congress required CTAs and CPOs to register with the CFTC. Last, Congress required CTAs and CPOs to undergo periodic educational training in cooperation with the National Futures Association (NFA), the designated self-regulatory organization for the managed futures industry.

Commodity trading advisers may invest in both exchange-traded futures contracts and forward contracts. The economic structure of forward contracts is highly similar to that of futures contracts, with the most major difference being that futures contracts are exchange-traded while forward contracts are usually traded over the counter. Forward contracts are private agreements. Therefore, they can have terms that vary considerably from the standard terms of exchange-listed futures contracts. Forward contracts accomplish virtually the same economic goal as a futures contract but with the flexibility of custom-tailored terms. However, futures contracts provide substantial protection against counterparty risk as a result of being backed by the exchange's clearinghouse, whereas forward contracts are exposed to full counterparty risk. In the remainder of this session, both types of contracts are referred to as futures contracts.

\section*{Three Ways to Access Managed Futures}
There are three ways to access the skill-based investing of the managed futures industry:

\begin{enumerate}
  \item Public commodity pools

  \item Private commodity pools

  \item Individually managed accounts

\end{enumerate}

Commodity pools are investment funds that combine the money of several investors for the purpose of investing in the futures markets. Public commodity pools are open to the general public for investing in much the same way that a mutual fund sells its shares to the public. In the United States, public commodity pools must file a registration statement with the SEC (Securities and Exchange Commission) before distributing shares in the pool to investors. An advantage of public commodity pools is the low minimum investment and the high liquidity that they provide for investors, allowing them to withdraw their investments with relatively short notice (compared to other hedge fund strategies).

Private commodity pools are funds that invest in the futures markets and are sold privately to high-net-worth investors and institutional investors. They are similar in structure to hedge funds and are increasingly considered a subset of the hedge fund marketplace. Commodity pools are managed by a general partner. In the United States, the general partner for the pool must typically register with the CFTC and the NFA as a CPO. However, there are exceptions to the general rule. Private commodity pools are organized privately to avoid lengthy or burdensome initial regulatory requirements, such as registration with the SEC in the United States, and to avoid ongoing reporting requirements, such as those of the CFTC in the United States. Otherwise, their investment objective is the same as that of a public commodity pool. Advantages of private commodity pools are usually lower fees and greater flexibility to implement investment strategies.

The CPOs for either public or private pools typically hire one or more CTAs to manage the money deposited with the pool. Commodity trading advisers (CTAs) are professional money managers who specialize in the futures markets. Some CPOs act as a fund of funds, diversifying investments across a number of CTA products. Like CPOs, CTAs in the United States must register with the CFTC and the NFA before managing money for a commodity pool. In some cases, a managed futures investment manager is registered as both a CPO and a CTA. In this case, the general partner for a commodity pool may also act as its investment adviser.

In addition, wealthy investors and institutional investors may use a managed account. A managed account (or separately managed account) is created when money is placed directly with a CTA in an individual account rather than being pooled with other investors. When large enough to be cost-effective, managed accounts offer numerous advantages over pooled arrangements. These separate accounts have the advantage of representing narrowly defined and specific investment objectives tailored to the investor's preferences. With a managed account, the investor retains custody of the assets with the investor's regular broker and only needs to allow the CPO or CTA to exert trading authority in the account. Other advantages to the investor include transparency and control, which allow the investor to see all of the trading activity, as well as the ability to increase or decrease the leverage applied. Like hedge funds, CTAs and CPOs charge management fees and incentive fees. The standard hedge fund fees of 2 and 20 ( $2 \%$ management fee and $20 \%$ incentive fee) are equally applicable to the managed futures industry, although management fees can range from $0 \%$ to $3 \%$ and incentive fees can range from $10 \%$ to $35 \%$.


\end{document}