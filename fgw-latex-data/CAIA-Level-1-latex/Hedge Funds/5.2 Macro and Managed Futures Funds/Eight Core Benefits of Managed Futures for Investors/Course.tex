\documentclass[11pt]{article}
\usepackage[utf8]{inputenc}
\usepackage[T1]{fontenc}
\usepackage{amsmath}
\usepackage{amsfonts}
\usepackage{amssymb}
\usepackage[version=4]{mhchem}
\usepackage{stmaryrd}

\title{Reading }

\author{}
\date{}


\begin{document}
\maketitle
Eight Core Benefits of Managed Futures for Investors

Managed futures provide a number of benefits in terms of risk-return trade-offs to investors, eight of which are introduced here:

\begin{enumerate}
  \item Diversification. Managed futures constitute an alternative asset class that has achieved strong performance in both up and down equity, commodity, and currency markets, and has exhibited low correlation to traditional asset classes, such as stocks, bonds, cash, and real estate. Managed futures, when used in conjunction with traditional asset classes, may reduce risk while potentially increasing portfolio returns.

  \item Performance. Historically, managed futures have provided risk-return profiles comparable to those of many traditional asset classes and superior to those offered by long-only investments in commodities. For example, the historical Sharpe ratio of a diversified portfolio of managed futures could be four times higher than that of a long-only portfolio of commodities.

  \item Access to multiple markets. There are more than 150 liquid futures products across the globe, including stock indices, currencies, interest rates, fixed income, energies, metals, and agricultural products. CTAs are able to take advantage of potential opportunities in various asset classes in many geographical locations. The fundamental law of active management states that the information ratio of an investment increases as the breadth of the investment strategy increases (holding other variables constant). This means that CTAs have the potential to provide performance with a superior risk-return profile.

  \item Transparency. Futures prices are determined competitively and are marked to market daily. The fact that futures prices tend to be determined in single-price discovery markets in which everyone can see the limit order book and in which the settlement prices are, in most cases, tradable makes them more accurate and more reliable than prices determined in nearly any other market. The prices used to mark portfolios to market are not stale. There are no dark pools of liquidity, like those found in equity markets. There are no interpolation methods similar to those of some bond markets, in which only a handful of bonds actually trade on any given day. Moreover, there are no models needed to determine the value of structured securities. As a result, the returns experienced are real and have not been smoothed.

  \item Liquidity. Liquidity has already been mentioned, but only in the context of liquidating positions and extracting cash. In fact, transaction costs in futures are lower than in their underlying cash markets. As a result, the benefits of the kind of active management and trading that CTAs implement are available with less drag from market impact than one would incur with the same type of trading in underlying markets.

  \item Size. As an investment alternative, managed futures have been available since the 1970 s and experienced significant growth over the past several decades. As of 2021 the size of the market was estimated at about $\$ 350$ billion. This means the market has reached a level at which it can accommodate allocations from institutional investors.

  \item No withholding taxes. In a number of the world's stock and bond markets, foreign investors are taxed more heavily than are domestic investors. With futures, all of the tax benefits that accrue to domestic investors can be passed through to those who use futures in the form of simple cash/futures arbitrage.

  \item Very low foreign exchange risk. Futures on foreign assets or commodities have little exposure to foreign exchange risk. A futures contract has no net liquidating value. As a result, a long position in a European equity index futures contract has no exposure to the change in the price of the euro, whereas an investment in European equities exposes the investor not only to changes in the price of European stocks but to changes in the price of the euro as well. In the case of futures, the investor's currency risk is limited to the comparatively small amounts of margin that must be posted at exchanges around the world and to any realized profit or loss that has not yet been converted back into the investor's home currency.

\end{enumerate}

To understand this benefit, note that a position in a futures contract is similar to a long position in the same asset in the cash market, where the position is financed through borrowing. This means a futures position in a foreign-currency-denominated asset is similar to a cash position in the same asset with investment financed through borrowing in the same foreign currency. As a result, currency fluctuations will have equal effects on assets and liabilities of the investors, with zero net effect. For instance, from a Japanese investor's viewpoint, a position in Euro Stoxx futures makes or loses money only when the index rises or falls. A change in the yen price of the euro would, by itself, produce neither a gain nor a loss, because the investor has no cash position in euros.

In contrast, the yen return to a fully funded, currency-unhedged investment in Euro Stoxx would be, to a first approximation, the sum of the return on Euro Stoxx, as viewed by a euro-based investor, and the yen return on the euro. Conventional money managers are well aware of the problems raised by currency risk because currency volatility is potentially very large. During periods of increased uncertainty in global markets, currency volatility may contribute as much to the risk of a fully funded position as does the volatility of the underlying asset, the Euro Stoxx.

For CTAs, the only foreign currency risk associated with using futures to trade comes from the value of cash or collateral balances that are the result of either posting margin collateral or accumulating gains or losses in currencies in which the contracts are denominated. Because these balances tend to be small relative to the notional values of the positions taken, foreign currency risk is, for all practical purposes, separate from the risks associated with the underlying assets or commodities. This decoupling allows CTAs to take much more nuanced views on currency exposure than would be possible for most conventional money managers, for whom hedging currency exposure can be costly.


\end{document}