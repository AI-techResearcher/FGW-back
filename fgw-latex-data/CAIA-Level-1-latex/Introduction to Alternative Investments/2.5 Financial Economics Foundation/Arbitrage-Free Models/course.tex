\documentclass[11pt]{article}
\usepackage[utf8]{inputenc}
\usepackage[T1]{fontenc}
\usepackage{amsmath}
\usepackage{amsfonts}
\usepackage{amssymb}
\usepackage[version=4]{mhchem}
\usepackage{stmaryrd}

\begin{document}
\section*{Reading}
Arbitrage-Free Models

Arbitrage is the attempt to earn riskless profits (in excess of the risk-free rate) by identifying and trading relatively mispriced assets. The implications of arbitrage activities form an important foundation for understanding finance in general and financial markets in particular. This section discusses arbitrage-free pricing models -a key technique in finance that has revolutionized both the field of finance and the world's financial markets.

\section*{Underlying Concept of Arbitrage-Free Models}
An arbitrage-free model is a financial model with relationships derived by the assumption that arbitrage opportunities do not exist, or at least do not persist. The term arbitrage is sometimes used to describe attempts to earn profits that require the bearing of substantial risk due to uncertainty. For example, an equity portfolio manager might claim to be "arbitraging" the valuation differences between growth stocks and value stocks, but the manager is likely taking nontrivial risk. In its purest sense, often termed pure arbitrage, true arbitrage requires no risk bearing.

Pure arbitrage opportunities exist when identical assets can be traded at different prices, allowing the arbitrageur to buy at the lower price, sell at the higher price, and profit when arbitrage activities force the two identical assets to trade at identical prices. Arbitrage-free pricing models are based on the assumption that in the absence of transaction costs, taxes, or other trading restrictions, identical assets must trade at identical prices. In imperfect markets, arbitrageurs speculate that prices of identical assets will converge through time.

Arbitrage-free modeling provides a framework for understanding pricing relationships under idealized conditions. For example, in the absence of trading costs, if a euro is worth 1.10 Canadian dollars and a Canadian dollar is worth 1.10 U.S. dollars, then a euro will tend to be worth 1.21 U.S. dollars. We deduce this from the knowledge that any value other than 1.21 U.S. dollars would allow an arbitrage profit.

Arbitrage-free modeling is an important tool in modern financial analysis. For more than 50 years, finance has been applying arbitrage-free modeling to more and more assets and financial derivatives. Financial experts have done this by better identifying the relationships that exist among assets. This progress has not only changed the study of finance but has also dramatically changed the functioning of financial markets, as evidenced by the tremendous use of financial derivatives to manage risk.

A key externality of arbitrage activities is that they tend to drive similar assets toward similar prices which, in turn, improves global economic decisions. Better asset prices serve as signals of more accurate information to producers and consumers, which, in turn, enables supply and demand to be balanced at more efficient levels.

\section*{Applications of Arbitrage-Free Models}
Arbitrage-free financial models vary in their complexity. In the next section, we discuss arbitrage-free pricing models that involve virtually instantaneous transactions, such that the arbitrage activity is concluded within seconds. In the section following that, we discuss arbitrage-free models that involve carrying positions for potentially extended periods of time. However, true to the purest definition of arbitrage, the models that are discussed are limited to those models containing little or no risk.

Arbitrage-free pricing models are used in the analysis of interest rates, foreign exchange rates, derivatives, and other areas, such as cash-and-carry trades. Arbitragefree pricing models are relative pricing models. A relative pricing model prescribes the relationship between two prices. A trivial relative pricing model would specify that the price of a troy ounce of gold should sell for about $9.7 \%$ more than an avoirdupois ounce because a troy ounce of gold is about $9.7 \%$ larger. Note that this relative pricing model implies nothing about the overall price level of gold.

An absolute pricing model attempts to describe a value or a price level based on its underlying economic factors. For example, the price of a share of common stock typically involves substantial uncertainty with regard to its future growth. Attempts to model the stock's price (such as by using a dividend growth model) are absolute pricing models, since they estimate a price based on the stock's underlying fundamental factors. Absolute pricing models tend to be imprecise, since the model is based on bold assumptions and estimates about which investors have highly heterogeneous beliefs. However, relative pricing models are typically quite precise. It is the precision of relative pricing models that drives the usefulness of arbitrage-free pricing models. In effect, arbitrage-free pricing models tend to be used wherever relative pricing models are well developed and accurate.

\section*{Arbitrage-Free Pricing in Spot Markets}
The spot market or cash market is any market in which transactions involve immediate payment and delivery: The buyer immediately pays the price, and the seller immediately delivers the product. Technically speaking, virtually all transactions involving financial securities have deferred delivery generated by the settlement period. But deferred delivery of spot (or cash) transactions is usually quite short and exists merely for convenience in facilitating the procedures necessary to settle the transaction.

Arbitrage-free pricing in spot markets involves identifying two sets of transactions with identical outcomes and requiring that their prices be equal. For example, consider an investor wishing to exchange euros for yen. In the spot foreign exchange market, the investor may find that one euro can be exchanged for 140 yen. However, there are numerous sets of transactions for converting euros to yen. For example, the investor may find that one euro can be converted to 1.40 U.S. dollars and that each U.S. dollar can then be converted into 100 yen. Of course, there are many other multiple-transaction paths that would lead to the same result: converting euros to yen. An arbitrage-free pricing model of the foreign exchange rates would describe the relationships that must exist between all of the exchange rates such that no investor or speculator could earn a profit through instantaneous trading among the currencies.

The skeleton of this arbitrage-free pricing model and other more sophisticated models is based on two steps: (1) identify two economically equivalent sets of assets or transactions, and (2) set their values equal to form a relationship based on the underlying determinants of their values. The next section extends this concept to the passage of time.

\section*{Carry Trades with and without Hedging}
Carry trades are typically a set of long and short positions intended to generate perceived benefits through time, such as enhanced return, as the positions are "carried." Carry trades can either be hedged or be exposed to the risks of price changes.

For an unhedged example, consider an investor who observes that a one-year default-free bond in a particular foreign currency offers a $5 \%$ yield, whereas a defaultfree bond in the investor's domestic currency with the same maturity offers a yield of only $4 \%$. The investor shorts the domestic bond (i.e., borrows in the domestic currency) at a cost of $4 \%$ and locks in a $5 \%$ yield in the foreign currency by purchasing the foreign bond with the borrowed cash. The carry trade offers an interest spread of $1 \%$ but is exposed to the risk that the foreign currency will weaken in value relative to the domestic currency. If the foreign currency weakens by more than $1 \%$ per year over the lifetime of the trade, the losses will exceed the $1 \%$ per year net income. In fact, to the extent that interest rate differentials reflect expectations of different inflation rates, the investor should expect the foreign currency to weaken by an amount that offsets the interest rate spread on a risk-adjusted basis.

The investor faces the risk that the proceeds of the foreign bond may be insufficient to settle the short position in the domestic bond at the end of the trade. The investor may decide to hedge the risk of this carry trade by locking in the exchange rate ahead of time between the foreign and domestic currencies. Specifically, the investor could use derivatives (such as a forward contract, discussed in the session entitled Derivatives and Risk-Neutral Valuation) to lock in the rate to exchange the principal amount received in the foreign currency when the long position in the foreign bond matures for the amount due in the domestic currency. The key to the hedge is that it must allow the investor the opportunity to exchange the proceeds of the long position to cover the obligation of the short position at a prenegotiated value. However, since the investor is fully hedged against risk, the investor should only be able to receive the riskless return in an informationally efficient market.


\end{document}