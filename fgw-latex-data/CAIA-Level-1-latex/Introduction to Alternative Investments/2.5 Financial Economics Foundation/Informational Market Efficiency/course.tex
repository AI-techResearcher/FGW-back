\documentclass[11pt]{article}
\usepackage[utf8]{inputenc}
\usepackage[T1]{fontenc}
\usepackage{amsmath}
\usepackage{amsfonts}
\usepackage{amssymb}
\usepackage[version=4]{mhchem}
\usepackage{stmaryrd}

\begin{document}
\section*{Reading}
Informational Market Efficiency

Financial economics serves as a vital foundation to asset pricing and the understanding of alternative investments. This session discusses informational market efficiency, asset pricing, forward contracts, and options.

The concept of informational market efficiency is especially important in the management of alternative investments. Informational market efficiency refers to the extent to which asset prices reflect available information. An informationally efficient market is a market in which assets are traded at prices that equal their values based on all available information. The concept of informational market efficiency is sometimes referred to as efficient market theory or the efficient market hypothesis.

In practice, all financial markets display at least some informational market efficiency, but no financial market is perfectly efficient. For example, many trades of large equities on the U.S. stock exchanges occur at one-cent intervals, implicitly indicating at least some degree of mispricing. It is more useful to describe markets as displaying varying degrees of informational market efficiency rather than attempting to divide markets into those that are and those that are not informationally efficient.

\section*{Further Definitions of Informational Market Efficiency}
Definitions of informational market efficiency often extend beyond the terse definition that "prices reflect available information." For example, informationally efficient markets are sometimes described as markets in which the net present values (NPVs) of all investment decisions are zero.

Further, informationally efficient markets are often described as markets in which investors are unable to use information to consistently earn superior risk-adjusted returns. Note that investors bearing higher risk should tend to earn consistently higher returns, and that any investor bearing risk might occasionally earn high returns. However, in an informationally efficient market, investors cannot earn higher expected returns without bearing additional risk. Note that this is less a definition of an efficient market than it is an implication of an efficient market.

An informationally efficient market is often described as a market in which prices follow a "random walk." However, markets can be informationally efficient without following a random walk, and in theory, the opposite is also true: Prices in a totally irrational market could follow a random walk while being informationally inefficient.

It may be helpful to note that the term efficient is used in several distinct ways in investments. For example, an efficient portfolio typically denotes a portfolio that offers an unsurpassed combination of risk and return, and an economy that allocates its resources very well is said to be efficient. Accordingly, it is probably useful to specify informational market efficiency when the term is being used to denote the extent to which market prices reflect available information.

\section*{Forms or Levels of Informational Market Efficiency}
Informational market efficiency is often discussed in the context of forms or levels that are related to information sets. First, weak form informational market efficiency (or weak level) refers to market prices reflecting available data on past prices and volumes (i.e., historical trading data). Weak form efficiency addresses the issue of whether technical analysis can be useful in earning consistent and superior risk-adjusted returns.

The concept of semistrong form informational market efficiency (or semistrong level) refers to market prices reflecting all publicly available information (including not only past prices and volumes but also any publicly available information such as financial statements and other underlying economic data). Semistrong form efficiency is designed to address the issue of whether technical analysis and, especially, fundamental analysis can be useful in earning consistent and superior riskadjusted returns.

Finally, the concept of strong form informational market efficiency (or strong level) refers to market prices reflecting all publicly and privately available information. Strong form efficiency is designed to address the issue of whether any attempts to earn consistent and superior risk-adjusted returns can be successful, including insider trading.

Two clarifying details can be helpful. First, note that the three forms do not alter the general meaning of informational efficiency. The only thing that changes between the three levels or forms is the information set. Second, note that the information sets moving from weak form to semistrong form to strong form are cumulative. If weak form efficiency is violated, then all three forms will be violated, because the semistrong and strong forms include the information set used in the weak form. However, violation of strong form efficiency does not imply violation of the weak or semistrong forms.

The purpose of these three forms is to simplify and structure discussions of informationally efficient markets. Although the information sets are in fact cumulative, the three levels are often casually linked directly to the three major trading strategies: technical analysis to the weak form, fundamental analysis to the semistrong form, and insider trading to the strong form. The strong form is often criticized as being superfluous because it would seem to appear almost by definition that market prices cannot reflect information that is not publicly available. But an argument can be made that if insider trading generates consistent abnormal profits to insiders, then outsiders would perceive their informational disadvantage and would refuse any trading strategy other than an indexed buy-and-hold strategy. Put differently, strong form efficiency may not be a stable outcome, because if insiders consistently engage in NPV $>0$ trading, it means that others irrationally persist in engaging consistently in NPV $<0$ trading.

\section*{Informational Market Efficiency and "Efficient Inefficiency"}
Informational market efficiency is the state in which available information regarding an asset is quickly reflected in the market price of that asset. For example, when does the market price of an equity, such as Tesla Inc., reflect the value of a new technology developed by the firm? Does the stock price rise when the idea for the technology is created, when the idea is made public, when the firm announces an investment to deploy the technology, when the technology is proved reliable, or when the firm begins receiving cash flows from sales based on the technology? In an informationally efficient market, the answer is that the stock price reflects all potential cash flows (with their attendant probabilities), the moment the information regarding those cash flows is revealed to the marketplace. In such a market, no investor is able to consistently earn superior risk-adjusted returns based on available information, because the information is instantaneously reflected in market prices when it becomes publicly available.

A clear understanding of the implications of informational market efficiency is vital to being an effective overseer of assets. Informational market efficiency is a theoretical idea rather than a precise description of actual markets. No asset market is perfectly efficient. Actual markets should be viewed as exhibiting different degrees of market efficiency. But perfect informational market efficiency is an important ideal. By way of analogy, consider a plumb line. A plumb line is a vertical line generally approximated using a suspended string with a weight attached at the bottom. A plumb line can be an important method of ensuring that a building's framework is well constructed. In practice, however, no building has perfect beams or walls. Similarly, the concept of perfect informational market efficiency creates a reference point against which market inefficiencies can be identified and the convergence of prices to the theoretically correct price can be forecast. In other words, perfect market efficiency is how financial analysts predict how prices should behave-allowing traders to identify mispriced assets and estimate their expected return and risk. Skill-based traders base their trades on perceived departures of actual asset prices from their informationally efficient prices.

How do empirics and economic reasoning inform asset allocators and their overseers about the extent to which various asset markets are informationally efficient? Markets tend to be more informationally efficient to the extent that they (1) are being traded by large numbers of well-informed and financially sound traders competing for profits, (2) contain securities for which substantial amounts of reliable information are made broadly and quickly available, and (3) are subject to minimal transactions costs, taxes, and other impediments to trade. Large markets in modern economies with institutions that support free trade tend to exhibit high degrees of informational market efficiency.

The proposition that markets are perfectly efficient, however, is inconsistent with rational investing. If markets were perfectly efficient, no trader could earn a superior profit by performing analyses using available information. Traders performing analysis would be wasting their valuable time processing information and would lose wealth relative to buy-and-hold investors because of higher trading levels and increased transaction costs. In the long run, it is only the existence of market inefficiencies that incentivizes analysts to use available information to drive markets toward efficiency.

Markets become more efficient through the efforts of speculators and other traders to identify mispriced assets and then to buy those perceived as being underpriced and to sell those perceived as being overpriced. The best traders are successful; they gain wealth, and they exert increasing influence on the pricing of assets.

The enigma as to how markets can become efficient when efficiency destroys the incentives to process information has led to the proposition that markets tend toward being efficiently inefficient. ${ }^{1}$ See, for example, Lasse Heje Pedersen, Efficiently Inefficient (Princeton, NJ: Princeton University Press, 2015). The idea is that each market tends toward its own equilibrium degree of informational inefficiency, where that amount of inefficiency balances the marginal costs of additional skillbased trading with the marginal revenues from the skill-based trades.

Empirical studies of market efficiency reveal varying degrees of it in different markets and tend to indicate that opportunities to exploit particular inefficiencies decay through time as each successful trading strategy attracts additional capital. The empirical and theoretical evidence together suggest that skill-based trading strategies are more likely to be successful when (1) executed by the most skilled traders in any market and (2) executed in relatively new markets or with relatively new securities that have less competition among skill-based traders.

\section*{Six Factors Driving Informational Market Efficiency}
The overall driver of informational market efficiency is greater competition among informed buyers and sellers. Thus, markets tend to attain higher degrees of informational market efficiency when there are more traders using all available information, and when those traders can transact with low costs. Informed traders will search to buy underpriced assets and sell (or short sell) overpriced assets, driving assets with similar risk toward offering equal expected returns (ignoring tax treatment differentials and other imperfections).

But what underlying factors cause the competition or analysis that drives prices toward informationally efficient levels? Let's look at six major factors. The first four factors serve to facilitate competition and to enhance liquidity; the last two factors facilitate better analysis.

\begin{enumerate}
  \item The greater the value of the assets being traded, the greater the competition for potential profits and losses from mispricing, within limits. Higher profit potential motivates market participants to use more information and better analysis. Everything else equal, a $\$ 100$ trade mispriced by $1 \%$ transfers only $\$ 1$ of wealth between traders, whereas a $\$ 1,000,000$ trade mispriced by only $0.1 \%$ transfers $\$ 1,000$ of wealth. However, very large asset values, such as huge equity deals, may reduce competition if there are relatively few traders who have the resources to acquire the assets.

  \item Greater trading frequency for the assets increases competition by providing greater incentives for investors, speculators, and arbitrageurs to analyze information and attempt to make favorable trades. Securities that are traded very infrequently typically have large bid-ask spreads due to the reduced profit potential for traders to benefit from mispricing.

  \item Low levels of trading frictions facilitate higher competition by encouraging arbitrage and speculation with the lowering of total trading costs. Reduced trading frictions include lower transaction costs, such as brokerage fees, exchange fees, regulatory fees, and taxes.

  \item Fewer regulatory constraints on trading also tend to lead to improved informational market efficiency by expanding competition and trading. Examples of regulatory constraints that may inhibit competition include restrictions on short selling and leverage.

  \item Assets will also tend to trade at prices closer to their informationally efficient values when there is easier access to better information, as better information facilitates better financial analysis. In the United States, the Securities and Exchange Commission has as one of its primary goals requiring public companies to disclose meaningful information to the public.

  \item Assets will also tend to trade at prices closer to their informationally efficient values when there is less uncertainty about their valuation. In other words, better valuation methods lead to better analysis. For example, the development of sound option pricing models in the 1970s led to improved informational market efficiency in options markets.

\end{enumerate}

\section*{Factors Influencing Informational Efficiency in Alternative Asset Markets}
As introduced in the session entitled What Is an Alternative Investment?, alternative assets differ substantially from traditional assets. Many of these differences can cause the informational efficiency of alternative asset markets to differ from the informational efficiency of traditional markets.

Let's take a look at how these differences relate to the six factors that drive market efficiency, discussed in the previous section. Both traditional and alternative asset markets are quite diverse with regard to the first four factors. In other words, there are large, heavily traded markets, and small, thinly traded markets, in both traditional and alternative asset markets.

But it is primarily with regard to the fifth and sixth factors that many alternative markets possess features that lend themselves to less efficient pricing: substantial nonpublic information and substantial uncertainty with regard to valuation methods. The practices and tools for investing in traditional assets tend to be better developed and more widely accepted. Market participants tend to better understand the relationship between traditional asset values (such as bond prices) and information (such as expected inflation rates) than the relationship between alternative asset values (such as intellectual property values) and information (such as technological innovations).

The complex trading strategies inherent in some alternative investments rely on the discovery and exploitation of market inefficiencies in order to be successful. Hedge funds, discussed in detail in the sessions, Structure of the Hedge Fund Industry through Funds of Hedge Funds, tend to implement highly sophisticated trading strategies with frequent use of short positions, leverage, and high turnover. These strategies require the exceptional skills that are possessed only by top managers. The relatively low number of traders with the skills, models, data, and other resources needed to compete in the hedge fund arena increases the potential for the persistence of inefficient pricing. In contrast, long-only trading in traditional assets is accessible to numerous traders.

Private equity is another alternative investment that is accessible to a relatively limited number of traders and that requires highly specialized tools. Fewer investors are in the financial position to accommodate the specialized analytical tools, high minimum investments, and illiquidity of many private equity opportunities, meaning that the number of competitors may be limited. Thus, markets for traditional investments, such as publicly traded equity markets, may be more informationally efficient than markets for alternative assets, such as private equity.

An understanding of market informational efficiency, and especially the degree to which various markets may or may not be informationally efficient, is a vital tool in the practice of alternative investing.


\end{document}