\documentclass[11pt]{article}
\usepackage[utf8]{inputenc}
\usepackage[T1]{fontenc}
\usepackage{amsmath}
\usepackage{amsfonts}
\usepackage{amssymb}
\usepackage[version=4]{mhchem}
\usepackage{stmaryrd}

\begin{document}
\section*{APPLICATION A}
Using the CAPM equation, when the risk-free rate is $2 \%$, the expected return of the market is $10 \%$, and the beta of asset $\mathrm{i}$ is 1.25 , what is the expected return of asset i?

\section*{Answer and Explanation}
Apply Equation 1 to solve the CAPM equation for the expected return of asset i. Subtract $10 \%$ (expected return of the market) by $2 \%$ for a difference of $8 \%$. Multiply $8 \%$ by 1.25 (beta of the asset) for a product of $10 \%$. Lastly, add the risk- free rate of $2 \%$ to $10 \%$ for a sum of $12 \%$, which is the expected return of asset $\mathrm{i}$.

\section*{APPLICATION B}
Returning to the previous example in which the risk-free rate is $2 \%$ and the beta of asset $\mathrm{i}$ is 1.25 , if the actual return of the market is $22 \%$, the ex post CAPM model would generate a return due to non-idiosyncratic effects of $27 \%$ for the asset: $2 \%+[1.25(22 \%-2 \%)]$.

\section*{Answer and Explanation}
This is a similar to Application A, but instead of the using expected return, actual returns are used in the model. We need to apply a slightly modified Equation 3 as shown below:

$$
R_{i t}=R_{f}+\beta_{i}\left(R_{m t}-R_{f}\right)+\varepsilon_{i t}
$$

Subtract $22 \%$ (actual return of the market) from $2 \%$ (the risk-free rate) for a difference of $20 \%$. Multiply $20 \%$ by 1.25 for a product of $25 \%$. Then add $2 \%$ to $25 \%$ for a sum of $27 \%$, which is the assets return due to non-idiosyncratic effects.

If the asset actually returned $30 \%$ (i.e. $R_{i t}=30 \%$ ), then 3\% difference between $27 \%$ (assets return due to non-idiosyncratic effects) and $30 \%$ (actual asset return), would be attributable to idiosyncratic return, $\varepsilon_{i t}$ Thus, $\varepsilon_{i t}=3 \%$.


\end{document}