\documentclass[11pt]{article}
\usepackage[utf8]{inputenc}
\usepackage[T1]{fontenc}
\usepackage{amsmath}
\usepackage{amsfonts}
\usepackage{amssymb}
\usepackage[version=4]{mhchem}
\usepackage{stmaryrd}

\begin{document}
\section*{Reading}
Single-Factor Equity Pricing Models

An asset pricing model is a framework for specifying the return or value of an asset based on its risk, as well as future cash flows. Although asset pricing models include the term pricing in their name, they are focused on the returns on assets rather than their prices. Also, the term is usually used to describe the returns of equities rather than assets such as bonds.

This section reviews single-factor asset pricing and discusses the distinction between ex ante asset pricing and ex post asset pricing. Asset pricing models are not simply mathematical exercises; they are ways of expressing the most fundamental issues related to investing: the nature of the risks and returns of investment opportunities.

\section*{Single-Factor Asset Pricing}
The central theme of asset pricing involves return, systematic risk, and diversification. The capital asset pricing model (CAPM) provides one of the easiest and most widely understood examples of single-factor asset pricing by demonstrating that the risk of the overall market index is the only risk that offers a risk premium. The CAPM is a general equilibrium model, meaning that it prices all assets rather than simply describing one or more relative pricing relationships.

\section*{FOUNDATION CHECK}
This section assumes basic familiarity with the capital asset pricing model, including its underlying assumptions, the intuition of the model, the division of systematic and diversifiable risk, the interpretation of beta, and the estimation of beta.

Equation 1 provides the most common representation of the CAPM:


\begin{equation*}
E\left(R_{i}\right)=R_{f}+\beta_{i}\left[E\left(R_{m}\right)-R_{f}\right] \tag{1}
\end{equation*}


where $E\left(R_{i}\right)$ is the expected return on asset $i, \beta_{i}$ is the market beta of asset $i,\left(R_{m}\right)$ is the expected return on the market portfolio, and $R_{f}$ is the riskless rate of return.

The CAPM is frequently and correctly criticized for failing to explain and predict financial returns accurately. Nevertheless, this section discusses the CAPM as a foundation for developing more complex models and the concepts crucial to the analysis of alternative investments.

Equation 1 indicates that the expected return of any asset (the left side of the equation) has two parts: a risk-free rate to compensate the investor for the time value of money $\left(R_{f}\right)$ and a risk premium to compensate the investor for bearing the risk. The asset's risk premium, $\beta_{i}\left[E\left(R_{m}\right)-R_{f}\right]$, is the product of the asset's risk, or beta, and the market risk premium, meaning the amount investors demand for bearing each unit of risk. The market return is the return of the market portfolio. The market portfolio is a hypothetical portfolio containing all tradable assets in the world (except riskless financial assets). Each asset in the market portfolio is held in a quantity based on its market weight. The market weight of an asset is the proportion of the total value of that asset to the total value of all assets in the market portfolio. Thus, if the combined market value of all shares of XYZ Corporation is $\$ 250$ billion, and if the combined market value of all investable assets in the world is $\$ 250$ trillion, then the market weight of XYZ's equity would be $0.10 \%$.

The CAPM is an example of a single-factor asset pricing model. A single-factor asset pricing model explains returns and systematic risk using a single risk factor. Whereas the CAPM describes the entire economy, other single-factor models may simply describe relative prices and returns among a subset of the economy. For instance, consider an analyst modeling the returns of a group of REITs (real estate investment trusts) that have somewhat similar underlying assets. Equation 2 represents a REIT-based single-factor asset pricing model that differs in important ways from the CAPM:


\begin{equation*}
E\left(R_{i}\right)=a_{i}+\beta_{i}\left[E\left(R_{\text {index }}\right)\right] \tag{2}
\end{equation*}


where $E\left(R_{i}\right)$ is the expected return on $R E I T_{i}, a_{i}$ is a constant, $\beta_{i}$ is the beta of $R E I T_{i}$, and $E\left(R_{i n d e x}\right)$ is the expected return on an index of REITs.

Note that the model in Equation 2 does not specify that all assets must be included in the index, or that the constant is the riskless rate. The beta in Equation 2 describes the behavior of a REIT with respect to an index of REITs, which would clearly differ from the beta from the CAPM, which describes the behavior of an asset with respect to the market portfolio.

Thus, the CAPM is a specialized case of a single-factor asset pricing model. The CAPM is the very important case that describes an economy in which all investors diversify perfectly among all assets and achieve an equilibrium in which all investors allocate their assets between two portfolios: the market portfolio and the riskless portfolio.

Within the context of single-factor asset pricing models such as the CAPM, the next two sections discuss the distinction between ex ante asset pricing and ex post asset pricing.

\section*{Ex Ante Asset Pricing}
Equation 1 is primarily a cross-sectional representation of the CAPM that focuses on the expected returns of asset $i$ rather than the realized returns of asset $i$ subscripted for time ( $t$ ). Equation 1 is the expectational (i.e., ex ante) form of the CAPM. Ex ante models, such as ex ante asset pricing models, explain expected relationships, such as expected returns. Ex ante means "from before." Ex ante models provide an understanding of how return expectations or requirements are formed.

The expected return expresses the central tendency of asset is return. The actual return of asset $i$ in a particular time period may differ from the expected return either because the market earned more or less than expected or because asset $i$ experienced an unexpected and idiosyncratic change in price.

The ex ante form of the CAPM makes two powerful prescriptions that are especially relevant to an analysis of alternative investments. The first is the assertion that any and all rewards for bearing risk should only be available from bearing market risk, which can be fully measured by an asset's beta relative to the market portfolio. The second assertion is that investors should not be able to earn any additional expected return from bearing any other type of risk. The first assertion is driven by the single-factor nature of the CAPM, and the second assertion is common to equilibrium asset pricing models. In an equilibrium asset pricing model, participants do not seek to change their positions to exploit perceived pricing errors, because there are no discernible pricing errors based on available information, meaning there are no arbitrage opportunities.

The implications of the ex ante form of the CAPM are vast. If the CAPM were true, then every investor would hold all risky assets in proportion to their size. Risk-averse investors would hold a greater portion of their portfolio in risk-free assets, and risk-tolerant investors would hold a greater portion of their wealth in the risky market portfolio. Although individual investors might allocate different total amounts to the market portfolio, every investor would be exposed to exactly the same risk factor: the risk that the market portfolio will change in price. In the idealized world of the CAPM, no investor tries to beat the market by overweighting or underweighting any risky assets or by trying to time the market (i.e., trying to buy and sell assets immediately before favorable price changes).

The importance of alternative investments as a distinct category of investing must therefore emanate from the insufficiency of the CAPM to describe financial markets. This is because if the CAPM were true, all investors would hold the same portfolio of risky assets, and no further analysis or management would be required. To motivate a nontrivial approach to alternative investment management, we must relax some of the assumptions on which the CAPM is based. In other words, for there to be a need to analyze alternative investments, the CAPM must be an insufficient description of asset pricing. Alternative investment analysis must focus on assets for which prices are not well described by the CAPM and must implicitly or explicitly use models that differ from the CAPM.

The CAPM is derived from the assumption that many of the real-world features that are linked to alternative investments do not exist. The CAPM is typically derived assuming that no single trader can affect security prices, that all investors can focus exclusively on the market value of their wealth at the end of the same single period, that all assets are publicly traded, that all investors can short sell limitlessly, that all investors can borrow limitlessly at the risk-free rate, that there are no taxes or transaction costs, that all investors care only about the mean and variance of an asset's return distribution, and, in most cases, that all investors have equal expectations about security returns.

A foundation for alternative investment analysis must begin with ideas of how assets are priced when the CAPM's assumptions do not hold. In other words, what risks other than the CAPM beta might be compensated? If different risks are rewarded, are they rewarded with equally attractive risk premiums? If some securities are not publicly traded, how do their risks and returns compare and contrast with the risks and returns of publicly traded securities? If superior knowledge or skill can enhance expected returns, how would assets be priced? Understanding these important questions is critically linked to understanding asset pricing, the distinctions between ex ante and ex post pricing models, and the analysis of alternative investments.

\section*{Ex Post Asset Pricing}
The previous description of the CAPM focused on the expected return of an asset. Expected returns were shown to depend on a common or systematic factor. Systematic return is the portion of an asset's return driven by a common association. Systematic risk is the dispersion in economic outcomes caused by variation in systematic return. Idiosyncratic return is the portion of an asset's return that is unique to an investment and not driven by a common association. Idiosyncratic risk is the dispersion in economic outcomes caused by investment-specific effects. This section focuses on realized returns and the modeling of risk.

Actual returns deviate from expected returns due to unexpected effects. The unexpected portions of returns result from systematic and idiosyncratic risks. Systematic effects occur when a systematic risk factor is higher or lower than expected. Idiosyncratic effects are all effects that are not systematic. The ex ante form of the CAPM does not include an added expected return from idiosyncratic effects, since the expected value (i.e., expected return) of all idiosyncratic effects must be zero. This is because idiosyncratic risk is diversifiable. If idiosyncratic risk bearing offered a risk premium, then investors could receive a higher expected return (rather than simply a lower total risk) from simply holding a diversified portfolio-which would represent an unsustainable arbitrage opportunity.

This section discusses an ex post (meaning "from afterward" or realized) form of the CAPM. An ex post model describes realized returns and provides an understanding of risk and how it relates to the deviations of realized returns from expected returns.

The realized return of an asset differs from its expected return due to systematic and idiosyncratic effects, which are illustrated as the right side of the following equation:


\begin{equation*}
R_{i t}-R_{f}=\beta_{i}\left(R_{m t}-R_{f}\right)+\varepsilon_{i t} \tag{3}
\end{equation*}


The left side of the equation is the realized excess return of asset $i$ in time period $t$. The excess return of an asset refers to the excess of deficiency of the assets relative to the periodic risk-free rate. The terms between the equal sign and the plus sign reflect the effect of the market's realized return in time period $t$, or the effect of systematic risk on the realized return of asset $i$ in time period $t$. To the extent that the realized return of the market differs from its expected return, an asset with a nonzero beta realizes a return that differs from its expected return proportional to its beta. Finally, $\varepsilon_{i t}$, the term to the far right, is the portion of the excess return that is due to the effect of idiosyncratic risk. Idiosyncratic returns include any effect on the return of asset $i$ in time period $t$ other than that which is correlated with the return of the market, such as the impact of firm-specific news. Taking the expected value of each side of the ex post CAPM equation and rearranging the terms returns the equation to the ex ante form of the CAPM (Equation 1).

Two essential attributes of the ex post CAPM are that (1) the return from idiosyncratic risk, $\varepsilon_{i t}$, has an expected value of zero (otherwise, it would appear in the ex ante form of the CAPM), and (2) the return from idiosyncratic risk is not linearly correlated with the return of the market, because any such effects are captured through the beta of the asset. In this case, the asset pricing model is being used with its true idiosyncratic return component, $\varepsilon_{i t}$, not an estimate, such as the residuals from a regression equation. Linear regression residuals are, by definition, uncorrelated with the regression's independent variables.

Equation 3, the ex post CAPM equation, can be viewed as both a cross-sectional and a time-series model, since one or more variables on each side are subscripted both by time $(t)$ and by subject ( $I$ ). Thus, the model might be used to describe the time-series properties for a single stock or might be used across many firms during a single time period in a cross-sectional study.

Equation 4 provides insight into risk. It is formed by taking the variance of both sides of Equation 3, assuming that $R_{f}$ and $\beta_{i}$ are constant and that the correlation between $R_{m t}$ and $\varepsilon_{i t}$ is zero:


\begin{equation*}
\sigma_{i}^{2}=\beta_{i}^{2} \sigma_{m}^{2}+\sigma_{\varepsilon}^{2} \tag{4}
\end{equation*}


where $\sigma_{i}^{2}$ is the variance of the returns of asset $i, \sigma_{m}^{2}$ is the variance of the returns of the market index, and $\sigma_{\varepsilon}^{2}$ is the variance of the idiosyncratic returns of asset $i$. The left side of Equation 4 is the total risk of asset $i$. The term $\beta_{i}^{2} \sigma_{m}^{2}$ is the portion of the total risk that is attributable to the asset's systematic risk, and $\sigma_{\varepsilon}^{2}$ is the portion attributable to the idiosyncratic risk. The idiosyncratic risk vanishes when enough assets are added to a portfolio.

Although the ex post returns in Equation $4\left(R_{i t}, R_{f}\right.$, and $R_{m t}$ ) can be observed, the beta of the investment, $\beta_{i}$, is never observed, and therefore $\varepsilon_{i t}$ can only be estimated. When empirical tests of Equations 3 and 4 are performed, the measured idiosyncratic return in Equation 3 and the variance of the idiosyncratic risk in Equation 4 contain estimation errors to the extent that the estimated beta differs from the true beta of the investment.

This session discusses several asset pricing issues in the context of the CAPM because the CAPM provides a relatively simple representation of the concept of systematic and idiosyncratic risk and return. But the CAPM is generally faulted for its inability to describe the real world accurately, especially its inability to describe the behavior of alternative investments. Alternative investment analysis often focuses on the potential for multiple sources of systematic risk and on the potential to invest such that the expected idiosyncratic return, $E\left(\varepsilon_{i t}\right)$, is positive.


\end{document}