\documentclass[11pt]{article}
\usepackage[utf8]{inputenc}
\usepackage[T1]{fontenc}

\begin{document}
\section*{Reading}
The Three Primary Theories of the Term Structure of Interest Rates

There are three primary theories (i.e., hypotheses) for the shape of the term structure of spot interest rates (default-free). The theories propose explanations of the relationship between interest rates corresponding to different longevities. These theories are vital tools that cut to the heart of fixed-income investment management. The following sections refer to each explanation of the term structure as a "theory," but each is also well known as a "hypothesis."

\section*{The Unbiased Expectations or Pure Expectations Theory}
In a risk-neutral world, risk premiums are not required by market participants to bear interest rate risks. In that world, the unbiased expectations theory (i.e., the pure expectations theory) would hold. The unbiased expectations theory hypothesizes that all fixed-income securities offer the same expected return over the same time interval (i.e., there are no risk premiums), therefore serving as a useful tool in risk-neutral modeling in which all interest rates are formed purely on interest rate expectations. Put differently, under the unbiased expectations theory the expected value of every fixed-income security is expected to grow through time at the same rate over the same time interval and the shape of the term structure is driven purely by interest rate expectations (as opposed to being partly driven by risk aversion). Under the unbiased expectations theory the term structure should not be consistently upward sloping-which is inconsistent with historical observations.

\section*{The Liquidity Preference or Liquidity Premium Theory}
Longer-term bonds tend to experience greater price volatility than shorter-term bonds, leading to the hypothesis that longer-term bonds are riskier and therefore require higher expected rates of return (i.e., higher risk premiums). Given risk aversion in a well-functioning market, the liquidity preference theory (i.e., the liquidity premium theory) would hold. The liquidity preference theory hypothesizes that longer-term fixed-income securities offer higher expected returns over the same time interval as shorter-term bonds, that risk premiums are positive and increasing in the bond's longevity, that all interest rates are formed based on both interest rate expectations and risk premiums, and that fixed-income management reflects a trade-off between risk and return. The liquidity preference theory hypothesizes that the expected return on zero-coupon fixed-income securities is an increasing function of the security's maturity, and that the shape of the term structure is formed as the sum of the term structure that would exist in a risk-neutral world (i.e., under the unbiased expectations theory) and the risk premiums associated with each maturity. The consistent upward slope to the term structure is consistent with the liquidity preference theory.

\section*{The Market Segmentation or Preferred Habitat Theory}
Recall that the liquidity preference theory hypothesizes that fixed-income securities offer monotonically increasing expected returns (higher risk premiums) to securities with longer maturities. The market segmentation theory (i.e., preferred habitat theory) is based on an assumption that there may be localized imbalances in the supply and demand for bonds with different longevities. Specifically, some investors, such as pensions and insurance companies, may prefer or better tolerate the risks of longer-term bonds while others may prefer holding short-term bonds. Similarly, borrowers have their preferred longevities for obtaining funding.

The market segmentation theory hypothesizes that the preferred habitats of borrowers and lenders influence the expected returns of each maturity range, resulting in varying risk premiums and varying expected returns across maturity ranges that form humps and other non-monotonic shapes that are not eliminated by arbitrageurs (because the market is segmented). Economic theorists argue that the activities of speculators willing to form hedges that are long bonds within relatively underpriced maturity ranges and short bonds within relatively overpriced maturity ranges should minimize or eliminate expected return differentials based on habitat preferences.

\section*{Managerial Implications of the Three Term-Structure Theories}
The three theories of the term structure prescribe different fixed-income strategies. The unbiased expectations hypothesis implies that borrowing and lending decisions should focus on issues of convenience such as cash-flow matching because all longevity-related choices offer equal expected returns to lenders and costs to borrowers. The liquidity premium theory asserts that lenders should seek longer longevities until the marginal aversion to risk offsets the higher expected returns, while borrowers should seek shorter maturities until risks associated with cash flow mismatches (funding risks) offset the lower expected costs. Finally, the market segmentation hypothesis introduces the complexities that different longevities offer expected returns to lenders and costs to borrowers that are driven by supply and demand factors that differ across maturity ranges and that may vary substantially through time.


\end{document}