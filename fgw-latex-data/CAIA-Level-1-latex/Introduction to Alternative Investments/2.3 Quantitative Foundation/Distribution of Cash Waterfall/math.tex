\documentclass[11pt]{article}
\usepackage[utf8]{inputenc}
\usepackage[T1]{fontenc}
\usepackage{amsmath}
\usepackage{amsfonts}
\usepackage{amssymb}
\usepackage[version=4]{mhchem}
\usepackage{stmaryrd}

\begin{document}
\section*{APPLICATION A}
Fund $A$ at the end of its term has risen to a total net asset value (NAV) of $\$ 300$ million from its initial size of $\$ 200$ million. Assuming no hurdle rate and an $80 \% / 20 \%$ carried-interest split, the general partner is entitled to receive carried interest equal to how much? The answer is $\$ 20$ million.

\section*{EXPLANATION}
Total profit (ending NAV minus Initial NAV), no hurdle rate, and the carried interest split are needed to determine how much the general partner is entitled to receive. In this case, $\$ 300$ million minus $\$ 200$ million equals $\$ 100$ million. $\$ 100$ million in profit multiplied by $20 \%$, which is the percentage of carried interest the general partner is entitled to, equals $\$ 20$ million.

\section*{APPLICATION B}
Fund B terminates and ultimately returns $\$ 132$ million to its limited partners, and the total initial size of the fund was $\$ 100$ million. Assuming a carried interest rate of $20 \%$, the general partner is entitled to receive carried interest equal to how much? The answer is $\$ 8$ million. Note that if $\$ 32$ million is the profit only to the LP, the total profit of the fund was higher.

\section*{EXPLANATION}
The answer is found by solving the following equations: LP profit $=0.8 \times$ total profit; so $\$ 32$ million $=0.8 \times$ total profit; therefore, total profit $=\$ 40$ million. The second equation is GP carried interest $=0.2 \times$ total profit; therefore, carried interest $=\$ 8$ million.\\
We are given only the net profit to the limited partners, $\$ 32$ million (i.e. $\$ 132$ million minus $\$ 100$ million). To determine the carried interest that the general partner is entitled to, the total profit is needed. Therefore, we divide $\$ 32$ million by $80 \%$, which is the carried interest percentage that the limited partners are entitled to. The quotient is $\$ 40$ million which indicates the total profit. Multiplying the total profit by the $20 \%$, which is the carried interest percentage that the general partners are entitled to, equals $\$ 8$ million in carried interest to the general partners. Again, the annual management fee is set to zero in this problem for simplicity.

\section*{APPLICATION C}
Consider a fund that makes two investments, A and B, of $\$ 10$ million each. Investment A is successful and generates a $\$ 10$ million profit, whereas Investment B is a complete write-off (a total loss). Assume that the fund managers are allowed to take $20 \%$ of profits as carried interest. How much carried interest will they receive if profits are calculated on a fund-as-a-whole (aggregated) basis, and how much will they receive if profits are calculated on a deal-by-deal (individual transaction) basis?

\section*{EXPLANATION}
On a fund-as-a-whole basis, Investment A generated a profit of $\$ 10$ million, while Investment B lost $\$ 10$ million. Therefore, the net of the two investments in the whole fund is $\$ 0$. Therefore, there is no carried interest to distribute to either the limited partners or general partners.

On a deal-by-deal basis, Investment A generated a profit of $\$ 10$ million and the general partners are entitled to their proportion of the carried interest, which is $20 \%$. Therefore, $\$ 10$ million multiplied by $20 \%$ for a product of $\$ 2$ million. Investment B does not generate a profit; in fact it loses $\$ 10$ million. Thus, there is no carried interest and limited partners and general partners do not receive a distribution.

\section*{APPLICATION D}
Consider a fund that calculates incentive fees on a fund- as-a-whole basis and makes two investments, $A$ and $B$, of $\$ 10$ million each. Investment $A$ is successful and generates a $\$ 10$ million profit after three years. Investment B is not revalued until it is completely written off after five years. Assume that the fund managers are allowed to take $20 \%$ of profits as carried interest calculated on an aggregated basis. How much carried interest will they receive if there is no clawback provision, and how much will they receive if there is a clawback provision?

\section*{EXPLANATION}
With no clawback provision, Investment A generates $\$ 10$ million of profit after three years. Therefore, the fund managers can take profits after three years when Investment A generated the $\$ 10$ million profit. Fund managers are entitled to $\$ 10$ million multiplied by $20 \%$, the carried interest percentage for general partners, for a product of $\$ 2$ million. Now, 2 years later when Investment B fails to generate a profit the fund managers do not receive any more carried interest and do not have to give back the prior $\$ 2$ million distributed in carried interest from Investment $A$. This is the result of the fund not having a clawback provision.

With a clawback provision, the scenario begins similarly. Investment A generates $\$ 10$ million of profit after three years. Therefore, the fund managers can take profits after three years when Investment A generated the $\$ 10$ million profit. Fund managers are entitled to $\$ 10$ million multiplied by $20 \%$, the carried interest percentage for general partners, for a product of $\$ 2$ million. However, when Investment B fails the $\$ 2$ million distributed to general partners is returned to the limited partners (in theory!) because Investment A profit plus Investment B profit equals zero (i.e. there is no combined profit). If there was a combined profit, then the general partners would be entitled to $20 \%$ of that combined profit, which may require either an additional distribution to the general partners or a clawback (i.e. if the combined profit was below $\$ 10$ million.)

\section*{APPLICATION E}
Consider a $\$ 10$ million fund with $20 \%$ incentive fees that lasts a single year and earns a $\$ 2$ million profit. Ignoring a hurdle rate, the fund manager would receive $\$ 400,000$, which is $20 \%$ of $\$ 2$ million. But with a hard hurdle rate of $10 \%$, the fund manager receives the $20 \%$ incentive fees only on profits in excess of the $10 \%$ return, meaning $\$ 200,000$.

\section*{EXPLANATION}
This application describes a $\$ 10$ million fund that generated a $\$ 2$ million profit, with a $10 \%$ hurdle rate, where the general partners receive $20 \%$ incentive fees (also called carried interest or performance fees). Ignoring the $10 \%$ hurdle rate, the fund managers received $\$ 2$ million multiplied by $20 \%$ or $\$ 400,000$. With the $10 \%$ hurdle rate in place, the fund managers only receive the $20 \%$ incentive fee on profits in excess of a $10 \%$, that is in excess of $10 \%$ multiplied by $\$ 10$ million or $\$ 1$ million. With the hurdle rate the general partners receive $20 \%$ multiplied by $\$ 1$ million for an incentive fee of $\$ 200,000$. Keep in mind that if the fund had returned $\$ 1$ million or less, the general partners would not be entitled to any incentive fee.

\section*{APPLICATION F}
Fund A with an initial investment of $\$ 20$ million liquidates with $\$ 24$ million cash after one year. The hurdle rate is $15 \%$, and the incentive fee is $20 \%$. What is the distribution to the fund manager if the fund uses a hard hurdle? What is the distribution to the fund manager if the fund has a soft hurdle and a $50 \%$ catch-up rate?

\section*{Explanation}
Since the fund is liquidating this is not simply a distribution of profits. Thus the first $\$ 20$ million (i.e. the initial investment) is distributed to investors or limited partners. The additional $\$ 4$ million (\$24 million minus $\$ 20$ million) is the profit that is subject to the hurdle rate and incentive fees, if applicable.

With the hard hurdle rate in effect we need to apply the $15 \%$ hurdle rate to the $\$ 20$ million initial investment, $15 \%$ multiplied by $\$ 20$ million for a product of $\$ 3$ million. Therefore, limited partners are entitled to $\$ 3$ million before general managers can receive the incentive fee on the remainder. In this case, the fund generated a profit of $\$ 4$ million the fund managers will collect an incentive fee on the difference, $\$ 4$ million minus $\$ 3$ million or $\$ 1$ million. $\$ 1$ million multiplied by $20 \%$ (the fund managers' incentive fee) equals $\$ 200,000$. The limited partners receive the other $\$ 800,000$ or $\$ 1$ million multiplied by .8 as well as the other $\$ 3$ million in profit, for a total profit of $\$ 3.8$ million. At the end of liquidation, fund managers will receive $\$ 200,000$ and limited partners will receive $\$ 23.8$ million (i.e. \$20 million of initial investment and $\$ 3.8$ million in profit).

Let's consider this same application with a soft hurdle rate of $15 \%$ and a $50 \%$ catch up rate. This scenario begins similarly. We need to apply the $15 \%$ hurdle rate to the $\$ 20$ million initial investment, $15 \%$ multiplied by $\$ 20$ million for a product of $\$ 3$ million. Therefore, limited partners are entitled to $\$ 3$ million before general managers can receive the incentive fee on the remainder. The fund generated a profit of $\$ 4$ million the fund managers will collect fees on the difference, $\$ 4$ million minus $\$ 3$ million or $\$ 1$ million. In this scenario, the $50 \%$ catch up provision is utilized until the fund managers receive $20 \%$ of total profits. After the $20 \%$ of total profits is received using the catch up provision, the fund managers would collect $20 \%$. Therefore, the fund managers will collect $50 \%$ of the $\$ 1$ million left over \%. Therefore, the fund managers will collect 50\% of the \$1 million left over\\
after the hurdle rate is satisfied, or \$500,000. \$500,000 divided by \$4 million is 12.5\%, which means that the fund managers were unable to fully catch up. If the\\
fund has returned \$5 million, the fund managers would have been about to fully catch up (i.e. the catch-up provision would be satisfied). If the fund had earned\\
over \$5 million in profit the catch up provision would have been satisfied and the 20\% incentive fee would apply for fund managers.


\end{document}