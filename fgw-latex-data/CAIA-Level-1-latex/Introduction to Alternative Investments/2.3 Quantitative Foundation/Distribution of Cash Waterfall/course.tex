\documentclass[11pt]{article}
\usepackage[utf8]{inputenc}
\usepackage[T1]{fontenc}
\usepackage{amsmath}
\usepackage{amsfonts}
\usepackage{amssymb}
\usepackage[version=4]{mhchem}
\usepackage{stmaryrd}

\begin{document}
\section*{Reading}
Distribution of Cash Waterfall

Limited partnerships, including private equity funds and hedge funds, have provisions for the allocation of cash inflows between general partners (GPs) and limited partners (LPs). Provisions related to the distribution waterfall are often the most complex parts of the limited partnership agreement. The waterfall is a provision of the limited partnership agreement that specifies how distributions from a fund will be split and how the payouts will be prioritized. Specifically, the waterfall details what amount must be distributed to the LPs before the fund manager or GPs can take a share from the fund's profits.

One important reason LPs need to understand the distribution waterfall is because of its impact on managerial incentives and, consequently, on the behavioral drivers of the fund's performance. Familiarizing themselves with the design of the waterfall's terms and conditions is one of the few opportunities LPs have to anticipate and manage risk. The waterfall's design always produces effects (sometimes unintended ones) as it drives the motivation and attitude, sense of responsibility, accountability, and priorities of fund managers.

\section*{Cash Waterfalls}
The distribution of cash waterfalls has specialized terms, which tend to differ between private equity and hedge funds. This section introduces most of the major terms that are used in the remaining sections.

Cash inflows to a fund in excess of the costs of investment and the expenses of the fund represent the waterfall that is distributed to GPs and LPs. Excess revenue above expenses is referred to as net cash flow or profit. In calculating profit, management fees are deducted, but any fees that are based on profitability are not deducted. (Management fees are usually deducted from the fund, regardless of profitability.)

Carried interest is synonymous with an incentive fee or a performance-based fee and is the portion of the profit paid to the GPs as compensation for their services, above and beyond management fees. Carried interest is typically up to $20 \%$ of the profits of the fund and becomes payable once the LPs have achieved repayment of their original investment in the fund, plus any hurdle rate.

A hurdle rate specifies a return level that LPs must receive before GPs begin to receive incentive fees. When a fund has a hurdle rate, the first priority of cash profits is to distribute profits to the LPs until they have received a rate of return equal to the hurdle rate. Thus, the hurdle rate is the return threshold that a fund must return to the fund's investors, in addition to the repayment of their initial commitment, before the fund manager becomes entitled to incentive fees. The term preferred return is often used synonymously with hurdle rate-a return level that LPs must receive before GPs begin to receive incentive fees.

A catch-up provision permits the fund manager to receive a large share of profits once the hurdle rate of return has been achieved and passed. A catch-up provision gives the fund manager a chance to earn incentive fees on all profits, not just the profits in excess of the hurdle rate. A catch-up provision contains a catch-up rate, which is the percentage of the profits used to catch up the incentive fee once the hurdle is met. A full catch-up rate is $100 \%$. To be effective, the catch-up rate must exceed the rate of carried interest.

Vesting is the process of granting full ownership of conferred rights, such as incentive fees. Rights that have not yet been vested may not be sold or traded by the recipient and may be subject to forfeiture. Vesting is a driver of incentives. Vesting can be pro rata over the investment period, over the entire term of the fund, or somewhere in between, such as on an annual basis.

A clawback clause, clawback provision, or clawback option is designed to return incentive fees to LPs when early profits are followed by subsequent losses. A clawback provision requires the GP to return cash to the LPs to the extent that the GP has received more than the agreed profit split. A GP clawback option ensures that if a fund experiences strong performance early in its life and weaker performance at the end, the LPs get back any incentive fees until their capital contributions, expenses, and any preferred return promised in the partnership agreement have been paid.

\section*{The Compensation Scheme}
A key element of the managerial compensation structure is the nature of the incentives that align interests between fund managers and their investors. Investors and fund managers have an agency relationship in which investors are the principals and fund managers are their agents. The compensation scheme is the set of provisions and procedures governing management fees, general partner investment in the fund, carried-interest allocations, vesting, and distribution. As with all agency relationships, compensation schemes should be designed to align the interests of the principals (the LPs) and the agents (the GPs) to the extent that the alignment is cost-effective. It is generally cost-ineffective to try to maximize the alignment of GP and LP interests. For example, requiring huge investments into the partnership by general partners might initially appear to be an effective method of aligning LP and GP interests. However, GPs with a large proportion of their wealth invested in a single fund may manage the fund in an overly risk-averse manner.

The partnership agreement provisions, as well as other terms and conditions, such as investment limitations, transfers, withdrawals, indemnification, and the handling of conflicts of interest, tend to look quite similar across fund agreements. Surprisingly, fund terms have been relatively stable across the market cycles. The explanation for this phenomenon is that both fund managers and their investors have sufficient negotiation power to reject terms sought by the other side that differ substantially from terms widely used in the market, but not so much leverage as to move the market in one direction or the other.

Management fees are regular fees that are paid from the fund to the fund managers based on the size of the fund rather than on the profitability of the fund. The purpose of management fees is to cover the basic costs of running and administering the fund. These costs are mainly the salaries of investment managers and backoffice personnel; expenses related to the development of investments; travel and entertainment expenses; and office expenses, such as rent, furnishings, utilities, and supplies. During the investment period, management fees are calculated as a percentage of the size of the LP commitment, typically between $1 \%$ and $2.5 \%$, depending on fund size. Management fees may be charged at a lower rate or on the invested NAV of the fund after the investment period concludes or when a successor fund is formed. Although the calculation of the management fee is relatively simple and fairly objective, there are controversies surrounding the finer details.

General partners have an opportunity to earn a variety of fees from their portfolio companies. Phalippou (2017) lists a number of fees, including director fees, monitoring fees, transaction fees, restructuring fees, broken deal fees, banking fees, and advisory fees. ${ }^{1}$ Ludovic Phalippou, Private Equity Laid Bare (CreateSpace Independent Publishing Platform, 2017). Limited partners would prefer that these fees count as management fee offsets, where all fees earned by general partners would reduce the management fee owed to the GP by the LPs.

The general partners' investment in the fund is the amount of capital they contribute to the fund's pool of capital. GPs typically invest a significant amount of capital in their funds, usually at least $1 \%$ of total fund capital, which is treated the same way as the capital contributed by limited partners. There are a number of reasons for this. For example, the GPs contribute a meaningful amount of capital to ensure their status as a partner of the fund for income tax reasons. More important, however, is that they contribute substantial personal wealth to the fund to help align the interests of fund managers and their investors. For all of the calculation examples that follow, the GPs' own investment in the fund is not being considered, because it has the same payoff as that of the limited partners. In other words, in this volume, the computations of the amount of cash being distributed to GPs ignore their ownership interest, since that ownership interest receives cash in the same manner as the LPs. Limited partners prefer that the GPs contribute their investment in cash at the formation of the fund, as GPs earning their commitment through forgone management fees is not an as-effective alignment of interest.

\section*{Incentive-Based Fees}
Incentive-based or performance-based fees are a critical part of the compensation structure. Carried interest, as discussed earlier, is an incentive-based fee distributed from a fund to the fund's manager. The term carried interest tends to be used in private equity and real estate; the term incentive fee is more often used in hedge funds. Management fees are paid regardless of the fund's performance and therefore fail to provide a powerful incentive to produce exceptional investment results. Excessive and quasi-guaranteed management fees stimulate tentative and risk-averse behavior, such as following the herd. Consequently, the carried interest, meaning the percentage of the profit paid to fund managers, is the most powerful incentive to align interests and create value. The most common carriedinterest split is $80 \% / 20 \%$ (a.k.a. $80 / 20$ ), which gives the fund manager a $20 \%$ share in the fund's net profits and is essential to attracting talented and motivated managers. These fees are asymmetric, as a fund manager shares in the gains of the investors, but does not compensate investors for any portion of their losses. (Note that the following examples ignore management fees for the sake of simplicity.)

\section*{Aggregating Profits and Losses}
In the case of multiple projects within private equity funds, two approaches are used for determining profits and distributing incentive fees. Carried interest can be fund-as-a-whole carried interest, which is carried interest based on aggregated profits and losses across all the investments, or can be structured as deal-by-deal carried interest. Deal-by-deal carried interest is when incentive fees are awarded separately based on the performance of each individual investment.

Participating in every investment's profit, or deal-by-deal carried interest, can be problematic because the general partner can make profits on successful investments while having little exposure to unsuccessful transactions. As the limited partners take the bulk of the capital risk, this approach significantly weakens the alignment of interests. A fund-as-a-whole carried-interest approach protects the interests of the LPs but may be less effective in attracting talented managers. The fund-as-a-whole scheme may entail the risk of frustrating the fund managers, as their rewards may be deferred for years until all deals can be aggregated. Carriedinterest distribution is typically one of the most intensively negotiated topics. The amount of the payment is often not as much of an issue as the timing of the payment. In practice, carried interest schemes include elements of both approaches in order to circumvent their respective limitations.

\section*{Clawbacks and Alternating Profits and Losses}
Clawbacks are relevant to funds that calculate carried interest on a fund-as-a-whole basis. The idea of typical clawback provisions is that incentive fees distributed to managers are returned when a firm experiences losses after profits so that the total incentive fees paid, ignoring the time value of money, are equal to the incentive\\
fees that would be due if all profits and losses had occurred simultaneously. Funds experience early profits and late losses in two primary instances. In private equity funds, it is possible that a few of the projects in which the fund has invested may successfully terminate and generate large cash inflows and profits to the fund. Other projects may fail at a later date, thereby generating large losses or write-offs. An important issue when a fund experiences large gains early in its life, followed by subsequent losses, is whether incentive fees paid on the early profits will be returned to the LPs.

Another instance in which losses follow profits is more common in the hedge fund industry, where market conditions or managerial decision-making can cause strategies to be highly successful in one time period and then highly unsuccessful in a later time period. In this case, the fund earns high profits followed by large losses. In both cases, it is possible that incentive fees, or carried interest, could be paid during the earlier profitable stage, even though subsequent losses could cause the investment to have no profit over its entire lifetime. Thus, a limited partner could end up paying incentive fees for an investment that lost money over its lifetime. Clawback provisions are designed to address this problem for limited partners.

The goal of clawback provisions is to protect the economic split agreed between the GP and LPs. The clawback provision is sometimes called a giveback or a lookback, because it requires a partnership to undergo a final accounting of all of its capital and profit distributions at the end of a fund's lifetime. Clawback provisions are the opposite of vesting. Vesting of fees is the process of making payments available such that they are not subject to being returned.

A clawback provision is a promise to repay overdistributions, but such a promise is only as good as the creditworthiness of the GP. The GP is normally organized as a limited liability vehicle with no assets other than the interest in the fund. In the partnership agreements of many funds, the clawback provision simply binds the GP and requires his or her cooperation and financial support.

The sentiment that clawbacks are worthless is not uncommon. Situations arise in which LPs are unable to receive the clawbacks they are owed. Attempting to enforce the clawback provisions may lead to years of litigation without resulting in any return of cash. The simplest and, from the viewpoint of LPs, most desirable solution is to ensure that the GP does not receive carried interest until all invested capital has been repaid to investors. With this approach, however, it can take several years before the fund's team sees any gains, and it could be unacceptable or demotivating to the fund managers. An accepted compromise for securing the clawback obligation is to place a fixed percentage of the fund manager's carried interest proceeds into an escrow account as a buffer against potential clawback liability.

Clawbacks typically refer to GP clawbacks, or corrective payments to prevent a windfall to the fund manager. However, it is also possible for LPs to receive more than their agreed percentage of carried interest. Consequently, some partnership agreements also address so-called LP clawbacks.

\section*{Hard Hurdle Rates}
A hurdle rate, or preferred return, specifies that a fund manager cannot receive a share in the distributions until the limited partners have received aggregate distributions equal to the sum of their capital contributions as well as a specified return, known as the hurdle rate. In other words, a hurdle rate specifies a return level that LPs must receive before GPs begin to receive incentive fees. This section details hurdle rates and discusses a hard hurdle rate. A hard hurdle rate limits incentive fees to profits in excess of the hurdle rate.

The sequence of cash distributions with a hard hurdle rate is as follows:

\begin{enumerate}
  \item Capital is returned to the limited partners until their investment has been repaid.

  \item Profits are distributed only to the limited partners until the hurdle rate is reached.

  \item Additional profits are split such that the fund manager receives an incentive fee only on the profits in excess of the hurdle rate.

\end{enumerate}

\section*{Soft Hurdles and a Catch-Up Provision}
A soft hurdle rate allows fund managers to earn an incentive fee on all profits, given that the hurdle rate has been achieved. Returning to the example of a one-year $\$ 10$ million fund with a hurdle rate of $10 \%$ and profits of $\$ 2$ million, a soft hurdle rate of $10 \%$ allows the fund manager to receive $20 \%$ of the entire $\$ 2$ million profit, or $\$ 400,000$. As long as the resulting share to the limited partners allows a return in excess of the hurdle rate, then the hurdle rate can be ignored in terms of computing the incentive fee. The limited partners receive $\$ 1.6$ million, which is a $16 \%$ return.

The soft hurdle in this case allows the fund manager to receive an incentive fee on the entire profit. A soft hurdle has a catch-up provision that can be viewed as providing the fund manager with a disproportionate share of excess profits until the manager has received the incentive fee on all profits. The sequence of cash distributions with a soft hurdle rate is as follows:

\begin{enumerate}
  \item Capital is returned to the limited partners until their investment has been repaid.

  \item Profits are distributed only to the limited partners until the hurdle rate is reached.

  \item Additional profits are split, with a high proportion going to the fund manager until the fund manager receives an incentive fee on all of the profits.

\end{enumerate}

Once the fund manager has been paid an incentive fee on all previous profits, additional profits are split using the incentive fee. This is called a catch-up provision.

\section*{Incentive Fee as an Option}
Incentive fees are long call options to GPs, who receive the classic payout of a call option: If the assets of the fund rise, they receive an increasing payout, and if the assets of the fund remain constant or fall, they receive no incentive fee. The underlying asset is the fund's net asset value, and the time to expiration of the option is the time until the next incentive fee is calculated, at which time a new option is written for the next incentive fee. In the absence of a hurdle rate, the strike price of the call option is the net asset value of the fund at the start of the period or the end of the last period in which an incentive fee was paid, whichever is greater. The GPs pay for this call option by providing their management expertise.

A hurdle rate may be viewed as increasing the strike price of the incentive fee call option. A hurdle rate increases the amount by which the net asset value of the fund must rise before the fund manager receives an incentive fee. The higher the hurdle rate, the lower the value of the call option.

As a call option, incentive fees provide fund managers with a strong incentive to generate profits. The call option moves in-the-money when the net asset value of a fund rises to the point of providing a return in excess of any hurdle rate. The call option moves out-of-the-money when the net asset value of the fund falls below the point of providing a return in excess of any hurdle rate. When the option is below or near its strike price, the incentive fees provide the fund manager with an incentive to increase the risk of the fund's assets. The effect of increased risk is to increase the value of the call option. If the risks generate profits, the fund manager can benefit through high incentive fees. If the risks generate losses, the effect on the fund manager is limited to receiving no incentive fee, ignoring clawbacks.

When the incentive fee call option is deep-in-the-money, the fund manager benefits less from an increase in the risk of the underlying assets. The consequences of net asset value changes to the fund manager are more symmetrical when the option is deep-in-the-money, meaning when large incentive fees are likely. Risk aversion may motivate the fund manager to lessen the risk of the underlying assets when the incentive fee option is deep-in-the-money.

It can be argued that the multifaceted incentives generated by the optionlike character of incentive fees are perverse. The LPs prefer fund managers to take risks based on market opportunities and the risk-return preferences of the LPs. However, incentive fees can motivate fund managers to base investment decisions on the resulting risks to their personal finances. In summary, incentive fees can cause decisions involving risk to be based on the degree to which an option is in-the-money, near-the-money, or out-of-the-money.


\end{document}