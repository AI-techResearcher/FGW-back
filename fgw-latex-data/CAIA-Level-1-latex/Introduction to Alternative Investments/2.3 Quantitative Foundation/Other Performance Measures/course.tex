\documentclass[11pt]{article}
\usepackage[utf8]{inputenc}
\usepackage[T1]{fontenc}
\usepackage{amsmath}
\usepackage{amsfonts}
\usepackage{amssymb}
\usepackage[version=4]{mhchem}
\usepackage{stmaryrd}

\begin{document}
\section*{Reading}
Other Performance Measures

The performance measurement of illiquid investments, such as private equity, can pose substantial problems due to the limited points in time for which market values are available. In addition to IRR and MIRR, analysts often examine value ratios unadjusted for time value. This section discusses three such ratios.

\section*{Ratios as Performance Measures}
The distribution to paid-in (DPI) ratio, or realized return, is the ratio of the cumulative distribution to investors to the total capital drawn from investors, and can be loosely viewed as a non-annualized measure of income (actually, distributions) in the numerator to total investment in the denominator.


\begin{equation*}
\mathrm{DPI}_{T}=\frac{\sum_{t=0}^{T} D_{t}}{\sum_{t=0}^{T} C_{t}} \tag{1}
\end{equation*}


where $D_{\mathrm{t}}$ is the distribution in year $t$ and $C_{t}$ is the contribution in year $t$.

The residual value to paid-in (RVPI) ratio, or unrealized return, at time $T$ is the ratio of the total value of the unrealized investments at time $T$ to the total capital drawn from investors during the previous time periods, and can be loosely viewed as a measure of capital gain or loss, with a ratio of one indicating that, ignoring prior distributions, the investment has neither gained or lost value relative to the total contributions.


\begin{equation*}
\operatorname{RVPI}_{T}=\frac{\mathrm{NAV}_{T}}{\sum_{t=0}^{T} C_{t}} \tag{2}
\end{equation*}


The total value to paid-in (TVPI) ratio, or total return, is a measure of the cumulative distribution to investors plus the total value of the unrealized investments relative to the total capital drawn from investors, and is the sum of the income (DPI) and capital gain or loss (RVPI).


\begin{equation*}
\mathrm{TVPI}_{T}=\frac{\sum_{t=0}^{T} D_{t}+\mathrm{NAV}_{T}}{\sum_{t=0}^{T} C_{t}}=\mathrm{DPI}_{T}+\mathrm{RVPI}_{T} \tag{3}
\end{equation*}


Note that these ratios do not take the time value of money into account. Further, the RVPI and the TVPI depend on net asset values, which in the case of illiquid assets are usually professional estimates rather than observed market values. Their estimations are the most problematic components of return evaluation and so DPI, which is measured using only distributed capital, is seen as being the more reliable measure for mature funds.

\section*{The Public Market Equivalent (PME) Method}
A key approach to benchmarking private equity is the PME method. The Public Market Equivalent (PME) method uses a publicly traded securities index that is believed to have a similar risk exposure to private equity as a return target and requires or finds the corresponding premium over public equity (e.g., 300 to 500 basis points) for a private equity investment using the investment's cash contributions (calls), distributions, and terminal value. The PME method uses the returns of a public equity market index as a base for the fund's reinvestment rate and opportunity cost of capital. Specifically, the PME method finds the added (or reduced) cashweighted return for a private equity investment obtained by investing in the private equity investment rather than by investing in a stock market index.

For example, consider the following highly simplified example of $\$ 100\left(C_{0}\right)$ invested for two years in three alternatives: PE Fund \#1, PE Fund \#2, and a public market equity index. At the end of two years, PE Fund \#1 returns $\$ 121\left(C_{2}\right)$, PE Fund \#2 returns $\$ 169\left(C_{2}\right)$, and an investment in the market index returns $\$ 144\left(C_{2}\right)$. The internal rates of return (IRRs) for the three opportunities are included in the next exhibit.

Simplified PMEs for Two Funds with Two

Cash Flows

\begin{center}
\begin{tabular}{|l|c|c|c|r|}
\hline
Description & $c_{0}$ & $c_{2}$ & IRR & PME \\
\hline
PE Fund \#1 & $-\$ 100$ & $\$ 121$ & $10 \%$ & $-10 \%$ \\
\hline
Market Index & $-\$ 100$ & $\$ 144$ & $20 \%$ & $0 \%$ \\
\hline
PE Fund \#2 & $-\$ 100$ & $\$ 169$ & $30 \%$ & $+10 \%$ \\
\hline
\end{tabular}
\end{center}

The rightmost column in the above exhibit is the added or subtracted performance of the PE fund relative to the public market using the PME method. For each PE fund, the PME indicates the performance (IRR) of the opportunity relative to the performance of the market index. Note that PE Fund \#1 underperformed the market index by $10 \%$ per year (even though it earned 10\%), whereas PE Fund \#2 outperformed the market index by only $10 \%$ per year (even though it earned 30\%).

This introduction provides a foundation on which to build a full understanding of the PME method. The example is extremely simple due to there being only two cash flows: one investment and one distribution. It demonstrates the essential intuition of the PME approach. The exhibit above shows that if the $\$ 100$ had been invested in the public equity market, it would have grown at $20 \%$ to a value of $\$ 144$. But when invested in the private equity market, it would have grown either at $10 \%$ to $\$ 121$ or at $30 \%$ to $\$ 169$. The public market provides a benchmark against which to evaluate both funds. It indicates that PE Fund \#1 underperformed the public equity market by $10 \%$ per year and PE Fund \#2 outperformed the public market by $10 \%$ per year.

Although the PME method is easy to illustrate in the simplified example of two cash flows, it becomes more technically challenging when it is applied to investments with multiple cash investments (capital calls), multiple distributions, and cash-weighted returns. It is in cases of multiple cash inflows or outflows where cashweighted return analysis is necessary to provide valuable comparisons.

Application of the PME method involving more than two cash flows requires simulation of the performance of an investment in a public equity market index using the actual cash flows pertaining to an investment in a private fund. Topic 7 of the CAIA Level II curriculum explains the application of the PME method (and other methods of evaluating investments) in detail, including variations of the method and challenges with implementing each variation. The Level II curriculum also discusses alternatives to cash-weighted returns. including multiples.


\end{document}