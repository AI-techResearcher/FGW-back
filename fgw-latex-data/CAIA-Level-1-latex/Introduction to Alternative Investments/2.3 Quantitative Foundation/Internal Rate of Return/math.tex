\documentclass[11pt]{article}
\usepackage[utf8]{inputenc}
\usepackage[T1]{fontenc}
\usepackage{amsmath}
\usepackage{amsfonts}
\usepackage{amssymb}
\usepackage[version=4]{mhchem}
\usepackage{stmaryrd}

\begin{document}
\section*{APPLICATION A}
Investment A is expected to cost $\$ 100$ and to be followed by cash inflows of $\$ 10$ after one year and then $\$ 120$ after the second year, when the project terminates. The IRR is based on anticipated cash flows and is an anticipated lifetime IRR.

\section*{Answer and Explanation}
To solve this problem, we will use Equation 2:

$$
C F_{0}+\frac{C F_{1}}{(1+I R R)^{1}}+\frac{C F_{2}}{(1+I R R)^{2}}+\frac{C F_{3}}{(1+I R R)^{3}}+\ldots+\frac{C F_{T}}{(1+I R R)^{T}}=0
$$

It is much easier to solve this problem with your financial calculator doing the search. However, to perform the search by hand, note that CF0 =-\$100, CF1=\$10, and CF2 $=\$ 120$. Putting this into the equation:

$$
-100+\frac{\$ 10}{(1+I R R)^{1}}+\frac{\$ 120}{(1+I R R)^{2}}=0
$$

A method to solve for IRR that does not use the advanced features of the calculator is the "trial and error" method. That is put in an interest rate (as a decimal) for IRR and solve. If the NPV is higher than zero, increase the interest rate. If the NPV is lower than zero, decrease the interest rate. Continue guessing the rate until the NPV converges “close enough" to zero. In this situation, let's guess $10 \%$.

$$
-100+\frac{\$ 10}{(1+.10)^{1}}+\frac{\$ 120}{(1+.10)^{2}}=0
$$

This works out to have a NPV of $\$ 8.26$. Therefore, we need to increase the interest rate. Let's try an interest rate of $15 \%$.

$$
-100+\frac{\$ 10}{(1+.15)^{1}}+\frac{\$ 120}{(1+.15)^{2}}=0
$$

With an interest rate of $15 \%$, the NPV is (\$0.56). The NPV is less than zero, so the IRR is lower than $15 \%$. Using an interest rate of $14.66 \%$ (a contrived "guess" that is the answer), as shown below, the NPV is zero. Therefore, the IRR is about $14.66 \%$.

$$
-100+\frac{\$ 10}{(1+.1466)^{1}}+\frac{\$ 120}{(1+.1466)^{2}}=0
$$

The IRR of the investment is $14.7 \%$.

\section*{APPLICATION B}
Fund B expended $\$ 200$ million to purchase investments and distributed $\$ 30$ million after one year. At the end of the second year, it is being appraised at $\$ 180$ million.

\section*{Answer and Explanation}
To solve this problem, we will use Equation 2:

$$
C F_{0}+\frac{C F_{1}}{(1+I R R)^{1}}+\frac{C F_{2}}{(1+I R R)^{2}}+\frac{C F_{3}}{(1+I R R)^{3}}+\ldots+\frac{C F_{T}}{(1+I R R)^{T}}=0
$$

Which result to the IRR of the fund is a since-inception IRR (or interim IRR) of $2.7 \%$.

\section*{APPLICATION C}
Investment $\mathrm{C}$ was purchased three years ago by BK Fund for $\$ 500$. In the three years following the purchase, the investment distributed cash flows to the investor of $\$ 110, \$ 120$, and $\$ 130$. Now in the fourth year, the investment has been appraised as being worth $\$ 400$. The IRR of the investment is based on realized previous cash flows and a current appraised value.

\section*{Answer and Explanation}
To solve this problem, we will use Equation 2:

$$
C F_{0}+\frac{C F_{1}}{(1+I R R)^{1}}+\frac{C F_{2}}{(1+I R R)^{2}}+\frac{C F_{3}}{(1+I R R)^{3}}+\ldots+\frac{C F_{T}}{(1+I R R)^{T}}=0
$$

Which result to the IRR may be described as an interim IRR and is $15.0 \%$.


\end{document}