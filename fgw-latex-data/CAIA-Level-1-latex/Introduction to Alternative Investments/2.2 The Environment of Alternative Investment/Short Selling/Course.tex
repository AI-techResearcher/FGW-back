\documentclass[11pt]{article}
\usepackage[utf8]{inputenc}
\usepackage[T1]{fontenc}
\usepackage{amsmath}
\usepackage{amsfonts}
\usepackage{amssymb}
\usepackage[version=4]{mhchem}
\usepackage{stmaryrd}

\begin{document}
Short Selling

Short selling financial assets is the process of borrowing securities from a securities lender, selling the securities at their market price, and eventually purchasing identical securities in the market to extinguish the loan from the securities lender. Short selling is an important part of many alternative investment strategies. Therefore, a solid understanding of short selling is vital to a complete understanding of alternative investments.

\section*{The Institutional Mechanics of Short Selling}
Owners of stocks or other securities often leave the actual stock certificates with their brokerage firms or other financial institutions for safekeeping. Financial institutions often lend these securities to short sellers to generate income.

For example, suppose the stock of Neveright is currently selling for $\$ 50$ per share. Some speculators believe that the share price of Neveright will soon fall to $\$ 40$ per share and wish to take advantage of this situation. They place orders to short sell shares of Neveright through their brokerage firms. Their brokerage firms lend shares to their own customers seeking to short sell or they arrange to borrow shares from other financial institutions.

Customers of various brokerage firms hold millions of shares of Neveright in street name. Street name refers to the brokerage practice of having the direct legal ownership of customer securities held in the name of the brokerage firm on behalf of the customers rather than having the legal ownership of the shares reside directly with the economic owners of the securities (i.e., the customers of the brokerage firm). The brokerage firms lend shares in Neveright to the speculators that in turn get sold into the financial markets and likely deposited in other brokerage firms by the new owners of the shares.

Suppose that while the speculators maintain their short positions in Neveright, the firm pays a $\$ 1$ per share dividend. Neveright distributes dividends to the current owners of record for the firm's shares, not to the previous shareholders who lent out their securities. The borrower of the securities must compensate the lender of the securities for the dividends on the securities that the corporation distributes during the holding period of the loan. Substitute dividends are cash flows paid by share borrowers to share lenders to compensate the lenders for the distributions paid by the corporation while the loan of stock is outstanding.

Eventually the speculators close their short positions by purchasing the stock in the open market and returning the shares to the lender. Since there is no upper limit to the price of a stock, short selling subjects the short seller to unlimited liability.

\section*{The Mechanics of Short Selling to the Short Seller}
Consider a speculator (or hedger) that borrows 100 shares of Neveright from a broker and sells them in the market for $\$ 50$. The speculator must have and post collateral to guarantee repayment of the loan. The speculator receives $\$ 5,000$ (less commissions) from the sale and has a liability representing 100 shares of Neveright that must eventually be returned.

If the stock falls to $\$ 40$, the speculator may buy 100 shares of Neveright for $\$ 40$ a share (for a total of $\$ 4,000$ less commission) and return the 100 shares to the securities lender for a profit before commission and other fees of $\$ 1,000$. However, if the stock climbs to $\$ 60$ and the speculator closes the position, the speculator would suffer a loss of $\$ 1,000$ plus commissions. Because there is no upper limit to the price of Neveright stock, potential losses to short selling are unlimited.

The speculator is also responsible for substitute dividends (discussed earlier) and various costs, including the following:

\begin{enumerate}
  \item The borrower of the short position posts collateral equal to the price of the assets plus margin, also known as a haircut, usually of $2 \%$. Thus, if Fund A borrows $\$ 100,000$ of stock from ABC Brokerage Firm and short sells that stock into the market, Fund A must place the proceeds of the sale (i.e., $\$ 100,000$ ) and $2 \%$ more (i.e., $\$ 2,000$ ) as collateral to provide protection to the lender against the risk that the borrowed stock will rise in price at the same time that Fund A becomes unable to fulfill its obligation to return the stock.

  \item The lender of the securities earns interest on the collateral but typically offers the borrower of the securities a rebate. A rebate is a payment of interest to the securities' borrower on the collateral posted. A typical rebate is a little less than current short-term market interest rates (e.g., the general collateral rate less $0.25 \%)$. The goal of the securities lender is to receive a spread between the interest rate the lender is able to earn on the collateral and the rebate paid to the securities borrower. This compensation to the lender for their services is a cost to the borrower. Note that the securities lender takes the risk that the borrower will default and be unable to return the shares at the same time that the collateral will be insufficient to repurchase the shares in the marketplace.

  \item The substitute dividend payments from the short seller (securities borrower) to the securities lender is not an economic cost to the short seller to the extent that the stock paying the dividend falls in price by an amount equal to the dividend. The short seller's gain from the stock price drop offsets the dividend to the lender-part of the concept of dividend irrelevancy. Dividend irrelevancy is the proposition that, in the absence of imperfections such as income taxation that penalized dividends, the distribution of corporate dividends does not alter shareholder wealth.

\end{enumerate}

Most securities lending is performed on an overnight basis, wherein securities lenders may demand return of the shares at any time and may require regular adjustment of the collateral amount to reflect the current market price of the borrowed securities. However, some short sales can be performed as term loans of\\
perhaps six months, wherein the lender agrees not to demand return of the securities until the term has ended.

\section*{Special Situations Involving Short Selling}
There are special risks to having short positions in equity securities, especially for stocks that are popular targets of short sellers. As the quantity of a stock's outstanding shares being lent to short sellers increases, the competition to find new stock to borrow increases. Entities that hold the stock put that stock "on special." In this context, a special stock is a stock for which higher net fees are demanded when it is borrowed. To the short seller, this means receiving a smaller rebate. For example, general collateral stocks, which are stocks not facing heavy borrowing demand, may earn a $2 \%$ rebate when risk-free rates are at $2 \%$, whereas stocks on special may earn zero rebates or even negative rebates, wherein borrowers must pay the lenders money in addition to the interest that the lender is earning on the collateral.

When numerous speculators establish highly similar large positions, it is often referred to as a crowded trade. In the case of traders establishing large short positions, the trade is often termed a crowded short. The security being shorted can become a special stock, and in extreme cases, the security can only be made available for short sales at extraordinarily high borrowing rates as high as $20 \%$ or more.

When the inventory of stock available to borrowers becomes extremely tight, short sellers may find their position bought in, meaning the broker revokes the borrowing privilege for that specific stock and requires the trader to cover the short position. If shares cannot be borrowed through another lender on affordable terms, this leaves a convertible arbitrage manager without a hedge to the convertible bond position, which is likely to lead the trader to sell the bond to reduce the stock market risk of the portfolio.

Short sellers should monitor the availability of shares trading in the market to ensure that they can be purchased without substantially increasing the market price when they are needed to cover a short position. Short sellers need to be aware of the possibility of a short squeeze. A short squeeze occurs when holders of short positions are compelled to purchase shares at increasing prices to cover their positions due to limited liquidity. As the ratio of shares being sold short increases relative to the total number of freely floating shares, it becomes increasingly difficult to borrow additional shares, and the potential for a short squeeze increases. Several hedge fund managers being forced to buy in and cover their short positions simultaneously can put upward pressure on the price of the shorted security. The upward movement of the stock price may cause other short sellers to cover their positions, putting even more upward pressure on the stock price. As more and more hedge fund managers scramble to cover their short positions, the price of the underlying stock can rise rapidly, leaving the last few hedge fund managers squeezed out of their positions at especially elevated prices.

Another potentially huge complexity from short selling is that the lender of the security may demand that the shares be returned. Most securities are lent on a shortterm basis, with the lender retaining the right to demand that the shares be returned at any moment. Usually when this happens, the broker simply arranges for another securities lender to loan shares so that the short seller maintains a seamless exposure.

But especially in times of overall market turbulence, or in times of turbulence for a particular stock, the shares become difficult to borrow. In those cases, the short seller may be forced to cover the position. This means that the short seller must purchase the shares in the market so that they can be returned to the securities lender. The short seller is therefore forced to close his position during a period of turbulence rather than at a time of his choosing. Note that an investor with a long position in a stock does not face this risk. The short seller's potential problem of being forced to liquidate a position is especially acute during a short squeeze.


\end{document}