\documentclass[11pt]{article}
\usepackage[utf8]{inputenc}
\usepackage[T1]{fontenc}

\begin{document}
\section*{Reading}
The Participants

This session provides an introduction to the environment of alternative investing, including the participants, the financial markets, liquid alternatives, and taxation. Its focus is on explaining the purposes and functions of these components so that readers gain an understanding of why the investing environment is structured the way it is and how the different components interact.

Participants can be divided into four major categories: the buy side, the sell side, outside service providers, and regulators. This section briefly describes the primary roles of the first three categories of participants; the primary role of regulators is discussed in the Regulatory Environment lesson.

\section*{The Buy Side}
Buy side refers to the institutions and entities that buy large quantities of securities for the portfolios they manage. Buy side entities include asset owners and asset managers. The buy side contrasts with the sell side (detailed in the The Sell Side section), which focuses on distributing securities to the public. Examples of buy-side institutions follow, with an emphasis on the perspective of alternative investing.

Plan sponsors: A plan sponsor is a designated party, such as a company or an employer, that establishes a health care or retirement plan (pension) for employees that has special legal or taxation status, such as a 401(k) retirement plan in the United States or a superannuation fund in Australia. Plan sponsors are companies or other collectives that establish the health care and retirement plans for the benefit of the organization's employees or members. Plan sponsors are responsible for determining membership parameters and investment choices and, in some cases, providing contribution payments in the form of cash or stock (or both). In many cases, one individual, the plan trustee, is designated with overall responsibility for managing the plan's assets, whereas the plan administrator is charged with overseeing the plan's day-to-day operations. Both the trustee and the administrator are identified in the plan's summary plan description. As described in CAIA Level II, these funds can be designed as defined contribution plans or defined benefit plans. In defined contribution plans, employees take the investment risk, while that risk is borne by the employer in defined benefit plans.

Foundations and endowments: A foundation is a not-for-profit organization that donates funds and support to other organizations for its own charitable purposes. An endowment is a fund bestowed on an individual or institution (e.g., a museum, university, or hospital) to be used by that entity for specific purposes and with principal preservation in mind.

Family office and private wealth: Family office and private wealth institutions are private management advisory firms that serve ultra-high-net-worth investors. A family office is a group of investors joined by familial or other ties who manage their personal investments as a single entity, usually hiring professionals to manage money for members of the office.

Sovereign wealth funds: Sovereign wealth funds are state-owned investment funds held for the purpose of future generations and/or to stabilize the state currency. These funds may emanate from budgetary and trade surpluses, perhaps through exportation of natural resources and raw materials such as oil, copper, or diamonds. Because of the high volatility of resource prices, unpredictability of extraction, and exhaustibility of resources, sovereign wealth funds are accumulated to help provide financial stability and future opportunities for citizens and governments. Sovereign wealth funds can also be built using the proceeds of international trade, such as countries with a large surplus in their trade balance generated by a large export sector.

Private limited partnerships: Private limited partnerships are a form of business organization that potentially offers the benefit of limited liability to the organization's limited partners (similar to that enjoyed by shareholders of corporations) but not to its general partner. For tax purposes, limited partnerships tend to flow taxable revenue and expenses directly through to their partners rather than being taxed at the partnership level.

Private investment pools: Hedge funds, funds of funds, private equity funds, managed futures funds, commodity trading advisers (CTAs), and the like are private investment pools that focus on serving as intermediaries between investors and alternative investments. Most U.S. funds are structured as limited partnerships and offer incentive-based compensation schemes to their managers. These limited partnerships are usually managed by the general partner, while most of the invested funds are provided by the limited partners. These structures are detailed in the Alternative Investment Structures lesson.

Separately managed accounts: Separately managed accounts (SMAs) are individual investment accounts offered by a brokerage firm and managed by independent investment management firms. The relationship between an investment adviser and a client to whom advice is provided is typically documented by a written investment management agreement. SMAs can be thought of as being similar to pooled investment arrangements, such as mutual funds, in that a customer pays a fee to a money manager for managing the customer's investment, but SMAs tend to be differentiated from funds in five major ways:

\begin{enumerate}
  \item A fund investor owns shares of a company (the fund) that in turn owns other investments, whereas an SMA investor actually owns the invested assets as the owner on record.

  \item A fund invests for the common purposes of multiple investors, whereas an SMA may have objectives tailored to suit the specific needs of its only investor, such as tax efficiency.

  \item A fund is often opaque to its investors to promote confidentiality; an SMA offers transparency to its investor.

  \item Fund investors may suffer adverse consequences from other investors' redemptions (withdrawals) and subscriptions (deposits), but an SMA provides protection from these liquidity issues for its only investor.

  \item Whereas the fund structure may allow investors to have limited liability, the SMA format may allow losses to be greater than the capital contribution when leverage or derivatives are used.

\end{enumerate}

From an investor's perspective, the advantages of the first four distinctions typically outweigh the disadvantages of the last distinction. However, fund managers prefer the simplicity and convenience of pooled arrangements (funds).

Mutual funds ('40 Act funds): Mutual funds, or ' 40 Act funds, are registered investment pools offering their shareholders pro rata claims on the fund's portfolio of assets. In the United States, mutual funds that offer their shares for sale to the public are known as ' 40 Act funds due to the regulations that permit their offering by registered investment advisers: the U.S. Investment Company Act of 1940 . In recent years, ' 40 Act funds have increasingly offered alternative asset\\
exposures through these retail fund structures. A general discussion is provided in the lesson, Alternative Investments, along with more specific discussions throughout Topics 2 through 5.

UCITS (Undertakings for Collective Investment in Transferable Securities): Regulation of public funds in Europe centers on the concept of UCITS. UCITS allow retail access and marketing of hedge-fund-like investment pools, somewhat akin to the retail access and marketing of ' 40 Act funds in the United States. UCITS conform to European regulations such that the product can be sold throughout the various members of the EU. They are subject to strict regulations, including investment restrictions, diversification requirements, minimum size requirements, an annual audit, and meeting standards involving the promoters and other parties related to the UCITS creation, distribution, and management.

Master limited partnerships: Master limited partnerships (MLPs) are publicly traded investment pools that are structured as limited partnerships and that offer their owners pro rata claims. Like equities, MLP units are traded on major stock exchanges, but they have legal and tax structures similar to those of private limited partnerships.

\section*{The Sell Side}
In contrast to buy-side institutions, sell-side institutions, such as large dealer banks, act as agents for investors when they trade securities. Sell-side institutions make their research available to their clients and are more focused on facilitating transactions than on managing money.

Large dealer banks: Large dealer banks are major financial institutions, such as Goldman Sachs, Deutsche Bank, and the Barclays Group, that deal in securities and derivatives. Although based on the same economic principles as typical retail banks, large dealer banks are much bigger and more complex. The macroeconomic impact of a large dealer bank failure may be more widespread because of the central role this type of bank plays in the economic system at large. Generally, most large dealer banks act as intermediaries in the markets for securities, repurchase agreements, securities lending, and over-the-counter (OTC) derivatives. In addition, large dealer banks are often engaged in proprietary trading.

Large dealer banks also have large asset management divisions that cater to the investment management needs of institutional and wealthy individual clients. This involves custody of securities, cash management, brokerage, and investment in alternative investment vehicles, such as hedge funds and private equity partnerships. Some of these types of banks operate internal hedge funds and take on private equity partnerships as part of their business management service. In this role, the bank acts as a general partner with limited-partner clients.

The role of dealer banks in the primary market is to intermediate between issuers and investors, to provide liquidity, and to act as underwriters of investments. In secondary securities markets, large dealer banks trade with one another and with brokers/dealers directly over the computer or the phone, as well as play an intermediating role of facilitating trades.

Large dealer banks also engage in proprietary trading. Proprietary trading occurs when a firm trades securities with its own money in order to make a profit. Large dealer banks serve as counterparties to OTC derivatives such as options, forwards, and swaps that require the participation of counterparty dealer who meets customer demand by taking the opposite side of a desired position. Dealers may accept the risk or use a matched book dealer operation, in which the dealer lays off the derivative risk by taking an offsetting position.

As part of their business management activities, large dealer banks are active as prime brokers that offer professional services specifically to hedge funds and other large institutional customers. (Prime brokers are discussed in more detail in the Outside Service Providers section.) Several large dealer banks have ventured into off-balance-sheet financing methods, a practice that involves a form of accounting in which large expenditures are kept off the company's balance sheet through various classification methods. Companies use off-balance-sheet financing to keep their debt-to-equity and leverage ratios low. In addition to their special role in the financial system, large dealer banks share many of the same responsibilities as conventional commercial banks, including deposit taking and lending to corporations and consumers.

Brokers: Also on the sell side are retail brokers that receive commissions for executing transactions and that have research departments that make investment recommendations. Advantages of using brokers include their expertise in the trading process, their access to other traders and exchanges, and their ability to facilitate clearance and settlement. Because brokers play the role of middlemen in the trading process, traders can use broker services when they want to remain anonymous to other traders.

Typically, traders can manage their order exposure by breaking up large trades and distributing them to different brokers or by asking a single broker in charge of the entire trade to expose only parts of the order, so that the full size remains unknown to other traders. Brokers also often represent limit orders for clients (i.e., orders placed with a brokerage to buy or sell shares at a specified price or better). In this event, brokers monitor the markets on behalf of their clients and make decisions based on client limit and stop orders when the markets change.

The brokerage firm's proprietary trading operations involve the firm's own account, called the house account. Other sources of broker revenue include soft commissions, payments for order flow, interest on margin loans, short interest rebates (on short sales), underwriting fees when the firm helps clients sell securities, and mergers and acquisitions (M\&A) fees. The major cost of running a brokerage firm is labor: the brokers and other employees who provide the firm's services to clients.

Brokerage firms and other firms with major investment activities organize their activities into three major operations: (1) front office, (2) back office, and (3) middle office. Front office operations involve investment decision-making and, in the case of brokerage firms, contact with clients. Back office operations play a supportive role in the maintenance of accounts and information systems used to transmit important market and trader information in all trading transactions, as well as in the clearance and settlement of the trades. Middle office operations form the interface between the front office and the back office, with a focus on risk management.

\section*{Outside Service Providers}
Other major participants in the world of alternative investments are outside service providers, such as prime brokers, accountants, attorneys, and fund administrators. Alternative investment funds rely on outside service providers for their successful creation and operation. Their roles are briefly discussed here.

Prime brokers: Prime brokers allow an investment manager to carry out trades in multiple financial instruments at multiple broker-dealers while keeping all cash and securities at a single firm, and have the following primary functions: clearing and financing trades for clients, providing research, arranging financing, borrowing and lending securities, and producing portfolio accounting. Prime brokers offer a range of services.

Prime brokers have a powerful tool in their ability to make margin calls. The prime broker can demand that the fund manager deposit more cash into its prime brokerage account to support its leveraged trading, and that the fund manager liquidate outstanding portfolio positions to raise cash to deposit with the prime broker. The ultimate threat is that the prime broker can seize collateral from the hedge fund manager and liquidate the collateral to raise cash.

Accountants and auditors: The accounting firm providing services to a hedge fund or to another alternative investment fund should include an experienced auditor and tax adviser. During the creation of the fund or investment vehicle, the accounting firm provides services largely parallel to those of an attorney: reviewing legal documents to ensure that accounting methods and allocations are appropriate and feasible, and that relevant tax issues have been addressed. The accountant helps prepare partnership returns and the necessary forms for the investors in the fund to report their shares of partnership income, deductions, gains, and losses (e.g., Schedule K-1 in the United States). The adviser also provides tax-related advice to the fund throughout the year and may be called on as a consultant on structuring and compensation issues for the principals of the general partner. The auditor performs a year-end audit of the fund, including the review of security pricing, and presents the results of this audit to the fund and its investors. Accountants usually cooperate with the prime broker and fund administrator to gather the necessary information for audits and tax returns.

Attorneys: An attorney helps determine the best legal structure for a fund's unique investment strategies, objectives, and desired investors. The attorney takes care of filing any documents required by the government (federal or other levels) and creates the legal documents necessary for establishing and managing a hedge fund or another alternative investment. The attorney can offer guidance on marketing a hedge fund or another alternative investment in full compliance with all legal requirements, as well as on operational issues, such as personal trading. For example, in the United States, an attorney can provide advice regarding Securities and Exchange Commission (SEC) rules governing the use of testimonials, performance statistics, and prior performance statistics.

Fund administrators: Many hedge funds and other alternative investment funds now engage a fund administrator to be responsible for bookkeeping, thirdparty information gathering, and securities valuation functions for all of their funds, both onshore and offshore. The fund administrator has ten roles: (1) maintains a general ledger account, (2) marks the fund's books, (3) maintains its records, (4) carries out monthly accounting, (5) supplies its monthly profit and loss (P\&L) statements and calculates its returns, (6) verifies asset existence, (7) independently calculates fees, (8) provides an unbiased, third-party resource for price confirmation on security positions, (9) produces a monthly capital account statement for investors, and (10) apportions fund income or loss among them. The administrator takes over the duties of day-to-day accounting and bookkeeping so that managers can focus on maximizing the portfolio's returns. The administrator can also be an important source of information for the auditor and tax adviser in completing required audits and tax returns.

Hedge fund infrastructure: Hedge funds can require a complicated infrastructure and extensive technological systems. The hedge fund infrastructure may have three main financial components: (1) platforms, (2) software, and (3) data providers. Financial platforms are systems that provide access to financial markets, portfolio management systems, accounting and reporting systems, and risk management systems. Financial software may consist of prepackaged software programs and computer languages tailored to the needs of financial organizations. Some funds use open-source software, and others pay licensing fees for proprietary software. For a hedge fund, most of the raw material that goes into its strategy development and ongoing investment process is in the form of data. Financial data providers supply funds primarily with raw financial market data, including security prices, trading information, indices, and company-specific information such as income statements and balance sheets. The amount of data is dictated by the investment style. Nonetheless, most hedge fund managers are required to keep abreast of market developments and macroeconomic news.

Due to legal implications, directly marketing alternative investment vehicles can be problematic. One method of indirectly marketing private funds is to report a fund's performance to an index provider, especially if the fund's performance is attractive. Index providers compile indices of prices that assist fund managers in evaluating performance.

Consultants: Consultants may be hired by pensions, endowments, or high-net-worth individuals to provide a number of roles and services that center on advice, analysis, and investment recommendations. Clients rely on consultants to offer unbiased analysis of money managers' investment performance, as well as advice on how to best allocate funds. Clients expect their consultants to help them lay out the parameters of their investment objectives by setting out a plan for allocating assets within the framework of their objectives and risk tolerance. Consultants work closely with their clients to monitor the performance of investments while continuing to play an advisory role in a client's choice of other service providers.

Consultants are increasingly being used to serve the role of chief investment officer in small organizations. The role of an outsourced chief investment officer (OCIO) ranges from performing all of the decision-making duties of an in-house chief investment officer to a reduced role of assisting staff with a subset of decisions.

Consultants have traditionally been compensated for their services in one or more of the following ways: fees from their clients, fees from proprietary investment products, or compensation packages from the money managers for whom they generate business. Consulting conflicts of interest can emerge when consultants are compensated by money managers because this form of payment can detract from the ability to offer independent advice to clients and encourage the consultant to favor the money managers offering compensation. Further, the compensation that consultants receive from money managers is undisclosed and can be quite substantial. Some consultants waive their regular consulting fee, giving the impression that their services are free.

Consultants' integrity and expertise are vital parts of the consultant-client relationship because many clients rely on their consultants to set out the best investment plan for their purposes and hire the best money managers to oversee those investments. A third compensation approach has emerged in which consultants use their expertise in manager selection and risk management to serve as fund-of-funds managers to their clients. This arrangement avoids explicit fees to the investors for the consulting advice, and offers the potential that the consultants will act with substantial objectivity in the selection of managers.

Depositories and custodians: Depositories and custodians are very similar entities that are responsible for holding their clients' cash and securities and settling clients' trades, both of which maintain the integrity of clients' assets while ensuring that trades are settled quickly. The Depository Trust Company (DTC) is the principal holding body of securities for traders all over the world and is part of the Depository Trust and Clearing Corporation (DTCC), which provides clearing, settlement, and information services. The National Securities Clearing Corporation is the DTCC's second major subsidiary in the United States. The DTCC also created the Fixed Income Clearing Corporation (FICC). The European Central Counterparty Limited (EuroCCP) is the major depository for clients in European trading markets, and offers European clients the same clearing and settlement services as those offered by the DTCC to American traders.

Banks: A commercial bank focuses on the business of accepting deposits and making loans, with modest investment-related services. An investment bank focuses on providing sophisticated investment services, including underwriting and raising capital, as well as other activities such as brokerage services, mergers, and acquisitions.

Hedge funds may enlist the services of a commercial bank to facilitate the flow of both investment- and non-investment-related capital. In addition, hedge funds may use their commercial bank for loans, credit enhancement, and/or lines of credit. In the United States, the commercial banking and investment banking functions tend to be separated by regulation. Germany uses universal banking, which means that German banks can engage in both commercial and investment banking. Also unlike the United States, a large portion of German firms are privately funded and have two governance bodies: the Vorstand, or management board, and the Aufsichtsrat, or supervisory board.

Although the Japanese financial system seems superficially similar to the American system, banks are much more influential in Japan than they are in the United States, and cross-ownership is far more common. Japanese banks can hold common stock, and Japanese corporations can hold stock in other Japanese firms. A keiretsu is a group of firms in different industries bound together by cross-ownership of their common stock and by customer-supplier relationships. The 10 largest Japanese banks (known as city banks) are responsible for funding approximately one-third of all investments in the country. As in Germany, large banks play an active role in monitoring the decisions of the borrowing firm's management and have significant power to seize collateral, as both trustee and direct lender.

In the United Kingdom, there are two main types of banks: clearing banks, which are similar to American commercial banks, and merchant banks, similar to American investment banks. As in the United States, UK banks are not strongly involved in the firms with which they do business, and substantial stock ownership by banks is prohibited.


\end{document}