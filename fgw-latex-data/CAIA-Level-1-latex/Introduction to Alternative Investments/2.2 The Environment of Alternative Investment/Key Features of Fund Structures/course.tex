\documentclass[11pt]{article}
\usepackage[utf8]{inputenc}
\usepackage[T1]{fontenc}
\usepackage{amsmath}
\usepackage{amsfonts}
\usepackage{amssymb}
\usepackage[version=4]{mhchem}
\usepackage{stmaryrd}

\begin{document}
\section*{Reading}
Key Features of Fund Structures

This section reviews key features of limited partnerships that are common to various investment partnerships, including private equity funds, hedge funds, and other private placements.

\section*{The Four Key Partnership Documents}
There are four key documents used in establishing and managing a hedge fund, private equity fund, or other private partnership: (1) private-placement memoranda (a.k.a. the offering memorandum), which include the formal description of an investment opportunity that complies with securities regulations; (2) a partnership agreement (a.k.a. a limited partnership agreement), which is a formal written contract creating a partnership; (3) a subscription agreement, which is an application submitted by an investor who desires to join a limited partnership; and (4) a management company operating agreement, which is an agreement between members related to a limited liability company and the conduct of its business as it pertains to the law.

The limited partnership agreement (LPA) defines its legal framework and its terms and conditions. These are the terms that the limited partners are agreeing to when making a private investment, so it is important that this document is read carefully and all of the terms are understood. In many cases, private equity managers are not fiduciaries, so limited partners should not assume that the general partner has the best interest of their investors in mind.

The LPA has two main categories of clauses: (1) investor protection clauses and (2) economic terms clauses. Investor protection clauses cover investment strategy, including possible investment restrictions, key-person provisions, termination and divorce, the investment committee, the LP advisory committee, exclusivity, and conflicts. Economic terms clauses include management fees and expenses, the GP's contribution, and the distribution of cash (i.e., the waterfall). The distribution waterfall defines how returns are split between the LP and GP and how fees are calculated. LPAs are continuously evolving, given the increasing sophistication of fund managers and investors, new regulations, and changing economic environments. In essence, the LPA lays out conditions aimed at both aligning the interests of fund managers with their investors and discouraging the GP from cheating (moral hazard), lying (adverse selection), or engaging in opportunism (holdup problem) in whatever form. Limited partners may wish to review the Private Equity Principles document from the Institutional Limited Partners Association (ILPA) to understand best practices and the important terms where the interests of LPs and GPs may diverge.

Moral hazard and adverse selection take place when there is asymmetric information between two parties (e.g., LPs and GPs). Adverse selection takes place before a transaction is completed, when the decisions made by one party cause less desirable parties to be attracted to the transaction. For example, if an LP decides to seek GPs that charge very low fees and offer funds with very favorable terms, the LP is likely to attract unskilled GPs that claim to be skilled.

Moral hazard, in contrast, takes place after a transaction is completed and can be defined as the changes in behavior of one or more parties as a result of incentives that come into play once a contract is in effect. For example, without proper monitoring, a GP may take excessive risk in order to increase the potential performance fee, or an unskilled manager may decide not to take substantive risk and just collect the management fee.

In economics, the holdup problem is a situation in which two parties (in this case, a GP and an LP) refrain from cooperating due to concerns that they might give the other party increased bargaining power and thereby reduce their own profits. Incentives are designed so that the fund manager's focus is on maximizing terminal wealth and performance and ensuring that contractual loopholes are not exploited (e.g., by producing overly optimistic interim results). It is through the proper alignment of the economic interests of investors and managers, not just through the LPA covenants, the advisory boards, or the committees composed of LPs, that one can eliminate many of the problems associated with the principal-agent relationship, especially in those scenarios that cannot be foreseen. To be successful, the structure must address management fees, performance-related incentives, hurdle rates, and, importantly, the fact that GPs make a substantial personal investment in each fund. Additional clauses may be required to cover reinvestments and clawbacks, as well as noneconomic terms such as key-person provisions, joint and several liability, and disclosure obligations. Together, these clauses provide LPs with moderate but sufficient control over the management of the fund. LPs are encouraged to understand all of the terms of their investment and negotiate carefully with GPs to ensure that the interests of the GP and the LP are truly aligned, as the original contract presented is likely to have terms that favor the GP. These key LPA features are described in the following sections.

\section*{Corporate Governance in Private Funds}
The law and the LPA define and restrict the degree of control LPs have over the activities of GPs. Such controls relate, for example, to waiving or accepting investment restrictions, extending the investment period or fund duration, handling key-person-related issues, or participating in an LP advisory committee (LPAC), whose responsibilities are defined in the LPA and normally relate to dealing with conflicts of interest, reviewing valuation methods, and any other consents predefined in the LPA. LPs can make decisions with either a simple majority (e.g., the decision to extend the investment period or the fund's duration) or a qualified majority (e.g., the decision to remove the GP without cause). A qualified majority is generally more than $75 \%$ of LPs in contrast to the over $50 \%$ required for a simple majority.

Occasionally, LPs may be offered positions on the investment committee. However, it is not clear whether LPs should actually take on this role. In limited partnership structures, an overactive LP could become reclassified as a GP, thereby losing limited liability. Generally, international industry professionals recognize that fund managers should make investment and divestment decisions without the direct involvement of investors, so as not to dilute the responsibility of the manager, create potential conflicts of interest with nonparticipating investors, or expose LPs to the risk of losing their limited liability. Also, investors do not normally have the legal rights or the required skills and experience to make such decisions.

Another important element of corporate governance is reporting to LPs. Various PE associations or industry boards have released guidelines for valuation and reporting. The obligation to disclose in compliance with these guidelines is increasingly being made part of contractual agreements. While some GPs reduce the level of detail provided to the bare minimum and share it with all LPs, others share different levels of detail depending on the specific type of investor.

\section*{Investment Objectives, Fund Size, and Fund Terms}
In LPAs, the description of investment objectives should be specific but not too narrow. Investors should not attempt to put overly restrictive limits on a fund manager's flexibility, which could block the fund manager's ability to profit from unanticipated opportunities. Further, with uncertain investments, severe information asymmetry, and difficulties in monitoring and enforcing restrictions, fund managers may simply find ways around narrow restrictions.

The fund's size, in terms of capital committed by LPs, needs to be in line with these investment objectives. However, various factors, such as the management resources required or the number of potential opportunities, implicitly set a minimum or maximum size of the fund.

In the case of private equity, fund lives or terms are typically seven to 10 years, with possible extensions of up to three years. This represents a trade-off among better investment returns, sufficient time to invest and divest, and the degree of illiquidity still acceptable for investors.

Normally, the extension of a fund's life is approved annually by a simple majority of LPs or members of the LPAC, one year at a time versus two or more at once, during which time management fees are either reduced or eliminated altogether to stimulate quick exits.

Normally, proceeds are distributed to investors as soon as is feasible after the realization of or distribution of a fund's assets, but in some cases, LPs grant fund managers the discretion to reinvest some of the proceeds that are realized during the investment period.

\section*{Management Fees and Expenses}
In private equity, compensation is overwhelmingly performance driven. Management fees provide a base compensation so that the fund manager can support the ongoing activities of funds, and there is a consensus that GPs should not be able to make significant profits on management fees alone. These fees need to be based on reasonable operating expenses and salaries, and be set at a level modest enough to ensure that the fund manager is motivated primarily by the carried interest, but sufficient to avoid the manager's departure to greener fields.

\section*{Global Regulations and Fund Structures}
We present a brief overview here of global regulations and fund structures. These topics will be covered at a more comprehensive level in the CAIA Level II curriculum. The Markets in Financial Instruments Directive (MiFID) is an EU law that establishes uniform regulation for investment managers in the European Economic Area (the EU plus Iceland, Norway, and Liechtenstein). The MiFID is one of the primary pieces of European legislation dealing with regulation of investment services, including management services. The MiFID II is a revision directed toward extending the reach of MiFID to cover gaps in the 2007 document as well as addressing emerging issues, such as lack of transparency in trading occurring in dark pools.

In July 2011, the Alternative Investment Fund Managers Directive (AIFMD) came into force. This directive applies to alternative investment fund managers (AIFMs) that are located in the EU or, if located outside the EU, manage either EU funds or market funds (whether EU or non-EU) in the EU. An AIFM includes any legal or natural person whose regular business is to manage one or more alternative investment funds (AIFs). An AIF is any collective investment that invests in accordance with a specified policy, except UCITS. This captures hedge funds, private equity funds, infrastructure funds, real estate funds, and non-UCITS retail funds, whether open-ended or closed-ended and whether listed or not.

Hedge fund activity and hedge fund regulation vary tremendously outside of the United States and the EU. For example, the Australian Securities and Investment Commission (ASIC) does not regulate hedge funds differently from other managed funds. " "Changing Rules: The Regulation, Taxation and Distribution of Hedge Funds around the Globe," PricewaterhouseCoopers, June 2009. Domestic hedge funds in Australia are usually organized as unit trusts, and foreign hedge funds are foreign investment funds (FIFs). Taxation is a relatively important and complex issue in Australian hedge fund ownership.

SICAVs and SICAFs are European fund structures, often domiciled in Luxembourg. Société d'Investissement à Capital Variable (SICAV) funds are publicly traded openend funds, while Société d'Investissement à Capital Fixe (SICAF) funds are publicly traded closed-end funds. Many of these fund vehicles are organized under the UCITS framework, which includes restrictions on leverage, liquidity, and concentration of assets in the fund's portfolio. Other SICAV and SICAF funds are organized as specialized investment funds (SIFs), which give more flexibility to private equity and hedge fund managers, as the investment strategies are not as limited as under the UCITS framework.

While UCITS funds are also used in Ireland, ICAVs are also available. The Irish Collective Asset Management Vehicle (ICAV), launched by new regulations in 2015, is rapidly attracting assets. ICAVs have a goal of allowing very flexible investment vehicles and an ease of structuring new funds under simplified compliance rules. "ICAV: Irish Collective Asset-Management Vehicle," Deloitte, 2015. This flexibility allows for both open-end and closed-end structures and may be designed to be tax compliant for U.S. investors.

In the Cayman Islands, funds may be structured in a variety of ways, including as limited companies, unit trusts, or limited partnerships. Exempted vehicles are typically managed outside of the Cayman Islands, even though they are domiciled and regulated in Cayman. Cayman vehicles tend to require a minimum investment of US\$100,000, which limits the ownership to larger and hopefully sophisticated investors. The Cayman Islands Monetary Authority (CIMA) does not place restrictions on the transparency, leverage, or other investment risk features of fund vehicles. " "Establishing Investment Funds in the Cayman Islands," Deloitte, May 2018. Fund directors and service providers are often located in Cayman, whereas fund management activities take place in the home country of the fund manager. Master-feeder structures are one of the most common features found in Cayman funds.


\end{document}