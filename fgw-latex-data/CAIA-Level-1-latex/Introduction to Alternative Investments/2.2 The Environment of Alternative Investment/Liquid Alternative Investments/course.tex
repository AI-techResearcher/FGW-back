\documentclass[11pt]{article}
\usepackage[utf8]{inputenc}
\usepackage[T1]{fontenc}
\usepackage{amsmath}
\usepackage{amsfonts}
\usepackage{amssymb}
\usepackage[version=4]{mhchem}
\usepackage{stmaryrd}

\begin{document}
\section*{Reading}
Liquid Alternative Investments

As their name implies, liquid alternatives are investment vehicles that offer alternative strategies in a form that provides investors with liquidity through opportunities to sell their positions in a market. Many major alternative investments, such as private equity or hedge funds, have historically been illiquid and opaque private placements held by high-net-worth and institutional investors. Liquid alternative investments are innovative products that provide access for all investors to the same or similar strategies in an exchange-traded and transparent format.

But the nature of the liquidity offered by liquid alternatives might better be described as "offering retail access" rather than "being able to be converted into cash quickly," the reason being that many alternatives, such as managed futures funds and structured products, have offered daily liquidity for years but are not commonly viewed as liquid because the products have been predominantly accessible only to institutional and high-net-worth investors.

\section*{The Spectrum of Liquid Alternatives Products}
Liquid alternatives span a spectrum of alternative assets and strategies, with more innovations expected to emerge. A popular investment vehicle in the United States that illustrates liquid alternatives well is real estate investment trusts (REITs). REITs hold real estate as their underlying assets. Private REITs are common. But many REITs, especially the largest, are owned through publicly traded shares. The underlying assets of many large REITs are large private real estate properties, such as office buildings, retail properties, health care facilities, and apartment complexes. Large real estate properties are often owned by institutions, directly or through limited partnerships. REITs offer retail access of similar properties to large and small investors alike. Even though the underlying real estate properties are illiquid, the shares in the REITs offer investors high levels of liquidity. Many REITs also hold liquid real estate assets, such as mortgage securities. REITs are further discussed in the Real Estate Equity session.

Real estate in general and REITs in particular have been popular in the United States for so long that some experts may not view REITs as liquid alternatives. Many discussions of liquid alternatives focus on more recent innovations that provide liquid investment vehicles for small investors to obtain exposure to classic alternative investment strategies, such as hedge fund strategies. Specifically, these new liquid alternatives include the offering of hedge fund and managed futures strategies through liquid mutual funds, such as ' 40 Act funds in the United States and UCITS in the EU.

Liquid alternatives tend to have substantial fee structure differences, which are discussed later in this section. Liquid alternatives differ with the extent to which their investment strategies match the investment strategies of privately placed alternative investments. In this regard, there are five distinct types of liquid alternative funds:

\begin{enumerate}
  \item Unconstrained clones: These liquid funds follow virtually the same strategy as private placement products with underlying liquid assets, such as some hedge funds or managed futures funds.

  \item Constrained clones: These liquid funds implement a similar strategy as private placement products but are limited in risk exposure by leverage, concentration, or liquidity constraints.

  \item Liquidity-based replication products: These liquid funds are designed to mimic illiquid private placement investments, using liquid securities as proxies.

  \item Skill-based replication products: These liquid funds are designed to mimic a highly skilled private placement strategy using a simplified and more mechanical strategy.

  \item Absolute return or diversified products: These liquid funds are designed to offer absolute returns and/or diversifying returns not directly related to opportunities historically available in private placements and potentially inconsistent with alternative strategies as typically deployed.

\end{enumerate}

The last category refers to products being touted as liquid alternatives that are long-only mixes of traditional investment strategies that offer returns that have exhibited relatively low correlation with the overall market. These products lack the innovation, leverage, short positions, illiquidity, and skill-based active trading that have been the hallmark of alternative investment for decades. They tend to be offered by institutions with expertise in traditional investments that are responding to investor preferences for investment products that offer diversification relative to traditional equity and bond markets.

\section*{Growth and Growth Factors in Liquid Alternatives}
Prior to the financial crisis of 2007-09, global assets under management in liquid alternatives totaled less than $\$ 100$ billion. The performance success of some alternative investment strategies during the financial crisis, such as managed futures and global macro funds, led retail investors to welcome the opportunity to diversify into those strategies and other alternative investment strategies as retail products became widely available.

By 2018, liquid alternatives had soared to over $\$ 800$ billion in global AUM. Nevertheless, the proportion of assets in mutual funds that is devoted to alternative assets is only a few percentage points.

Continued growth could be driven by two primary factors. First, retail investors may continue to diversify into alternative strategies to lessen their percentage exposure to traditional stock and bond strategies. Second, the shift of retirement assets from a focus on defined benefit plans to defined contribution may mean that retail access to alternative investments will increase.

\section*{Three Constraints against Achieving Alternative Investment Benefits through Liquid Products}
Some alternative investment strategies appear unable to be implemented through liquid retail structures, such as U.S. mutual funds or UCITS funds. There are three primary constraints.

\begin{enumerate}
  \item Leverage: The sophisticated hedge fund strategies discussed in Topic 5 often require substantial use of leverage, which is restricted within U.S. mutual funds by regulation. Specifically, there is a $300 \%$ asset coverage rule that requires a mutual fund to have assets totaling at least three times the total borrowings of the fund, thus limiting borrowing to $33 \%$ of assets. UCITS restrictions are even tighter.

  \item There are regulatory constraints on concentration (i.e., lack of diversification).

  \item There are illiquidity constraints (e.g., no more than $15 \%$ of a ' 40 Act fund or $10 \%$ of a UCITS fund can be invested in illiquid assets) that prevent substantial inclusion of private equity in open-end mutual funds.

\end{enumerate}

These three regulatory issues are a primary reason why such alternative investments are organized through private placements. It should be noted that to qualify as a private placement vehicle, funds are severely limited as to the number of investors permitted. The severe limits on the number of investors lead fund managers to require large initial investment sizes, which steer the products away from small retail investors and toward large institutional investors.

Other hedge fund strategies appear quite tractable for delivery through retail products. For example, the returns of managed futures funds and hedge funds holding other liquid underlying assets can easily be delivered through retail products as long as the strategies do not require high leverage or concentration. The Funds of Hedge Funds session discusses the creative ways that multialternative mutual funds can be structured so as to facilitate the delivery of a large subset of hedge fund strategies through retail products.

A highly researched and debated approach to delivering hedge-fund-like strategies without necessarily using sophisticated management teams or illiquid securities is hedge fund replication. Hedge fund replication is the attempt to mimic the returns of an illiquid or highly sophisticated hedge fund strategy using liquid assets and simplified trading rules.

Another method of delivering alternative investment strategies through retail vehicles is the use of a closed-end mutual fund structure. Closed-end mutual fund structures provide investors with relatively liquid access to the returns of underlying assets even when the underlying assets are illiquid.

\section*{Four Factors Determining Performance of Liquid Alternatives Compared to Private Placements}
Liquid alternatives are relatively new products with limited historical return data. Accordingly, there is especially high uncertainty with regard to the extent to which liquid alternatives will generate return enhancement or diversification benefits comparable to the results achieved in the past for institutional investors in private placements.

Returns from private placement vehicles and liquid alternatives may differ primarily due to four important factors, two of which relate to investment flexibility and two of which relate to fees:

\begin{enumerate}
  \item The permissible investment strategies differ. Private placements often enjoy important flexibility with regard to leverage (including the magnitude of short positions) and concentration (lack of diversification).

  \item Similarly, private placements may be able to generate higher returns due to their investment flexibility to hold more illiquid assets, thereby potentially receiving higher liquidity premiums.

  \item Fees differ between liquid alternatives and private placements. Liquid alternatives tend to have lower fees because most ' 40 Act funds do not have incentive fees, especially asymmetric incentive fees wherein managers benefit from sharing upside profits but are limited in their exposure to downside losses. UCITS funds and Canadian investment funds do allow asymmetric incentive fees.

  \item Managerial skill may differ. The higher potential fees from the asymmetric incentive fees of private placements may attract managers with greater skill. Some liquid alternative funds implement simplified trading rules rather than hiring sophisticated management teams. Note that three of the four factors favor private placement vehicles as being more likely to generate attractive returns. The only factor favoring liquid alternatives is lower fees.

\end{enumerate}

\section*{Empirical Analysis of Liquid Alternative Investment Performance}
Comparing the performance between private and public alternative funds in which the strategies match can be an effective way to estimate the risk and return differentials. A 2013 study by Cliffwater (discussed further in the Funds of Hedge Funds session) compared funds and concluded that, on average, liquid alternative funds have lower risks and slightly to moderately lower average returns than limited partnership (or LP) funds that employ the same strategy. "Performance of Private versus Liquid Alternatives: How Big a Difference?" Cliffwater, June 2013.

This brief overview of liquid alternatives lays a foundation for more detailed discussions on the underlying assets and investment strategies of the funds. Liquid alternatives are further discussed in the context of real assets in the sessions, Other Real Assets and Real Estate Assets and Debts, hedge funds in the Funds of Hedge Funds session, and private equity in the Private Equity Funds session.


\end{document}