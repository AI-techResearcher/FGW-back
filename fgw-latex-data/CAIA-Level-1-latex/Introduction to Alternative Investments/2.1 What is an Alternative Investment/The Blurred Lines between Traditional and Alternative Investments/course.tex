\documentclass[11pt]{article}
\usepackage[utf8]{inputenc}
\usepackage[T1]{fontenc}

\begin{document}
\section*{Reading}
The Blurred Lines between Traditional and Alternative Investments

The previous sections defined the category of alternative investments by describing the investments that are or are not commonly thought of as alternative. But the question remains as to what the defining characteristics of investments are that cause them to be classified as alternative. For example, why is private equity considered an alternative investment but other equities are considered traditional investments? What is the key characteristic or attribute that differentiates these equities? The answer is that traditional equities are listed on major stock exchanges whereas private equity is not. In this case, traditional equities possess the characteristic of being publicly traded, while private equity does not.

The lines between traditional and alternative assets are not distinct and universal. The next exhibit, The Blurred Lines Between Traditional and Alternative Assets depicts the four categories of alternative investments (in the left column), assets that are sometimes listed as alternative and sometimes as traditional (in the middle column), and assets viewed only as traditional (in the right column). The Blurred Lines Between Traditional and Alternative Assets illustrates the lack of clear lines between alternative and traditional assets. This curriculum will focus on those assets that are most universally described as alternative.

The Blurred Lines Between Traditional and Alternative Assets

\begin{center}
\begin{tabular}{|lll|}
\hline
Alternative Investments & Assets Often Characterized as Traditional or Alternative & Analogous Traditional Assets \\
\hline
Hedge funds & Liquid alternative mutual funds & Ordinary mutual funds \\
Private equity & Closed-end funds with illiquid holdings & Public equities \\
Real assets & Public real estate and public equities of corporations with performance dominated & Public equities with performance dominated by \\
 & by stable positions in real assets & managerial decisions \\
Complex structured & Simple structured products offering relatively stable and common risk and return & Simple derivatives used as part of a strategy with \\
products & characteristics & stable risk exposures \\
\hline
\end{tabular}
\end{center}

Note in The Blurred Lines Between Traditional and Alternative Assets that hedge fund-like returns are now available in publicly traded "liquid alternative" mutual funds. Not all hedge fund strategies are available through these public mutual funds-that is, the U.S. Investment Company Act of 1940 (the ' 40 Act) or Undertakings for Collective Investment in Transferable Securities (UCITS) funds-because of regulatory limits on leverage and illiquidity. So-called ' 40 Act (1940 Act) funds are those regulated under the U.S. Investment Company Act of 1940. Some alternative strategies are available through closed-end fund structures.

Private equity is inherently illiquid and generally is not available via ' 40 Act funds or other public investment pools, although several closed-end structures, such as business development corporations, hold private equity as their underlying investments. Closed-end fund structures can use modest amounts of leverage and illiquid underlying assets because the investment companies are not generally required to redeem investor shares on demand.

Among the category of real assets, real estate is most often characterized as both traditional and alternative, especially when the real estate is accessed through publicly traded investment pools, such as REITs. Some publicly traded common stocks with value primarily derived from holdings of natural resources, such as oil reserves, mineral rights, or land, are often classified as real assets. However, to the extent that a stock's value is driven by such managerial expertise as marketing, trading of assets, or technology, the returns will not be dominated by the values of underlying real assets and the stocks are more appropriately viewed as traditional operating firms.

Finally, the category of structured products varies from rather simple financial derivatives that are often classified as traditional investments (e.g., credit default swaps) to more complex derivatives, such as collateralized loan obligations, that are usually classified as alternative. Furthermore, some financial derivatives, such as futures contracts and forward contracts, can be used to replicate traditional asset exposures and thus clearly fall within the realm of traditional investing. For example, a portfolio of cash plus a long position in a forward contract on an equity index synthetically replicates a long position in the equities underlying the index. The decision of whether to classify a structured product as an alternative investment should be based on the extent to which the product offers nontraditional risk and return exposures and requires investment management methods that differ markedly from traditional investment management methods.


\end{document}