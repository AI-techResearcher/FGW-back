\documentclass[11pt]{article}
\usepackage[utf8]{inputenc}
\usepackage[T1]{fontenc}
\usepackage{amsmath}
\usepackage{amsfonts}
\usepackage{amssymb}
\usepackage[version=4]{mhchem}
\usepackage{stmaryrd}
\usepackage{hyperref}
\hypersetup{colorlinks=true, linkcolor=blue, filecolor=magenta, urlcolor=cyan,}
\urlstyle{same}

\title{Reading }

\author{}
\date{}


\begin{document}
\maketitle
A History of Alternative Investing: The U.S. Case

The next exhibit, Popular Institutional Quality Assets, 1890-Present provides a general overview of the investments typically held by institutional investors, such as banks, pension funds, endowments, and insurance companies. Throughout much of the twentieth century, each institutional-quality investment was evaluated primarily on the safety of its income and principal and tended to be evaluated on a standalone basis.

Popular Institutional Quality Assets, 1890-Present

\begin{center}
\begin{tabular}{ll|}
\hline
$1890-1920$ & Government debt, real estate, mortgages, preferred stock \\
$1920-1950$ & Add high-quality corporate bonds, domestic equities, agricultural debt \\
$1950-1980$ & Add average-quality corporate bonds, international equities \\
$1980-$ Present & Add high-yield debt, small stocks, structured products, private equity, hedge funds, real assets \\
\end{tabular}
\end{center}

Beginning in the 1950s and 1960s, modern portfolio theory established the mechanics and advantages of diversification. Modern portfolio theory evaluates risk on a portfolio basis-formalizing the idea that much risk can be diversified away by holding a broad mix of available investments. In the 1980s and 1990s, the appropriateness of investments for institutions increasingly began to be evaluated on a portfolio basis.

The change in law and investment practices from evaluating risk on a standalone basis to a portfolio-as-a-whole basis is evidenced in the previous exhibit.

Beginning in the 1980s, inclusion of such assets as small-company stocks, low-quality corporate bonds, and alternative assets became more common among financial institutions, such as banks, pension funds, endowment funds, and insurance companies. Evaluated on a standalone basis, many of these assets had little or no reliable income and were at risk for loss of the original investment. But when held in a portfolio, these relatively high-risk investments could lower the total risk of the portfolio because of their ability to provide improved diversification.

The previous exhibit indicates that institutions usually did not hold common stocks prior to 1920 . Most institutional-quality investments more than 100 years ago were those secured by tangible assets, such as real estate.

The underlying determinants of economic performance are changing with increasing speed. Take, for instance, the composition of the major stocks in the United States. In early 1901, the Dow Jones Industrial Average included 12 stocks: 10 common stocks and two preferred stocks. Almost every one of the 12 stocks was a commodity producer (copper, sugar, tobacco, paper, lead, coal, leather, rubber, and steel).

By 1960, the top seven Fortune 500 firms in the United States were General Motors Company, Standard Oil Company of New Jersey, Ford Motor Company, General Electric, U.S. Steel, Mobil, and Gulf Oil. The list included three oil companies and one steel company as well as two automobile manufacturers and one electrical equipment manufacturer.

Now, top firms are dominated by services and technology. The top five U.S. firms in terms of market capitalization in 2022 were Apple Inc, Microsoft Corporation, \href{http://Amazon.com}{Amazon.com} Inc, Tesla Inc, and Alphabet Inc. With the exception of Tesla, these firms are often viewed as asset-lite with inventory and fixed assets representing a small proportion of their balance sheets. Clearly, it is inappropriate to view traditional assets as solid and alternative assets as speculative.

Investments closely tied to commodity prices are now viewed as alternative investments, yet they constituted most of the industrial investment opportunities in 1900. With such dramatic and increasingly rapid changes in the components of an economy, it is difficult to conclude that conservative and traditional investment principles consist of maintaining unchanging investment practices. In effect, sticking with traditional investment practices moves the risk and diversification of a portfolio through time as the economy underlying the investment opportunities shifts. It is only through a dynamic approach to asset allocation (one that adjusts to new industries, securities, and other economic changes) that a portfolio can maintain good principles of diversification.


\end{document}