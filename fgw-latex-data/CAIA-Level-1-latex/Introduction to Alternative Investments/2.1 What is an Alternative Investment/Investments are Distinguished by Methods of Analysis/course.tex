\documentclass[11pt]{article}
\usepackage[utf8]{inputenc}
\usepackage[T1]{fontenc}

\begin{document}
\section*{Reading}
Investments are Distinguished by Methods of Analysis

The previous section outlined return characteristics of alternative investments that distinguished them from traditional investments: diversifying, illiquid, inefficient, and non-normal. Alternative investments can also be distinguished from traditional investments through the methods used to analyze, measure, and manage their returns and risks. As in the previous case, the reasons for the difference lie in the underlying structures: Alternative investments have distinct regulatory, securities, trading, compensation, and institutional structures that necessitate distinct methods of analysis. Public equity returns are extensively examined using both theoretical analysis and empirical analysis. Theoretical models, such as the capital asset pricing model, and empirical models, such as the Fama-French three-factor model, detailed later in the CAIA curriculum, are examples of the extensive and highly developed methods used in public equity return analysis. Analogously, theories and empirical studies of the term structure of interest rates and credit spreads arm traditional fixed-income investors with tools for predicting returns and managing risks. But alternative investments do not tend to have an extensive history of well-established analysis, and in many cases the methods of analysis used for traditional investments are not appropriate for these investments due to their structural differences.

Alternative investing requires alternative methods of analysis. In summary, a potential definition of an alternative investment is any investment for which traditional investment methods are clearly inadequate. There are four main types of methods that form the core of alternative investment return analysis.

\section*{Return Computation Methods}
Return analysis of publicly traded stocks and bonds is relatively straightforward, given the transparency in regularly observable market prices, dividends, and interest payments. Returns to some alternative investments, especially illiquid investments, can be problematic. One major issue is that in many cases, a reliable value of the investment can be determined only at limited points in time. In the extreme, such as in most private equity deals, there may be no reliable measure of investment value at any point in time other than at termination, when the investment's value is the amount of the final liquidating cash flow. This institutional structure of infrequent trading drives the need for different return computation methods.

Return computation methods for investments are driven by their structures and can include such concepts as internal rate of return (IRR), the computation of which over multiple time periods uses the size and timing of the intervening cash flows rather than the intervening market values. Also, return computation methods for many alternative investments may take into account the effects of leverage. While traditional investments typically require the full cash outlay of the investment's market value, many alternative contracts can be entered into with no outlay other than possibly the posting of collateral or margin or, as in the case of private equity, commitments to make a series of cash contributions over time. In the case of no investment outlay, the return computations may use alternative concepts of valuation, such as notional principal amounts. While IRR is commonly used in traditional investments, especially in the case of multiple cash contribution commitments, alternative investments such as private equity may use other methods. The Quantitative Foundations session provides details regarding return computation methods, such as IRR, that facilitate analysis of alternative investments. The sessions on private equity explain more commonly used methods, including Public Market Equivalent (PME).

\section*{Statistical Methods}
The traditional assumption of near-normal returns for traditional investments offers numerous simplifications. First, the entire distribution of an investment with normally or near-normally distributed returns can be specified with only two parameters: (1) the mean of the distribution, and (2) the standard deviation, or variance, of the distribution. Much of traditional investment analysis is based on the representation of an investment's return distribution using only the mean and standard deviation. Further, numerous statistics, tests, tables, and software functions are readily available to facilitate the analysis of a normally distributed variable.

But as indicated previously in this session, many alternative investments exhibit especially non-normally distributed returns over medium- and long-term time intervals. Non-normality is usually addressed through the analysis of higher moments of the return distributions, such as skewness and kurtosis. Accordingly, the analysis of alternative investments typically requires familiarity with statistical methods designed to address this non-normality caused by institutional structures like thin trading, securities structures like tranching, and trading structures like alternating risk exposures. An example of a specialized method is in risk management: While a normal distribution is symmetrical, the distributions of some alternative investments can be highly asymmetrical and therefore require specialized risk measures that specifically focus on the downside risks. The Measures of Risk and Performance session introduces some of these methods.

\section*{Valuation Methods}
Fundamental and technical methods for valuing traditional securities and potentially identifying mispriced securities constitute a moderately important part of the methods used in traditional investments. In traditional investments, fundamental equity valuation tends to focus on relatively healthy corporations engaged in manufacturing products or providing services, and tends to use methods such as financial statement analysis and ratio analysis. Many hedge fund managers use the same general fundamental and technical methods in attempting to identify mispriced stocks and bonds. However, hedge fund managers may also use methods specific to alternative investments, such as those used in highly active trading strategies and strategies based on identifying relative mispricings. For example, a quantitative equity manager might use a complex statistical model to identify a pair of relatively overpriced and underpriced stocks that respond to similar risk factors and are believed to be likely to converge in relative value over the next day or two. Additionally, alternative investing tends to focus on the evaluation of fund managers, while traditional investing tends to focus more on the valuation of securities.

Methods for valuing some types of alternative investments are quite distinct from the traditional methods used for valuing stocks and bonds. Here are several examples:

\begin{itemize}
  \item Alternative investment management may include analyzing active and rapid trading that focuses on shorter-term price fluctuations.
  \item Alternative investment analysis often requires addressing challenges imposed by the inability to observe transaction-based prices on a frequent and regular basis. The challenges in illiquid markets relate to determining data for comparison (i.e., benchmarking), since reliable market values are not continuously available.
  \item Alternative investments such as real estate, private equity, and structured products tend to have unique cash flow forecasting challenges.
  \item Alternative investments such as some intellectual property and private equity funds use appraisal methods that are estimates of the current value of the asset, which may differ from the price that the asset would achieve if marketed to other investors.
\end{itemize}

These specialized pricing and valuation methods are driven by the structures that determine the characteristics of alternative investments.

\section*{Portfolio Management Methods}
Finally, issues such as illiquidity, non-normal returns, and increased potential for inefficient pricing introduce complexities for portfolio management techniques. Most of the methods used in traditional portfolio management rely on assumptions such as the ability to transact quickly, relatively low transaction costs, and often the ability to confine an analysis to the mean and variance of the portfolio's return.

In contrast, portfolio management of alternative investments often requires the application of techniques designed to address such issues as the non-normality of returns and barriers to continuous portfolio adjustments. Non-normality techniques may involve skewness and kurtosis, as opposed to just the mean and variance. In traditional investments, the ability to transact quickly and at low cost often allows for the use of short-term time horizons, since the portfolio manager can quickly adjust positions as conditions change. The inability to trade some alternative investments, like private equity, quickly and at low cost adds complexity to the portfolio management process, such as liquidity management, and mandates understanding of specialized methods. Finally, alternative investment portfolio management tends to focus more on the potential for assets to generate superior returns.


\end{document}