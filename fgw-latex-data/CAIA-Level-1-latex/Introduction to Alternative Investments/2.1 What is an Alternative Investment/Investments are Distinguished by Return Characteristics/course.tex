\documentclass[11pt]{article}
\usepackage[utf8]{inputenc}
\usepackage[T1]{fontenc}

\begin{document}
\section*{Reading}
Investments are Distinguished by Return Characteristics

A popular way of distinguishing between traditional and alternative investments is by their return characteristics. Investment opportunities exhibiting returns that are substantially distinct from the returns of long only position in traditional stocks and bonds might be viewed as being alternative investments. Stock returns in this context refer to the returns of publicly traded equities; similarly, bond returns refer to the returns of publicly traded fixed-income securities.

\section*{Diversification}
An investment opportunity with returns that are uncorrelated with or only slightly correlated with traditional investments is often viewed as an alternative investment. An attractive aspect of this lack of correlation is that it indicates the potential to diversify risk. In this context, many alternative investments are referred to as diversifiers. A diversifier is an investment with a primary purpose of contributing diversification benefits to its owner. Absolute return products are investment products viewed as having little or no return correlation with traditional assets, and have investment performance that is often analyzed on an absolute basis rather than relative to the performance of traditional investments. The term absolute returns in this context should not be confused with the mathematical use of the term absolute value to indicate a numerical value that is always nonnegative. Diversification can lower risk without necessarily causing an offsetting reduction in expected return and is therefore generally viewed as a highly desirable method of generating improved risk-adjusted returns.

Sometimes alternative assets are viewed as synonymous with diversifiers or absolute return products. But clearly most types of investments, such as private equity, REITs, and particular styles of hedge funds, have returns that are at least modestly correlated with public equities over medium-to long-term time horizons and are still viewed as alternative investments. Accordingly, this non-correlation-based view of alternative investments does not provide a precise demarcation between alternative and traditional investments. Nevertheless, having distinct returns is often an important characteristic in differentiating alternative investments from traditional investments.

Alternative investments may be viewed as being likely to have return characteristics that are different from stocks and bonds, as demonstrated by their lack of correlation with stocks and bonds. The distinctions between traditional and alternative investments are also indicated by several common return characteristics found among alternative investments that either are not found in traditional investments or are found to a different degree. The following three sections discuss the most important potential return characteristic distinctions.

\section*{Illiquidity}
Traditional investments have the institutional structure of tending to be frequently traded in financial markets with substantial volume and a high number of participants. Therefore, their returns tend to be based on liquid prices observed from reasonably frequent trades at reasonable levels of volume. Many alternative investments are illiquid. In this context, illiquidity means that the investment trades infrequently or with low volume (i.e., thinly). Illiquidity implies that returns are difficult to observe due to lack of trading, and that realized returns may be affected by the trading decisions of just a few participants. Other assets, often termed lumpy assets, are assets that can be bought and sold only in specific quantities, such as a large real estate project. Thin trading causes a more uncertain relationship between the most recently observed price and the likely price of the next transaction. Generally, illiquid assets tend to fall under the alternative investment classification, whereas traditional assets tend to be liquid assets. However, liquid assets can be found inside alternative investment structures such as hedge funds and structured products.

The risk of illiquid assets may be compensated for by higher returns. An illiquid asset can be difficult or expensive to sell, as thin volume or lockup provisions prevent the immediate sale of the asset at a price close to its potential sales value. The urgent sale of an illiquid asset can therefore be at a price that is considerably lower than the value that could be obtained from a long-term comprehensive search for a buyer. Given the difficulties of selling and valuing illiquid investments, many investors demand a risk premium, or a price discount, for investing in illiquid assets. While some investors may avoid illiquid investments at all costs, others specifically increase their allocation to illiquid investments in order to earn this risk premium.

\section*{Inefficiency}
The prices of most traditional investments are determined in markets with relatively high degrees of competition and therefore with relatively high informational efficiency. In this context, competition is described as numerous well-informed traders able to take long and short positions with relatively low transaction costs and with high speed. Efficiency (i.e., informational efficiency) refers to the tendency of market prices to reflect all available information.

Efficient market theory asserts that arbitrage opportunities and superior risk-adjusted returns are more likely to be identified in markets that are less competitively traded and less informationally efficient. (Market efficiency is detailed in the Financial Economics Foundations session.) Many alternative investments have the institutional structure of trading at inefficient prices. Inefficiency refers to the deviation of actual prices from valuations that would be anticipated in an efficient market. Informationally inefficient markets are less competitive, with fewer investors, higher transaction costs, and/or an inability to take both long and short positions. Accordingly, alternative investments may be more likely than traditional investments to offer returns driven by pricing inefficiencies.

\section*{Non-Normality}
To some extent, the returns of almost all investments, especially the short-term returns on traditional investments, can be approximated as being normally distributed. The normal distribution is the commonly discussed bell-shaped distribution, with its peaked center and its symmetric and diminishing tails. The return distributions of most investment opportunities become nearer to the shape of the normal distribution as the time interval of the return computation nears zero and as the probability and magnitude of jumps or large moves over a short period of time decrease. However, over longer time intervals, the returns of many alternative investments exhibit non-normality, in that they cannot be accurately approximated using the standard bell curve. The non-normality of medium-and long-term returns is a potentially important characteristic of many alternative investments.

What structures cause non-normality of returns? First and foremost, many alternative investments are structured so that they are infrequently traded; therefore, their market returns are measured over longer periods of time. These longer time intervals combine with other aspects of alternative investment returns to make alternative investments especially prone to return distributions that are poorly approximated using the normal distribution. These irregular return distributions may\\
arise from several sources, including (1) securities structuring, such as with a derivative product that is nonlinearly related to its underlying security or with an equity in a highly leveraged firm, and (2) trading structures, such as an active investment management strategy alternating rapidly between long and short positions.

Non-normality of returns introduces a host of complexities and lessens the effectiveness of using methods based on the assumption of normally distributed returns. Many alternative investments have especially non-normal returns compared to traditional investments; therefore, the category of alternative investments is often associated with non-normality of returns.


\end{document}