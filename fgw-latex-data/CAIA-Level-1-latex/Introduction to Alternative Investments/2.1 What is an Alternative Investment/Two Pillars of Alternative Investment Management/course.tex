\documentclass[11pt]{article}
\usepackage[utf8]{inputenc}
\usepackage[T1]{fontenc}
\usepackage{multirow}

\DeclareUnicodeCharacter{00D7}{$\times$}

\begin{document}
\section*{Reading}
Two Pillars of Alternative Investment Management

For well over a century, the set of assets deemed to be traditional institutional assets has changed dramatically. Institutional investors must decide which new asset classes to include in their portfolio and when to include those assets. Conservatism in allocating to "new" institutional investment classes runs the risk of missing out on improved diversification and perhaps missing stellar early first-mover returns (i.e., high returns resulting from institutions pouring new money into an asset class that is increasingly viewed as appropriate for them to hold). Boldly venturing into asset classes not previously included in institutional portfolios, however, also exposes institutions to the risk of underperforming their more conservative peers.

The challenge for an institutional investor is to decide, as skillfully as possible, which new types of assets to include in a portfolio and which to exclude. How does an investor make such difficult decisions? This section advocates relying on two pillars: empirical analysis and economic reasoning.

\section*{Empirical Analysis}
Investment literature abounds with the warning that "future investment performance" should not "be inferred from or predicted based on past investment performance." This overstated warning is taken from Rule 156 of the Securities Act of 1933. In practice, much-if not most-investment analysis and decision-making is ultimately based on historical risk-adjusted performance. For example, it is primarily through historical observation that investment managers have developed opinions on the extent to which investing in equities differs from investing in bonds.

Conversely, the empirical methods used to explain the performance of alternative investments tend not to be as reliable and developed as the methods used to evaluate the performance of traditional investments. Investment professionals seek investments that can enhance the risk-adjusted performance of portfolios. But with huge numbers of potential strategies and powerful tools to backtest performance, it is risky to select opportunities based on empirical analysis alone.

\section*{Economic Reasoning}
Historical analysis alone is insufficient for determining asset allocations. For example, in the late 1990s the performance of U.S. growth stocks was consistently and strongly positive. The outstanding performance of this sector generated historic returns with extremely attractive statistics: high mean returns, very low variances, and virtually no major drawdowns. The empirical results were so uniformly positive that they led one major investment research firm specializing in mutual funds to assign attractively low risk ratings to many U.S. equity growth funds-then came the dot-com crash of 2000 , when plummeting values of growth stock funds resulted in huge losses for their investors.

Investors should be especially skeptical of empirical analyses that sound too good to be true. Economic reasoning can serve as a reliable reality check. Does solid theory support the contention that a particular asset will enhance risk-adjusted return? The addition of any new type of alternative investment into a well-diversified portfolio should be supported to the greatest extent possible by both empirics and theory.

Philosophers debate the two major approaches to the acquisition of knowledge (i.e., theory and empirics). Rationalists argue that most or all knowledge is ultimately understood through reasoning. Empiricists argue that reasoning is derived from observation. We would argue that both are needed.

The crux of the matter is that best practices in alternative investing include striving to make decisions that are supported by both sound analysis of past performance data and careful economic reasoning. This goal is executed with a balance of the two pillars: (1) evidence based on objective analysis of empirical data and (2) evidence based on an unbiased assessment of theories based on economic reasoning.

\section*{Viewing Alternative Assets in a 2×2 Framework}
The four major categories of alternative investments (real assets, hedge funds, private equity/credit, and structured products) can be roughly viewed along two dimensions, as illustrated in the next exhibit, A 2×2 Framework of Alternative Assets.

A 2×2 Framework of Alternative Assets

\begin{center}
\begin{tabular}{|c|c|c|c|}
\hline
 & \multicolumn{2}{|c|}{Trading Characteristics} &  \\
\hline
 &  & Publicly Traded & Privately Traded \\
\hline
\multirow{3}{*}{Primary Goal} & Enhanced Returns & Hedge Funds & Private Equity/Credit \\
\hline
 &  &  &  \\
\hline
 & Diversification/Risk Mgmt. & Structured Products & Real Assets \\
\hline
\end{tabular}
\end{center}

The distinction between publicly and privately traded assets along the horizontal dimension is rather straightforward. However, the distinction along the vertical dimension is less definitive. The top two boxes contain assets that tend to be selected more for their potential ability to deliver enhanced returns or alpha (i.e., to serve as return enhancers). In other words, superior risk-adjusted return is often the primary goal of these assets. The bottom two boxes contain assets that tend to be favored to a substantial degree by their ability to diversify a portfolio (i.e., diversifiers) or to serve as a tool to manage risk.

The distinction between the return enhancers and the diversification/risk tools is not sharp-most alternative assets are generally regarded as offering degrees of both return enhancement and risk management. Nevertheless, the previous exhibit can serve as a discussion point for the crucial decision of asset allocation.


\end{document}