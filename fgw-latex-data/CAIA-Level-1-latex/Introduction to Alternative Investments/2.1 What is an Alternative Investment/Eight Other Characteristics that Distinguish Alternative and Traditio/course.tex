\documentclass[11pt]{article}
\usepackage[utf8]{inputenc}
\usepackage[T1]{fontenc}

\begin{document}
\section*{Reading}
Eight Other Characteristics that Distinguish Alternative and Traditional Investments

Previous sections discuss differentiating alternative and traditional investments by their return characteristics and the methods by which they are analyzed. This section discusses eight additional characteristics or factors that often play a role in distinguishing alternative and traditional investments.

\begin{enumerate}
  \item Regulatory factors in the context of investing refer to the role of government, including both regulation and taxation, in influencing the nature of an investment. For example, hedge funds (but not their managers) are often less regulated and typically must be formed in particular ways to avoid higher levels of regulation. Taxation is another important feature of government influence that can motivate the existence of some investment products and plays a major role in the transformation of underlying asset cash flows into investment products.

  \item Structuring refers to the partitioning of claims to cash flows through leverage and securitization. Securitization is the process of transforming asset ownership into tradable units. Cash flows may be securitized simply on a pass-through basis (i.e., a pro rata or pari passu basis). Cash flows can also be partitioned into financial claims with different levels of risk or other characteristics, such as the timing or taxability of cash flows. The use of securities and security structuring transforms asset ownership into potentially distinct and diverse tradable investment opportunities. The nature of this transformation drives and shapes the nature of the resulting investments, the characteristics of the resulting returns, and the types of methods that are needed for investment analysis. On the other hand, lack of easily tradable ownership units can drive the selection and implementation of investment methods.

  \item Trading strategies refer to the role of an investment vehicle's investment managers in developing and implementing trading strategies that alter the nature of an investment. A buy-and-hold management strategy will have a minor influence on underlying investment returns, while an aggressive, complex, fast-paced trading strategy can cause the ultimate cash flows from a fund to differ markedly from the cash flows of the underlying assets. The trading strategy embedded in an alternative asset such as a fixed-income arbitrage hedge fund is often the most important factor in determining the investment's characteristics.

  \item Compensation structures refer to the ways that organizations distribute rewards. In the case of a hedge fund, compensation structures would include the financial arrangements contained in the limited partnership formed by the investors and the entity used by the fund's managers. Such arrangements usually determine the exposure of the fund's investment managers to the financial risk of the investment, the fee structures used to compensate and reward managers, and the potential conflicts of interest between parties. Compensation structures within investments, especially alternative investments, have implications for the agency costs generated by owner-manager relationships.

  \item Institutional factors refer to the financial markets (and their policies, such as restrictions on short selling, leverage, and trading) and financial institutions related to a particular investment, such as whether the investment is publicly traded. Public trading or the listing of a security is an essential driver of an investment's nature. For example, some hotels are owned by investors as shares of publicly traded corporations, such as Hyatt and Marriott, which are usually considered to be traditional investments. Other hotels, such as those owned by investors as real estate investment trusts (e.g., Host Hotels \& Resorts Inc.) and those held privately (e.g., Omni Hotels), are usually considered to be alternative investments. Other institutional structures can determine whether an investment is regularly traded, is held by individuals at the retail level, or tends to be traded and held by large financial institutions such as pension funds and foundations.

  \item Information asymmetries refer to the extent to which market participants possess different data and knowledge. In traditional investments, most securities are regulated and are required to disclose substantial information to the public. Many alternative investments are private placements, and therefore the potential for large information asymmetries is greater. These information asymmetries raise substantial issues for financial analysis and portfolio management.

  \item Incomplete markets refer to markets with insufficient distinct investment opportunities. The lack of distinct investment opportunities can prevent market participants from implementing an investment strategy that satisfies their exact preferences, such as their preferences regarding risk exposures. In an ideal world, securities could be costlessly created to meet every investor need. For example, an investor may desire an insurance contract that contains a specific clause regarding payouts, but regulations may prohibit such clauses. Or perhaps a contract with regard to a potential risk may be subject to unacceptable moral hazard. Moral hazard is risk that the behavior of one or more parties will change after entering into a contract. As a result of this inability to contract efficiently, the investor might be unable to diversify perfectly. Trading restrictions in some alternative investments, such as large minimum investment sizes, can be viewed as exacerbating the problem of incomplete markets and the investment challenges that accompany them.

  \item Innovation is the application of creativity. In this context, innovation has three major influences. First, especially in venture capital, innovation from nascent enterprises raises challenges regarding methods of cash flow projections, financial analysis, and portfolio management that distinguish the study of alternative investments from the study of traditional investments. Second, substantial degrees of innovation permeate the institutions surrounding alternative investments such as new structured products and derivatives. These innovations tend to inject new opportunities and challenges more in alternative investing than in traditional investing. Finally, innovation often serves as the source of superior returns in alternative assets, especially private equity.

\end{enumerate}

\end{document}