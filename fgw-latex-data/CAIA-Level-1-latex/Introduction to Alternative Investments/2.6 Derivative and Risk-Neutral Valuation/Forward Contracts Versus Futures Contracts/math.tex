\documentclass[11pt]{article}
\usepackage[utf8]{inputenc}
\usepackage[T1]{fontenc}
\usepackage{amsmath}
\usepackage{amsfonts}
\usepackage{amssymb}
\usepackage[version=4]{mhchem}
\usepackage{stmaryrd}

\begin{document}
\section*{APPLICATION A}
Futures contracts on crude oil are often denominated in 1,000-barrel sizes. In other words, each contract calls for the holder of a short position at the delivery date of the futures contract to deliver to the long side 1,000 barrels of the specified grade of oil using stated delivery methods. Assume that a trader establishes a long position of five contracts in crude oil futures at the then-current futures market price of $\$ 100$ per barrel. Both the trader on the long side of the contract and the trader on the short side of the contract post collateral (margin) of, say, $\$ 10$ per barrel. At the end of the day, the market price of the futures contract falls to $\$ 99$. How much money will each side of the contract have (assuming that the required collateral was the only cash and that there were no other positions)?

\section*{Answer and Explanation}
The difference between the current futures market price of $\$ 100$ and the future futures market price of $\$ 99$ is $\$ 1$. Since each contract represent 1,000 barrels, and the long side purchased 5 contracts. Therefore, the long side has 5,000 barrels of crude oil that has moved against its long position by $\$ 1$ each. The long position lost 5,000 multiplied by $\$ 1$ or $\$ 5,000$. The short position is exactly the opposite so it gained $\$ 5,000$. The futures margin is $\$ 10$ per barrel. So each position, both long and short, posted $\$ 10$ multiplied by 5,000 or $\$ 50,000$ of margin. Since we already determined that this futures price move to $\$ 99$ from $\$ 100$, impacted each position by $\$ 5,000$, we know that the long position decreased by $\$ 5,000$ to $\$ 45,000$. While the short position increase by $\$ 5,000$ to $\$ 55,000$.

\section*{APPLICATION B}
To lock in sales prices for its anticipated production, HiHo Silver Mining Company wishes to take short positions in five silver futures contracts, settling in each quarter for the next four quarters (20 contracts total).

\section*{Answer and Explanation}
The initial margin requirement is $\$ 11,000$ per contract. HiHo Silver Mining Company is purchasing 20 futures contracts. The product of 20 by $\$ 11,000$ is $\$ 220,000$, which is the collateral needed to establish the position.

\section*{APPLICATION C}
Returning to the previous example of an oil trader with a long position of five contracts established at an initial futures price of $\$ 100$ per barrel, the five contracts call for delivery of 5,000 barrels (five contracts $x 1,000$ barrels). The trader posts exactly the required initial margin of $\$ 50,000$ ( $\$ 10,000$ per contract). Suppose that the maintenance margin requirement is $\$ 25,000$ ( $\$ 5,000$ per contract) and that the price of oil drops $\$ 6$ per barrel. What is the trader's margin balance after the price decline? Also, describe any margin call that might be made and what it would require.

\section*{Answer and Explanation}
The trader's margin balance is $\$ 20,000$. We arrive at that answer by multiplying the price drop, $(\$ 6)$, by 5,000 barrels (the quantity in barrels covered by the 5 futures contracts of 1,000 barrels). Now, add $(\$ 30,000)$ to the $\$ 50,000$ required initial margin balance for a sum of $\$ 20,000$. Since the margin balance is below the maintenance margin requirement of $\$ 25,000$, the trader would receive a margin call, which would require an additional $\$ 30,000$ to bring the margin back to the initial margin requirement.


\end{document}