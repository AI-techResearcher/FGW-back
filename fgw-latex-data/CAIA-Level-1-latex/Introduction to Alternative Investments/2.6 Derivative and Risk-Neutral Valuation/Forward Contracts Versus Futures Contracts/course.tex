\documentclass[11pt]{article}
\usepackage[utf8]{inputenc}
\usepackage[T1]{fontenc}
\usepackage{amsmath}
\usepackage{amsfonts}
\usepackage{amssymb}
\usepackage[version=4]{mhchem}
\usepackage{stmaryrd}

\begin{document}
\section*{Reading}
Forward Contracts Versus Futures Contracts

Forward contracts were discussed in moderate detail in the previous lessons in this session. This section discusses futures contracts on underlying financial assets and physical assets (i.e., commodities).

Both forward contracts and futures contracts are binding agreements for the purchase or sale of an asset but with deferred exchanges of the asset and the cash.

\section*{Trading Differences between Forward Contracts and Futures Contracts}
In introductory material, the terms forward contract and futures contract are often used interchangeably due to their similarities: The hallmark of both contracts is the deferred delivery, and both contracts are priced with similar principles. One major distinction between the two is that forward contracts are typically over-thecounter (OTC) contracts, whereas futures contracts are exchange traded. Since futures contracts are traded on an organized exchange, they share the same advantages as other listed securities: a central marketplace and transparent pricing. Compared to most forward contracts, futures contracts also enjoy clearinghouse security, uniform contract size and terms, and daily liquidity.

Forward contracts are ad hoc contracts negotiated between two parties, with flexibility regarding the details to help meet the needs and preferences of each party. As exchange-traded contracts, futures contracts are standardized. Each futures contract trades with a relatively high degree of uniformity with regard to the quantity and quality of the underlying asset and the location and time of delivery.

The standardization of futures contracts permits active trading and liquidity. At any point in time, the long futures position holder can close a position by establishing an identical short position (so that the long position and short position net to zero). Similarly, the short futures position can close a position by entering an offsetting long position. The outstanding quantity of unclosed contracts is known as open interest. If the buyer (i.e., long) of the futures contract does not wish to take delivery of the underlying asset, the buyer closes out the long futures position at the prevailing market price of the contract by taking on a short position. Similarly, if the holder of a short position does not wish to deliver the underlying asset, the holder can establish an offsetting long position prior to delivery. Only a very small percentage of futures contracts (usually less than 1\%) result in delivery of the underlying asset. The point is that the primary purpose of futures (and forward) contracts is to exchange risks, rather than to serve as vehicles for arranging physical transfers of goods. The idea is that by using futures markets to manage risk, a party can take or make delivery of physical goods using the cash market with the lowest transportation or other costs.

Forward contracts are over-the-counter contracts between two parties that contain the terms and conditions agreed on by the two parties. These terms and conditions include how much, if any, collateral is required; the size of the contract; and the delivery details (including time, quality, and location). Since the contracts are not standardized, there are usually no market prices that can be observed to directly value the position. If the holder of a long or short position in a forward contract wishes to terminate or hedge the exposure, there is no ready secondary market of identical contracts available. The entity wishing to terminate the exposure to a forward may attempt to negotiate an exit with the counterparty to the forward or establish a new forward contract with another party, which will serve to offset the risk. Whereas long and short positions in the same futures contract will close a position, the same is not true for forward contracts. Because forward contracts are specific to a given counterparty, a transaction can only be closed with the same counterparty. Although a long and short forward position with two different counterparties will neutralize market exposure, counterparty risk remains. Nevertheless, the flexibility of forward contracts makes them very popular. The most prominent forward market is the currency forward market, which is substantially more liquid than the currency futures market.

Many of the distinctions between forward and futures contracts may disappear over time. Due to the Dodd-Frank Act in the United States and new regulations throughout the world, market structures are changing. If OTC markets are required to offer greater transparency and participate in a central clearing system, forwards will become more like exchange-traded futures contracts.

\section*{The Mechanics of Marking-to-Market}
A critical distinction between most futures and forward contracts is that futures contracts are marked-to-market. The term marked-to-market means that the side of a futures contract that benefits from a price change receives cash from the other side of the contract (and vice versa) throughout the contract's life. The cash exchanges resulting from positions being marked-to-market are intended to cause each side of the derivative to have a zero market value at the end of each day. The reason that each contract has a zero value at the end of the day is that the price at which the commodity is promised to be delivered is adjusted to the current futures price as a result of the marking-to-market process.

The following example provides a closer examination of the process of marking-to-market. Consider a trader who establishes a long position in a gold futures contract at $€ 1,000$ per ounce on Monday morning. The trader has promised to buy gold for $€ 1,000$ per ounce unless the trader closes the position by establishing a short position that offsets the original long position prior to the required delivery date. Suppose that the gold futures contract rises in price to close on Monday afternoon at $€ 1,005$ per ounce. In effect, the futures exchange collects $€ 5$ per ounce from the trader who established the short position and delivers $€ 5$ per ounce into the account of the trader who established the long position. Now the futures contract calls for delivery of the gold at $€ 1,005$ per ounce. Suppose that on Tuesday the futures contract falls to $€ 998$ per ounce. The exchange then takes $€ 7$ per ounce out of the account of the trader with the long position and delivers $€ 7$ per ounce to the trader with the short position (assuming that they both continue to hold their respective positions). The contract would then be changed to call for delivery of the gold at $€ 998$ per ounce.

The process continues each day until delivery day. Suppose that at the delivery date the price of gold has risen to $€ 1,500$. The holder of the long position must now pay $€ 1,500$ per ounce for the gold. But recall that the trader entered a contract to buy gold at $€ 1,000$, not $€ 1,500$. The final economic result is accomplished because, throughout the life of the contract, there was a net transfer of $€ 500$ per ounce from the short side of the contract to the long side of the contract through the markingto-market process as the closing futures price of gold rose from $€ 1,000$ per ounce to $€ 1,500$ per ounce. The long position effectively combines the $€ 500$ of marked-tomarket profit with the original promise to pay $€ 1,000$ and delivers $€ 1,500$ in exchange for the gold. The short position effectively nets the $€ 500$ loss accrued from marking-to-market from the $€ 1,500$ received at delivery to receive the promised net value of $€ 1,000$ per ounce.

The net result is the same: Both sides of the trade perform as originally promised unless one or both close their positions prior to delivery.

An exchange-traded futures contract can be viewed as a forward contract that is settled in cash at the end of each day (i.e., marked-to-market) and then restruck at the prevailing price for new futures contracts. Thus, the long position in the first example began with a contract to buy gold at $€ 1,000$ per ounce and ended with a contract to buy gold at $€ 1,500$ per ounce. During the price move from $€ 1,000$ to $€ 1,500$, the holder of the long position in the contract received $€ 500$ from the holder of the short position. If the holder of the long position takes delivery of the gold at $€ 1,500$, the net cost will be the originally agreed-upon price of $€ 1,000$ (when the $€ 500$ of receipts from marking-to-market profits are included). Correspondingly, the short position holder delivers gold at $€ 1,500$ but nets only $€ 1,000$ after considering the mark-to market losses of $€ 500$. In advanced pricing models, the impact of interest rates on the marking-to-market process is included in the original pricing of the futures contract. In this discussion, these minor interest effects were ignored.

\section*{Marking-to-Market and Counterparty Risk}
Each side of a derivative contract refers to the other side of the contract as its counterparty to the contract. Forward contracts and, to a lesser extent, futures contracts expose each party to the risk that the counterparty holding the other side of the contract will default on its obligations. This risk of failure of the counterparty to perform contractual duties is known as counterparty risk and is discussed in greater detail in subsequent sessions.

The importance of the marking-to-market process is to avoid the counterparty risk known as the crisis at maturity. A crisis at maturity is when the party owing a payment is forced at the last moment to reveal that it cannot afford to make the payment or when the party obligated to deliver the asset at the original price is forced to reveal that it cannot deliver the asset. The key point is that the potential for a crisis at maturity creates uncertainty throughout the life of the contract when information is asymmetric. Rather, through the marking-to-market process, the party accruing an increasingly expensive obligation to the other party is forced each day to deliver the necessary funds or to reveal any financial problem.

Consider the previous example of a contract to deliver gold at $€ 1,000$. When the market price of gold soared from $€ 1,000$ to $€ 1,500$, the holder of an unhedged short position would be required to deliver the gold at a loss of $€ 500$. In the absence of a marking-to-market process, the holder of the long position would be incurring larger and larger counterparty risk as the price of gold soared. With marking-to-market, the short position would settle a portion of the loss each day that the price of gold rose, thus avoiding the crisis at maturity.

If a party does not have the financial resources to meet the requirements of daily marking-to-market, the party's position is closed into the market, and a new counterparty to the position takes over. Hence, daily marking of a position to market typically limits counterparty risk to one day's price movement.

During the marking-to-market process, financial settlement of the contract effectively takes place daily throughout the contract's life rather than simply at the delivery date. In essence, a long-term futures contract is a string of daily contracts that is restruck every day. Marking-to-market of exchange-traded futures contracts minimizes counterparty risk. In addition to the protection provided by the marking to-market process, the exchange's clearing mechanism combines capital from all exchange members to guarantee the trades of any individual members who may default on their obligations. However, the failure of a large futures commission merchant (FCM), such as Lehman Brothers Europe, could create counterparty risk, depending on the jurisdiction and the legal segregation of the assets.

As an OTC-traded product, forward contracts are not usually marked-to-market and are therefore subject to greater counterparty risk. Some market participants prefer the forward market because of the lack of a marking-to-market process. Although forwards have greater counterparty risk than futures do, corporate users may prefer to participate in the forward market to avoid the volatility that futures positions can create in a firm's cash flow and financial statements.

\section*{Marking-to-Market and the Time Value of Money Effect on Risk}
A critical difference between futures and forward contracts is that the marking-to market feature of futures contracts accelerates the receipt of profits and losses relative to forward contracts. This acceleration has two distinct effects: one on risk and the other on pricing.

Let's first examine the effect of marking-to-market on risk. Acceleration of cash flows due to marking-to-market is tantamount to higher price volatility and higher risk.

For example, consider the difference between being long a futures contract and being long a forward contract on oil. For simplicity, let's assume that although the contract is a one-year contract, due to an important announcement in the first week of the contract the price of oil will either rise by $\$ 10$ or fall by $\$ 10$ per barrel. A $\$ 10$ rise in the oil price in the first week generates a $\$ 10$ profit for the long side of either the futures contract or the forward contract. But the long side of the futures contract receives that $\$ 10$ profit in the form of cash during the first week through the marking-to-market process, whereas the long side of the forward contract receives the profits as cash at settlement in one year. If the price were to fall, the long side of the futures contract would pay $\$ 10$ in one week, whereas a forward contract payment for the loss would be deferred until delivery in one year.

The marking-to-market process effectively requires participants to pay as they go. Paying now rather than later increases the present value, and therefore futures contracts have higher price risk than otherwise identical forward contracts.

\section*{Marking-to-Market and the Time Value of Money Effect on Prices}
The second effect of the marking-to-market process can be to alter the market price of a futures contract relative to an otherwise identical forward contract. At inception, there should be no difference between the price of a futures contract and an otherwise identical forward contract if interest rate changes are uncorrelated with the spot price underlying the contracts.

To understand this complex issue, consider otherwise identical futures and forward contracts with underlying assets that contain no systematic risk and therefore offer no expected profit to the long position and no expected loss to the short position. Because of the marking-to-market process, the futures contract will generate daily cash flows between the long side and the short side as the futures price changes through time. The expected value of these cash flows is zero, since the underlying asset contains no systematic risk.

However, the expected discounted value of these cash flows will be positive to the long side of the contract if the interest rate is positively correlated with the spot price underlying the futures contract. If the interest rate and spot price are positively correlated, then the long position in the futures contract will receive cash flows from the marking-to-market process, which will be invested at a high interest rate (because high spot prices and high interest rates will tend to occur together). Conversely, the long side will deliver payments due to the marking-to-market process when the spot price falls, at which time the interest rate will tend to be low (due to the assumed positive correlation between spot prices and interest rates).

The net result is that with positive correlation between spot prices and interest rates, the long side of a futures contract tends to receive marking-to-market cash flows when interest rates move higher and tends to deliver marking-to-market cash flows when interest rates move lower. This asymmetric relationship, which tends to benefit the long side, forces the price of the futures contract above the price of an otherwise equivalent forward contract.

Conversely, with a negative correlation between spot prices and interest rates, the long side of a futures contract tends to deliver marking-to-market cash flows when interest rates move higher and tends to receive marking-to-market cash flows when interest rates move lower. This asymmetric relationship, combined with the opportunity cost of money, forces the price of the futures contract below the price of an otherwise equivalent forward contract when the spot price is negatively correlated with interest rates.

In summary, the price of a contract that is marked-to-market will be greater than, equal to, or less than the price of an otherwise identical contract that is not marked-to-market depending on whether interest rates are positively correlated, uncorrelated, or negatively correlated with the spot price of the contract's underlier.

\section*{Futures Trading and Initial Margin}
Market participants in futures contracts are required to make a collateral deposit of a size determined by the futures exchange. The collateral deposit made at the initiation of a long or short futures position is called the initial margin. This margin requirement is a small percentage of the full purchase price of the underlying commodity, usually less than $10 \%$. Margin requirements are set by the exchanges, are subject to change, and are expressed as currency per contract. For example, at a particular point in time, the initial margin requirement for each futures contract on silver might be $\$ 11,000$. This means that the entity initiating a long or short position in silver futures must have $\$ 11,000$ of available collateral per silver futures contract being traded to enter the order and establish the position. Thus, a jewelry-manufacturing firm wishing to take a long position in 10 silver contracts would have to have $\$ 110,000$ of available collateral to place the trade order.

The initial margin reduces counterparty risk by ensuring the payment of daily losses on futures market positions (except in the case of very extreme price movements). Any collateral deposits for forward contracts are determined through negotiations between the parties.

\section*{Marking-to-Market and Maintenance Margin}
When commodity prices change substantially, the promise of the long position to pay for delivery or the promise of the short position to make delivery could be placed in peril. To protect the integrity of the contracts, futures exchanges require that positions be marked-to-market, as discussed previously. After initiation of the position (which is done subject to initial margin requirements), market participants with open futures positions are subject to maintenance margin requirements. A maintenance margin requirement is a minimum collateral requirement imposed on an ongoing basis until a position is closed. Like the initial margin, the maintenance margin is expressed as units of currency per contract and is usually set at $75 \%$ to $80 \%$ of the initial margin. If the collateral of a market participant falls below the maintenance margin requirement, typically due to the marking-to-market of losses, a margin call is issued. A margin call is a demand for the posting of additional collateral to meet the initial margin requirement. If the investor cannot meet the margin call, the futures commission merchant has the right to liquidate the investor's positions in the account. (The positions may be closed at market prices without the investor's direction.) This daily process ensures that promises to make and take delivery have reduced counterparty risk.

Returning to the example of the jewelry manufacturer with a long position in 20 silver contracts, assume that the position was established at a futures price of $\$ 25$ per ounce and that each contract called for delivery of 5,000 ounces. Thus, the manufacturer has promised to buy 100,000 ounces ( 20 contracts) at $\$ 25$ per ounce, for a total purchase price of $\$ 2,500,000$. Now suppose that the market price of the futures contract drops from $\$ 25$ to $\$ 24$. As holder of a long position, the jewelry manufacturer has lost $\$ 1$ per ounce, and its position has dropped in value by $\$ 100,000$ (based on all 100,000 ounces underlying the 20 contracts). The futures exchange marks the position to market by transferring $\$ 100,000$ out of the account of the jewelry manufacturer and placing it into the accounts of entities with short positions in silver futures contracts. The silver manufacturer now has $\$ 100,000$ less cash in its account, but now its promise is to buy the silver at $\$ 24$ an ounce rather than $\$ 25$ per ounce.

Suppose that the jewelry manufacturer originally had only enough collateral to meet the initial margin requirement of $\$ 220,000$. After the $\$ 100,000$ loss due to the marking-to-market process, the account contains only $\$ 120,000$. If the required maintenance margin is not met, the jewelry manufacturer will receive a margin call and will be required to post an additional $\$ 100,000$ in collateral to return the account to meeting the initial margin requirement and to prevent a forced closure of its\\
positions. The process continues on a daily basis to provide assurances that each trader's obligations will be met. The exchange or the broker can alter margin requirements during a contract's lifetime, often in response to changes in past or anticipated volatility.

Futures contracts have other characteristics that differ from forward contracts, including transparent pricing and, usually, higher liquidity. Although these differences are often important, to focus on the basic principles of commodities futures, the remainder of this session generally ignores the distinction between forward and futures contracts, usually using the terms interchangeably.


\end{document}