\documentclass[11pt]{article}
\usepackage[utf8]{inputenc}
\usepackage[T1]{fontenc}
\usepackage{amsmath}
\usepackage{amsfonts}
\usepackage{amssymb}
\usepackage[version=4]{mhchem}
\usepackage{stmaryrd}

\begin{document}
\section*{Reading}
Forward Contracts on Rates

Forward contracts on interest rates and currency exchange rates facilitate risk management of funding costs and currency conversions by operating firms and others. They are also foundational to many alternative investment strategies that use the contracts to engineer desired risk exposures.

\section*{Forward Rate Agreements}
The Financial Economics Foundations session demonstrated how an investor with a long position in one zero-coupon bond and a short position in another zerocoupon bond could be viewed as having locked in a borrowing rate or a lending rate during the time interval between the maturities of the bonds (ignoring default risk). This section discusses financial derivatives that accomplish the same objective of locking in borrowing and lending rates (and currency conversion rates). Two common short-term rates that serve as the reference rates in forward rate agreements are LIBOR for U.S. dollars and Euribor for Euros. A reference rate is a market rate specified in contracts such as a forward contract that fluctuates with market conditions and drives the magnitude and direction of cash settlements.

A forward rate agreement (FRA) is a cash-settled contract in which one party agrees to offer a specified or fixed rate (the FRA rate), such as an interest rate on a specified principal amount and over a specified time in the future (or a currency exchange rate at a specified time in the future) while the other party agrees to provide that rate. In the case of a forward rate agreement on an interest rate, the payer in the contract effectively agrees to pay the interest rate over a specified time interval based on a specified notional amount while the receiver in the contract effectively agrees to receive that rate. The term notional principal is used to indicate that the principal amount is not actually exchanged, but rather serves to scale the size of the rate-related payments. The contract is settled with a payment based on the amount by which the FRA rate differs from the actual market rate at the time of settlement (the reference rate) multiplied by the notional amount.

The buyer of the FRA uses the FRA for protection against interest rate increases by locking in a fixed future rate (the FRA rate) to borrow (e.g., meet funding needs) rather than pay whatever subsequent borrowing rates occur (e.g., LIBOR). The FRA buyer receives cash at settlement when the reference rate exceeds the FRA rate and makes payment to the FRA seller when the reference rate is less than the FRA rate.

For example, consider a three-month FRA (to be settled in several years) with an FRA rate of $5 \%$ and a notional value of $\$ 1,000,000$. At the time of settlement the actual market interest rate (LIBOR) rises to 6\%. The FRA would require the FRA seller to pay the buyer $\$ 2,500$ because LIBOR, the reference rate, is above the FRA, found as follows: $[\$ 1,000,000 \times(3 \mathrm{mos} . / 12 \mathrm{mos}$.) $\times(6 \%-5 \%)]$. Had the reference rate fallen to $4 \%$, the buyer would pay the seller $\$ 2,500$. In practice, the cash settlement amount is usually based on a discounted value (i.e., the $\$ 2,500$ would be discounted for the length of the loan-three months in this example) and would be payable at the start of the period when the market rate is observed. The example illustrates the important ability of an FRA to allow financial entities to control their borrowing costs and lending revenues.

\section*{Forward Rate Agreements and Implied Forward Interest Rates}
The Financial Economics Foundations session detailed the concept of implying forward default-free interest rates from observing default-free spot rates. In a perfect market, the FRA rate will be equal to:


\begin{equation*}
F_{T-t}=\left[\left(T \times r_{T}\right)-\left(t \times r_{t}\right)\right] /(T-t) \tag{1}
\end{equation*}


In practice, continuous compounding is not used, nothing is truly default-free, and markets are not perfect. Nevertheless, Equation 1 works well as an approximation and it conveys the primary point of risk-neutral pricing of forward contracts: Initial forward contract prices and rates are not driven toward equaling expected future spot prices and rates. Rather, they are implied from spot rates. Equation 1 is quite intuitive. The forward rate is shown to be the difference between the longer-term interest rate and the shorter-term interest rate, with each rate being averaged over its longevity. For example, if the five-year rate is $5 \%$ and the four-year rate is $4 \%$, the forward rate on a one-year security settling in four years must be $9 \%$ (i.e., $25 \%-16 \%$ ).

\section*{Forward Rates and Their Extensions}
The point being made in this session is that given the prices (the previous section) or rates (this section) from the spot market for the underlying cash instruments (i.e., the prices of shorter-term and longer-term securities), there is only one arbitrage-free price or rate of the forward contract that spans the maturity dates. Arbitrage-free price and rate relationships will hold in a perfectly efficient market, but no market is perfectly efficient. When actual market prices deviate from arbitrage-free prices, investors may use skill-based strategies that attempt to earn superior profits by anticipating that relative prices will tend to revert toward their arbitrage-free levels. Relative value hedge fund strategies (discussed in the session entitled Relative Value Hedge Funds) are examples of such strategies.

Another type of financial derivative is a swap. A swap is a string of forward contracts grouped together that vary by time to settlement. Thus, a commodity swap is a portfolio of commodity forwards. Typically, the settlement times are equally spaced. For example, an oil refinery might regularly need to purchase crude oil. Rather than bear the risk of fluctuating oil prices, the refinery may decide to lock in the purchase price of the oil by entering various forward contracts to purchase the oil at prespecified prices (i.e., to swap cash for oil). Instead of entering into a series of separate forward contracts, the refinery may enter into a single swap that calls for\\
quarterly or monthly exchanges through time at prices set at the initiation of the swap. Swaps are discussed in greater detail in the Credit Risk and Credit Derivatives session.

Some alternative investment strategies use swaps to manage risk exposures. The risk of a swap is formed from the risk exposures of the forward contracts that comprise it. Thus, like forward contracts, interest rate swaps allow for the transfer of interest rate risk. Analogously, commodity swaps and currency swaps allow for the transfer of commodity risks and currency risks.


\end{document}