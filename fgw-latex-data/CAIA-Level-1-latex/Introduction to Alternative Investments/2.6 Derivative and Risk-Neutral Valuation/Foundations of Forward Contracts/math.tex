\documentclass[11pt]{article}
\usepackage[utf8]{inputenc}
\usepackage[T1]{fontenc}
\usepackage{amsmath}
\usepackage{amsfonts}
\usepackage{amssymb}
\usepackage[version=4]{mhchem}
\usepackage{stmaryrd}

\begin{document}
\section*{APPLICATION A}
Nine-month riskless securities trade for $\$ 97,000$, and 12 -month riskless securities sell for $\$ P$ (both with $\$ 100,000$ face values and zero coupons). A forward contract on a three-month, riskless, zero-coupon bond, with a $\$ 100,000$ face value and a delivery of nine months, trades at $\$ 99,000$.

What is the arbitrage-free price of the 12 -month zero-coupon security (i.e., $P$ )?

The 12-month bond must sell for $\$ 96,030$ to prevent arbitrage.

\section*{Answer and EXPLANATION}
The 12-month bond offers a ratio of terminal wealth to investment of $(\$ 100,000 / P)$. The ninemonth bond reinvested for three months using the forward contract offers $(\$ 100,000 / \$ 97,000)(\$ 100,000 / \$ 99,000)$. Setting the two returns equal and solving for $P$ generates $P=\$ 96,030$.

Therefore, the 12-month bond must sell for $\$ 96,030$ to prevent arbitrage.

Note that there is nothing special about the use of three months, nine months and twelve months in this example. View the exercise as having a maximum length (in this case twelve months) that is broken in to two sub-intervals. The idea is that the price ratio (terminal value/initial value) of the entire time interval must equal the product of the two price ratios of the subintervals.


\end{document}