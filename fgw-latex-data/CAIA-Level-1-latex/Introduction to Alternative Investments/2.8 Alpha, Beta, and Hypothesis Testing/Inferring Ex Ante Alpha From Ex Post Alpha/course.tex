\documentclass[11pt]{article}
\usepackage[utf8]{inputenc}
\usepackage[T1]{fontenc}
\usepackage{amsmath}
\usepackage{amsfonts}
\usepackage{amssymb}
\usepackage[version=4]{mhchem}
\usepackage{stmaryrd}

\begin{document}
\section*{Reading}
Inferring Ex Ante Alpha From Ex Post Alpha

One of the most central functions of alternative investment analysis is the process of attempting to identify ex ante alpha. Ex ante alpha estimation would be simplified if the expected returns of all assets could be observed or accurately estimated. In practice, expectations of returns on risky assets vary from market participant to market participant. In fact, the existence of ex ante alpha comes from different investors having different expectations of risk-adjusted return.

A key method of identifying ex ante alpha for a particular investment fund is a thorough and rigorous analysis of the manager and the manager's processes and methods. Analysis of historical data should typically also play a role, though not too large a role, in identification of ex ante alpha. In this section, these empirical methods are discussed. Empirical methods estimate ex ante alpha through attempting to differentiate between the roles of luck and skill in generating past riskadjusted returns. The objective of these empirical analyses is to understand how much, if any, of an investment's past returns are attributable to skill and might be predicted to recur.

\section*{Two Steps to Empirical Analysis of Ex Ante Alpha}
Two critical steps are used to identify ex ante alpha from historical performance. First, an asset pricing model or benchmark must be used to divide the historical returns into the portions attributable to systematic risks (and the risk-free rate) and those attributable to idiosyncratic effects. Second, the remaining returns, meaning the idiosyncratic returns (i.e., ex post alpha), should be statistically analyzed to estimate the extent, if any, to which the superior returns may be attributable to skill rather than luck.

The first step, identifying ex post alpha, requires the specification of an ex post asset pricing model or benchmark and can be challenging. Ex post alpha estimation is the process of adjusting realized returns for risk and the time value of money. Ex post alpha is not perfectly and unanimously measured, because it relies on accurate specification of systematic risks and estimation of the effects of those systematic risks on ex post returns.

Given estimates of ex post alpha (idiosyncratic returns), the second step is the statistical analysis of the superior or inferior returns to differentiate between random luck and persistent skill. The second part of this session (starting with the Using Statistical Methods to Locate Alpha lesson) discusses hypothesis testing and statistical inference. The idea is that, given a set of assumptions with regard to the statistical behavior of idiosyncratic returns, historical returns can be used to infer central tendencies. If historical risk-adjusted returns are very consistently positive or negative, the analyst can become increasingly confident that the underlying investment offered a positive or negative alpha.

\section*{Lessons about Alpha Estimation from a Fair Casino Game}
To frame the discussion of the role of idiosyncratic risk and model misspecification in alpha estimation, we discuss a hypothetical scenario in which skill is clearly not a factor, such as in the casino game roulette. This simplified scenario enables a clearer illustration of the challenges raised by model misspecification. Model misspecification is any error in the identification of the variables in a model or any error in identification of the relationships between the variables. Model misspecification inserts errors in the interpretation and estimation of relations.

For example, assume that there is a perfectly balanced roulette wheel in a casino with perfectly honest employees and guests. For simplicity, the payouts of all bets are assumed to be fair gambles rather than gambles offering the house an advantage. In other words, every possible gamble has an expected payout equal to the amount wagered, meaning an expected profit or loss of zero. Gamblers use a variety of strategies, and they wager different amounts of money.

Based on these assumptions, a model can be derived that states that the expected gain or loss to each gambler should be $\$ 0$ and $0 \%$. By assumption, any realized gambling returns that differ from zero will be based purely on luck. When the actual profits and losses to the gamblers at the end of a day are observed, some gamblers ended up winning large amounts of money, some gamblers lost a lot of money, and many gamblers won or lost smaller amounts.

Based on the assumption that the roulette wheel is perfectly balanced, all of the observed profits and losses are idiosyncratic (i.e., all ex post alphas were generated by luck, since all ex ante alphas were zero).

Let's assume that there is a researcher who believes that some gamblers have skill in predicting the outcomes of the roulette wheel. That researcher would hypothesize that some or all of the observed profits were due to that skill and should thus be viewed as ex ante alpha. The researcher decides to perform statistical tests to identify the skilled gamblers.

Even in this simplified example, it would be easy for the researcher to make incorrect inferences. For example, assume that thousands of gamblers were observed. The researcher might focus on the gambler who won the most money, conclude that the odds were extraordinarily low that a gambler could win so much money in one night, and therefore falsely conclude that the chosen gambler was skilled. Another researcher might expand the search to multiple nights and multiple casinos and find a gambler with even higher winnings. But in this example, no level of winnings can prove that skill was involved, because skill was eliminated by assumption.

Unfortunately, some financial analysts use the analogous approach to analyze investment opportunities. They examine the past returns from a large set of investment pools and conclude that the top-performing funds must have achieved that success through skill. This example highlights the challenges faced in investment analysis. Does ex ante alpha exist in a particular market? Do we have models that can accurately separate ex post alpha from systematic risk bearing? Finally, will our statistical tests enable us to differentiate between idiosyncratic outcomes (luck) and ex ante alpha (skill)?


\end{document}