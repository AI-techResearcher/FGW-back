\documentclass[11pt]{article}
\usepackage[utf8]{inputenc}
\usepackage[T1]{fontenc}

\begin{document}
\section*{Reading}
Ex Ante Alpha Estimation and Return Persistence

Numerous investment advertisements warn that "past performance is not indicative of future results." That admonition would be true with regard to alpha if markets were perfectly efficient. But there is no doubt that inefficiencies exist and that abnormally good and bad performance has been predictable based on past data in many instances. However, there are also many instances in which investors have incorrectly used past performance to indicate future results.

Abnormal return persistence is the tendency of idiosyncratic performance in one time period to be correlated with idiosyncratic performance in a subsequent time period. This section focuses on return persistence in interpreting idiosyncratic return and identifying ex ante alpha.

\section*{Separating Luck and Skill with Return Persistence}
Assume that a reasonably accurate performance attribution has distinguished returns due to systematic risks from those due to idiosyncratic risks. The next step is to attribute the idiosyncratic returns to their sources: luck, skill, or both. Proper attribution of the idiosyncratic returns (the ex post alpha) to luck or skill is typically a statistical challenge.

Attempting to identify ex ante alpha through an abnormal return persistence procedure can be summarized in the following three steps:

\begin{enumerate}
  \item Estimate the average idiosyncratic returns (ex post alpha) for each asset in time period 1.

  \item Estimate the average idiosyncratic returns (ex post alpha) for each asset in time period 2.

  \item Statistically test whether the ex post alphas in time period 2 are correlated with the ex post alphas in time period 1.

\end{enumerate}

\section*{Interpreting Estimated Return Persistence}
A statistically significant positive correlation between average idiosyncratic returns in consecutive periods implies positive return persistence. To the extent that the return model has been correctly specified, consistent and statistically significant positive correlation would lead to increased confidence that managerial skill has driven some or all of the investment results.

Note that this approach differs markedly from the more common approach of using a single time period to identify top returns and assuming that the top returns were driven by skill. However, just because an investment experiences positive return persistence in two consecutive periods does not prove that the returns are based on skill. The most that a researcher can do is use careful statistical testing to develop increased confidence that persistence has been successfully identified.

The later part of this session discusses hypothesis testing with statistics and the care that should be used in constructing tests and interpreting their results.


\end{document}