\documentclass[11pt]{article}
\usepackage[utf8]{inputenc}
\usepackage[T1]{fontenc}
\usepackage{amsmath}
\usepackage{amsfonts}
\usepackage{amssymb}
\usepackage[version=4]{mhchem}
\usepackage{stmaryrd}

\begin{document}
Ex Ante Versus Ex Post Alpha

Although in a very general sense there is consensus in the alternative investment community regarding the general meaning of alpha as superior risk-adjusted performance, the term is often used interchangeably for two very distinct concepts. Sometimes alpha is used to describe any high risk-adjusted returns, and sometimes it is used to describe superior returns generated through skill alone. This section distinguishes these two views of alpha using the terms ex ante alpha and ex post alpha. Considerable confusion regarding alpha originates from the failure to distinguish between these different uses of the term alpha.

\section*{Ex Ante Alpha}
Ex ante alpha is a term that is not commonly used in industry or academics; rather, it is used in this curriculum to denote an issue of critical importance in understanding alpha. Ex ante alpha is the expected superior return if positive (or inferior return if negative) offered by an investment on a forward-looking basis after adjusting for the riskless rate and for the effects of systematic risks (beta) on expected returns. Ex ante alpha is generated by a deliberate over- or underallocation to mispriced assets based on investment management skill. Simply put, ex ante alpha indicates the extent to which an investment offers a consistent superior riskadjusted investment return.

In the context of the single-factor market model, ex ante alpha may be viewed as the first term on the right-hand side of the following equation:


\begin{equation*}
E\left(R_{i t}-R_{f}\right)=\alpha_{i}+\beta_{i}\left[E\left(R_{m t}\right)-R_{f}\right] \tag{1}
\end{equation*}


where $\alpha_{i}$ is the ex ante alpha of asset $i$.

In a perfectly efficient market, $\alpha_{i}$ (alpha) in this equation would be zero for all assets. The use of a single-factor market model in Equation 1 and throughout most of this session is for simplicity. A multifactor model would simply insert a set of beta terms and factor returns in Equation 1 in addition to or in place of the market beta and market factor.

Equation 1 is described as representing a single-factor market model rather than the CAPM because the CAPM implies that no competitively priced asset would offer a positive or negative ex ante alpha, since every asset would trade at a price such that its expected return would be commensurate with its risk. In practice, market participants often seek expected returns that exceed the expected return based on systematic risk, a goal that is illustrated in Equation 1 using the term $\alpha_{i}$.

In practice, ex ante alpha is typically a concept rather than an observable variable. This can be seen from Equation 1 in a number of ways. First, $\beta_{i}$ is a sensitivity that must be estimated with approximation. If the true value of $\beta_{i}$ is not clear, then the true value of $\alpha_{i}$ cannot be known. Second, all of the expected returns in Equation 1 , except the risk-free rate, are unobservable and must be estimated. Thus, ex ante alpha can only be estimated or predicted. A positive ex ante alpha is an expression of the belief that a particular investment will offer an expected return higher than investments of comparable risk in the next time period. As an illustration, consider the manager of an equity market-neutral hedge fund who desires to maximize ex ante alpha while maintaining a beta close to zero. The manager's strategy creates a hedge against systematic risk factors while attempting to exploit abnormal performance of individual stocks within the same sector or industry. Once the ex ante alpha of each stock is estimated, the portfolio is built using an optimization process seeking to maximize the positive alpha of long positions and the negative alpha of short positions, while requiring the systematic risk exposures of the long portfolio to match the short portfolio. The intended result is a zero-beta, or marketneutral, portfolio with a high ex ante alpha.

\section*{Ex Post Alpha}
The ex ante alpha discussed in the previous section is a common interpretation of the term alpha. This section provides details about another potential interpretation of the term: the ex post alpha. As in the case of ex ante alpha, ex post alpha is a term used primarily for the purposes of this curriculum.

Ex post alpha is the return, observed or estimated in retrospect, of an investment above or below the risk-free rate and after adjusting for the effects of beta (systematic risks). Whereas ex ante alpha may be viewed as expected idiosyncratic return, ex post alpha is realizedidiosyncratic return. Simply put, ex post alpha is the extent to which an asset outperformed or underperformed its benchmark in a specified time period. Ex post alpha can be the result of luck or skill. Unlike ex ante alpha, ex post alpha can usually be estimated with a reasonable degree of confidence.

Considerable and valid disagreement exists with describing the concept of ex post alpha as being a type of alpha. The reason is that alpha is sometimes associated purely with skill, whereas ex post alpha can be generated by luck. Nevertheless, the use of the term to describe past superior performance is so common that it is labeled as such throughout this book. In the context of the single-factor market model, ex post alpha may be viewed as the last term on the right-hand side of the following equation $\left(\varepsilon_{i t}\right)$ :


\begin{equation*}
R_{i t}-R_{f}=\beta_{i}\left(R_{m t}-R_{f}\right)+\varepsilon_{i t} \tag{2}
\end{equation*}


Note that Equation 2 refers to theoretical values rather than actual values estimated using a linear equation or other statistical technique. Some analysts would correctly refer to $\varepsilon_{i t}$ as the idiosyncratic return or the abnormal return and might object to having the return labeled as any type of alpha, because there might be no reason to think of the return as being generated by anything other than randomness or luck. Nevertheless, many other analysts use the term alpha synonymously with idiosyncratic return or abnormal return; therefore, the term ex post alpha is used here to distinguish the concept from the other interpretation of alpha (ex ante alpha).

In this example, Trim Fund must have been lucky, because the fund outperformed its benchmark by 125 basis points despite the managers being unskilled. Alphabased analysis typically involves two steps: (1) ascertaining abnormal return performance (ex post alpha) by controlling for systematic risk, and (2) judging the extent to which any superior performance was attributable to skill (i.e., was generated by ex ante alpha). The more problematic issue can often be in the second step of the analysis, differentiating between the potential sources of the ex post alpha: luck or skill.

A key difference between ex ante and ex post alpha is that ex ante alpha reflects skill, whereas ex post alpha can be a combination of both luck and skill. For example, a manager might have enough skill to select a portfolio that is $1 \%$ underpriced but that happens to experience some completely unexpected good news that results in the portfolio outperforming other assets of similar risk by $11 \%$. The manager had an ex ante alpha of $1 \%$ (purely skill) and an ex post alpha of $11 \%$ ( $1 \%$ from skill plus $10 \%$ from luck).

When discussing alpha, many analysts do not explicitly differentiate between the ex ante and ex post views. If an analyst identifies an alpha of $5 \%$ because a fund's risk-adjusted returns were $5 \%$ higher the previous year than the risk-adjusted returns of other funds, then in this book's terminology, the alpha is an ex post alpha. However, if the analyst expects that a fund will have a $5 \%$ higher expected return than other funds of similar risk, then in this book's terminology, the analyst believes that the fund has an ex ante alpha of $5 \%$ and that the fund's superior return is probably attributable to the better skill of the manager in selecting superior investment opportunities.


\end{document}