\documentclass[11pt]{article}
\usepackage[utf8]{inputenc}
\usepackage[T1]{fontenc}
\usepackage{amsmath}
\usepackage{amsfonts}
\usepackage{amssymb}
\usepackage[version=4]{mhchem}
\usepackage{stmaryrd}

\begin{document}
\section*{APPLICATION A}
Consider the Sludge Fund, a fictitious fund run by unskilled managers that generally approximates the S\&P 500 Index but does so with an annual expense ratio of 100 basis points (1\%) more than other investment opportunities that mimic the S\&P 500.

\section*{Answer and EXPLanation}
Using Equation 1 and assuming that the S\&P 500 is a proxy for the market portfolio, the ex ante alpha of Sludge Fund would be approximately -100 basis points per year. This can be deduced from assuming that $\beta_{\mathrm{i}}=1$ and that $\left[E\left(R_{i t}\right)-\right.$ $\left.E\left(R_{m t}\right)\right]=-1 \%$ due to the expense ratio. Sludge Fund could be expected to offer an ex ante alpha, meaning a consistently inferior risk-adjusted annual return, of $-1 \%$ per year. This example illustrates that ex ante alpha can be negative to indicate inferior expected performance, although alpha is usually discussed in the pursuit of the superior performance associated with a positive alpha.

The application to this solution can be deduced as $\alpha_{i}=-1 \%$. Let's consider the facts. The expected Fund return and the expected Market return are the same, except that the fund has an expense ratio of $1 \%$. The beta of the fund to the market is 1 . Therefore, the only difference between the market and the fund is a $1 \%$ expense ratio that weighs on the fund performance. Therefore, the $\alpha_{i}=-1 \%$.

Now let's calculate this mathematically by manipulating Equation 1 :

$$
\begin{aligned}
\mathrm{E}\left(\mathrm{R}_{i, t}-\mathrm{R}_{f}\right) & =\alpha_{i}+\beta_{i}\left[E\left(R_{m, t}\right)-R_{f}\right] \\
\alpha_{i} & =\beta_{i}\left[E\left(R_{m, t}\right)-R_{f}\right]-E\left(R_{i, t}-\text { Expense Ratio }-R_{f}\right)
\end{aligned}
$$

With the modified Equation 1, it becomes clearer how the expense ratio impacts alpha. Let's solve for $\alpha_{i}$.

\section*{APPLICATION B}
Consider the Trim Fund, a fund that tries to mimic the S\&P 500 Index and has managers who are unskilled. Unlike the Sludge Fund from the previous section, Trim Fund has virtually no expenses. Although Trim Fund generally mimics the S\&P 500, it does so with substantial error due to the random incompetence of its managers. However, the fund is able to maintain a steady systematic risk exposure of $\beta_{\mathrm{i}}=1$.

\section*{Answer and Explanation}
Last year, Trim Fund outperformed the S\&P 500 by 125 basis points. Using Equation 2, assuming that $\beta_{\mathrm{i}}=1$ and that $\left(R_{i t}-R_{m t}\right)=+1.25 \%$, it can be calculated that $\varepsilon_{i t}=+1.25 \%$. Thus, Trim Fund realized a return performance for the year that was $1.25 \%$ higher than its benchmark, or its required rate of return. In the terms of this session, Trim Fund generated an ex post alpha of 125 basis points, even though the fund's ex ante alpha was zero.

Let's examine this application from a mathematic context as opposed to deducing the ex-post alpha from the given information.

To solve ex-post alpha we need to rearrange Equation 2:

$$
\begin{gathered}
R_{i t}-R_{f}=\beta_{i}\left(R_{m t}-R_{f}\right)+\varepsilon_{\mathrm{it}} \\
\frac{R_{i t}-R_{f}}{\beta_{i}}=\left(R_{m t}-R_{f}\right)+\varepsilon_{\mathrm{it}} \\
\varepsilon_{\mathrm{it}}=\frac{R_{i t}-R_{f}}{\beta_{i}}-R_{m t}+R_{f}
\end{gathered}
$$

Now, not all the values are given. However, we know that given $\beta_{\mathrm{i}}=1$, then $R_{i t}-R_{m t}=1.25 \%$ so we can use example figures that reflect that difference. $\beta_{\mathrm{i}}=1 R_{m t}=$ $10 \%, \mathrm{R}_{\mathrm{it}}=11.25 \%$, and $\mathrm{R}_{\mathrm{f}}=0 \%$. Now using these figures, we can calculate $\varepsilon_{\mathrm{it}}$ :

$$
\varepsilon_{\mathrm{it}}=\frac{0.1125-0}{1}-0.10+0
$$

As shown, $\varepsilon_{\text {it }}=0.0125$


\end{document}