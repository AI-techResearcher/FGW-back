\documentclass[11pt]{article}
\usepackage[utf8]{inputenc}
\usepackage[T1]{fontenc}

\begin{document}
\section*{Reading}
Return Drivers

The term return driver represents the investments, the investment products, the investment strategies, or the underlying factors that generate the risk and return of a portfolio. A conceptually simplified way to manage a total portfolio is to divide its assets into two groups: beta drivers and alpha drivers. Briefly, in the context of a portfolio, an investment that moves in tandem with the overall market or a particular risk factor is a beta driver. An investment that seeks high returns independent of the market is an alpha driver.

For example, consider an investor who owns a portfolio consisting of one mutual fund indexed to the FTSE 100 and one market-neutral fund with offsetting long and short exposures that attempts to earn superior rates without bearing systematic risk. The allocation to the FTSE 100 Index fund is a beta driver, since the holding will generate systematic risk but will not offer ex ante alpha. That allocation is designed simply to harvest the average higher returns of bearing beta (systematic) risk. The allocation to the market-neutral fund is an alpha driver, since it is an attempt to earn superior rates of return through superior security selection rather than through systematic risk bearing.

Viewed from a portfolio management context, various investments and investment strategies can be viewed as alpha drivers, beta drivers, or mixtures of both. Alternative investing tends to focus more on alpha drivers, whereas traditional investing tends to focus more on beta drivers.

\section*{Beta Drivers}
Beta drivers capture market risk premiums, and good or pure beta drivers do so in an efficient (i.e., precise and cost-effective) manner. Beta drivers capture risk premiums by bearing systematic risk.

Bearing beta risk as defined by the CAPM has been extremely lucrative over the long run. The long-term tendency of beta drivers to earn higher returns from equity investments than are earned on risk-free investments is attributed to the equity risk premium. The equity risk premium (ERP) is the expected return of the equity market in excess of the risk-free rate. This risk premium may be estimated from historical returns or implied by stock valuation models, such as through the relationship between stock prices and forecasts of earnings.

Especially in the United States, stocks have outperformed riskless assets tremendously, and these high historical returns form the equity risk premium puzzle. The equity risk premium puzzle is the enigma that equities have historically performed much better than can be explained purely by risk aversion, yet many investors continue to invest heavily in low-risk assets. Based on the data of the past 100 years or so, it seems that most investors are foolish not to place more of their money in equities rather than riskless assets. There is no consensus, however, on whether the superior equity returns of the past century that generated the high equity premium will persist in magnitude through the twenty-first century.

\section*{Passive Beta Drivers as Pure Plays on Beta}
Passive investing, such as employing a buy-and-hold strategy to match a benchmark index, is a pure play on beta: simple, low cost, and with a linear risk exposure. A linear risk exposure means that when the returns to such a strategy are graphed against the returns of the market index or another appropriate standard, the outcomes tend to lie on a straight line. Options and investment strategies with shifting betas have nonlinear risk exposures.

A passive beta driver strategy generates returns that follow the up-and-down movement of the market on a one-to-one basis. In this sense, pure beta drivers are linear in their performance compared to a financial index.

Some managers can deliver beta drivers for annual fees of as little as a few basis points per year, whereas others may charge more than a half percent per year and deliver performance before fees that is virtually identical to that of a pure beta driver. Asset gatherers are managers striving to deliver beta as cheaply and efficiently as possible, and include the large-scale index trackers that produce passive products tied to well-recognized financial market benchmarks. These managers build value through scale and processing efficiency.

\section*{Alpha Drivers}
Alpha drivers seek excess return or added value through generating returns that exceed the returns on investments of comparable risk. Many alternative assets fall squarely into the category of alpha drivers. They tend to seek sources of return less correlated with traditional asset classes, which reduces risk in the entire portfolio in the process. Alpha drivers are the focus of much alternative investing. Alternative investments are often touted as being able to generate greater combinations of return and risk by providing return streams that have relatively low correlation with traditional stock and bond markets but comparable average returns.

\section*{Product Innovators and Process Drivers}
Historically, most investment pools were mixes of beta drivers and alpha drivers. In other words, the funds derived considerable return variation from bearing substantial systematic risk but implemented active investment strategies intended to generate alpha. In recent decades, the distinction between alpha drivers and beta drivers has increased. Thus, much of the asset management industry has moved into the tails of the alpha driver-beta driver spectrum. At one end of the spectrum are product innovators, which are alpha drivers that seek new investment strategies offering superior rates of risk-adjusted return. At the other end are passive indexation strategies, previously described as asset gatherers, which offer beta exposure as efficiently as possible without any pretense of alpha seeking.

Another development among beta drivers is the growth of process drivers. Process drivers are beta drivers that focus on providing beta that is fine-tuned or differentiated. As an example, these index trackers have introduced a large number and wide variety of exchange-traded funds (ETFs) that track specific sectors of the market rather than broadly diversifying across most or all sectors. For example, many new ETFs provide beta for a particular market-capitalization range, industry, asset class, or geographic market. These process drivers carve up systematic risk exposure into narrower risk factors as they identify investors desiring targeted risk exposures.

The increased difficulty for a fund manager to capture alpha or to compete with the extremely low-cost asset gatherers has put pressure on beta drivers with high fees. It has been argued that some managers following a pure beta driver strategy do not disclose the true nature of their strategy accurately, perhaps because it\\
would be difficult to justify their high fees when their performance before fees is virtually indistinguishable from that of other beta drivers with fees near zero.


\end{document}