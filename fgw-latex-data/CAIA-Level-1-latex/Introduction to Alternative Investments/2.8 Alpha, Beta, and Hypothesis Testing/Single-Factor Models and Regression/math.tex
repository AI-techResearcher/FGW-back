\documentclass[11pt]{article}
\usepackage[utf8]{inputenc}
\usepackage[T1]{fontenc}
\usepackage{amsmath}
\usepackage{amsfonts}
\usepackage{amssymb}
\usepackage[version=4]{mhchem}
\usepackage{stmaryrd}

\begin{document}
\section*{APPLICATION A}
Consider a regression with an alpha estimate of $0.5 \%$ (with a standard error of $0.3 \%$ ) and a beta estimate of 1.1 (with a standard error of 0.3 ). Are the regression paramet ers statistically significant?

\section*{Answer and EXPLANATION}
We are looking for a $5 \%$ statistical significance of an alpha estimate and a beta estimate with a confidence level of $95 \%$. We need to know that the alpha estimate is $0.5 \%$ with a standard error of 0.3 and the beta estimate is 1.1 with a standard error of 0.3 . In addition, we need to know the $z$ - score of a $5 \%$ confidence level is 1.96 . Now, we need to calculate the $t$-statistic of the alpha estimate and beta estimate and compare those numbers with the $z$-score of a $95 \%$ confidence level or 1.96 . If the estimate's $t$-statistic exceeds the $z$-score for the set confidence interval, the estimate is statistically significant. If the estimate's $t$-statistic is less than the $z$ score for the set confidence interval, the estimate is not deemed statistically significant.

To calculate the $t$-statistic for the alpha estimate we need to divide 0.5 by 0.3 which equals 1.67 .1 .67 is less than the $z$-score of 1.96 for the set confidence interval. The alpha estimate is not deemed statistically significant.

To calculate the $t$-statistic for the beta estimate we need to divide 1.1 by 0.3 which equals 3.67 .3 .67 is greater than the $z$-score of 1.96 for the set confidence level. The beta estimate is deemed statistically significant.


\end{document}