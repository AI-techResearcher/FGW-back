\documentclass[11pt]{article}
\usepackage[utf8]{inputenc}
\usepackage[T1]{fontenc}
\usepackage{amsmath}
\usepackage{amsfonts}
\usepackage{amssymb}
\usepackage[version=4]{mhchem}
\usepackage{stmaryrd}

\begin{document}
\section*{APPLICATION A}
Consider the following data on Target Fund: $\beta=1.5$ and its expected return is $14 \%$. Assume that the expected return of the market is $11 \%$ and that the riskfree rate is 3\%. During the next year, the market earns $8 \%$ and Target Fund earns 7\%. What was: (I) the Fund's ex ante alpha, (2) the Fund's ex post alpha, (3) the Fund's return that was skill, and (4) the Fund's return that was luck?

1.

\section*{EXPLANATION}
Inserting the market's expected return, the Fund's beta, and the risk-free rate into the ex ante CAPM generates the expected return of an efficiently priced asset with a beta of I.5:

$$
\mathrm{E}(\mathrm{R})=3 \%+[1.5(11 \%-3 \%)] \Rightarrow \mathrm{E}(\mathrm{R})=15 \%
$$

Because the Fund's expected return is only $14 \%$, it represents an ex ante alpha of $-1 \%$.

\begin{enumerate}
  \setcounter{enumi}{1}
  \item Given the market return was $5 \%$ more than the risk-free return of $3 \%$, an asset with a beta of 1.5 should have earned $7.5 \%(5 \% \times 1.5)$ more than the riskless rate (i.e., $10.5 \%$ ). Because it earned $7 \%$, it had an ex post alpha (s) of $-3.5 \%$.

  \item The answer to (1) means that the Fund is expected to underperform by $1 \%$. Therefore, $-1 \%$ of the actual subsequent return was skill.

  \item The Fund underperformed by $-3.5 \%$. Since the expected loss on the fund is $-1 \%$ based on skill (perhaps due to fees), the return attributed to bad luck (negative idiosyncratic return) was $-2.5 \%$

\end{enumerate}

\end{document}