\documentclass[11pt]{article}
\usepackage[utf8]{inputenc}
\usepackage[T1]{fontenc}
\usepackage{amsmath}
\usepackage{amsfonts}
\usepackage{amssymb}
\usepackage[version=4]{mhchem}
\usepackage{stmaryrd}

\begin{document}
\section*{Reading}
Return Attribution, Alpha, and Beta

Return attribution (performance attribution) was introduced in the session entitled Measures of Risk and Performance. This section focuses on return attribution and distinguishing between the effects of systematic risk (beta), the effects of skill (ex ante alpha), and the effects of idiosyncratic risk (luck).

\section*{A Numerical Example of Alpha}
For simplicity, consider an example that uses a single-factor market model and for which expected returns are known. Assume that Fund A trades unlisted securities that are not efficiently priced, has a beta of 0.75 , and has an expected return of $9 \%$. Additionally, assume that the expected return of the market is $10 \%$ and that the risk-free rate is $2 \%$. During the next year, the market earns $18 \%$ and Fund $A$ earns $17 \%$.

Given these assumptions, we can answer the following questions:

\begin{itemize}
  \item What was the fund's ex ante alpha?
  \item What was the fund's ex post alpha?
  \item What was the amount of ex post alpha that was luck?
  \item What was the amount of the ex post alpha that was skill?
\end{itemize}

First, the ex ante alpha is found as the intercept of the ex ante version of the single-factor market model, in this case a CAPM-style model. Inserting the market's expected return, the fund's beta, and the risk-free rate into Equation 1 generates the required return, $E\left(R_{A}{ }^{\star}\right)$, for Fund $\mathrm{A}$ in an efficient market:

$$
\begin{aligned}
& E\left(R_{A}^{*}\right)-2 \%=[0.75(10 \%-2 \%)] \\
& E\left(R_{A}^{*}\right)=8 \%
\end{aligned}
$$

The return of $8 \%$ is the expected return that investors would require on an asset with a beta of 0.75 , which is also the expected return that Fund $A$ would offer in an efficient market. The ex ante alpha of Fund $A$ is any difference between the expected return of Fund $A$ and its required return:

$$
\text { Ex Ante Alpha }=\text { Expected Return }- \text { Required Return } \rightarrow 9 \%-8 \% \rightarrow 1 \%
$$

Thus, Fund A offers $1 \%$ more return than would be required based on its systematic risk (i.e., an ex ante alpha of $1 \%$ ). Next, the ex post alpha is found from the ex post version of the single-factor market model. Inserting the two realized returns, the beta and the risk-free rate, into Equation 2 generates the following:

$$
17 \%-2 \%=[0.75(18 \%-2 \%)]+\varepsilon \rightarrow \text { Ex Post Alpha }(\varepsilon)=15 \%-12 \%=+3 \%
$$

The analysis indicates that even though Fund A underperformed the market portfolio prior to risk adjustment, it performed 3\% better than assets of similar risk. Thus, in the terminology introduced earlier in the session, the ex post alpha (idiosyncratic return) was $3 \%$.

Finally, since the analysis assumes that the fund offers an expected superior return, or ex ante alpha, of $1 \%$, then $1 \%$ (i.e., one-third) ex post alpha of $3 \%$ could be said to be attributable to skill and $2 \%$ (i.e., two thirds of 3\%) attributable to good luck (positive idiosyncratic return).

In practice, true beta and expected returns are difficult to estimate. The beta is necessary to estimate either ex ante alpha or ex post alpha. The expected returns are necessary only to estimate ex ante alpha and to distinguish between luck and skill. It is common for a return attribution analysis to estimate ex post alpha but not consider ex ante alpha, and not estimate the distinction between luck and skill.

\section*{Three Types of Model Misspecification}
The previous example assumed that the investment's systematic risks were fully and accurately captured in a single market beta and that the single-factor market model was accurate. Errors in estimating alpha can result from model misspecification, including misspecification of a benchmark. Three primary types of model misspecification can confound empirical return attribution analyses:

\begin{enumerate}
  \item Omitted (or misidentified) systematic return factors is the failure to include relevant factors in an analysis of returns such as momentum or size.

  \item Misestimated betas is estimation error due to randomness or econometric errors such as failure to correct for heteroskedasticity.

  \item Nonlinear risk-return relation error is the failure to model nonlinearity such as quadratic or cubic effects.

\end{enumerate}

In each case of misspecification, the component of the return attributable to systematic risk is not precisely identified. Because systematic risks have a positive expected return, omitting a significant risk factor or underestimating a beta tends to overstate the manager's skill by attributing beta return to alpha.

The bias caused by omitted systematic return factors in estimating alpha can be illustrated as follows. Assume that a fund's return is driven by four betas, or systematic factors. If an analyst ignores two of the factors (e.g., factor 3 and factor 4), then the estimate of the idiosyncratic return will, on average, contain the expectation of the two missing effects, both of which would have positive expected values.

In the second case of model misspecification, misestimated betas, when the systematic risk, or beta, of a return series is over- or underestimated, the return attributable to the factors is also over- or underestimated. Underestimation of a beta is a similar but less extreme case of omitting a beta.

The final major problem with misspecification is when the functional relationship between a systematic risk factor and an asset's return is misspecified. For example, most asset pricing models assume a linear relationship between risk factors and an asset's returns. If the true relationship is nonlinear, such as in the case of options, then the linear specification of the relationship generally introduces error into the identification of the systematic risk component of the asset's return.

\section*{Beta Nonstationarity}
Beta nonstationarity is one reason why return can be attributed to systematic risk with error. Beta nonstationarity is a general term that refers to the tendency of the systematic risk of a security, strategy, or fund to shift through time. For example, a return series containing leverage is generally expected to have a changing systematic risk through time if the leverage changes through time. An example is the stock of a corporation with a fixed dollar amount of debt. As the assets of the firm rise, the leverage of the equity falls (or if the assets fall, leverage of the equity rises), causing the beta of the equity to shift.

A type of beta nonstationarity that is sometimes observed in hedge funds is beta creep. Beta creep is when hedge fund strategies pick up more systematic market risk over time. When assets pour into hedge funds, it might be expected that the managers of the funds will allow more beta exposure in their portfolios in an attempt to maintain expected returns in an increasingly competitive and crowded financial market. This causes the creeping effect: Over time, as the alphas of available investment opportunities decline, the amount of systematic risk in portfolios will creep upward.

The betas of funds may also be nonstationary because of market conditions, such as market turmoil, rather than changes in the fund's underlying assets. In periods of economic stress, the systematic risks of funds have been observed to increase. Beta expansion is the perceived tendency of the systematic risk exposures of a fund or asset to increase due to changes in general economic conditions. Beta expansion is typically observed in down market cycles and is attributed to increased correlation between the hedge fund's returns and market returns.

Another example of beta nonstationarity is market timing, the intentional shifting of an investment's systematic risk exposure by its manager. Consider the case of a skilled market timer. The fund manager takes on a positive beta exposure when his or her analysis indicates that the market is likely to rise and takes on a negative beta, or a short position, when he or she perceives that the market is likely to decline. This beta nonstationarity (or beta shifting) makes return attribution more problematic, since the level of beta between reporting periods would typically be very difficult to estimate accurately.

This market-timing example raises an interesting issue in the attribution of returns to alpha or beta. Assume for the sake of argument that a market-timing fund manager possesses superior skill in timing markets. The manager is successful at designing and implementing the strategy to generate superior returns but is unable to enhance returns through picking individual stocks. Would the fund's superior return better be described as alpha or beta?

At first glance, the answer may appear to be ex ante alpha, since the market timing manager's return is superior. But in each sub-period, the manager earns a rate of return commensurate with the fund's systematic risk exposure; that is, whether the fund's risk exposure is positive or negative, its returns are commensurate with risk. Thus, in each sub-period, the portfolio earns the predicted return and exhibits an ex post alpha of zero. However, when viewed over the full time period, the fund earns a high ex post alpha, since the portfolio outperformed the market through superior market timing.

This example illustrates an important lesson: Evaluation of investment performance over a full market cycle can alleviate difficulties with shifting betas and misspecified models. A full market cycle is a period of time containing a large representation of market conditions, especially up (bull) markets and down (bear) markets. Although use of a full market cycle does not eliminate return attribution difficulties, it can mitigate the impact of modeling misspecifications and estimation errors.

\section*{Can Alpha and Beta Be Commingled?}
The difficulty of identifying the return attributable to systematic risk is not limited to beta nonstationarity. Sometimes the line between alpha and beta can be blurred, even on a conceptual basis. Consider a specialized type of private equity transaction involving target firms in financial distress. An investment strategy directed at these opportunities requires sophisticated investors with keen negotiating skills and large amounts of available cash, since transactions must be made quickly. Very skilled investors can identify attractive opportunities, but the strategy requires exposures to systematic risks that cannot be hedged. One could argue that any superior return is ex ante alpha, since it takes superior skill to participate successfully in this market. However, one could also argue that the superior return is at least partially beta, since high returns are achieved only through bearing the systematic risk of the sector. Should highly attractive returns that require skill as well as the bearing of systematic risk be attributed to alpha or beta? Perhaps there is no clear answer, such as in trying to attribute the dancing superiority of a pair of competitive dancers to each performer. In some cases, performance may be better viewed as indistinguishably related to both.


\end{document}