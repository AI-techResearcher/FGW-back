\documentclass[11pt]{article}
\usepackage[utf8]{inputenc}
\usepackage[T1]{fontenc}
\usepackage{amsmath}
\usepackage{amsfonts}
\usepackage{amssymb}
\usepackage[version=4]{mhchem}
\usepackage{stmaryrd}

\begin{document}
\section*{Reading}
Overview of Beta and Alpha

The Derivatives and Risk-Neutral Valuation session discussed a number of measures of the price risk of options using Greek letters, such as delta, theta, and gamma. Greek letters and other similar-sounding words, such as vega, are not limited to option analysis. This session begins with a detailed discussion of alpha and beta. Alpha and beta are central concepts within alternative investment analysis. Consider the following hypothetical example of a discussion of investment performance:

During an investment committee meeting, the chief investment officer ( $\mathrm{CIO})$ comments on the performance of a convertible arbitrage fund named MAK Fund: "MAK generated an alpha of $8 \%$ last year and $10 \%$ two years ago. I think we can expect an alpha of $4 \%$ next year." A portfolio manager debates the point: "MAK Fund takes positions in convertible bonds with high credit risk. I think that MAK's alpha during the last two years was really beta." The CIO replies: "But MAK is delta hedged. And even though the fund is long gamma, is there really any beta in being long gamma?"

The preceding example illustrates how Greek letters are often used in investments to represent key concepts. This session focuses on alpha and beta, two critical concepts in the area of alternative investments. In a nutshell, alpha represents, or measures, superior return performance; and beta represents, or measures, systematic risk. A primary purpose of this session is to explore their meanings and nuances. The second purpose of this session is to discuss hypothesis testing, since alpha and beta are generally estimated rather than observed.

\section*{Beta}
In the CAPM (capital asset pricing model), the concept of beta is precisely identified: Each asset has one beta, and the beta is specified as the covariance of the asset's return with the return of the market portfolio, divided by the variance of the returns of the market portfolio. This is also the definition for a regression coefficient in a simple linear regression of an asset's returns on the returns of the market portfolio. Intuitively, beta is the proportion by which an asset's excess return moves in response to the market portfolio's excess return (the return of the asset minus the return of the riskless asset). If an asset has a beta of 0.95 , its excess return can be expected, on average, to increase and decrease by a factor of 0.95 relative to the excess return of the market portfolio.

But beta has a more general interpretation outside the CAPM, both within traditional investment analysis and especially within alternative investment analysis. Beta refers to a measure of risk, or the bearing of risk, wherein the underlying risk is systematic (shared by at least some other investments and usually unable to be diversified or fully hedged without cost) and is potentially rewarded with expected return. Outside the CAPM model, assets can have more than one beta, and a beta does not have to be a measure of the response of an asset to fluctuations in the entire market portfolio.

The Financial Economics Foundations session detailed the idea of beta in asset pricing models. For example, when a particular investment, such as private equity, locks the investor into the position for a considerable length of time, is this illiquidity a risk that is rewarded with extra expected return? If so, then a benchmark should reflect that risk and reward, and a beta measuring that illiquidity and a term reflecting its expected reward should be included as an additional factor in an ex ante asset pricing model.

In alternative investments, the term beta can be used to refer to any systematic risk for which an investor might be rewarded. The term can apply to a specific systematic risk, from a single-factor or a multifactor model, or to the combined effects of multiple systematic risks from multiple factors. Beta is commonly used in phrases such as "This strategy has no beta," "Half of the manager's return was (from) beta," and "That product is a pure beta bet."

Bearing beta risk is generally viewed as a source of higher expected return. The attempt to earn consistently higher returns without taking additional systematic risk leads to the topic of the next section: alpha.

\section*{Alpha}
Alpha refers to any excess or deficient investment return after the return has been adjusted for the time value of money (the risk-free rate) and for the effects of bearing systematic risk (beta). For an investment strategy, alpha refers to the extent to which the skill, information, and knowledge of an investment manager generate superior risk-adjusted returns (or inferior risk-adjusted returns in the case of negative alpha).

The measurement of alpha, and even the existence of alpha, is an important issue in investments in general and in alternative investments in particular. One person may believe that a high return was generated by skill (alpha), whereas another person may argue that the same return was a reward for taking high risks (beta) or a result of being lucky (idiosyncratic risk). Therefore, the concept of alpha and the estimation of alpha are inextricably linked to the view of how financial assets and financial markets function. Asset pricing models, discussed in detail in the sessions Financial Economics Foundations and Measures of Risk and Performance, are expressions of asset and market behavior. The demarcation between return from alpha, beta, and idiosyncratic risk depends on one's view of the return-generating process (or asset pricing model) as implicitly or explicitly expressed. If the return-generating process is misspecified and relevant beta risks are excluded from the analysis, then manager skill may be overstated, because the perceived alpha may include compensation for beta risks omitted from a benchmark or asset pricing model.

The concept of alpha originated with Jensen's work in the context of the CAPM. Jensen's analysis was a seminal empirical application of the single-factor market model. Jensen measured the net returns from mutual funds after accounting for the funds' returns based on the single-factor market model. He subtracted the single factor market model's estimated return from the actual returns, and what was left over (either positive or negative) was labeled alpha. However, the term alpha is not limited to the context of the CAPM. Regressions based on single-factor or multifactor market models are commonly performed with the value of the intercept referred to as alpha to reflect the common notation of the intercept of a linear regression.


\end{document}