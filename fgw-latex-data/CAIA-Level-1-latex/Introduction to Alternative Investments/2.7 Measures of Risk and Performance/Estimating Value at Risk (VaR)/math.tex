\documentclass[11pt]{article}
\usepackage[utf8]{inputenc}
\usepackage[T1]{fontenc}
\usepackage{amsmath}
\usepackage{amsfonts}
\usepackage{amssymb}
\usepackage[version=4]{mhchem}
\usepackage{stmaryrd}

\begin{document}
\section*{APPLICATION A}
Let's return to the example of JAC Fund's \$1 million holding of the ETF with an expected return of zero. Estimating roughly that the daily standard deviation of the ETF is $1.35 \%$, for a $99 \%$ confidence interval, the 10 -day VaR is found through substituting the known values into the equation:

$$
\begin{gathered}
2.33 \times \sigma \times \sqrt{\text { Days }} \times \text { Value } \\
2.33 \times 1.35 \% \times \sqrt{10} \times \& 1,000,000
\end{gathered}
$$

The first three values multiplied together produce the percentage change in the value that is being defined as a highly abnormal circumstance. In this case, the answer would be very roughly $10 \%$, indicating that there is a $1 \%$ chance that the ETF could fall $10 \%$ or more in 10 business days. This percentage is then multiplied by the position's value (the fourth term) to produce the dollar amount of the VaR. In the example, the $10 \%$ loss on the $\$ 1$ million stock holdings would produce a VaR of approximately $\$ 100,000$.

\section*{Answer and Explanation}
In order to solve this application we need to use Equation 1. In this case, with a 99\% confidence interval the z-score is 2.33. Following Equation 1, we multiply 2.33 by $1.35 \%$ by the square root of 10 (the days in the period) to get a product of $9.94 \% .9 .94 \%$ represents the percentage change in the value. To complete this solution we multiply $9.94 \%$ by $\$ 1,000,000$ for an answer of $\$ 99,469.44$. The $z$-score is a value that is assumed to be provided rather than being memorized or calculated.


\end{document}