\documentclass[11pt]{article}
\usepackage[utf8]{inputenc}
\usepackage[T1]{fontenc}
\usepackage{amsmath}
\usepackage{amsfonts}
\usepackage{amssymb}
\usepackage[version=4]{mhchem}
\usepackage{stmaryrd}

\begin{document}
\section*{Reading}
Risk-Adjusted Return Measures

The previous section focused on ratio-based performance measures. This section discusses three performance measures that are not return-to-risk ratios. Other performance measures exist, and some firms use performance measures unique to their particular firm. In practice, a variety of performance measures should be viewed in a performance review, each of which is selected to view performance from a relevant perspective.

\section*{Jensen's Alpha}
Jensen's alpha is based on the single-factor market model discussed in the Financial Economics Foundations session. In terms of expected returns, Jensen's alpha may be expressed as the difference between its expected return and the expected return of efficiently priced assets of similar risk. The return of efficiently priced assets of similar risk is usually specified using the single-factor market model, as shown in the following equation:


\begin{equation*}
\alpha_{p}=E\left(R_{p}\right)-R_{f}-\beta_{p}\left[E\left(R_{m}\right)-R_{f}\right] \tag{1}
\end{equation*}


The right-hand side expresses the alpha as the expected return of the portfolio in excess of the riskless rate and the required risk premium. Any return above the riskless rate and the required risk premium is alpha, which represents superior performance.

Jensen's alpha is a direct measure of the absolute amount by which an asset is estimated to outperform, if positive, the return on efficiently priced assets of equal systematic risk in a single-factor market model. It is tempting to describe the return in the context of the CAPM, but strictly speaking, no asset offers a nonzero alpha in a CAPM world, since all assets are priced efficiently. In practice, expected returns on the asset and the market, as well as the true beta of the asset, are unobservable. Thus, Jensen's alpha is typically estimated using historical data as the intercept (a) of the following regression equation adapted from the Alpha, Beta, and Hypothesis Testing session.


\begin{equation*}
R_{t}-R_{f}=a+b\left(R_{m t}-R_{f}\right)+e_{t} \tag{2}
\end{equation*}


where $R_{t}$ is the return of the portfolio or asset in period $t, R_{m t}$ is the return of the market portfolio in time $t$, $a$ is the estimated intercept of the regression, $b$ is the estimated slope coefficient of the regression, and $e_{t}$ is the residual of the regression in time $t$. The error term $e_{t}$ estimates the idiosyncratic return of the portfolio in time $t, b$ is an estimate of the portfolio's beta, and $a$ is an estimate of the portfolio's average abnormal or idiosyncratic return. Since the intercept, $\alpha$, is estimated, it should be interpreted subject to levels of confidence.

\section*{$M^{2}$ (M-Squared) Approach}
The $\mathrm{M}^{2}$ approach, or $\mathrm{M}$-squared approach, expresses the excess return of an investment after its risk has been normalized to equal the risk of the market portfolio. The first step is to leverage or deleverage the investment so that its risk matches the risk of the market portfolio. The superior (or inferior) return that the investment offers relative to the market when it has been leveraged or deleveraged to have the same volatility as the market portfolio is $\mathrm{M}^{2}$. A fund is leveraged to a higher level of risk when money is borrowed at the riskless rate and invested in the fund, and a fund is deleveraged when money is allocated to the riskless asset rather than invested in the fund.

Consider three funds with excess returns and volatilities as expressed in the second and third columns of the next exhibit. Note that the three funds differ in volatility (column 3), so their returns cannot be directly compared.

\begin{center}
\begin{tabular}{|lccccccc|}
\hline
(1) & \begin{tabular}{c}
(2) \\
Fxcess \\
Return \\
\end{tabular} & \begin{tabular}{c}
(3) \\
Fund \\
Volatility \\
\end{tabular} & \begin{tabular}{c}
(4) \\
Sharpe \\
Ratio \\
\end{tabular} & \begin{tabular}{c}
(5) \\
Portfolio \\
Weight \\
\end{tabular} & \begin{tabular}{c}
(6) \\
Portfolio \\
Volatility \\
\end{tabular} & \begin{tabular}{c}
(7) \\
Portfolio \\
ExcessReturn \\
\end{tabular} & \begin{tabular}{c}
(8) \\
Fund \\
$\mathbf{M}^{2}$ \\
\end{tabular} \\
\hline
A & $3 \%$ & $5 \%$ & .60 & $200 \%$ & $10 \%$ & $6 \%$ & $6 \%+R_{f}$ \\
B & $5 \%$ & $10 \%$ & .50 & $100 \%$ & $10 \%$ & $5 \%$ & $5 \%+R_{f}$ \\
C & $6 \%$ & $15 \%$ & .40 & $67 \%$ & $10 \%$ & $4 \%$ & $4 \%+R_{f}$ \\
\hline
\end{tabular}
\end{center}

\section*{Sample Computations of $\mathrm{M}^{2}$}
The Sharpe ratio in column 4 reveals that Fund A provides the best excess return per unit of standard deviation. The $\mathrm{M}^{2}$ approach shows Fund A's superior potential with a different metric in light of the opportunity provided by the market portfolio. Assuming that the volatility of the market portfolio is estimated to be $10 \%$, the first step of the $\mathrm{M}^{2}$ approach is to leverage or deleverage each of the funds into a total portfolio that has the same volatility as the market portfolio, which is $10 \%$. Columns 5, 6, and 7 indicate leveraging (Fund A) and deleveraging (Fund C) to create risk levels equal to that of the market. To invest in Fund A, which has a volatility\\
of $5 \%$, with a total volatility of $10 \%$, a manager would use 2:1 leverage, effectively allocating a weight of $+200 \%$ to Fund A and $-100 \%$ to the riskless asset, as indicated in column 5. To invest in Fund B with a total volatility of $10 \%$, the manager can simply allocate $100 \%$ of a portfolio to Fund B. Finally, to invest in Fund C with a total volatility of $10 \%$, the manager allocates $67 \%$ of the portfolio to Fund C and the remaining $33 \%$ to the riskless asset. Using leverage and deleverage, all three alternatives can be used to generate portfolios with the same expected volatility as the market, or $10 \%$, as indicated in column 6 . The excess returns of the portfolios, found by multiplying the alphas of the funds in column 2 by the weight of the fund in the portfolio column 5 , are shown in column 7.

The most attractive alternative, using Fund A with leverage, is the alternative with the highest excess return, since all three portfolios have the same volatility. The expected return of each portfolio is $\mathrm{M}^{2}$, which is shown in column 8 by adding the riskless rate to the excess return in column $7 ; \mathrm{M}^{2}$ provides an estimate of the expected return that an investor can earn using a specified investment opportunity and taking a level of total risk equal to that of the market portfolio. Equation 3 provides the formula for $\mathrm{M}^{2}$ :


\begin{equation*}
\mathrm{M}^{2}=R_{f}+\left\{\left(\sigma_{m} / \sigma_{p}\right)\left[E\left(R_{p}\right)-R_{f}\right]\right\} \tag{3}
\end{equation*}


where $R_{f}$ is the riskless rate, $\sigma_{m}$ is the volatility of the market portfolio, $\sigma_{p}$ is the volatility of the portfolio or asset for which $\mathrm{M}^{2}$ is being calculated, and $E\left(R_{p}\right)$ is the mean or expected return of the portfolio.

APPLICATION C

Consider the following information:

\begin{center}
\begin{tabular}{|l|c|c|}
\hline
 & Expected Return & Volatility \\
\hline
Market & $12 \%$ & $14 \%$ \\
\hline
Portfolio & $14 \%$ & $28 \%$ \\
\hline
Riskless Asset & $2 \%$ & $0 \%$ \\
\hline
\end{tabular}
\end{center}

$\mathrm{M}^{2}$ is found by inserting four of the values into Equation 3:

$$
\begin{array}{r}
\mathrm{M}^{2}=2 \%+\{(0.14 / 0.28)[0.14-0.02]\} \\
\mathrm{M}^{2}=2 \%+6 \%=8 \%
\end{array}
$$

It should be noted that there is an alternative formula for $\mathrm{M}^{2}$, sometimes called $\mathrm{M}^{2}$-alpha, which is slightly different from Equation 3. This alternative formula can be found both in Modigliani and Modigliani's original paper as well as in subsequent analyses by other authors. ${ }^{1}$ Franco Modigliani and Leah Modigliani, "Risk-Adjusted Performance," Journal of Portfolio Management 23, no. 2 (Winter 1997): 45-54. However, this text focuses on the $\mathrm{M}^{2}$ formula in Equation 3, which is more reflective of the totality of the original work.

The formula for $\mathrm{M}^{2}$ is an expected return or, in the case of an estimation using sample data, the mean return. Specifically, it is an estimated expected return on a strategy that uses borrowing or lending to bring the total volatility of the position equal to the volatility of the market portfolio. The first term on the right-hand side of the formula for $\mathrm{M}^{2}$ is the riskless rate, the compensation for the time value of money. The term in brackets is a risk premium specific to the portfolio or fund being analyzed. The ratio inside the first set of parentheses is the leverage factor that brings the volatility of the portfolio to the same level as the volatility of the market. That leverage factor is multiplied by the excess return of the underlying fund to form the excess return of the leveraged position.

\section*{Average Tracking Error}
An important concept in all investing is tracking error, discussed earlier in this session. Most applications of the concept of tracking error refer to it as the standard deviation of these differences. Thus, tracking error is most commonly viewed as a standard deviation. However, some sources use the term tracking error to refer generally to the differences through time between an investment's return and the return of its benchmark. When tracking error is used in the latter sense, the term average tracking error simply refers to the excess of an investment's return relative to its benchmark. In other words, it is the numerator of the information ratio.


\end{document}