\documentclass[11pt]{article}
\usepackage[utf8]{inputenc}
\usepackage[T1]{fontenc}
\usepackage{amsmath}
\usepackage{amsfonts}
\usepackage{amssymb}
\usepackage[version=4]{mhchem}
\usepackage{stmaryrd}
\usepackage{mathtools}

\begin{document}
\section*{APPLICATION A}
A portfolio is expected to earn $7 \%$ annualized return when the riskless rate is $4 \%$ and the expected return of the market is $8 \%$.

\section*{Answer and Explanation}
If the beta of the portfolio is 0.5 , the alpha of the portfolio is $1 \%$, found by substituting into Equation 1 and solving:

\$\$\\
\[
\begin{gathered}
\alpha_{\mathrm{P}}=\mathrm{E}\left(\mathrm{R}_{\mathrm{P}}\right)-\mathrm{R}_{f}-\beta_{\mathrm{P}}\left[\mathrm{E}\left(\mathrm{R}_{m}\right)-\mathrm{R}_{f}\right] \\
\alpha_{\mathrm{P}}=7 \%-4 \%-[0.5(8 \%-4 \%)]=1 \%
\end{gathered}
\]

\section*{APPLICATION B}
Consider a portfolio with $\mathrm{M} 2=4 \%$. The portfolio is expected to earn $10 \%$, whereas the riskless rate is only $2 \%$. What is the ratio of the volatility of the market to volatility of the portfolio?

\section*{Answer and Explanation}
In order to solve this problem, we need to manipulate Equation 3, like so:

$$
\begin{aligned}
& \mathrm{M}^{2}=\mathrm{R}_{f}+\left\{\left(\sigma_{m} / \sigma_{p}\right)\left[\mathrm{E}\left(\mathrm{R}_{p}\right)-\mathrm{R}_{f}\right]\right\} \\
& \mathrm{M}^{2}-\mathrm{R}_{f}=\left\{\left(\sigma_{m} / \sigma_{p}\right)\left[\mathrm{E}\left(\mathrm{R}_{p}\right)-\mathrm{R}_{f}\right]\right\} \\
& \mathrm{M}^{2}-\mathrm{R}_{f} /\left[\mathrm{E}\left(\mathrm{R}_{p}\right)-\mathrm{R}_{f}\right]=\left(\sigma_{m} / \sigma_{p}\right)
\end{aligned}
$$

Now, we can plug in the data provided $\mathrm{M}^{2}=4 \%, \mathrm{E}\left(\mathrm{R}_{p}\right)=10 \%$, and $\mathrm{R}_{f}=2 \%$ :

$$
\begin{gathered}
(0.04-0.02) /(0.1-0.02)=\sigma_{m} / \sigma_{p} \\
0.02 / 0.08=\sigma_{m} / \sigma_{p} \\
0.25=\sigma_{m} / \sigma_{p}
\end{gathered}
$$

The ratio of the volatilities is 0.25

\section*{APPLICATION C}
Consider the following information

\begin{center}
\begin{tabular}{|l|c|c|}
\hline
 & Expected Return & Volatility \\
\hline
Market & $12 \%$ & $14 \%$ \\
\hline
Portfolio & $14 \%$ & $28 \%$ \\
\hline
Riskless Asset & $2 \%$ & $0 \%$ \\
\hline
\end{tabular}
\end{center}

\section*{Answer and Explanation}
We are given all the information we need for this problem. To find $\mathrm{M}^{2}$, we must use Equation 3:

$$
\begin{aligned}
& \mathrm{M}^{2}=\mathrm{R}_{f}+\left\{\left(\sigma_{m} / \sigma_{p}\right)\left[\mathrm{E}\left(\mathrm{R}_{p}\right)-\mathrm{R}_{f}\right]\right\} \\
& \mathrm{M}^{2}=2 \%+\{(0.14 / 0.28)[0.14-0.02]\} \\
& \mathrm{M}^{2}=2+6 \%=8 \%
\end{aligned}
$$

\begin{center}
\begin{tabular}{|c|c|c|c|c|c|}
\hline
$M^{2}$ & \begin{tabular}{c}
Riskless \\
Rate \\
\end{tabular} & \begin{tabular}{c}
Market \\
Expected \\
Return \\
\end{tabular} & \begin{tabular}{c}
Market \\
Volatility \\
\end{tabular} & \begin{tabular}{c}
Portfolio \\
Expected \\
Return \\
\end{tabular} & \begin{tabular}{l}
Portfolio \\
Volatility \\
\end{tabular} \\
\hline
$8.00 \%$ & $2 \%$ & $12 \%$ & $14 \%$ & $14 \%$ & $28 \%$ \\
\hline
$11.50 \%$ & $3 \%$ & $10 \%$ & $15 \%$ & $20 \%$ & $30 \%$ \\
\hline
$35.00 \%$ & $5 \%$ & $10 \%$ & $30 \%$ & $35 \%$ & $30 \%$ \\
\hline
$10.40 \%$ & $2 \%$ & $14 \%$ & $18 \%$ & $16 \%$ & $30 \%$ \\
\hline
$8.00 \%$ & $4 \%$ & $13 \%$ & $15 \%$ & $12 \%$ & $30 \%$ \\
\hline
$15.00 \%$ & $10 \%$ & $20 \%$ & $30 \%$ & $15 \%$ & $30 \%$ \\
\hline
\end{tabular}
\end{center}


\end{document}