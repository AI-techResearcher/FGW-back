\documentclass[11pt]{article}
\usepackage[utf8]{inputenc}
\usepackage[T1]{fontenc}
\usepackage{amsmath}
\usepackage{amsfonts}
\usepackage{amssymb}
\usepackage[version=4]{mhchem}
\usepackage{stmaryrd}

\begin{document}
\section*{Reading}
Ratio-Based Performance Measures

There are two major types of performance measures. The first uses ratios of return to risk. With this method, return can be expressed in numerous ways in the numerator, and risk can be expressed in numerous ways in the denominator. This section discusses the most useful and common return-to-risk ratios. A second method for measuring performance involves estimating the risk-adjusted return of an asset that can be compared with a standard. This and other approaches are discussed in this section.

The numerator of ratio-based performance measures is based on the expected return or the average historical return of the given asset. The numerator usually takes one of three forms: (1) the asset's average return, (2) the asset's average return minus a benchmark or target rate of return, and (3) the asset's average return minus the riskless rate.

The denominator of the ratio can be virtually any risk measure, although the most popular performance measures use the most widely used risk measures, such as volatility (standard deviation) or beta. The risk measure may be an observed estimate of risk or the investor's belief regarding expected risk. This section discusses the most common ratio-based performance measures in alternative investment analysis.

\section*{The Sharpe Ratio}
The most popular measure of risk-adjusted performance for traditional investments and traditional investment strategies is the Sharpe ratio. The Sharpe ratio has excess return as its numerator and volatility as its denominator:


\begin{equation*}
S R=\left[E\left(R_{p}\right)-R_{f}\right] / \sigma_{p} \tag{1}
\end{equation*}


where $S R$ is the Sharpe ratio for portfolio $p, E\left(R_{p}\right)$ is the expected return for portfolio $p, R_{f}$ is the riskless rate, and $\sigma_{p}$ is the standard deviation of the returns of portfolio $p$. The numerator is the portfolio's expected or average excess return, where expected or average excess return is defined as expected or average total return minus the riskless rate.

The following examples further illustrate use of the Sharpe ratio.

The Sharpe ratio facilitates comparison of investment alternatives and the selection of the opportunity that generates the highest excess return per unit of total risk. However, the denominator of the Sharpe ratio (the standard deviation) does not reflect the marginal contribution of risk that occurs when an asset is added to a portfolio with which it is not perfectly correlated. In other words, the actual additional risk that the inclusion of an asset causes to a portfolio is less than the standard deviation whenever that asset helps diversify the portfolio. Accordingly, it can be argued that the Sharpe ratio should be used only on a stand-alone basis and not in a portfolio context.

It should be obvious that both the numerator and the denominator of the Sharpe ratio should be measured in the same unit of time, such as quarterly or annual values. The resulting Sharpe ratio, however, is sensitive to the length of the time period used to compute the numerator and the denominator. Note that the numerator is proportional to the unit of time, ignoring compounding. Thus, the excess return expressed as an annual rate will be two times larger than a semiannual rate and four times larger than a quarterly rate, ignoring compounding. However, the denominator is linearly related to the square root of time, assuming that returns are statistically independent through time:


\begin{equation*}
\sigma_{T}=\sigma_{1} \sqrt{T} \tag{2}
\end{equation*}


where $\sigma T$ is the standard deviation over $T$ periods; $\sigma 1$ is the standard deviation over one time period, such as one year; and $T$ is the number of time periods.

This formula assumes that the returns through time are statistically independent. Thus, a one-year standard deviation is only $\sqrt{2}$ times a semiannual standard deviation, and a one-year standard deviation is only twice $(\sqrt{4})$ the quarterly standard deviation. Thus, switching from quarterly returns to annualized returns roughly increases the numerator fourfold but increases the denominator only twofold, resulting in a twofold higher ratio.

If returns were perfectly correlated through time, the Sharpe ratio would not be sensitive to the time unit of measurement; it would be dimensionless. However, in a perfect financial market, returns are expected to be statistically independent through time, and in practice, returns are usually found to be somewhat statistically independent through time. The point is that Sharpe ratio comparisons must be performed using the same return intervals.

Sharpe ratios should be computed and compared consistently with the same units of time, such as with annualized data. Sharpe ratios can then be easily intuitively interpreted and compared across investments. However, Sharpe ratios ignore diversification effects and are primarily useful in comparing returns only on a standalone basis. This means that Sharpe ratios should typically be used when examining total portfolios rather than evaluating components that will be used to diversify a portfolio. Of course, if the investments being compared are well-diversified portfolios, then the Sharpe ratio is appropriate, since systematic risk and total risk are equal in well-diversified portfolios. It should be noted that in the field of investments, the term well-diversified portfolio is traditionally interpreted as any portfolio containing only trivial amounts of diversifiable risk.

Finally, a Sharpe ratio is only as useful as volatility is useful in measuring risk. In the case of normally distributed returns, the volatility fully describes the dispersion in outcomes. But in the many alternative investments with levels of skew and kurtosis that deviate from the normal distribution, volatility provides only a partial measure of dispersion. Thus, the Sharpe ratio is a less valuable measure of risk-adjusted performance for asset returns with non-normal distributions.

\section*{Four Important Properties of the Sharpe Ratio}
As detailed in the previous section, the Sharpe ratio has the following four important properties:

\begin{enumerate}
  \item It is intuitive. Using annual or annualized data, the Sharpe ratio reflects the added annual excess return per percentage point of annualized standard deviation.

  \item It is a measure of performance that is based on stand-alone risk, not systematic risk. Therefore, it does not reflect the marginal risk of including an asset in a portfolio when there is diversifiable risk.

  \item It is sensitive to dimension. The Sharpe ratio changes substantially if the unit of time changes, such as when quarterly rates are used rather than annualized rates.

  \item It is less useful in comparing investments with returns that vary by skew and kurtosis.

\end{enumerate}

The Sharpe ratio should be used with caution when measuring the performance of particular alternative investments, such as hedge funds. Research has shown that the Sharpe ratio may be manipulated (to the benefit of a hedge fund manager) using optionlike strategies that prevent extreme returns.

\section*{The Treynor Ratio}
Another popular measure of risk-adjusted performance for traditional investments and traditional investment strategies is the Treynor ratio, which differs from the Sharpe ratio by the use of systematic risk rather than total risk. The Treynor ratio has excess return as its numerator and beta as the measure of risk as its denominator:


\begin{equation*}
T R=\left[E\left(R_{p}\right)-R_{f}\right] / \beta_{p} \tag{3}
\end{equation*}


where $T R$ is the Treynor ratio for portfolio $p ; E\left(R_{p}\right)$ is the expected return, or mean return, for portfolio $p ; R_{f}$ is the riskless rate; and $\beta_{p}$ is the beta of the returns of portfolio $p$.

The Treynor ratio offers the intuition of estimating the excess return of an investment relative to its systematic risk. The Treynor ratio can be directly compared to the equity risk premium discussed in the Alpha, Beta, and Hypothesis Testing session.

Unlike the Sharpe ratio, the Treynor ratio should not be used on a stand-alone basis. Beta is a measure of only one type of risk: systematic risk. Therefore, selecting a stand-alone investment on the basis of the Treynor ratio might tend to maximize excess return per unit of systematic risk but not maximize excess return per unit of total risk unless each investment were well diversified. Beta does, however, serve as an appropriate measure of the marginal risk of adding an investment to a welldiversified portfolio. In this way, the Treynor ratio is designed to compare well-diversified investments and to compare investments that are to be added to a welldiversified portfolio. But the Treynor ratio should not be used to compare poorly diversified investments on a stand-alone basis. The Treynor ratio is less frequently applied in alternative investments, as beta is not an appropriate risk measure for many alternative investment strategies.

The Treynor ratio depends on the unit of time used to express returns. Generally, the beta of an asset (the denominator of the ratio) would be expected to be quite similar, regardless of the unit of time used to express returns. However, ignoring compounding, the quarterly returns would be expected to be one-quarter the\\
magnitude of annual returns, and monthly returns would be expected to be one-twelfth the magnitude of annual returns. Thus, the numerator is proportional to the time unit, and the denominator is roughly independent of the time unit, meaning that the ratio is proportional to the unit of time.

\section*{Four Important Properties of the Treynor Ratio}
As detailed in the previous section, the Treynor ratio has the following four important properties:

\begin{enumerate}
  \item It is highly intuitive. Using annual or annualized data, the Treynor ratio reflects the added annual excess return per unit of beta.

  \item It is a measure of performance that is based on systematic risk, not stand-alone risk. Therefore, it does not reflect the marginal total risk of including an asset in a portfolio that is poorly diversified.

  \item It is directly proportional to its dimension. The Treynor ratio varies directly with the unit of time used, such that ratios based on annualized rates tend to be four times larger than ratios based on quarterly rates.

  \item It is less useful in comparing investments with returns that vary by skew and kurtosis, because beta does not capture higher moments.

\end{enumerate}

\section*{The Sortino Ratio}
A measure of risk-adjusted performance that tends to be used more in alternative investments than in traditional investments is the Sortino ratio. The Sortino ratio subtracts a benchmark return, rather than the riskless rate, from the asset's return in its numerator and uses downside standard deviation as the measure of risk in its denominator:

Sortino Ratio $\left[E\left(R_{p}\right)-R_{\text {Target }}\right] / T S S D$

where $E\left(R_{p}\right)$ is the expected return, or mean return in practice, for portfolio $p ; R_{\text {Target }}$ is the user's target rate of return; and TSSD is the target semistandard deviation (or downside deviation), discussed earlier in the session.

As a semistandard deviation, the TSSD focuses on the downside deviations. As a target semistandard deviation, TSSD defines a downside deviation as the negative deviations relative to the target return, rather than a mean return or zero. Thus, the Sortino ratio uses the concept of a target rate of return in expressing both the return in the numerator and the risk in the denominator.

Even if the target return is set equal to the riskless rate, the Sortino ratio is not equal to the Sharpe ratio. Although they would share the same numerator, the denominator would be the same only when distributions were perfectly symmetrical. The point is that the emphasis of the Sortino ratio is the use of downside risk rather than the use of a target rate of return. To the extent that a return distribution is nonsymmetrical and the investor is focused on downside risk, the Sortino ratio can be useful as a performance indicator.

\section*{The Information Ratio}
The information ratio provides a sophisticated view of risk-adjusted performance. The information ratio has a numerator formed by the difference between the average return of a portfolio (or other asset) and its benchmark, and a denominator equal to its tracking error:


\begin{equation*}
\text { Information ratio }=\left[E\left(R_{p}\right)-R_{\text {Benchmark }}\right] / T E_{p} \tag{5}
\end{equation*}


where $E\left(R_{p}\right)$ is the expected or mean return for portfolio $p, R_{\text {Benchmark }}$ is the expected or mean return of the benchmark, and $T E_{p}$ is the tracking error of the portfolio $p$ relative to its benchmark return.

Tracking error, which was discussed earlier in this session, may be approximately viewed as the typical amount by which a portfolio's return deviates from its benchmark. Technically speaking, tracking error is the standard deviation of the differences through time of the portfolio's return and the benchmark return.

The numerator is the average amount by which the portfolio exceeds its benchmark return (if positive). Thus, the information ratio is the amount of added return, if positive, that a portfolio generates relative to its benchmark for each percentage by which the portfolio's return typically deviates from its benchmark.

Like the Sharpe ratio, the information ratio is sensitive to whether it is computed using annualized returns or periodic (e.g., quarterly) returns. The information ratio is higher when the portfolio's average return is higher, and lower when the portfolio deviates from its benchmark by larger amounts. Accordingly, the use of the information ratio is an attempt to drive the portfolio toward investments that track the benchmark well but consistently outperform the benchmark.

\section*{Return on VaR}
Value at risk (VaR) was detailed earlier in this session as a measure of potential risk for a specified time horizon and level of confidence. Return on VaR (RoVaR) is simply the expected or average return of an asset divided by a specified VaR (expressing VaR as a positive number):

$$
\mathrm{RoVaR}=E\left(R_{p}\right) / \mathrm{VaR}
$$

In cases in which VaR is a good summary measure of the risks being faced, RoVaR may be a useful metric. In such cases, the risks of the investment alternatives typically share similarly shaped return distributions that are well understood by the analysts using the ratio.


\end{document}