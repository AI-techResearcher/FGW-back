\documentclass[11pt]{article}
\usepackage[utf8]{inputenc}
\usepackage[T1]{fontenc}

\begin{document}
\section*{Reading}
Benchmarking and Performance Attribution

Benchmarking and performance attribution are especially important and challenging in alternative investment management. These two important topics are covered in greater detail in Level II.

\section*{Benchmarking Overview}
The starting point for analyzing the risk and return of an investment is often to compare the investment with a benchmark. Benchmarking, often referred to as performance benchmarking, is the process of selecting an investment index, an investment portfolio, or any other source of return as a standard (or benchmark) for comparison during performance analysis. Benchmarking is typically performed by investors and analysts external to an investment pool for the purpose of monitoring performance. Fund managers may be reluctant to adopt or declare a benchmark because they may believe that the performance of their investment strategy cannot be properly linked to a benchmark or may prefer the investment flexibility of not having their performance tied to a specific benchmark.

\section*{Types of Benchmarks}
The return on a benchmark is usually calculated as an average of the returns from a number of assets. There are two general types of benchmark returns that might be used in the analysis of fund performance: peer and index.

Peer benchmarks are based on the returns of a comparison or peer group. The peer group is typically a group of funds with similar objectives, strategies, or portfolio holdings. The group may include virtually all possible comparison funds, known as a universe group, or a sampling. Instead of using a peer group of highly similar funds, a comparison group may be formed that contains many or all of the underlying securities that a fund might have in its portfolio. Unlike indices, comparison groups and peer groups tend to be customized for the specific needs of an investor analyzing one or more holdings. Thus, a particular financial institution, such as a pension fund or a pension consulting firm, might create comparison groups to benchmark managers against similar funds. Often the mean or median return of the group is subtracted from the return of the fund being analyzed to estimate abnormal returns. Also, the return of a fund being analyzed might be displayed in a graph or table alongside all the returns from a comparison group, rather than simply summarized using the mean or median return. The returns of a fund relative to its peer group are often expressed as a ranking or percentile in relation to the group.

Indices such as the MSCI World Index, a highly diversified equity index including stocks from 23 developed countries, and the Russell 2000 Index are commonly used as benchmarks. Indices typically reflect weighted averages of the returns of a set of securities or funds. Indices tend to be used for a more general audience and are often available for use by a variety of investors to gauge the performance of an investment, a market, or a sector.

\section*{Performance Attribution}
Performance attribution, also known as return attribution, is the process of identifying the components of an asset's return or performance. Performance attribution seeks to separate the total return of an investment into quantities that can be linked to various determinants such as market factors (e.g., the return of one or more indices) and managerial factors (e.g., market timing and superior asset selection).

Benchmarking is a simpler, popular, and practical form of attributing return. In benchmarking, the return of an asset is simply divided into two components: the benchmark return and the active return. The active return is the deviation of an asset's return from its benchmark. The benchmark's return is subtracted from the asset's return for the same time period to form the active return. In effect, the benchmark return is attributed to the systematic performance of the asset, and the active return is attributed to the idiosyncratic performance of the asset.


\end{document}