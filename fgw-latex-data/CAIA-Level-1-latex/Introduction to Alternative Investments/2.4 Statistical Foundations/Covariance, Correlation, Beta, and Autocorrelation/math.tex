\documentclass[11pt]{article}
\usepackage[utf8]{inputenc}
\usepackage[T1]{fontenc}
\usepackage{amsmath}
\usepackage{amsfonts}
\usepackage{amssymb}
\usepackage[version=4]{mhchem}
\usepackage{stmaryrd}

\begin{document}
\section*{APPLICATION A}
A returns series indicates first-order autocorrelation of 0.6 and second-order autocorrelation of 0.2 . What is the partial second-order autocorrelation coefficient?

\section*{Answer and Explanation}
In order to determine the second-order autocorrelation coefficient, we must use Equation 8.

Second - Order Partial Autocorrelation Coefficient $=\frac{\rho_{2}-\rho_{1}^{2}}{1-\rho_{1}^{2}}$

Inserting 0.6 for $\rho_{1}$ and 0.2 for $\rho_{2}$ generates: $[0.2-(0.6 \times 0.6)] /([1-(0.6 \times 0.6)]=-0.25$\\
Note that even though the returns in periods $\mathrm{k}$ and $\mathrm{k}-2$ are positively correlated, that correlation is primarily driven by first-order correlation. The marginal second-order effect is captured in the partial autocorrelation coefficient and indicates a mean-reverting effect (once the first-order effects have been removed).


\end{document}