\documentclass[11pt]{article}
\usepackage[utf8]{inputenc}
\usepackage[T1]{fontenc}
\usepackage{amsmath}
\usepackage{amsfonts}
\usepackage{amssymb}
\usepackage[version=4]{mhchem}
\usepackage{stmaryrd}

\begin{document}
\section*{Reading}
Return Distributions

This session provides foundational material regarding statistical methods for the study of alternative investments in general and for the subsequent material in this curriculum in particular. The use of statistics in performing hypothesis tests is addressed in detail in the Alpha, Beta, and Hypothesis Testing session.

Risky assets experience unexpected value changes and therefore unexpected returns. If we assume that investors are rational, the more competitively traded an asset, the more these unexpected price changes may be random and unpredictable. Hence, asset prices and asset returns in competitively traded markets are typically modeled as random variables. Frequency and probability distributions therefore provide starting points for describing asset returns.

\section*{Ex Ante and Ex Post Return Distributions}
Ex post returns are realized outcomes rather than anticipated outcomes. Future possible returns and their probabilities are referred to as expectational or ex ante returns. A crucial theme in understanding the analysis of alternative investments is to understand the differences and links between ex post and ex ante return data.

Often, predictions are formed partially or fully through analysis of ex post data. For example, the ex ante or future return distribution of a stock index such as the S\&P 500 Index is often assumed to be well approximated by the ex post or historical return distribution. The direct use of past return behavior as a predictor of future potential return behavior requires two properties to be accurate. First, the return distribution must be stationary through time, meaning that the expected return and the dispersion of the underlying asset do not change. Second, the sample of past observations must be sufficiently large to be likely to form a reasonably accurate representation of the process. For example, equity returns were very high during the bull market decade of the 1990s, very low during the early years of the financial crisis (2007-08), and high in the decade subsequent to the crisis. Using any of these time periods in isolation would likely overstate or understate the realistic longrun equity market returns.

Taken together, the requirements for the past returns to be representative of the future returns raise a serious challenge. If the past observation period is long, the sample of historical returns will be large; however, it is likely that the oldest observations reflect different risks or other economic conditions than can be anticipated in the future. If the sample is limited to the most recent observations, the data may be more representative of future economic conditions, but the sample may be too small to draw accurate inferences from it.

For a traditional asset, such as the common stock of a large, publicly traded corporation, it may be somewhat plausible that the asset's past behavior is a reasonable indication of its future behavior. However, many alternative investments are especially problematic in this context. For example, historical data may not exist for venture capital investments in new technology or may be difficult to observe or to obtain in cases such as private equity, where most or all trades are not publicly observable. Especially in alternative investments such as hedge funds, return distributions are expected to change as the fund's investment strategies and use of leverage change through time. In these cases and many others, ex ante return distributions may need to be based on economic analysis and modeling rather than simply projected from ex post data.

Nevertheless, whether based on prior observations or on economic analysis, the return distribution is a central tool for understanding the characteristics of an investment. The normal distribution is the starting point for most statistical applications in investments.

\section*{The Normal Distribution}
The normal distribution is the familiar bell-shaped distribution, also known as the Gaussian distribution. The normal distribution is symmetric, meaning that the left and right sides are mirror images of each other. Also, the normal distribution clusters or peaks near the center, with decreasing probabilities of extreme events.

Why is the normal distribution so central to statistical analysis in general and the analysis of investment returns in particular? One reason is empirical: The normal distribution tends to approximate many distributions observed in nature or generated as the result of human actions and interactions, including financial return distributions. Another reason is theoretical: The more a variable's change results from the summation of a large number of independent causes, the more that variable tends to behave like a normally distributed variable. Thus, the more competitively traded an asset's price is, the more we would expect that the price change over a small unit of time would be the result of hundreds or thousands of independent financial events and/or trading decisions. Therefore, the probability distribution of the resulting price change should resemble the normal distribution. The formal statistical explanation for the idea that a variable will tend toward a normal distribution as the number of independent influences becomes larger is known as the central limit theorem. Practically speaking, the normal distribution is relatively easy to use, which may explain some of its popularity.

\section*{Log Returns and the Lognormal Distribution}
For simplicity, funds often report returns based on discrete compounding. However, log returns offer a distinct advantage, especially for modeling a return probability distribution. In a nutshell, the use of log returns allows for the modeling of different time intervals in a manner that is simple and internally consistent. Specifically, if daily log returns are normally distributed and independent through time, then the log returns of other time intervals, such as months and years, will also be normally distributed. The same cannot be said of simple returns. Let's take a closer look at why log returns have this property.

The normal distribution replicates when variables are added but not when they are multiplied. This means that if two variables, $x$ and $y$, are normally distributed, then the sum of the two variables, $x+y$, will also be normally distributed. But because the normal distribution does not replicate multiplicatively, $x \times y$ would not be normally distributed. Aggregation of discretely compounded returns is multiplicative. Thus, if $R_{1}, R_{2}$, and $R_{3}$ represent the returns for months 1,2 , and 3 using discrete compounding, then the product $\left[\left(1+R_{1}\right)\left(1+R_{2}\right)\left(1+R_{3}\right)\right]-1$ represents the return for the calendar quarter that contains the three months. If the monthly returns are normally distributed, then the quarterly return is not normally distributed, and vice versa, since the normal distribution does not replicate multiplicatively. Therefore, modeling the distribution of discretely compounded returns as being normally distributed over a particular time interval (e.g., monthly) technically means that the model will not be valid for any other choice of time interval (e.g., daily, weekly, annually).

However, the use of log returns, discussed in the Quantitative Foundations session, solves this problem. If $R_{1}^{m=\infty}, R_{2}^{m=\infty}$, and $R_{3}^{m=\infty}$ are monthly log returns, then the quarterly log return is simply the sum of the three monthly log returns. The normal distribution replicates additively; thus, if the log returns over one time interval can be modeled as being normally distributed, then the log returns over all time intervals will be lognormal as long as they are statistically independent through time.

Further, log returns have another highly desirable property. The highest possible simple (non-annualized) return is theoretically $+\infty$, while the lowest possible simple return for a cash investment is a loss of $-100 \%$, which occurs if the investment becomes worthless. However, the normal distribution spans from $-\infty$ to $+\infty$, meaning that simple returns, theoretically speaking, cannot truly be normally distributed; a simple return of $-200 \%$ is not possible. Thus, the normal distribution may be a poor approximation of the actual probability distribution of simple returns. However, log returns, like the normal distribution itself, can span from $-\infty$ to $+\infty$. There are two equivalent approaches to model returns that address these problems: (1) use log returns and assume that they are normally distributed, or (2) add 1 to the simple returns and assume that it has a lognormal distribution. A variable has a lognormal distribution if the distribution of the logarithm of the variable is normally distributed. The two approaches are identical, since the lognormal distribution assumes that the logarithms of the specified variable (in this case, $1+R$ ) are normally distributed.

In summary, it is possible for returns to be normally distributed over a variety of time intervals if those returns are expressed as log returns (and are independent through time). If the log returns are normally distributed, then the simple returns (in the form $1+R$ ) are said to be lognormally distributed. However, if discretely compounded returns $(R)$ are assumed to be normally distributed, they can only be normally distributed over one time interval, such as daily, since returns computed over other time intervals would not be normally distributed due to compounding.


\end{document}