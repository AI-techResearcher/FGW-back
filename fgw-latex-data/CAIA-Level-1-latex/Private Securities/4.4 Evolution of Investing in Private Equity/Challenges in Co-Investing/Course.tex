\documentclass[11pt]{article}
\usepackage[utf8]{inputenc}
\usepackage[T1]{fontenc}
\usepackage{amsmath}
\usepackage{amsfonts}
\usepackage{amssymb}
\usepackage[version=4]{mhchem}
\usepackage{stmaryrd}

\begin{document}
Challenges in Co-Investing

\section*{The Challenges of Co-Investing from the LP Perspective}
Co-investments offer potential benefits to both investors and GPs. However, not all investors and GPs will execute co-investments for several reasons. The challenges of co-investments for an investor are:

\begin{itemize}
  \item Need for in house resources and expertise. Building a co-investment program takes time, resources, and people. In addition to requiring staff with direct investing experience to source and evaluate co-investment opportunities, an investor will also need to have the legal and accounting infrastructure as well as an efficient investment decision process to quickly make investment decisions.
  \item Access to co-investments. According to a survey of LPs, ${ }^{1}$ Preqin Fund Manager Survey, August 2015 about a quarter of LPs do not get co-investment rights after requesting for them from GP. The top considerations for GPs when giving co-investment rights are the speed at which LP can decide on co-investment and the size of LP fund commitment.
  \item Reduced diversification. Co-investing is effectively taking on deal selection risk, which has a wider return dispersion than fund investing. When making a coinvestment decision, an investor is adding a specific exposure in a specific point in time as different sectors generate varying returns over time.
  \item Organizational constraints. Some organizations may have specific restrictions, such as restrictions on travel or setting up of overseas offices. These may hinder the investor's ability to perform on-site due diligence on deals or the type of deals an investor can pursue. An alternative is to outsource some of these functions, which could be more expensive than an in-house function. Furthermore, an investor may have budget constraints that prevent the compensation required to build a co-investment team.
\end{itemize}

\section*{The Challenges of Co-Investing from the GP Perspective}
Most GPs have offered co-investment rights to their LP, or in the process of considering. This is reflected in a Preqin survey from 2015 where about two-third of their surveyed GPs offer co-investment rights and a further $18 \%$ are considering doing so. Overall, the surveyed GPs believe that the benefits of offering co-investments outweigh the disadvantages. Only $2 \%$ of the surveyed GP stated that there are no benefits in offering co-investment rights to LPs. Nonetheless, whether these coinvestment rights translate into actual co-investments still faces some hurdles:

\begin{itemize}
  \item Mostly talk and no action. Despite a lot of LPs negotiating hard for co-investment rights, very few execute co-investment. According to a 2018 Hamilton Lane $^{2}$ Hamilton Lane (June 2019). Co-investing: The Struggle Is Real, p. 3. survey of GPs, over half the GPs surveyed say that less than a quarter of their LP who have asked for co-investment opportunities are transacting.
  \item Delay in deal process. A GP may have to go through round robin offerings of any co-investment opportunities to LPs. This can be a time and labor-intensive process. Furthermore, if there is a need for negotiations with co-investors on deal terms such as transaction fees, costs, or promote, the deal process can be further delayed.
  \item Additional costs/resources. With co-investors, GPs may face additional costs associated with reporting or setting up of a special-purpose vehicle to meet the co-investors' requirements.
  \item Negative impact on relationship with other LPs. GPs also need to manage how information about co-investment opportunities are distributed to LPs to prevent perceived bias. In addition, some believe there may be negative impact on relationships with LPs who are not offered co-investment rights.
\end{itemize}

\end{document}