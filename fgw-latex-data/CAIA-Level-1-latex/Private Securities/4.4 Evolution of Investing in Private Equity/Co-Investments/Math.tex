\documentclass[11pt]{article}
\usepackage[utf8]{inputenc}
\usepackage[T1]{fontenc}
\usepackage{amsmath}
\usepackage{amsfonts}
\usepackage{amssymb}
\usepackage[version=4]{mhchem}
\usepackage{stmaryrd}
\usepackage{multirow}

\begin{document}
\section*{APPLICATION A}
Question : An investor plans to make a $\$ 100$ million investment into an investment program. The underlying portfolio is expected to generate an IRR of $20 \%$ and an equity multiple of $2.49 x$ over hold a period of 5 years. The table below shows the return tabulated for 4 different investment programs:

\begin{center}
\begin{tabular}{|c|c|c|c|c|}
\hline
INVESTMENT PROGRAM & (1): Fund Investment & (2): Co-investments-Option 1 & (3): Co-investments-Option 2 & (4): Solo Investiong \\
\hline
 & Funds $(2 \% / 20 \%)$ & $1 \% / 10 \%$ & $0 \% / 20 \%$ & No Fees \\
\hline
Annual Management Fee (on invested equity) & $2.0 \%$ & $1.0 \%$ & $0.0 \%$ & $0.0 \%$ \\
\hline
Carried Interest (on proftt) & $20 \%$ & $10 \%$ & $20 \%$ & $0 \%$ \\
\hline
Cumulative Management Fees & 10.0 & 5.0 & - & - \\
\hline
\multicolumn{5}{|l|}{Return Comparison} \\
\hline
Net Equity Multiple (x) & 2.1 & 2.3 & 2.2 & 2.5 \\
\hline
Net IRR (\%) & $16 \%$ & $18 \%$ & $17 \%$ & $20 \%$ \\
\hline
\end{tabular}
\end{center}

a) What is the total amount of fees paid as a proportion of expected gross profit for each of the investment program?

b) Suppose the carried interest had a hurdle rate of $8 \%$ p.a. and the expected IRR failed to meet the hurdle rate, would Co-investment-Option 1 or Co-investment -Option 2 have a lower total fee?

c) Of the four investment programs, which requires highest level of involvement from the investor, and which requires the least level of involvement from the investor?

\section*{Answer and explanation}
a) The total cash flow from the underlying portfolio is $\$ 249 m(\$ 100 m \times 2.49 x)$ and the gross profit is $\$ 149 m(\$ 249 m-\$ 100 m)$.

The total fee as a proportion of the gross profit for each of the program is shown in the shaded row $M$ below.

b) If the IRR is below the hurdle rate, there will be no carry interest for the GP. Thus, Co-investment-Option 2 will have lower total fee than Co-investment-Option 1.

c) Of the 4 programs, solo investing requires the most involvement from the investor. Fund investment requires the least amount of involvement from the investor.

\begin{center}
\begin{tabular}{|c|c|c|c|c|c|}
\hline
A & Invested Equity (\$ mn) & 100 &  &  &  \\
\hline
\multirow[t]{3}{*}{B} & hold periods (years) & 5 &  &  &  \\
\hline
 & INVESTMENT PROGRAM: & (1): Fund Investment & (2): Co-investments-Option 1 & (3): Co-investments-Option 2 & (4): Solo Investing \\
\hline
 &  & Funds $(2 \% / 20 \%)$ & $1 \% / 10 \%$ & $0 \% / 20 \%$ & No fees \\
\hline
C & Annual Management Fee (on invested equity) & $2.0 \%$ & $1.0 \%$ & $0.0 \%$ & $0.0 \%$ \\
\hline
D & Carried Interest (on profit) & $20 \%$ & $10 \%$ & $20 \%$ & $0 \%$ \\
\hline
$\mathrm{E}=\mathrm{A} \times \mathrm{C} \times \mathrm{B}$ & Cumulative Management Fees & 10.0 & 5.0 & - & - \\
\hline
$F=K \times D$ & Carried Interest (on profit) & 27.8 & 14.4 & 29.8 & - \\
\hline
\multirow[t]{7}{*}{$G=E \times F$} & Total Mgmt Fee and carry & 37.8 & 19.4 & 29.8 & - \\
\hline
 & Return Comparison &  &  &  &  \\
\hline
 & Net Equity Multiple (x) & 2.1 & 2.3 & 2.2 & 2.5 \\
\hline
 & Net IRR (\%) & $16 \%$ & $18 \%$ & $17 \%$ & $20 \%$ \\
\hline
 & CALCULATIONS &  &  &  &  \\
\hline
 & Given: &  &  &  &  \\
\hline
 & Net IRR (\%) &  &  &  & $20.0 \%$ \\
\hline
$\mathrm{H}$ & Net Equity Multiple (x) &  &  &  & 2.49 \\
\hline
$\mathrm{I}=\mathrm{H} \times \mathrm{A}$ & Total CF & $\$ 248.8$ & $\$ 248.8$ & $\$ 248.8$ & $\$ 248.8$ \\
\hline
\multirow[t]{2}{*}{$J=I-A$} & Gross Profit & $\$ 148.8$ & $\$ 148.8$ & $\$ 148.8$ & $\$ 148.8$ \\
\hline
 & Less Cumulative Management Fee & $-\$ 10.0$ & $-\$ 5.0$ & $\$-$ & $\$-$ \\
\hline
\multirow[t]{2}{*}{$\mathrm{K}=\mathrm{J}-\mathrm{C}$} & Profit Mgmt Fee & $\$ 138.8$ & $\$ 143.8$ & $\$ 148.8$ & $\$ 148.8$ \\
\hline
 & Less: Carried Interest & $-\$ 27.8$ & $-\$ 14.4$ & $-\$ 29.8$ & $\$-$ \\
\hline
$L=K-F$ & Net Profit (at Carried Interest) & $\$ 111.1$ & $\$ 129.4$ & $\$ 119.1$ & $\$ 148.8$ \\
\hline
$M=G / J$ & $\%$ of Gross Profit & $25.4 \%$ & $13.0 \%$ & $20.0 \%$ & $0.0 \%$ \\
\hline
\end{tabular}
\end{center}


\end{document}