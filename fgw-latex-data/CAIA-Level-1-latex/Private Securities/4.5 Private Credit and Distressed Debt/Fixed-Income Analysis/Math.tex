\documentclass[11pt]{article}
\usepackage[utf8]{inputenc}
\usepackage[T1]{fontenc}
\usepackage{amsmath}
\usepackage{amsfonts}
\usepackage{amssymb}
\usepackage[version=4]{mhchem}
\usepackage{stmaryrd}

\begin{document}
\section*{APPLICATION A}
Question : A fixed-rate bond and a floating-rate note each have a five-year maturity. The fixed-rate bond has a duration of 4.5 years. The floating-rate note will reset its coupon to market rates in 0.5 years. Approximating bond price changes as the product of duration and interest rate shifts (times minus one), if continuously compounded interest rates increase by $1.5 \%$, how much will the price of the fixed-rate bond and floating-rate note decline?

\section*{Answer and explanation}
The price of the fixed-rate bond will drop by approximately $6.75 \%$ ( 4.5 duration $\times 1.5 \%$ rate rise). The price of the floating-rate note will drop by approximately $0.75 \%$ ( 0.5 duration $\times 1.5 \%$ rise).

This application is showing the relationships between changes in interest rates and a bond's duration. A bond's duration is the sensitivity to the changes in interest rates: the longer the duration, the greater the sensitivity. Remember, fixed rate bond durations tend to be close to their maturity, such as the five-year fixed-rate bond which has a duration of 4.5 years. The floating rate note also has a five-year maturity, but its duration is tied to the period in which the floating interest rate resets, which is 0.5 years. To calculate the change in both bonds' values when interest rates increase, simply multiply the change in rates by their respective durations:

Change in Bond Price $=$ Change in Interest Rates $\times-$ Duration

In the case of the fixed-rate bond, this equates to $+0.015 \times-4.5=-0.0675$. In the case of the floating-rate note, this equates to $+0.015 \times-0.5=-0.0075$.

\section*{APPLICATION B}
Question : A bond has a duration of exactly 5.0 and a stated annual yield of $4.00 \%$. Calculate the bond's modified duration for each of the following cases: annual compounding, semi-annual compounding, and continuous compounding.

\section*{Answer and explanation}
To solve this problem, we must use Equation 1:

$$
\text { Modified Duration }=\frac{\text { Duration }}{\left[1+\left(\frac{y}{m}\right)\right]}
$$

For annual compounding:

$$
\text { Modified Duration }=\frac{5.0}{\left[1+\left(\frac{0.04}{1}\right)\right]}=4.81
$$

For semiannual compounding:

$$
\text { Modified Duration }=\frac{5.0}{\left[1+\left(\frac{0.04}{2}\right)\right]}=4.90
$$

For continuously compounding:

$$
\text { Modified Duration }=\frac{5.0}{\left[1+\left(\frac{0.04}{0}\right)\right]}=5.0
$$

Notice continuously compounding puts " 0 " in the denominator of the $y / m$ function. This is because $m \rightarrow 0$ when continuously compounded, therefore the denominator is simply 0 .

Based on Equation 1, the modified duration $=5.0 /[1+(.04 / \mathrm{m})]$. Inserting values form of 1,2 , and infinity solves for the cases of annual compounding, semi-annual compounding, and continuous compounding, respectively, as 4.81, 4.90 and 5.00.


\end{document}