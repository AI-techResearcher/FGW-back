\documentclass[11pt]{article}
\usepackage[utf8]{inputenc}
\usepackage[T1]{fontenc}
\usepackage{amsmath}
\usepackage{amsfonts}
\usepackage{amssymb}
\usepackage[version=4]{mhchem}
\usepackage{stmaryrd}

\begin{document}
Direct Lending

Direct lending (also called market-based lending, shadow banking, or nonbank lending) is a transaction in which investors extend credit to borrowers outside of the traditional banking system. In direct lending, borrowers do not go to banks for their lending; rather, they obtain funding from loans originated by private equity, private credit, and hedge fund lenders. Because the borrowers don't have access to funds through bank loans, they might pay higher interest rates, but there are other benefits for the borrowers besides access to credit. Nonbank lenders tend to have more flexible terms and offer faster loan processing than banks.

The underwriting process for traditional loans typically focuses on credit analysis, which considers whether the cash flow of the firm is sufficient to service the debt. A portion of the private credit market is focused on asset-based lending, where lenders consider the value of the collateral rather than the strength of the firm's cash flows. Some direct lending might be unsecured, but most direct lending is secured or senior in the capital structure of the corporation. Revolving lines of credit can be secured by the value of inventory or accounts receivables, while term loans may be secured by the value of property, plant, and equipment.

Successful investors in the private credit market need to have skills that go beyond traditional credit analysis. Given that the borrowers are typically below investment grade, private credit firms need expertise in workouts and restructurings, knowing how to work with a borrower to get the loan back on track or knowing how to work though the legal system to protect the value of their loan. Perhaps the most valuable skill in the private credit market is sourcing deals by finding companies in need of financing without having a number of lenders bidding down the rate on the loan.

One key advantage to investors in direct-lending strategies is that fees charged by fund managers are typically assessed on invested capital, rather than on committed capital as is common for many private equity funds. These lower fees and higher cash yields make it unlikely for direct-lending funds to experience a Jcurve effect in which negative returns early in the life of the fund are followed by positive returns at the end of the fund when investments are exited.

Most of the direct-lending opportunity is to support middle-market corporations with revenues of up to $\$ 100$ million and EBITDA of up to $\$ 10$ million, with loans of up to $\$ 20$ million to $\$ 50$ million. Much of the origination activity is focused on first-lien and senior secured debt, placing the direct lender in control of a situation that moves toward distress. As the senior lender to a firm, investors can potentially seize the firm or specific assets, leaving equity holders and subordinated lenders with potentially large losses.

A growing subset of the direct-lending opportunity is peer-to-peer lending. Peer-to-peer lending is originating loans directly to consumers and is done by both institutional and retail investors who have an opportunity to originate consumer loans, often through an Internet-based underwriting and brokerage platform. Peerto-peer (P2P) lending can reduce the interest rates that consumers pay when they refinance other consumer credit, such as credit cards and student loans, with lower-cost P2P loans. Investors can earn a higher yield than on other forms of similarly risky credit if the spread between risk-free debt and credit card rates is wide. While the P2P business started as individuals lending to individuals, disintermediating banks from both savings and lending products, over two-thirds of the lending was funded by institutional investors by 2014. It is notable that some of the leading Internet-based brokerage firms in the P2P lending space fund loans for only $10 \%$ of potential borrowers, as the lenders are highly focused on funding only the most highly qualified borrowers seeking loans on these platforms.


\end{document}