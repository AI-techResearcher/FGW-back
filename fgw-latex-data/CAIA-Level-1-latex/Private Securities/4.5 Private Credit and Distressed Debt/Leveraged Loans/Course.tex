\documentclass[11pt]{article}
\usepackage[utf8]{inputenc}
\usepackage[T1]{fontenc}
\usepackage{amsmath}
\usepackage{amsfonts}
\usepackage{amssymb}
\usepackage[version=4]{mhchem}
\usepackage{stmaryrd}

\begin{document}
Leveraged Loans

Another asset class of fixed-income securities that private equity firms have moved into is leveraged loans, which are also referred to as senior loans or syndicated loans.

\section*{Leveraged Loan Basics}
Leveraged loans are syndicated bank loans to non-investment-grade borrowers. The term syndicated refers to the use of a group of entities, often investment banks, in underwriting a security offering or, more generally, jointly engaging in other financial activities. Loans made by banks to corporations can be divided into two general classes: (1) those made to companies with investment-grade credit ratings (BBB or Baa and above), and (2) those made to companies with non-investmentgrade credit ratings (BB or Ba and lower). This second class of loans refers to leveraged loans.

A leveraged loan is made to a corporate borrower that is leveraged-that is, a company that is not investment grade, often due to excess leverage on its balance sheet. Thus, the word leveraged refers to the use of leverage by the borrower. The loan has a second-lien interest after other senior secured loans. The second-lien loan market is often viewed synonymously with the leveraged loan market. Exact definitions of a leveraged loan vary. Generally, a loan is considered leveraged if (1) the borrower has outstanding debt that is rated below BBB by Standard \& Poor's or lower than Baa by Moody's, or (2) the loan bears a coupon that is in excess of 125 to 200 basis points over the London Interbank Offered Rate (LIBOR).

A leveraged loan for a firm without a credit rating is identified by having a coupon that is in excess of LIBOR by a particular number of basis points, which varies through time and by source. The standard for that spread should be linked to the spreads observed in credit markets for loans rated BB (Ba) or lower. In other words, a leveraged loan for an unrated firm would have a credit spread similar to the credit spreads on bank loans of firms with non-investment-grade credit ratings.

In many respects, leveraged loans are similar to high-yield debt or junk bonds in terms of credit rating and corporate profile. Many non-investment-grade corporations have both high-yield bonds and leveraged loans outstanding. Since private equity firms are accustomed to dealing with banks and other fixed-income investors to finance their buyouts, leveraged loans provide a natural extension of their financing business.

\section*{Growth in Leveraged Loans}
The growth of leveraged loans has been driven by the development and expansion of their secondary market. Secondary trading of leveraged loans improved substantially with the introduction of their credit ratings by recognized rating agencies. For example, Moody's began to assign credit ratings to bank loans in 1995 and has rated trillions of dollars of bank loans since. An active secondary market has encouraged banks to issue loans and has motivated institutions to invest in those loans. With the entry of institutional investors into this market through private equity vehicles, leveraged loans have become an accepted form of investing, and the rate of issuance of leveraged loans has surpassed that of high-yield bond financing.

Many large commercial banks have changed their business model from that of a traditional lender, in which the bank loans are kept on their balance sheets, to that of an originator and distributor of debt. These commercial banks are in the fee-generation business more than the asset-management business. Origination and distribution of bank loans allows these banks to both collect fees and manage their credit risk. In short, these commercial banks are capitalizing on their strengths: lending money, collecting loan fees, and then divesting the loans into a secondary market. The subsequent management of the resulting assets (the leveraged loans) is left to institutional investors who acquire the loans through the secondary market.

\section*{Liquidity and the Demand for Leveraged Loans}
Davies (2018a and 2018b) expresses concern for the leveraged loan market and its potential impact on liquidity, financing availability, and the outlook for future LBO activity. Although many loans have maturities of three to seven years, many corporate borrowers never fully retire their debt. With over $40 \%$ of leveraged loans being used to refinance existing debt, borrowers may not have the ability to pay off loans at maturity if and when credit conditions tighten and the refinancing window closes. In 2018, nearly $80 \%$ of newly issued leveraged loans were purchased by structured credit products such as collateralized loan obligations (CLOs) and openend mutual funds. Should the demand for CLOs decline and mutual fund investors seek to redeem their holdings, the availability of affordable financing for buyouts and refinancing will become challenging. However, leveraged loans typically have high priority in corporate reorganizations when they are senior loans, a potential source of high recovery rates, which may keep investor demand for the loans high.


\end{document}