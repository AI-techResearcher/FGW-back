\documentclass[11pt]{article}
\usepackage[utf8]{inputenc}
\usepackage[T1]{fontenc}
\usepackage{amsmath}
\usepackage{amsfonts}
\usepackage{amssymb}
\usepackage[version=4]{mhchem}
\usepackage{stmaryrd}

\begin{document}
Distressed Debt

Distressed debt investing involves purchasing the debt of companies that are in or near default.

\section*{Describing Distressed Debt}
Distressed debt is often defined as debt that has deteriorated in quality since issued and that has a market price less than half its principal value, yields 1,000 or more basis points over the riskless rate, or has a credit rating of CCC (Caa) or lower.

Distressed debt investors are usually equity investors "in debt's clothing." They are relatively unconcerned with coupon payments, debt service, and repayment schedules, being interested in distressed debt for the capital appreciation that can be achieved in various situations. They are sometimes viewed as vultures looking to swoop in, purchase cheap debt securities, convert them to stock, turn the company around, and reap the rewards of appreciation. As discussed in Session 4.1, Private Equity Assets, the risks are large because the underlying company is in some form of distress. Consequently, distressed debt investors are exposed to event risk that the company will not be able to emerge from bankruptcy protection or will otherwise fail.

Within the risk spectrum, private equity distressed debt investors fall between LBO firms and venture capital. Like LBO firms, distressed debt investors purchase securities of companies that have established operating histories. In most cases, these companies have progressed far beyond their IPO stage. However, unlike LBO firms that target successful but stagnant companies, distressed investing targets troubled companies. These companies have declined and may already be in bankruptcy proceedings. Like venture capital and LBO funds, distressed debt investors assume considerable business risk. A company's current problems might be due to poor execution of an existing business plan, an obsolete business plan, excessive leverage, or simply poor cash management. These problems are more likely to be fixable than in the case of a start-up company with a nonviable product.

Mezzanine debt is made equity-like primarily through equity kickers. Distressed debt becomes equity-like through potential default risk. As in the case of mezzanine debt, the idea that debt can be equity-like can be clarified using Merton's view of the capital structure of a firm. In that framework, corporate debt can be seen as being equal to the combination of a long position in the firm's assets and a short position in a call option on the firm's assets. Equation 1 in the Mezzanine Debt lesson illustrated this option view of corporate debt.

If the value of the firm's assets falls near or below the face value of the debt, the debt holders' short position in the call option moves out-of-the-money and becomes a smaller and smaller value relative to the debt holders' long position in the firm's assets. The further out-of-the-money the call option moves, the closer the value of the call option on the right-hand side of Equation 1 moves toward zero. Thus, the corporation's debt behaves increasingly like an unlevered long position in the firm's assets as the value of the firm's assets falls below the face value of the firm's debt.

\section*{The Supply of Distressed Debt}
Debt rarely becomes distressed because of some spectacular event that renders a company's products worthless overnight. Rather, a company's financial condition typically deteriorates over a period of time. The management of a company that was once established in the marketplace may become lacking in energy or rigid, unable or unwilling to cope with new market dynamics. This is where successful private equity managers earn superior returns. Revitalizing companies and implementing new business plans are their specialty. The adept distressed investor is able to spot these tired companies, identify their weaknesses, and bring a fresh approach to the table. By purchasing the debt of the company, the distressed debt investor creates a seat at the table and the opportunity to turn the company around.

Leveraged buyout firms are a major source for distressed debt, as the debt used to initiate the LBO often becomes distressed debt. There is a natural cycle between private equity and distressed debt investing. Since LBOs use a substantial amount of debt to take a company private, this debt burden sometimes becomes too much to bear, and the private company enters into a distressed situation. These so-called leveraged fallouts occur frequently, leaving large amounts of distressed debt in their wake. However, this provides an opportunity for distressed debt buyers to jump in, purchase nonperforming bank loans and subordinated debt cheaply, eliminate the prior private equity investors, and assert their own private equity ownership.

\section*{The Demand for Distressed Debt}
There is no standard model for successful distressed debt investing. Each distressed situation requires a unique approach and solution. Successful distressed debt investing entails selection of companies (credit risks) that are undervalued in the marketplace and intervention in the operations of the companies and in bankruptcy reorganizations to secure high returns.

One reason the distressed debt market is attractive to vulture and other investors is that it is an inefficient market. First, distressed debt is not publicly traded like stocks. Further, most distressed debt was originally issued in private offerings and sold directly to institutional investors seeking investment-grade debt. This debt lacked liquidity from the outset, and what little liquidity existed dried up when the company became distressed. This lack of liquidity can lead to bonds trading at steep discounts to their true value. Institutional investors uncomfortable with the increased risk of their positions in distressed bonds may need to sell their claims at depressed prices.

Sometimes investors use distressed debt as a way to gain an equity investment stake in a company. In these cases, the distressed debt owners agree to exchange their debt in return for stock in the company. At other times, distressed debt owners help the troubled company get back on its feet, thus earning a substantial return as their distressed debt recovers in value.

Finally, distressed debt is not always an entree into private equity; it can simply be an investment in an undervalued security. At these times, distressed debt buyers may serve as patient creditors. They buy the debt from anxious sellers at steep discounts and wait for the company to correct itself and for the value of the distressed debt to recover.

\section*{Expected Credit Loss Rate}
The annual default rate is the annual portion of debt issues that default by failing to pay principal and interest as scheduled or that experience a technical default when a company is unable to comply with the covenants, or terms of the loan outside the payment of principal and interest. The annual credit loss rate is the annual default rate multiplied by the losses on the debt that aren't recovered through bankruptcy, as illustrated in Equation 1.


\begin{align*}
\text { Credit Loss Rate } & =\text { Default Rate } \times \text { Loss Given Default } \\
& =\text { Default Rate } \times(1-\text { Recovery Rate }) \tag{1}
\end{align*}


The credit loss rate in Equation 1 is the expected annual losses to the portfolio from default losses expressed as a percentage of the portfolio's total initial value.

Like LBO funds and venture capital funds, distressed debt funds tend to run concentrated portfolios of companies. However, distressed debt investors tend to invest across industries as opposed to concentrating in a single industry. This may lead to better diversification than is found in VC funds. Distressed debt portfolios may be viewed as suffering credit losses at rates that are more than offset by income and recoveries from firms that turn around. Equation 2 illustrates this minimum criterion based on a credit spread, an expected default rate, and an expected loss rate:

Credit Spread $\geq$ Credit Loss Rate + Required Risk Premiums

Equation 2 illustrates the distressed investor's goal of receiving a credit spread (above the riskless rate) at least large enough to cover expected annual credit losses (the credit loss rate, which is the product of the default rate and the loss rate given default on the portfolio). Further, the investor likely requires a risk premium to compensate the investor for risks such as illiquidity and the uncertainty with regard to default rates (especially to the extent that default rates exhibit systematic risk).

Credit spreads can be observed at a specific point in time, but spreads can change dramatically during times of rising defaults and increasing risk in financial markets. Credit spreads can exceed $6 \%$ during times of increasing default rates and declining recovery rates. Further, both credit spreads and credit losses are cyclical. Investors require being paid more on the loan credit spread than it eventually costs for the credit loss rate to receive compensation for associated risks. Jenkins and Thomas (2017) encourage investors to consider a strategy of having lower allocations to risky debt late in the credit cycle when spreads are tight, but increasing allocations quickly after spreads have widened, which often occurs during times of crisis when spreads and defaults are rising and investors are selling debt to reduce leverage.

\section*{Four Distressed Debt Investment Strategies}
There are four broad strategic categories of investing in distressed debt securities.

The first approach is an active approach with the intent to obtain control of the company. These investors typically purchase distressed debt to gain control through a blocking position in the bankruptcy process with the goal of subsequent conversion into the equity of the reorganized company. This strategy of gaining control also seeks seats on the board of directors and even the chairmanship of the board. This is the riskiest and most time-intensive of the distressed investment strategies. Returns are expected in the $20 \%$ to $25 \%$ range, consistent with those for leveraged buyouts. Often, these investors purchase fulcrum securities.

The second general category of distressed debt investing seeks to play an active role in the bankruptcy and reorganization process but stops short of taking control of the company. Here, the principals may be willing to swap their debt for equity or for another form of restructured debt. An equity conversion is not required, because control of the company is not sought. These investors participate actively in the bankruptcy process, working with or against other creditors to ensure the most beneficial outcome for their debt. They may accept equity kickers such as warrants with their restructured debt. Their return target is in the $15 \%$ to $20 \%$ range, very similar to that of mezzanine debt investors.

Third, there are passive or opportunistic investors. These investors do not usually take an active role in the reorganization of the company and rarely seek to convert their debt into equity. These investors buy debt securities that no one else is eager to buy. These distressed debt buyers usually buy their positions from financial institutions that do not have the time or inclination to participate in the bankruptcy reorganization, from mutual funds that are restricted in their ability to hold distressed securities, and from investors with positions in high-yield bonds who do not want to convert a high cash yield into an equity position in the company.

Last, the distress debt investor is focused on the collateral value as opposed to the borrower. These investors seek to buy mainly nonperforming loans from financial institutions that no one else is eager to buy, thus buying the loans at a significant discount to collateral value. The investor then plays an active approach through a\\
debt-to-asset strategy where they 1) realize the value of the collateral through court enforcement, or 2) negotiate repayments with the borrower with a workout or discounted payoff, or 3) they seek to control the underlying collateral.

\section*{Risks of Distressed Debt Investing}
The main risk associated with distressed debt investing is business risk. Just because distressed debt investors can purchase the debt of a company at large discounts from face value does not mean it cannot go lower. This is the greatest risk to distressed debt investing: that a troubled company may ultimately prove to be worthless and unable to pay off its creditors. Although creditors often convert their debt into equity, the company may in the end not be viable as a going concern. If the company cannot develop a successful plan of reorganization, it simply continues its downward spiral. Purchasers of distressed debt must have long-term investment horizons. Workout and turnaround situations do not happen overnight; it may take several years for a troubled company to correct its course and appreciate in value.

It may seem strange, but traditional views of creditworthiness, such as probability of default, may not apply here. In other words, lack of creditworthiness is already established. Credit risk and other fixed-income-based views of risk are less relevant. The debt is already distressed and may already be in default. Consequently, failure to pay interest and debt service may have already occurred.

Instead, vulture investors consider the business risks of the company. They are concerned not with the short-term payment of interest and debt service but with the ability of the company to execute a viable business plan. From this perspective, it can be said that distressed debt investors are truly equity investors. They view the purchase of distressed debt as an equity-like investment in the company as opposed to a decision to become a fixed-income investor.

\section*{Five Observations on Vulture Investing}
Schultze (2012) wrote an insightful and entertaining book describing the vulture investing process. Schultze's insights are summarized here.

First, while some vulture investors may resent the label, Schultze embraces it, specifically comparing his role in the economy to the role of the vulture in the natural world. Schultze notes that buzzards are good for the environment, cleaning up the toxic waste of dead and decaying animals. Analogously, vulture investors help the economy by cleaning up after bankruptcies, recycling bad debt, and turning poorly run companies into new investments with greater potential profits and job growth. Vulture investors do dirty work, but it pays well, especially for those with the legal training and skill to understand and influence the bankruptcy process.

Second, Schultze describes a credit cycle as one of booms and busts exacerbated by government policy. The boom is inflated by loose monetary policy and cheap credit designed to clean up the last bust, but with the cheap credit actually encouraging the next boom. By patiently waiting for the point in the cycle when credit has fully deteriorated, Schultze contends that a vulture investor may be able to buy distressed debt at $20 \%$ of its face value and potentially earn a recovery value of $60 \%$ at the end of the bankruptcy process.

Third, Schultze claims that the most important skill is the identification of the fulcrum security, as the largest profits to vultures come through holding the postbankruptcy equity (through recovery) and not from the cash recovery values of debt senior to the fulcrum security.

Fourth, while previously the fulcrum security was usually a subordinated debt issue, the increasing debt load of corporations worldwide has changed capital structures so dramatically that the fulcrum security is often now a senior secured bank loan.

Finally, Schultze notes that the negative perception of vulture investors does not come from their purchase of impaired debt in a fire sale, but rather from originating debt with a loan-to-own mentality. Traditional credit analysis focuses on the cash flows of the firm, seeking to determine that the cash flow is sufficient to service the debt until its maturity. In contrast, loan-to-own strategies don't underwrite the loan focused on the borrower's ability to pay, but on making senior secured loans that are likely to default where the underwriter seeks to control the firm or a specific asset that is deemed to be worth more than the value of the loan. That is, a lender in a loan-to-own strategy seeks out and structures deals that are likely to default, preferring to take control of a firm or its assets rather than to be paid principal and interest on the loan. Rather than thinking of this financing as a lending transaction, the vulture in the loan-to-own scenario considers the financing an acquisition due to its high probability of default.


\end{document}