\documentclass[11pt]{article}
\usepackage[utf8]{inputenc}
\usepackage[T1]{fontenc}
\usepackage{amsmath}
\usepackage{amsfonts}
\usepackage{amssymb}
\usepackage[version=4]{mhchem}
\usepackage{stmaryrd}

\begin{document}
\section*{APPLICATION A}
Question : If $20 \%$ of the bonds in a portfolio default each year and if $60 \%$ of each defaulted bond's value is ultimately unrecoverable (i.e., $40 \%$ of the bond's cost is recovered), what would be the expected annual default losses as a percentage rate relative to the portfolio's value?

\section*{Answer and explanation}
The total annual loss due to default is $12 \%$, found from Equation $1: Credit Loss Rate = Default Rate× Loss Given Default
= Default Rate × (1 − Recovery Rate) =
[20 \% \times(1-40 \%)]=12 \%$.

\section*{APPLICATION B}
Question : If the expected recovery rate is $50 \%$ and the annual default losses as a percentage rate relative to the portfolio's value is $9 \%$, what would be the minimum credit spread that an investor would require if the investor seeks a $5 \%$ premium for bearing the risks associated with the portfolio's various risks?

\section*{Answer and explanation}
We are trying to find the credit spread an investor would be willing to receive in order to compensate them for the risk of default. Using Equation 2:

Credit Spread $\geq$ Loan Loss Rate + Required Risk Premiums

In this problem, the annual default loss rate is given as $9 \%$ of the portfolio's value, and the investor requires a 5\% risk premium for bearing the risk of that default. Therefore,

$$
14 \% \geq 9 \%+5 \%
$$

If the credit spread is lower than $14 \%$, the investor would not be properly compensated for taking on these risks. Anything over $14 \%$ would be additional yield in excess of what the investor would require.


\end{document}