\documentclass[11pt]{article}
\usepackage[utf8]{inputenc}
\usepackage[T1]{fontenc}
\usepackage{amsmath}
\usepackage{amsfonts}
\usepackage{amssymb}
\usepackage[version=4]{mhchem}
\usepackage{stmaryrd}

\begin{document}
Private Credit Performance and Diversification

As with any investment, investors should be concerned with how private credit and distressed investments will alter the risk and return of their portfolio. Specifically, it is imperative to understand how the asset allocation will change when new investments are made in alternative credit vehicles.

Munday et al. (2018) evaluated the risk and return characteristics of a variety of credit vehicles using data from the vintage years 2004 to 2016, which of course includes the global financial crisis of 2008 and 2009. The first observation is that reported valuations of mezzanine, distressed, and direct-lending strategies excluding mezzanine exhibited relatively low price volatility. However, this perception was due to the smoothing effects of a lack of liquidity on the valuations of these strategies, where either the loans were not regularly traded or were held to maturity by the originator of the loan. After adjusting for the smoothing of this illiquidity using autocorrelation techniques, the annual standard deviations of returns to mezzanine funds, direct-lending funds, and distressed funds were $5.1 \%, 6.7 \%$, and $13.5 \%$, respectively. In contrast, the standard deviations of the more liquid and highly traded high-yield funds, BDCs, and leveraged loans were $11.5 \%, 29.8 \%$, and $11.0 \%$, respectively. Volatility was also demonstrated in the degree of drawdowns in 2008 and 2009, with mezzanine losing 15\% and BDCs experiencing a temporary loss of value of $50 \%$.

The second observation is that there was a relatively high correlation of returns between distressed debt, BDCs, high-yield bonds, and leveraged loans, with most pairwise correlations varying between 0.8 and 0.9 . In contrast, direct lending strategies ex-mezzanine had a much lower correlation of just 0.2 to 0.3 relative to the high yield, BDC, and leveraged loan strategies.

The third observation is that mezzanine, leveraged loan, and high-yield investments had lower levels of return, averaging $7.7 \%, 5.5 \%$, and $8.2 \%$, respectively, over this time period. Distressed and direct-lending ex-mezzanine investments had higher returns, averaging $8.9 \%$ and $11.8 \%$, respectively, over the same time period.

While it appears that mezzanine, direct-lending ex-mezzanine, and distressed funds have lower risk, it is not accurate to compare the volatility and correlation of returns of these less liquid investments to the seemingly higher volatility and correlation of returns of the high-yield, BDC, and leveraged loan strategies. The role of illiquidity in understating asset price volatility is detailed in Level II of the CAIA curriculum. Nevertheless, it appears that over the given time period (which includes the global financial crisis), distressed and direct lending strategies, the most complex and illiquid investments in this sector, exhibit sustained outperformance relative to the more liquid fixed-income strategies.

In alternative credit strategies, it appears that illiquid strategies performed well. That is, investors who avoided being forced sellers and had the cash to buy at the bottom of the last credit cycle profited handsomely from this strategy, especially if they had the expertise to work through the bankruptcy process and hold equity assets at the end of the restructuring process. Hedge funds with shorter holding periods and less willingness to work through the restructuring process may be able to profit simply by buying from forced sellers such as pensions and insurance companies subject to regulations that prohibit the holding of distressed paper, openend funds that have significant investor redemptions, and investors with too much leverage who are forced by a counterparty to quickly reduce their degree of leverage. Each scenario has, at times, led investors to sell more than half of their illiquid fixed income holdings in less than a month. History shows that it is much better to be a buyer than a seller at the bottom of a fire sale.


\end{document}