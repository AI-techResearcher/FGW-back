\documentclass[11pt]{article}
\usepackage[utf8]{inputenc}
\usepackage[T1]{fontenc}
\usepackage{amsmath}
\usepackage{amsfonts}
\usepackage{amssymb}
\usepackage[version=4]{mhchem}
\usepackage{stmaryrd}

\begin{document}
Types of Fund Private Credit Vehicles

Four primary types of debt are detailed in this session: leveraged loans, direct lending, mezzanine debt, and distressed debt. While leveraged loans and direct lending are referred to as private credit, mezzanine and distressed debt instruments can be referred to as private equity due to their equity-like risks.

The private credit and distressed debt market grew from $\$ 200$ billion at the end of 2007 to over $\$ 600$ billion 10 years later. Some analysts believe that the global financial crisis and the subsequent regulations led directly to the growth of the private credit market.

During the global financial crisis of 2008, there were a number of bank bailouts in both the United States and Europe, leading to increased regulations, such as the Dodd-Frank Act in the United States and the Basel III framework in Europe. These new regulations require banks to comply with stress tests and capital adequacy requirements, making banks more accountable for the types of loans they make and for the types of risks that they take. As these regulations took effect, European banks reduced their balance sheets by EUR 600 billion, leading to five straight years of a decline in the amount of lending by banks to small and medium enterprises. If small companies aren't allowed to access bank credit, they need to have access to alternative lending or marketplace financing in order to run their firms, so as banks have largely backed off from making loans to small and medium enterprises in order to improve their risk-based capital requirements, hedge funds, private equity funds, and private credit funds have stepped in to make these loans.

Over $\$ 200$ billion of the private debt investments are held in dry powder, which are investments pledged to private debt investment firms that have not yet been lent out to a borrower or capital that has been committed but not yet called or invested. An oversupply of dry powder may reduce credit spreads and weaken covenant protections. An undersupply of dry powder may mean that there could not be enough private credit available if there is a surge of buyout and merger activity, combined with a continued need for borrowings at small and medium enterprises.

Because of the illiquid nature of private credit, there is only a limited opportunity to access these investments through funds containing mostly liquid assets such as open-end mutual funds. Private credit investments are made predominately through closed-end vehicles such as hedge funds, private equity funds, and business development companies (BDCs), described in the session, Private Equity Funds, and the relatively new structure of interval funds discussed next. One of the downsides to investing in publicly traded BDCs is that they are subject to wide swings in their premium or discount to net asset value. As such, BDCs often exhibit high price volatility, especially compared to other vehicles evaluated simply using monthly NAVs.

\section*{Interval Funds}
Interval funds are semi-liquid, semi-illiquid closed-end funds that do not trade on the secondary market but offer the opportunity for investors to redeem or exit their investments at regularly scheduled intervals. Investors can purchase these funds at a regular NAV, as frequently as on a daily or weekly basis, with a low minimum investment. Interval funds might have a five-to-seven-year stated life, a long holding period that allows the fund to invest in less liquid credits that are likely to have a higher yield than publicly traded bonds. Investors can access liquidity in interval funds, as the fund manager will offer to repurchase some of the outstanding shares of the fund at regular intervals. The redemption opportunities of interval funds offer a degree of liquidity to investors despite the fund's holding of illiquid assets.

For example, the fund may offer to purchase $5 \%$ of outstanding shares at the end of each calendar quarter. Any investor who wishes to exit the fund will be able to tender their shares at that repurchasing interval. If the manager offers to repurchase $5 \%$ and investors tender $5 \%$ or less of the total shares outstanding, all of those investors will be cashed out at the end of that interval. If there's an offer to repurchase $5 \%$ of the shares and investors tendered $10 \%$ of the outstanding shares, each investor would sell half of their shares to the fund that will be repurchased on a pro rata basis, leaving investors with half of their redemption request unfulfilled. Investors can tender those unsold shares at the next repurchase interval, at which time a new queue will form for the amount of shares the fund seeks to repurchase at that time.

\section*{Drawdown Funds}
A drawdown fund is a type of private equity fund (that can be used for private credit) in which investor commitments are called as needed (e.g., to fund investments or meet expenses), in essence providing partnership-like liquidity features in a fund structure. These funds may have an indefinite term or a fixed life, such as three or five years or longer. They can purchase bank loans or bonds as well as underwrite new debt, which is when the private debt fund lends directly to a company after carefully reviewing their financials and negotiating with the borrower. Banks typically underwrite a loan based on the ability to repay, as investors in investmentgrade debt simply seek to be repaid principal and interest as scheduled.

\section*{Funds with a Loan-to-Own Objective}
Some private equity funds underwrite debt with a loan-to-own objective. A loan-to-own investment occurs when the investor focuses on the value of the borrower's assets and the value of the company that could be repossessed if the borrower was unable to service the loan, not necessarily evaluating the ability of the company to pay back the principal and interest as scheduled. In some cases, the lenders may prefer that the company default in order for the loan to turn into an ownership stake in the firm or its assets.

\section*{Fulcrum Securities and Reorganization}
A private equity firm may seek to invest in a fulcrum security. A fulcrum security is the senior-most debt security in a reorganization process that is not paid in full with cash but rather is the security that is most likely to be repaid with equity in the reorganized firm. Some of the largest gains in private credit investing can occur from converting a debt security into an equity security. For example, if the fulcrum security is a second-lien debt issue that trades at $20 \%$ of par value, it might be converted into an equity stake in the new firm that is initially worth $30 \%$ of face value, but may appreciate over time, perhaps eventually exceeding the par value of the original borrowing.

The equity in the newly reorganized firm may be substantially undervalued during its first year, for several reasons. First, many investors, such as bond funds, banks, or insurance companies, may not be legally allowed to hold the equity recovered in the reorganization process. As such, they would quickly sell the equity holdings, which would initially depress the stock price. Outside investors may be slow to warm up to the new stock, as they are focused on the negative headlines of the\\
recently completed bankruptcy process, and perhaps underestimate the reduced risk and increased cash flow of the firm that just reduced or eliminated its liabilities. After a few quarterly reports as a new company, the lower-risk and higher-return nature of the stock reveals itself, leading to enhanced profits for the longterm investor in the fulcrum security that morphed into a highly successful stock holding. Of course, not all newly reorganized companies play out this way, as some enter into bankruptcy a second time within a few years, so investors need to consider how the cash flow and risk of the new company will evolve over time.

While a private equity fund might have a longer-term investment in the equity of the firm, hedge funds might have a shorter-term trading-oriented view, where they can profit in the role of a liquidity provider. Consider a pension plan or an insurance company, where a governmental entity or the state insurance regulator prohibits more than a $5 \%$ allocation to non-investment-grade debt with an outright prohibition on the ownership of nonperforming or distressed debt. The result of these regulations might be that as soon as the firm has defaulted or been downgraded, the pension plan or the insurance company must immediately sell the debt held in their portfolio. As pension funds and insurance companies become forced sellers, the traditional buyers of debt, such as mutual funds, may not be interested in buying this debt. With the reduced demand for the debt, hedge funds may step in to meet supply (i.e., provide liquidity). If the prices of the downgraded debt issues fall substantially, the hedge funds may be the only large buyers and they make relatively low bids, resulting in low market prices that cause pension funds and insurance companies to realize large losses. For example, the hedge funds may offer to purchase a defaulted debt issue at $20 \%$ of its face value, which can lead to a large profit if that debt issue eventually receives a recovery value of, say, $40 \%$.


\end{document}