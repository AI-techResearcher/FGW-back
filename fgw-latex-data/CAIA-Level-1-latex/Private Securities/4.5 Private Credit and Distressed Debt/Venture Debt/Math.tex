\documentclass[11pt]{article}
\usepackage[utf8]{inputenc}
\usepackage[T1]{fontenc}
\usepackage{amsmath}
\usepackage{amsfonts}
\usepackage{amssymb}
\usepackage[version=4]{mhchem}
\usepackage{stmaryrd}

\begin{document}
\section*{APPLICATION A}
Assume a startup is raising US\$10M. Their pre-money valuation is US\$20M. They are looking to raise $60 \%$ in equity, and $40 \%$ in venture debt with a warrant coverage of $7 \%$ of loan value.

What is the equity percentage of the warrants?

\section*{Answer and explanation}
Given the mix of $60 \%$ in equity and $40 \%$ in venture debt, the amounts are as follows:

\begin{itemize}
  \item Equity amount $=$ US\$6M (US\$10M $\times 60 \%)$
  \item Venture debt amount $=$ US $\$ 4 M$ (US\$10M $\times 40 \%)$
\end{itemize}

The amount of equity from warrants is as follows:

Equity amount $=$ Warrant coverage $\times$ Venture Debt Amount $=$ US\$0.28M $(7 \% \times$ US\$4M)

The post-money valuation is as follows:

Post-money valuation $=$ Pre-money valuation + equity raised + equity raised from warrant

$=$ US\$26.28M (US\$20M + US\$6M + US\$0.28M)

The equity percentage of the warrant is as follows:

Equity percentage $=$ Equity amount from warrant $/$ Post-money valuation

$=1.07 \%(\$ 0.28 \mathrm{M} / \$ 26.28 \mathrm{M})$

\section*{APPLICATION B}
A startup company raised their Series A round from A-Ventures one year ago and is now looking to raise a Series B round from a new investor, B-Ventures.

Both the founders of the startup and A-Ventures want to avoid being heavily diluted by the new capital. The amount to raise is US\$10 million. The pre-money valuation is US $\$ 20$ million. The founder's ownership after Series A is 60\%.

They are considering two scenarios:

\begin{itemize}
  \item Scenario 1: raising $100 \%$ equity
  \item Scenario 2 : raising a mix $60 \%$ equity and $40 \%$ venture debt with a warrant coverage of $7 \%$
\end{itemize}

How much equity does the founder give up in Scenario 1 and Scenario 2 ?

\section*{Answer and explanation}
Scenario 1: $100 \%$ equity. Raise US\$10M equity.

Post-money valuation $=$ Pre-money valuation + equity raised

$=$ US\$30M (US\$20M + US\$10M)

\begin{center}
\begin{tabular}{|l|l|l|}
\hline
Post Money & Ownership percentage & Workings \\
\hline
B-Ventures & $33.3 \%$ & US\$10M/US\$30M \\
\hline
Founders & $40.0 \%$ & $60 \% \times(1-33.3 \%)$ \\
\hline
A-Ventures & $26.7 \%$ & $(1-60 \%) \times(1-33.3 \%)$ \\
\hline
TOTAL & $100.0 \%$ &  \\
\hline
\end{tabular}
\end{center}

Founder gives up $20 \%(60 \%-40 \%)$ of company in Scenario 1.

Scenario 2: $60 \%$ equity and $40 \%$ venture debt

Equity raised $=$ US $\$ 6 \mathrm{M}(60 \% \times$ US\$10M)

Venture debt $=$ US\$4M $(40 \% \times$ US\$10M)

Post-money valuation $=$ Pre-money valuation + equity raised

$=U S \$ 26.28 \mathrm{M}(\mathrm{US} \$ 20 \mathrm{M}+\mathrm{US} \$ 6 \mathrm{M}+\mathrm{US} \$ 0.28 \mathrm{M})$

\begin{center}
\begin{tabular}{|l|l|l|}
\hline
Post Money & Ownership percentage & Workings \\
\hline
Warrant for venture lender & $1.07 \%$ & From Application A \\
\hline
B-Ventures & $22.8 \%$ & US\$6M/ US\$26.28M \\
\hline
Founders & $45.7 \%$ & $60 \% \times(1-(22.8 \%+1.07 \%))$ \\
\hline
A-Ventures & $30.4 \%$ & $(1-60 \%) \times(1-(22.8 \%+1.07 \%))$ \\
\hline
TOTAL & $100.0 \%$ &  \\
\hline
\end{tabular}
\end{center}

Founder gives up $14.3 \%$ (60\% - 45.7\%) of company in Scenario 2.

The founders would prefer to own a larger portion of the company and experience less dilution. They should choose to raise their B round using $60 \%$ equity and $40 \%$ venture debt, which leaves them with a $45.7 \%$ equity stake in the company. That is preferable to a $100 \%$ equity round, which leaves the founders owning just $40 \%$ of the equity in the firm.


\end{document}