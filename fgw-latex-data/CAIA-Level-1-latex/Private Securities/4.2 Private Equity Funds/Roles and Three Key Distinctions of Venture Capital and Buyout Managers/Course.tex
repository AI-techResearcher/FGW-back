\documentclass[11pt]{article}
\usepackage[utf8]{inputenc}
\usepackage[T1]{fontenc}

\begin{document}
Roles and Three Key Distinctions of Venture Capital and Buyout Managers

Depending on the strategy, the roles of the PE managers can differ markedly.

The PE market is evolving. A few decades ago, the supply of PE capital came primarily through a limited number of large PE firms. These PE firms obtained much of their financing from creating PE funds and offering limited partnership investments to institutions and wealthy investors. The PE market emphasized relationships. PE firms invested capital in deals within a moderately inefficient market and a relatively less competitive environment. Established PE firms also obtained their external capital within a relatively less competitive environment, wherein institutions wishing to invest in PE faced concerns over whether they would have access to promising deals.

In buyout transactions, a greater proportion of time and manpower is spent analyzing specific investments and adjusting the business model. Buyout managers look to apply their expertise to turn around underperforming businesses, improve profitable businesses, or optimize the companies' balance sheets and financing. They typically engage in hiring new management teams or retooling strategies. In an operating company, it is easier to give guidance to a seasoned management team, whereas in early-stage investments, one often needs to build and coach the management team from the ground up.

The major distinctions between VC and buyout fund managers can be summarized as being driven by these three goals: (1) Venture capitalists look to launch new or emerging companies, whereas buyout managers focus on leveraging an established company's assets; (2) venture capitalists back entrepreneurs, whereas buyout managers deal with experienced managers; and (3) venture capitalists often play a more active role in the companies in which they invest, by either sitting on the board of directors or becoming involved in the day-to-day management of the company.


\end{document}