\documentclass[11pt]{article}
\usepackage[utf8]{inputenc}
\usepackage[T1]{fontenc}
\usepackage{amsmath}
\usepackage{amsfonts}
\usepackage{amssymb}
\usepackage[version=4]{mhchem}
\usepackage{stmaryrd}

\begin{document}
Private Equity Liquid Alternatives

Liquid alternatives, or liquid alts, have emerged in a variety of alternative investment sectors, and PE is no exception. In the United States, business development companies serve as a prominent example of liquid access to PE.

\section*{Business Development Companies}
Business development companies (BDCs) are publicly traded funds with underlying assets typically consisting of equity or equity-like positions in small private companies. BDCs use a closed-end structure and trade on major stock exchanges, especially the NASDAQ in the United States.

BDCs are investment companies with a primary purpose of pooling financial assets and issuing pro rata claims against those assets. The key to investment companies is that they can avoid the double taxation of corporate profits. Investment companies holding listed financial assets were authorized in the United States by the Investment Company Act of 1940. Legislation in the United States allowing BDCs to qualify as investment companies and enjoy a pass-through income tax status originated from amendments to the '40 Act in 1980. However, BDCs did not become popular until much later (approximately 2012).

To be classified as a BDC and enjoy the accompanying benefits, such as avoiding corporate income tax, a BDC must provide significant managerial assistance to the firms that it owns and must invest at least $70 \%$ of its investments in eligible assets, as specified by the Securities and Exchange Commission (SEC). BDCs must invest primarily in small firms, must maintain no more than moderate leverage, must distribute at least $90 \%$ of their income, and must meet diversification requirements to avoid corporate income taxes.

BDCs enable liquid ownership of pools of illiquid PE, just as REITs can be used to provide liquid access to illiquid private real estate. The shareholders are subject to income tax on the distributed profits. Any profits retained at the BDC level are subject to corporate income tax. Therefore, most BDCs distribute almost all profits to shareholders to avoid the income tax on retained earnings.

Recent figures indicate that there are more than 40 publicly traded BDCs in the United States, with over $\$ 10$ billion of combined market value. A few of the largest BDCs fall into the mid-cap category in terms of total market capitalization. Over $90 \%$ of the BDCs fall into the small-cap category. BDCs are tracked by several indices and ETFs, including at least one ETF that is leveraged. The indices and ETFs use market weights or modified market weights and tend to cover virtually all listed US BDCs. As discussed in the next section, the market prices of BDCs reflect their underlying closed-end fund structure.

\section*{Business Development Companies as Closed-End Funds}
BDCs use a closed-end fund investment structure that transforms ownership of underlying fund assets into shares (tradable pro rata claims). A major attribute of a closed-end structure is that it facilitates liquid ownership of illiquid pools of assets much better than would an open-end structure. Closed-end funds are especially popular in facilitating ownership of municipal bonds, international stocks, and illiquid instruments.

An open-end mutual fund has serious flaws with regard to providing liquid access to investors when the fund holds large quantities of highly illiquid pools of assets. Open-end funds must redeem shares on a regular basis, which can be difficult when holding illiquid assets, such as those held by BDCs. The primary distinction of closed-end funds relative to open-end funds is that the closed-end investment company does not regularly create new shares or redeem old shares in order to meet the desire of investors to invest in the fund or divest from the fund. Therefore, closed-end funds avoid the problems caused in open-end funds by using inaccurate net asset values (NAVs), as measured by the investment company, to create and redeem shares. However, closed-end funds introduce another problem: When investors transact in the secondary market, the price per share that they receive from or pay to another investor may be highly subject to short-term supply and demand factors in the secondary market and can diverge from the fund's NAV.

When there is a surge in demand for closed-end fund shares from investors who wish to establish or expand positions in a particular closed-end fund, the market price of the closed-end fund must rise until the supply of shares meets the demand. The price rises to encourage increased supply of shares from sales by existing shareholders and to discourage demand for shares from prospective shareholders. Similarly, when there is a surge in supply of closed-end fund shares from investors who wish to exit their holdings, the market price must decrease to restore a balance between the supply and demand for the fund's shares in the secondary market.

Closed-end fund share prices are often viewed relative to the net asset value per share that is reported by the investment companies. For example, a closed-end fund that reports a net asset value of $\$ 20$ per share is said to be selling at a $5 \%$ premium if the market price is $\$ 21$. If the market price of the closed-end fund is $\$ 18.50$, the closed-end fund shares would be said to be selling at a discount of $7.5 \%$. The formula for the premium (or discount if negative) of a closed-end fund share price is shown in Equation 1:


\begin{equation*}
\text { Premium }(\text { or Discount })=(\text { Market Price } / \text { Net Asset Value })-1 \tag{1}
\end{equation*}


The left-hand side of Equation 1 is typically expressed as a percentage and is termed a discount if the value is negative.

As illustrated in the example, large temporary changes in supply and demand can cause substantial dislocations of the market price of a closed-end fund. Investors with positions in closed-end funds bear the risk that they will need to liquidate their shares at a time when the market price of the shares has been substantially reduced by selling pressures. Thus, returns from investing in closed-end fund structures are driven both by the returns of the underlying assets and the premiums or discounts of the fund shares when positions are established and closed.

During periods of severe illiquidity caused by a financial crisis or another major event, closed-end fund discounts have been observed to reach extreme levels. Owners of closed-end fund shares needing to exit their investment in a liquidity crisis experience the double loss of selling not only when NAVs are down but also when the market price of the closed-end fund is at an extreme discount to its NAV. Thus, investors in BDCs are potentially exposed to especially large losses in the event of liquidations during periods of severe illiquidity. Note that an open-end mutual fund with listed equities as underlying assets would allow liquidation at NAVs based on market prices.

Finally, it should be noted that for many closed-end fund structures, the underlying assets are market traded, and the computation of the fund's NAV is straightforward. The premium or discount of such funds tends to be a simple and effective indicator of the attractiveness of the fund's market price. However, in the cases of BDCs, REITs, and other funds with unlisted underlying assets, the reported NAVs are based on non-market valuations, such as professional appraisals or accounting standards. Listed PE securities such as BDCs, closed-end funds, and publicly traded PE firms are much like REITs: They offer the liquidity of public trading with underlying assets that are illiquid and that when held directly are presumed to offer premiums for illiquidity. When the NAVs are based on subjective valuation methods rather than current market prices, the premiums or discounts of the closed-end fund shares may be poor indicators of the attractiveness of the fund's shares, because the NAVs themselves may be flawed indicators of the actual underlying values.

\section*{Extending Closed-End Fund Pricing to Illiquid Alternatives}
As detailed in the previous section, the premiums and discounts of closed-end fund share prices are generally perceived as varying through time, based on both large purchases by entities attempting to enter positions and large sales by entities attempting to exit positions. The direct application of these supply and demand pressures to BDCs is straightforward, since BDCs use a closed-end fund structure. However, the principles may be even more applicable to the transaction prices of illiquid alternatives. For example, when the limited partner of a PE partnership or other illiquid alternative investment wants to exit an investment, how much of a "discount" might that seller be forced to offer in order to entice a prospective buyer to purchase the position?

Careful observation and understanding of the behavior of closed-end fund share prices provide indications of the effect of illiquidity on transaction prices. In the previous section, there was an example of a closed-end fund moving from trading at a $10 \%$ premium to a $12 \%$ discount. Although the transaction prices and underlying net asset values of private partnerships are usually not quoted on a daily basis, the concept of illiquid assets trading at premiums and discounts applies. Of course, privately traded PE may also be exchanged at depressed prices when there are numerous partners wishing to exit. Simply put, the realized returns of PE investors who must liquidate their positions may be low when liquidity is poor, whether or not the investment uses a closed-end structure.

\section*{Are Liquid Private Equity Pools Diversifiers?}
Most PE is not directly listed or publicly traded. PE is often described as offering substantial diversification benefits. However, the lack of market prices on PE makes substantiation of such claims difficult. Prices for PE based on illiquid trading data or professional judgment can be argued to be smoothed, and therefore analysis based on those data should be expected to underestimate true volatilities and correlations.

However, listed BDCs provide an opportunity to observe market prices of PE. As in the case of real estate and REITs, the market data on liquid alternatives can be analyzed to provide evidence regarding the correlations and volatilities of the underlying illiquid assets. A critical underlying issue is how the returns of liquid PE (e.g., BDCs) compare to the returns of illiquid PE (e.g., private partnerships) when the underlying assets are similar.

As indicated in the exhibit below, the BIZD ETF had high correlations of monthly returns with the monthly returns of both SPY and IWM. SPY is an ETF of large US stocks and IWM is an ETF of small US stocks. The correlation of the returns of BIZD with the small-cap index (IWM) was higher than the correlation with the large-cap index (SPY). As observed in numerous analyses of REITs, the liquid alternatives appear to take on correlations more closely with small-caps than with large-caps. The BIZD ETF had a volatility of monthly returns approximately midway between the volatilities of SPY and IWM. In theory, the volatility of the returns of BIZD should be driven by the volatilities and correlations of the returns of the small business ventures that underlie the BDCs being tracked by BIZD.

The results in the exhibit below are roughly analogous to the results for REITs, based on numerous studies involving a variety of time periods. Simply put, listed liquid alternative investment companies appear to exhibit return performance that is highly correlated with listed equities in general. In particular, performance is most highly correlated with equities of similar capitalization size. The exhibit below therefore provides evidence that BDCs do not serve as effective diversifiers relative to listed equities.

\section*{Are Liquid Private Equity Pools Return Enhancers?}
The next exhibit lists the average annualized returns of the three ETFs: BIZD, SPY, and IWM. The returns indicate that BIZD underperformed the S\&P 500 ETF and the Russell 2000 ETF. It should be pointed out that investors in BIZD incur two levels of fees. First, BIZD imposes fees at the ETF level and has an expense ratio substantially higher than the expense ratios of many large ETFs, such as SPY. Second, the portfolio companies (i.e., the underlying BDCs) also have potentially large expense ratios.

Return Analysis of BIZD, March 2013 to December 2021

\begin{center}
\begin{tabular}{|lccc|}
\hline
 & Mean Return & Volatility & Correlation to BIZD \\
\hline
BIZD & $7.4 \%$ & $21.1 \%$ & 1.00 \\
SPY & $16.0 \%$ & $13.3 \%$ & 0.75 \\
IWM & $12.2 \%$ & $18.3 \%$ & 0.82 \\
\hline
\end{tabular}
\end{center}

PE accessed through private limited partnerships is also subject to substantial management fees, including incentive fees. Perhaps the critical determinant of longterm PE performance is the quality of the management teams. Therefore, a key issue in determining whether liquid PE funds such as BDCs can provide return enhancement is whether the BDCs offer superior management teams that can successfully acquire and manage underlying business enterprises.


\end{document}