\documentclass[11pt]{article}
\usepackage[utf8]{inputenc}
\usepackage[T1]{fontenc}
\usepackage{amsmath}
\usepackage{amsfonts}
\usepackage{amssymb}
\usepackage[version=4]{mhchem}
\usepackage{stmaryrd}

\begin{document}
Leveraged Buyout Funds

The dominant type of buyout in terms of aggregate size is the LBO. This lesson provides details about LBOs and LBO funds.

\section*{LBO Fund Structures}
Leveraged buyout (LBO) funds distinguish themselves by the size of the companies they take private. Generally, they classify themselves as investing in smallcapitalization companies ( $\$ 100$ million to $\$ 1$ billion in sales revenue), mid-capitalization companies ( $\$ 1$ billion to $\$ 5$ billion in sales revenue), or large-capitalization companies ( $\$ 5$ billion and above in sales revenue). The large-cap category of LBOs also includes super-sized or mega LBOs.

Almost all LBO funds are structured as limited partnerships. This is very similar to the way that VC funds are established. LBO funds are run by a general partner, typically an LBO firm. All investment discretion and day-to-day operations vest with the general partner. LPs, as the name implies, have a very limited role in the management of the LBO fund. For the most part, LPs are passive investors who rely on the general partner to source, analyze, perform due diligence, and invest the committed capital of the fund.

The number of LPs in a PE fund is not fixed. Most PE funds have 20 to 50 LPs, but some have as few as five and others more than 50. PE funds have contractually set lifetimes-typically 10 years, with provisions to extend the limited partnership for one to two more years. Limited partners would prefer to not pay management fees after the stated life of the fund, such as 10 years, has passed. If LPs are not paying management fees during the extended life of the fund, GPs have an incentive to wrap up the fund more quickly. During the first five years of the partnership, deals are sourced and reviewed and partnership capital is invested. After companies are taken private, the investments are managed and eventually liquidated. As the portfolio companies are sold, taken public, or recapitalized, distributions are made to the LPs, usually in cash but sometimes in securities, as is often the case when an IPO is used to exit an investment.

\section*{Total Number, Size, and Implications of Buyout Fund Fees}
LBO firms have numerous ways to make money. First, there are the annual management fees, which range from $1.25 \%$ to $3 \%$ of investor capital. Incentive fees, or carried interest, usually range from $20 \%$ to $30 \%$ of the fund's total profits.

For arranging and negotiating an LBO, an LBO firm may also charge fees of up to $1 \%$ of the total selling price to the corporation it is taking private. As an example, Kohlberg Kravis Roberts \& Co. (KKR) earned \$75 million for arranging the buyout of RJR Nabisco and \$60 million for arranging the buyout of Safeway, Inc. Some LBO firms (i.e., GPs) keep all of these fees for themselves rather than sharing them with the limited partner investors. Other LBO firms split the transaction fees, with LPs receiving typically $25 \%$ to $75 \%$. Still other LBO firms include all of these fees as part of the profits to be split up among the general partner and the LPs. Not only do LBO firms earn fees for arranging deals, but they can also earn breakup fees if a deal fails.

In addition to these fees, LBO firms may charge a divestiture fee for arranging the sale of a division of a private company after the buyout has been completed. Further, an LBO firm may charge directors' fees to a buyout company if managing partners of the LBO firm sit on the company's board of directors after the buyout has occurred. The debate over PE fees has intensified in recent years, especially because as buyout funds have grown in size, the management fees of the funds have not been adjusted downward as a percentage.

When the buyout industry started, the $1 \%$ to $2 \%$ management fee was necessary to pay the expenses of the PE general partner. This fee covered travel expenses, utility bills, and the salaries of the general partner's staff. In short, the management fee was originally used to keep the PE manager afloat until the incentive fee could be realized, which often took several years. Now, however, PE funds have grown to immense size; $\$ 10$ billion funds are common. The PE manager now earns a considerable amount of profit from its management fees. This could blunt the incentive of the PE manager to seek only the most potentially profitable PE opportunities.

Let's take a simple example of typical fees. Assume a PE firm raises a $\$ 10$ billion buyout fund and charges a management fee of $1.5 \%$. This is a fee of $\$ 150$ million during the investment period of the fund. Assuming a 10-year life for a fully invested fund and an $8 \%$ discount rate for the time value of money, the present value of just the management fees to the PE firm is $\$ 1.006$ billion. With management fees like this, there could be a disincentive to take risks.

While managers of buyout funds generally offer their investors an $8 \%$ preferred return before they take a share of the profits for themselves, this is less common in the case of VC funds, at least in the United States. In this context, Fleischer puts forward an alternative explanation for the preferred return: It is more important as an incentive to properly screen the deal flow. ${ }^{1}$ Victor Fleischer, "The Missing Preferred Return," UCLA School of Law, Law \& Economics Working Paper 465 (February 22 , 2005). Without it, buyout fund managers could pursue a low-risk, low-return strategy, for example, by being inactive or by choosing companies that have little potential to generate large returns. A preferred return forces managers to make riskier investments to generate a return in line with the investor's targets. In the case of VC funds, however, investments are always risky, and the high-risk, high-reward strategy makes it meaningless to bother about preferred returns.

\section*{Agency Relationships and Costs}
The objectives of the senior managers of listed corporations may be very different from those of the corporation's equity owners. For instance, management of public firms may be highly concerned with keeping their jobs and presiding over a large empire. Conversely, shareholders want value creation (i.e., share price maximization). In agency theory, senior corporate managers are the agents and shareholders are the principals. Shareholders, as the owners of the company, delegate day-to-day decision-making authority to management with the expectation or hope that management will act in the best interests of the shareholders. However, in a large company, equity ownership may be so widely dispersed that the shareholders of the company may not be able to fully align managerial objectives with shareholder objectives or otherwise control management's natural tendencies. Thus, the separation of ownership and control of the corporation results in conflicts of interest and agency costs.

Agency costs come in two forms. First, there is the cost to better align management's goals with the value-creation goal of shareholders. These costs include the costs of monitoring management, which may include audits of financial statements, shareholder review of management perquisites, and independent reviews of management's compensation structure. Better alignment is also sought via the compensation arrangements. Compensation arrangements that are designed to better align shareholder and management objectives include stock options, bonuses, and other performance-based compensation. Second, agency costs can\\
include the erosion of shareholder value from agency conflicts that are too costly to resolve efficiently. These costs include the adverse effect on shareholders of managerial actions that are not in the best interests of shareholders. The optimal strategy regarding agency conflicts is to implement only those actions that have benefits that exceed their costs, not necessarily to minimize agency costs. Thus, conflicts of interest and agency costs are realities of doing business using structures that use agency relationships.

LBO firms replace a dispersed group of shareholders with a highly concentrated group of owners. The concentrated and private nature of the new shareholders helps incentivize the managers of the buyout firm to focus on maximizing shareholder wealth. Further, the management of the now private company is often given a substantial equity stake in the company that provides a strong alignment of interests between the management/agents of the company and its principals/shareholders. As the company's fortunes increase, so do the personal fortunes of the management team. The large incentives to an LBO's management team are often vital to the LBO's goal of unlocking value.

With a majority of the remaining equity of the once public, now private, company concentrated in the hands of the LBO firm, the interaction between equity owners and management becomes particularly important. After a company is taken private, LBO firms maintain an active role in guiding and monitoring the management of the company. After a transaction is complete, an LBO firm remains in continuous contact with company management. As the majority equity owner, the LBO firm has the right to monitor the progress of management, ask questions, and demand accountability.

\section*{LBO Auction Markets}
In the past, LBO deals were sourced by a single PE firm without any competitive bidding from other PE firms. The traditional model of PE was one in which a single PE firm approached a stand-alone public company about going private or approached a parent company with respect to spinning off a subsidiary. In this model, the Ione PE firm worked with the executive management of the public company or the parent company to develop a financing plan for taking the public company or a subsidiary private. Bringing this deal to fruition may have taken months or years, as the PE firm worked on building its relationship with the senior management of the company.

Whenever large sums of capital enter an informationally inefficient market, the inefficiencies begin to erode. An influx of investment in LBOs has led to the development of an auction market environment. Single-sourced deals are a thing of the past. Now, when a parent company decides to sell a subsidiary in an LBO format, it almost always hires an investment banker to establish an auction process. An auction process involves bidding among several PE firms, with the deal going to the highest bidder. This competitive bidding process can often involve several rounds and can result in less upside for the PE investor, yet it reflects the maturation of the PE industry.

\section*{LBOs, Club Deals, Benefits, and Concerns}
Another development in the PE market is club deals. In the past, LBO firms worked on exclusive deals, one-on-one with the acquired company. However, the large inflow of capital into the PE market and the increasing market capitalization of firms targeted for LBOs have forced LBO firms to work together in so-called clubs. In a club deal, two or more LBO firms work together to share costs, present a business plan, and contribute capital to the deal. There is considerable debate about whether club deals add or detract value. Both sellers of target companies and potential buyers can initiate club deals.

Some have posited that club deals tend to increase the number of potential buyers by enabling firms that could not individually bid on a target company to do so through a club. A fund might not have sufficient capital to purchase a target alone because of either restrictions on investing more than a specified portion of its capital in a single deal or the large size of the target. For example, a common restriction found in many limited partnership agreements limits PE funds from investing more than $25 \%$ of their total capital in any one deal. For some of the very large buyouts, club deals are necessary.

Another benefit of club deals is that they allow PE firms to pool resources for pre-buyout due diligence research, which can often be quite costly. In addition, club deals allow one PE firm to get a second opinion about the value of a potential acquisition from another member of the club.

Some have expressed concern that club deals could depress acquisition prices by reducing the number of firms bidding on target companies, because there may be more competition from numerous individual bidders than from a few clubs of bidders. There is also concern that in a club deal it is less clear who will take the lead in the business plan, which PE firm will sit on the board of directors of the private company, who will be responsible for monitoring performance, and who will negotiate with outside lenders to provide the debt financing for the LBO.

\section*{Three Factors Driving Buyout Risks Relative to Venture Capital Risks}
LBO funds have less risk than VC funds for three reasons.

First, LBO funds tend to purchase public companies that are established and mature. Typically, buyouts target successful but undervalued companies. These companies generally have long-term operating histories, generate a positive cash flow, and have established brand names and identities with consumers. Also, the management teams of the companies have an established track record. Therefore, assessment of key employees is easier than assessment of a new team in a VC deal. VC funds face the substantial business risks associated with start-up companies.

Second, LBO funds tend to be less specialized than VC funds. While LBO firms may concentrate on one sector from time to time, they tend to be more diversified in their choice of targets. Their target companies can range from movie theaters to grocery stores. Therefore, although they maintain smaller portfolios than traditional long-only managers, they tend to have greater diversification than their VC counterparts.

Third, the eventual exit strategy of a new IPO is much more likely for an LBO than for a VC deal. This is because the buyout company already had publicly traded stock outstanding. A prior history as a public company, demonstrable operating profits, and a proven management team make an IPO for a buyout firm much more feasible than an IPO for a start-up venture.


\end{document}