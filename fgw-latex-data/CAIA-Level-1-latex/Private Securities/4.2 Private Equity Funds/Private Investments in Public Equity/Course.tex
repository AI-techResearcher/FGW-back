\documentclass[11pt]{article}
\usepackage[utf8]{inputenc}
\usepackage[T1]{fontenc}
\usepackage{amsmath}
\usepackage{amsfonts}
\usepackage{amssymb}
\usepackage[version=4]{mhchem}
\usepackage{stmaryrd}

\begin{document}
Private Investments in Public Equity

Private investments in public equity (PIPE) transactions are privately issued equity or equity-linked securities that are placed outside of a public offering, are exempt from registration, and are used as vehicles for publicly traded companies to issue additional equity shares (or other securities) in their firms. Investors purchase the securities directly from a publicly traded company in a private transaction. In other words, the "public" part of the name reflects that they are vehicles for publicly traded companies to issue additional equity shares (or other securities) in their firms. The "private" part of the name reflects that the securities are sold directly to investors, who usually cannot trade them in secondary markets for a specified period of time, frequently three to six months. PIPEs are often used as a substitute for secondary offerings of shares that are immediately publicly traded. Since 2001, the proceeds of PIPEs and secondary equity offerings (SEOs) in the U.S. market are approximately equal.

\section*{Characteristics and Types of Securities Issued through PIPEs}
In the United States, PIPE issuers can be anything from small companies listed in the over-the-counter market to large companies listed on the New York Stock Exchange (NYSE). Some PIPE transactions involve small, nascent corporations of the type that interest venture capitalists. They are also often issued by small to medium-size firms that may face difficulties, expenses, or delays in using public security offerings. The typical profile is a company with a market capitalization of under $\$ 500$ million that seeks an equity infusion of between $\$ 10$ million and $\$ 75$ million. However, some PIPE transactions involve established public companies, the domain of the buyout market.

PIPE deals offer investors a variety of securities that can be issued including the following:

\begin{itemize}
  \item Privately placed common stock: The greater the illiquidity, the greater the discount on the PIPE's issue price.
  \item Registered common stock: The advantage to the investor is that it can acquire a block of stock at a discount to the public market price for the registered common stock. This is particularly appealing for PE firms that have large chunks of cash to commit to companies.
  \item Convertible preferred shares or convertible debt: Conversion prices embedded in preferred stock and convertible debt tend to be lower than the conversion prices on publicly traded instruments. In addition, the issuer of the PIPE usually commits to register the equity securities within the next six months. This feature is particularly appealing for PE firms to the extent that they are able to purchase cheap equity with a ready-made exit strategy.
  \item Equity line of credit: An equity line of credit (ELC) is a contractual agreement between an issuer and an investor that enables the issuer to sell a formula-based quantity of stock at set intervals of time.
\end{itemize}

\section*{Buyer and Seller Motivations for PIPEs}
The greatest advantages for the issuing company of a PIPE are: (1) that the company can quickly raise capital (a PIPE transaction can be completed in just a few weeks) without the need for a lengthy registration process (which can take up to nine months), and (2) larger companies view PIPEs as a cheaper process for raising capital quickly, especially from a friendly investor.

Further, the management of the issuing company does not need to be distracted with the prolonged road show that typically precedes a public offering of stock. Management can remain focused on the operations of the business while receiving an equity infusion that strengthens the balance sheet.

The documentation required for a PIPE is relatively simple, compared to a registration statement. Typically, all that is needed is an offering memorandum that summarizes the terms of the PIPE, the business of the issuer, and the intended uses of the PIPE proceeds.

Another reason PE firms are interested in PIPEs is that they allow the PE firm to gain a substantial stake in the company, even control, at a discount. This is very enticing to PE firms, which normally have to pay a premium for a large chunk of a company's equity.

\section*{Traditional PIPEs and Structured PIPEs}
The biggest distinction between PIPEs is that of traditional PIPEs and structured PIPEs. The large majority of PIPE transactions are traditional PIPEs, in which investors can buy common stock at a fixed price. Most traditional PIPE transactions are initiated using convertible preferred stock or convertible debt with a fixed price at which the securities can be converted into common stock. The conversion price is the price per share at which the convertible security can be exchanged into shares of common stock, expressed in terms of the principal value of the convertible security. The conversion ratio is the number of shares of common stock into which each convertible security can be exchanged. The conversion ratio and the conversion price are inversely related measures of the same concept.

Having a fixed conversion price or conversion ratio limits the amount of dilution to existing shareholders. Also, the convertible preferred stock or debt may provide the investor with dividends and other rights in a sale, merger, or liquidation of the company that are superior to the residual claims of the existing stockholders.

Structured PIPEs include more exotic securities, like floating-rate convertible preferred stock, convertible resets, and common stock resets. These PIPEs have a floating conversion price that can change depending on the price of the publicly traded common stock. They are sometimes referred to as floating convertibles because the conversion price of the convertible preferred stock or debt floats up or down with the company's common stock price.

\section*{Toxic PIPEs}
In the past, structuring of PIPEs led to the creation of toxic PIPEs and so-called death spirals. A toxic PIPE is a PIPE somewhat popular years ago with adjustable conversion terms that can generate accelerating levels of shareholder dilution in the event of declining prices in the firm's common stock. Floating convertibles received a bad reputation because, unlike standard convertible bonds or preferred stocks, which get converted at a fixed conversion price, the conversion price for toxic PIPEs adjusts downward whenever the underlying common stock price declines. The drop in stock price leads to a drop in the conversion price, which can lead to a substantial dilution of shareholder value.

For example, under a structured PIPE, if the stock price of the issuer declines in value, the PIPE investor receives a greater number of shares upon converting the PIPE. Expressed differently, the conversion price of the PIPE declines commensurately with the underlying stock price. This can lead to a situation that is potentially poisonous to the issuing company's financial health. A toxic PIPE can generate the following sequence:

\begin{itemize}
  \item A company with a weak balance sheet and uncertain cash flows cannot issue additional publicly traded shares of its common stock.
  \item PE investors agree to provide more capital in return for structured PIPEs that can be converted into stock at a floating conversion rate and at a discount to the common stock price.
  \item The stock price of the company falls. The price decline may be triggered by PE investors short selling the publicly traded stock of the company to hedge their purchase of the PIPE, by a decline in the company's profitability, or both. A large downward movement in the stock price is the catalyst that can turn a structured PIPE into a toxic PIPE.
  \item The downward pressure on the company's common stock price triggers larger and larger conversion ratios for the PIPE investors (i.e., lower conversion prices), resulting in greater and greater dilution of the common stock of the company as the PIPE investors are granted an ever-growing number of shares representing an ever-rising percentage ownership in the firm.
  \item Prospects for greater dilution of the company's stock drive the market price of the stock further downward. The lower stock price again forces the company to reduce the conversion price for the PIPEs into common stock at lower and lower prices.
  \item The process of lower conversion prices, greater dilution, and lower share prices repeats in a downward death spiral.
  \item Ultimately, the PIPE investors exercise their conversion rights at greatly depressed conversion prices and either sell their converted shares (obtained at a large discount) or take control of the company using the large number of new shares.
\end{itemize}

Although this scenario sounds improbable, some PIPE transactions led to poisonous results for the issuing company. Although structured PIPEs still exist, both investors and companies receiving PIPE financing have become much more sophisticated regarding the details and floating conversion rates in toxic PIPEs. The learning experiences of the late 1990s and early 2000s led to a PIPE market with more sensible deals and less likelihood of perverse incentives.


\end{document}