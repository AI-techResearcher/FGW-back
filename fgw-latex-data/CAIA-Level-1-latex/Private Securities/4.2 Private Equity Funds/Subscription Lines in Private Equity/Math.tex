\documentclass[11pt]{article}
\usepackage[utf8]{inputenc}
\usepackage[T1]{fontenc}
\usepackage{amsmath}
\usepackage{amsfonts}
\usepackage{amssymb}
\usepackage[version=4]{mhchem}
\usepackage{stmaryrd}

\begin{document}
\section*{APPLICATION A}
Question : Scenario 1, Consider a buyout fund that called $\$ 750$ from its LPs to invest in a portfolio company in 2010 . In 2013, the fund received $\$ 150$ from the portfolio company and distributed it to its LPs. Finally, in 2015 the portfolio company was sold with $\$ 1,500$ distributed to its investors.

\begin{center}
\begin{tabular}{|r|r|r|r|r|r|r|}
\hline
 & 2010 & 2011 & 2012 & 2013 & 2014 & 2015 \\
\hline
LP Cash flow & -750 & 0 & 0 & 150 & 0 & 1,500 \\
\hline
\end{tabular}
\end{center}

What is the IRR and total profit from the deal?

\section*{Answer and explanation}
IRR $=18 \%$\\
Total Profit $=\$ 900$

\section*{APPLICATION B}
Question :  Consider a buyout fund that called $\$ 750$ from its LPs to invest in a portfolio company in 2010 . In 2013, the fund received $\$ 150$ from the portfolio company and distributed it to its LPs. Finally, in 2015 the portfolio company was sold with $\$ 1,500$ distributed to its investors. Assume the fund decided to use a SLOC to invest in the portfolio company. This gives the fund an advantage by providing an earlier closing to the seller than if the fund had to wait to call capital from their LPs. The SLOC has a simple annual interest rate of $2 \%$. Rather than calling investor capital in 2010, the SLOC delayed investor capital calls until 2012. What is the IRR and total profit from the deal using the SLOC?

\begin{center}
\begin{tabular}{|l|r|r|r|r|r|r|}
\hline
 & 2010 & 2011 & 2012 & 2013 & 2014 & 2015 \\
\hline
 & 0 & 0 & 0 & 150 & 0 & 1,500 \\
\hline
SLOC-Interest &  &  & -30 &  &  &  \\
\hline
Capital Commitment &  &  & -750 &  &  &  \\
\hline
LP Cash flow & 0 & 0 & -780 & 150 & 0 & 1500 \\
\hline
\end{tabular}
\end{center}

\section*{Answer and explanation}
IRR $=31 \%$. This is higher than Scenario 1 because the use of SLOC delayed the capital call from LPs.

Total Profit $=\$ 870$. This is lesser than Scenario 1 because of the interest paid on SLOC: $\$ 30(\$ 750 \times 2 \% \times 2$ years)


\end{document}