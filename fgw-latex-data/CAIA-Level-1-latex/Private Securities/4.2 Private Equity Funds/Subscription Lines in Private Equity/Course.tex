\documentclass[11pt]{article}
\usepackage[utf8]{inputenc}
\usepackage[T1]{fontenc}
\usepackage{amsmath}
\usepackage{amsfonts}
\usepackage{amssymb}
\usepackage[version=4]{mhchem}
\usepackage{stmaryrd}
\usepackage{hyperref}
\hypersetup{colorlinks=true, linkcolor=blue, filecolor=magenta, urlcolor=cyan,}
\urlstyle{same}

\begin{document}
Subscription Lines in Private Equity

A subscription line of credit (SLOC), also referred to as a line of credit, is a liquidity management tool used for the interim financing needs of a PE fund and secured by unfunded capital commitments of investments in the fund. The amount of credit availability is generally based on borrowing up to $90 \%$ of the unfunded capital commitments of LPs, which are generally investment-grade institutional investors. Subscription lines are generally structured as senior secured revolving facilities with 2- to 3-year terms.

For the General Partner (GP), the benefits of using a subscription line include:

\begin{itemize}
  \item Improved transaction execution certainty. Subscription lines allow for quick access to capital for acquisitions, allowing the GP to better control the timing of closing transactions.
  \item Better cash flow management. GPs can smooth out the cash flow and make capital calls from LPs in larger batches instead of smaller and more frequent calls. This may result in a reduced J-curve for the LP.
  \item Reduced administrative burden. Avoid making capital calls from LPs if the timing of the closing of the investment does not coincide with current cash flow availability.
  \item Enhanced IRR performance metric, but slightly reducing multiple-based performance metric. The use of an SLOC can delay capital calls from LPs. The typical result is that the IRR of the fund rises while the fund multiple declines due to the cost of the financing.
\end{itemize}

Lenders are attracted to providing such facility because:

\begin{itemize}
  \item Secured: Subscription lines are secured by unfunded capital commitment of investors in the fund. These investors are typically institutional investors with significant financial strength.
  \item Diversification: The facility also provides a more diverse base supporting repayment than typical corporate credit.
  \item Fee earning opportunity. The lender has multiple fee streams from upfront fees, drawn and undrawn fees, and the loan margin.
\end{itemize}

Subscription lines have been used by private markets for decades. But the usage and the scope of subscription lines have increased and widened in recent years. The global fund subscription line market is estimated to be over US $\$ 300$ bn based on lender's commitment for 2017, with the estimated size of subscription lines markets in the US, Europe, and Asia being $\$ 200$ bn, $\pounds 65$ bn, and $\$ 30$ to $\$ 35$ bn, respectively (Russell et al. 2019). ${ }^{1}$ Russell, Emma, Zoë Connor, and Emily Fuller (2019). Comparing the European, U.S., and Asian fund finance markets. In Fund Finance 2019, page 121. Extracted from: \href{https://www.acc.com/sites/default/files/resources/}{https://www.acc.com/sites/default/files/resources/} upload/GLI FF3 eEdition.pdf There is also strong demand from different types of PE funds on a global basis. These include buyouts, real estate, infrastructure, energy, shipping, mezzanine, healthcare, and other types of private equity funds.

The use of subscription lines is considered a double-edged sword for LPs. SLOCs can help reduce the operational burden, as capital calls can be batched together instead of smaller more frequent calls. It also has the potential to enhance the IRR, as LP capital calls may be delayed, and investment returns can be generated on borrowed money. This represents an inherent conflict of interest for the GP in the use of SLOC. Albertus and Denes $(2020)^{2}$ Albertus, James F., and Matthew Denes, Private Equity Fund Debt: Capital Flows, Performance, and Agency Costs (May 26, 2020). Available at SSRN: \href{http://dx.doi.org/10.2139/ssrn}{http://dx.doi.org/10.2139/ssrn}. 3410076 found that SLOCs distort performance measures that are sensitive to cash flow timing, and the use of SLOCs is more common among poorly performing funds. On the other hand, LPs can continue to invest the uncalled capital in other investments, where the return can potentially exceed the cost of the SLOC.

LPs should recognize that the use of SLOCs could create perverse incentives for GPs. Depending on the subscription agreement, carried interest can be calculated using IRR or multiple-based metrics. For an IRR-based hurdle, the use of SLOCs will enhance the IRR. For multiple-based metrics, the use of SLOC will slightly lower the multiple-based metric. While the higher IRR may be good for the LP, it lowers the total profit of the fund due to the amount of interest and fees incurred to access the SLOC. To monitor whether GPs are using subscription lines to enhance fund IRRs, ILPA (2020) ${ }^{3}$ ILPA (2020). Enhancing Transparency Around Subscription Lines of Credit. An ILPA publication. recommends the quarterly disclosure of details of subscription lines, including:

\begin{itemize}
  \item Total size of facility
  \item Total balance of facility
  \item Average number of days outstanding of each drawdown
  \item Net IRR with and without the use of the facility
\end{itemize}

\end{document}