\documentclass[11pt]{article}
\usepackage[utf8]{inputenc}
\usepackage[T1]{fontenc}
\usepackage{amsmath}
\usepackage{amsfonts}
\usepackage{amssymb}
\usepackage[version=4]{mhchem}
\usepackage{stmaryrd}

\begin{document}
Private Equity Secondary Markets and Structures

This lesson on PE funds briefly covers four topics that involve the trading mechanisms and organizational structures of PE investment.

\section*{The Secondary Market for Private Equity Partnerships}
This section discusses secondary trading of PE limited partnership interests. Many PE interests are organized as limited partnerships. The lack of registration and public trading makes the purchase and sale of LP interests less liquid than listed stock or other registered securities. Investors in PE are typically subject to a 10-year lockup period. Investors wishing to liquidate their investment before these exits will need to access the secondary market.

Three primary reasons to sell limited partnership interests: Most investors considering the acquisition of an investment take into account its liquidity, which means the ability to sell the investment without needing to offer a substantial discount from the value that would be obtained in a liquid market. There are three primary reasons that a PE investor may need to sell part of a portfolio:

\begin{enumerate}
  \item To raise cash for funding requirements: For example, a pension fund may need to generate cash to fund retirement benefits for pension recipients or meet capital calls.

  \item To trim the risk of the investment portfolio: During the global financial crisis, many large investors decided that they needed to strategically adjust the risk profiles of their investment portfolios.

  \item To rebalance the portfolio from time to time: This is a form of active portfolio management in which allocations to asset classes are changed resulting in a partial liquidation of an asset class.

\end{enumerate}

These three reasons are about the motivation to sell secondary PE interests and not about the value of the underlying investment. Without a secondary market, these liquidity needs might be more difficult or expensive to meet, leading an institution to allocate less to PE or to demand higher expected returns in compensation. With an active secondary market, the institution not only can meet liquidity needs at lower costs but also can better understand and manage its risk exposure through the information and opportunities provided by the market and its prices.

The role of GPs in secondary market transactions: There are two potential problems other than unfavorable price execution with selling limited partnerships into the secondary market. First, the general partner's permission (sign on) may be necessary to consummate the exchange. Second, GPs usually do not like to see their investors sell their limited partnership interests to outside third parties and may be unlikely to invite that limited partner to join in future PE funds that the general partner sponsors.

Buyers of secondary market limited partnership interests: From a buyer's perspective, there are several advantages to a secondary purchase of PE limited partnerships: (1) the investor might gain exposure to a portfolio of companies with a vintage year that is different from the investor's existing portfolio (facilitating vintage year diversification); (2) secondary interests typically represent an investment with a PE firm that is further along in the investment process than a new PE fund and may be closer to harvesting profits from the private portfolio; (3) purchasing the secondary interest of a limited partner who wishes to exit a PE fund may be a way for another investor to gain access to future funds offered by the general partner; and (4) the buyer may see greater potential for cash flows from the secondary portfolio than current primary investments. Simply stated, this is opportunistic buying, especially if the limited partnership interests are trading at substantial discounts.

\section*{Private Equity, Hedge Funds, and Six Fee Differences}
Competition for capital and for deals is forcing changes in the PE and hedge fund industries. Particularly, there is an increasingly blurry line between hedge funds and PE firms. Hedge fund managers are now bidding for operating assets in open competition with PE firms.

Hedge funds are moving into PE for diversification and their desire to apply their skills to new areas. In particular, the issues are based on differing fee structures for hedge fund managers compared to PE fund managers. The following list summarizes the six major differences between typical hedge fund incentive fees and typical PE fund incentive fees:

\begin{enumerate}
  \item Hedge fund incentive fees are front loaded. PE fund fees tend to be collected at the termination of deals.

  \item Hedge fund incentive fees are based on changes in net asset value whether the gains are realized or unrealized. PE fund fees are based on realized values of exited positions.

  \item Hedge fund incentive fees are collected on a regular basis, either quarterly or semiannually. PE fund incentive fees tend to be collected at the time of an event, such as exit.

  \item Investor capital does not need to be returned first to collect incentive fees in a hedge fund. PE funds typically do not distribute incentive fees until the original investor capital has been repaid.

  \item Hedge funds often have no provisions for the clawback of management or incentive fees. PE funds typically have clawback provisions requiring the return of fees on prior profits when subsequent losses are experienced.

  \item Hedge funds rarely have a preferred rate (hurdle rate) of return (e.g., 6\%) that must be exceeded before the hedge fund manager can collect an incentive fee. Most PE funds have a hurdle rate.

\end{enumerate}

In sum, the deal terms for a hedge fund are much more favorable to managers than are those for PE fund managers. Another consideration is that hedge funds with hurdle rates tend to have lower hurdle rates than PE funds in the computation of incentive fees. Most PE funds target returns in the $20 \%$ range, whereas hedge funds aim to beat a cash index plus some premium (e.g., LIBOR plus 6\%). This provides hedge fund managers with a competitive advantage against PE firms when bidding for operating assets, since lower hurdle rates provide hedge fund managers with an incentive to bid more aggressively than PE firms.

\section*{Publicly Traded Private Equity Firms and Their Governance}
There are relatively few publicly traded PE strategies in the liquid alternative space. Those strategies that are available tend to use the closed-end fund format, as the relatively permanent capital of those structures matches the constrained liquidity of the underlying investments. However, publicly traded PE firms offer investors exposure to PE and earning carried interest from high returns to those firms generated from the underlying fund investments.

Publicly traded shares of PE asset management companies include major firms such as Apollo, Ares, Blackstone, Carlyle, KKR, and Oaktree. PE exposure to these firms and others can be achieved through ETFs such as the Invesco Global Listed PE. Exposure can also be achieved through open-end mutual funds and listed companies, such as Onex, that hold stakes in a large number of private companies.

Drury discusses the governance issues raised by publicly held PE firms in detail. ${ }^{1}$ Lloyd L. Drury III, "Publicly-Held PE Firms and the Rejection of Law as a Governance Device," 16 University of Pennsylvania Journal of Business Law 57 (2013). The following are among Drury's observations:

\begin{enumerate}
  \item Listed PE firms tend to be organized as limited partnerships and LLCs rather than corporations, which, among other things, can be used to reduce personal liability for managers to LPs and be less friendly to the interests of the non-insiders.

  \item The organizational structures used by listed PE firms tend to retain control of the firm by insiders (managers) as opposed to other shareholders.

  \item The structures and jurisdictions selected by listed PE firms tend to reduce or even waive the management's fiduciary duties to its shareholders.

  \item Listed PE firms tend to exclude normal corporate controls by shareholders through the board of directors and opt out of governance rules promulgated by stock exchanges.

\end{enumerate}

As discussed in the next section, these governance issues can be argued to be attempts to preserve the strength of traditional PE firms: strong management that is highly incentivized to take risks in unlocking value. On the other hand, the governance structures used by listed PE firms may place non-inside shareholders at uncompensated peril.

\section*{The Battle between Private Equity Governance Structures}
One or more legal structures-both private and public-can be inserted between PE investors and underlying private enterprises. On the private end of the privatepublic spectrum are the purely private PE firms such as Bain Capital. On the public end of the spectrum are ETFs that invest in listed PE firms that, in turn, invest directly in enterprises. But there are increasingly blurred lines, with private PE firms deciding to go public and with listed firms taking large private PE exposures through direct investments in private PE firms.

The PE landscape can be viewed as a competition or battle between public and private governance structures at various levels, with advantages and disadvantages to each. This section discusses three potentially important issues over which private and public access have varying strengths and weaknesses: diversification, liquidity (listing), and regulation.

Diversification as a double-edged sword: Diversification reduces risk. Well-diversified investors can require lower rates of return on capital, thereby offering a competitive advantage to firms with diversified investors. A top reason that PE firms seek capital through listing is to attract substantial capital at relatively low cost. However, one of the primary advantages to PE management is that the top managers are incentivized to focus their considerable strengths and energies into creating value because their wealth is disproportionately concentrated in the investments that they are overseeing. One reason that the private PE structure is so successful is the skill and devotion of its poorly diversified management teams (i.e., who have concentrated exposures to the investments).

Liquidity (Listing) as a double-edged sword: Liquidity in PE involves not only the ability to exit an investment quickly at a non-discounted price, it also means not having to meet capital calls on committed capital. Listed investments also offer investors the confidence that the market price is likely to be better protected on the downside due to the ability of short sellers to constrain listed securities from being overpriced. In other words, the presence of short sellers provides some level of confidence that market prices are deemed to be reasonable by at least some investment professionals. Prices of private deals do not offer this protection. However, public listing involves substantial fees and has been argued to cause the prices of listed PE to take on the volatility of the entire stock market during times of crisis, as evidenced by the contrast between the reactions of public REITs and private real estate during the global financial crisis. Public listing provides visibility, which can help enterprises market their products and signal positive reputations, perhaps even building loyalty from customers who are also shareholders. However, public listing invites scrutiny such as nuisance shareholder proposals and government intrusions.

Regulation as a double-edged sword: On the one hand, strong regulations tend to generate better information revelation by the firm to the public and may provide some protection to investors against fraud and other abuses. On the other hand, regulations can be burdensome, expensive, and may constrain top managers from fully unlocking potential value.


\end{document}