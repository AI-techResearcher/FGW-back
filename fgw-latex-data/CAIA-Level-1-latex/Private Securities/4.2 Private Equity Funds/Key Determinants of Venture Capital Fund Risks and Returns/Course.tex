\documentclass[11pt]{article}
\usepackage[utf8]{inputenc}
\usepackage[T1]{fontenc}
\usepackage{amsmath}
\usepackage{amsfonts}
\usepackage{amssymb}
\usepackage[version=4]{mhchem}
\usepackage{stmaryrd}

\begin{document}
Key Determinants of Venture Capital Fund Risks and Returns

VC funds require different management from most other investment funds and contain different risks. This lesson summarizes key issues.

\section*{Access as a Key to Enhanced Returns}
The key to enhanced returns in VC investing by institutions is accessing the top-tier VC fund managers. There is substantial evidence that return performance is very persistent in the PE industry. The VC firms that are successful tend to form a series of VC funds through time. The VC managers that perform well in one VC fund tend to perform well in their next VC fund.

The likelihood of return persistence in VC is different from that of other asset classes, including large-cap public equities, in which the marketplace is much more liquid and competitive. Superior management teams of public equities are accessible to everyone; virtually anyone can purchase equity in such public firms. The result tends to be that the share price of public firms with highly successful management teams is bid up to a market price that reflects the likelihood that the management team will continue to be successful. In a competitive and efficient market, superior shareholder returns would not be likely to persist when managers continue to excel as expected. However, VC funds are not funded at market prices driven through competition.

Superior performance in a PE firm's most recent funds is usually viewed as a predictor that the firm's next fund will also generate superior performance. Quite a bit of this performance persistence can be explained by the reputation of the general partner managing the VC fund. The best VC firms attract the very best entrepreneurs, business plans, and investment opportunities. The most successful VC firms have an established track record of getting start-up companies to an initial public offering. Their track record allows them to attract investment capital from their LPs, as well as proprietary deal flow from start-up companies seeking VC. The general partner of a better-performing VC fund is more likely to raise a follow-on fund and to raise larger funds than a VC firm that performs poorly. GPs of the most successful PE funds can pick and choose among numerous investors desiring to participate in their next fund.

\section*{Three Dimensions to Diversifying Venture Capital Risk}
Achieving diversification, especially vintage-year and industry diversification, is the key to reducing risk. Given the cyclical nature of the overall economy and of PE in particular, a deal's vintage year can be an important determinant of the deal's success. Institutions investing in PE funds often analyze their holdings with respect to vintage years. They may seek diversification of their fund holdings across vintage years, or they may seek higher returns by strategically allocating with respect to vintage years based on their market view. Kaplan and Schoar find evidence of a boom and bust cycle in which high PE returns encourage new investment, which leads to reduced subsequent returns. ${ }^{1}$ Steven Kaplan and Antoinette Schoar, "Private Equity Performance: Returns, Persistence, and Capital Flow," Journal of Finance 60, no. 4 (2005): 1791-1823, doi:10.1111/j.1540-6261.2005.00780.x. Presumably, in vintage years when there were large investment amounts and greater competition for deals, the higher deal prices eventually led to lower returns. An institution that concentrates investments in a particular vintage year runs the risk that the vintage year will turn out few winning ventures. Accordingly, investors should diversify into PE funds of different vintage years.

Further dimensions over which investors should consider diversifying their VC portfolios include industry and geography.

\section*{Three Main Risks and the Required Risk Premiums for Venture Capital}
Most venture capitalists are long-term investors who expect to earn a premium from VC that is about 400 to 800 basis points over the returns of the public stock market, depending on the VC stage of financing. This risk premium can be viewed as providing compensation for three main risks.

First, there is the business risk of a start-up company. Although some start-ups successfully make it to the initial public offering stage, many more do not succeed. Venture capitalists must anticipate earning a return that sufficiently compensates them for bearing the risk of potential corporate failure. Although public companies can also fail, VC is unique in that the investor takes on this business risk before a company has had the opportunity to fully implement its business plan.

Second, there is substantial liquidity risk. There is no liquid public market for trading VC interests. The secondary trading that does exist is generally limited to exchange among a small group of other PE investors. This is a fragmented and thus inefficient market. The tailored nature of a venture capitalist's holdings is unlikely to appeal to more than a very select group of potential buyers. Consequently, the sale of an interest in a VC fund is not an easy task. Further, another VC firm may not have the time or ability to perform as thorough a due diligence process as the initial investing firm. Thus, a secondary sale often requires a substantial pricing discount.

Third, there may be an idiosyncratic risk due to the lack of diversification associated with a VC portfolio. The capital asset pricing model (CAPM) shows that the only risk that investors should be compensated for is the risk of the general stock market, or systematic risk. This is because unsystematic or company-specific risk can theoretically be diversified away. However, the CAPM is predicated on securities being freely transferable and infinitely divisible, and portfolios being fully diversifiable. Since the lack of liquidity in VC severely impairs transferability, some venture capitalists are not well diversified, and they bear substantial idiosyncratic risk. In the case of numerous investors who are not highly diversified, the CAPM does not hold, and idiosyncratic risk may be rewarded.

VC firms have become increasingly specialized as a result of the intensive knowledge base required to invest in the technology, telecommunications, and biotechnology industries, and specialization has expanded further to include the stage of investment in the life cycle of a start-up company. Unfortunately, specialization leads to concentrated portfolios, the very anathema of reduced risk through diversification. This concentration leads to the need for higher risk premiums.


\end{document}