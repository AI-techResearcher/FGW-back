\documentclass[11pt]{article}
\usepackage[utf8]{inputenc}
\usepackage[T1]{fontenc}
\usepackage{amsmath}
\usepackage{amsfonts}
\usepackage{amssymb}
\usepackage[version=4]{mhchem}
\usepackage{stmaryrd}

\begin{document}
\section*{APPLICATION A}
Question : A VC fund manager raises $\$ 100$ million in committed capital for his VC fund. The management fee is $2.5 \%$. To date, only $\$ 50$ million of the raised capital has been called and invested in start-ups. What would be the annual management fee?

\section*{Answer and Explanation}
The annual management fee that the manager collects is $\$ 2.5$ million $(2.5 \% \times \$ 100$ million), even though not all of the capital has been invested.

\section*{APPLICATION B}
Question : Assume a $\$ 200$ million contribution by the LPs in the first year to fund an investment, a $6 \%$ hurdle rate, a $100 \%$ catch-up, an $80 / 20$ carry split, and the sale of the investment by the fund in the third year for $\$ 300$ million. What is the total cash flow to the LPs and the total cash flow to the GPs?

\section*{Answer and Explanation}
We are told the LPs contribute $\$ 200$ million at the beginning of the period, which turns into $\$ 300$ million by the third year. With a $6 \%$ hurdle rate, the investment must clear $6 \%$ per year. If we ignore compounding in this problem, that amounts to $6 \% \times 2 \times \$ 200$ or $\$ 24$ million that the LPs must make prior to GPs collecting any carried interest.

The carried interest split is allocated 20\% to the GPs and 80\% to the LPs. Since there is a $100 \%$ catch-up provision in place for the GPs of the fund, and the fund more than exceeded the $\$ 24$ million required return, the GPs are entitled to their catch-up provision. This is where the math seems complicated, but think of it like this: because the carried interest split is 80/20, the LPs have effectively already "earned" a portion of their $80 \%$, or $\$ 24$ million, we must then "catch-up" the GPs so that they can earn their $20 \%$. Mathematically, this looks like this: $(\$ 24$ million $\times 20 \%) /(1-20 \%)=\$ 6$ million.

At this point $\$ 30$ million $(\$ 24+\$ 6$ ) of the $\$ 100$ million in total profit has been accounted for, which means we have $\$ 70$ million left to distribute. Now that all rates and catch-up provisions have been handled, we can simply distribute the carried interest between the LPs and GPs per the 80/20 agreement. This would mean $\$ 56$ million ( $80 \% \times 70$ million) would be distributed to the LPs and $\$ 14$ million $(20 \% \times \$ 70$ million) to the GPs.

In total, $\$ 20$ million would be distributed to the GPs (\$6 million catch-up $+\$ 14$ million residual profits) and $\$ 80$ million to the LPs (\$24 million hurdle rate $+\$ 56$ million residual profits).

So to summarise, assume that the hurdle rate is not compounded and ignore all other expenses such as management fees. The preferred return to the LPs is $\$ 24$ million ( $\$ 200$ million $\times 6 \% \times 2$ ), leaving $\$ 76$ million after also returning the $\$ 200$ million to the LPs. The $100 \%$ catch-up to the GPs for $80 \% / 20 \%$ split is $\$ 6$ million $(\$ 24$ million $\times 20 \% /(1-20 \%)$ ). The residual is $\$ 70$ million, found as $\$ 300$ million $-(\$ 200$ million $+\$ 24$ million $+\$ 6$ million), and is split $80 \% / 20 \%$.

\begin{center}
\begin{tabular}{|lllc|}
\hline
 & To LPs & To GPs & Total \\
\hline
Return of capital & $\$ 200$ million &  & $\$ 200$ million \\
\hline
\end{tabular}
\end{center}

\begin{center}
\begin{tabular}{|lllr|}
\hline
 & To LPs & To GPs & \multicolumn{1}{c|}{Total} \\
\hline
Preferred return to LPs & $\$ 24$ million &  & $\$ 24$ million \\
Catch-up for GP &  & $\$ 6$ million & $\$ 6$ million \\
80/20 split of residual & $\$ 56$ million & $\$ 14$ million & $\$ 70$ million \\
Total profit & $\$ 80$ million & $\$ 20$ million & $\$ 100$ million \\
\hline
\end{tabular}
\end{center}


\end{document}