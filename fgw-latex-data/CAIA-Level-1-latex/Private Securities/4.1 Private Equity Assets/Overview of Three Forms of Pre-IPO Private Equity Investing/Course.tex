\documentclass[11pt]{article}
\usepackage[utf8]{inputenc}
\usepackage[T1]{fontenc}
\usepackage{amsmath}
\usepackage{amsfonts}
\usepackage{amssymb}
\usepackage[version=4]{mhchem}
\usepackage{stmaryrd}

\begin{document}
Overview of Three Forms of Pre-IPO Private Equity Investing

A key investment strategy is to invest in firms that grow into publicly traded corporations.

Most large publicly traded firms began as small, nascent enterprises. For example, Apple was founded as Apple Computer Company in April 1976 by Steve Wozniak, Steve Jobs, and Ronald Wayne. Ronald Wayne soon sold his $10 \%$ share in the company for $\$ 500$. In August 2018 Apple's market capitalization reached $\$ 1$ trillion. This lesson discusses three major forms of investing in private companies with a goal of exiting the investment through an IPO (initial public offering).

\section*{Venture Capital to Initial Public Offering}
Virtually every attempt to start a new business is venture capital, from the smallest retail store to the largest energy exploration. In terms of numbers, most of these ventures are financed fully by their founders, with little or no capital from others. However, this section of the curriculum is about investing in institutional-quality alternative investments, which form the vast majority of the total financial value of venture capital.

Venture capital (VC), the best known of the private equity categories, is early financing for young firms with high potential growth that do not have a sufficient track record to attract investment capital from traditional sources, like public markets or lending institutions. Entrepreneurs develop business plans and then seek investment capital to implement those plans, since start-up companies are unlikely to produce positive cash flow or earnings for several years. The equity stakes that venture capitalists initially acquire begin as a substantial but minority position in the company. Control by VC investors is not absolute.

A VC project is primarily distinguished by its small size, lack of revenues, and high risk. The typical investment into a VC project is $\$ 5$ million or more, with a company value of $\$ 10$ million to $\$ 100$ million. The eventual goal of VC investors is to work with the original owners (typically the founders) to build products, revenues, and income to the point of the firm going public via an IPO and, eventually, for the VC investor to exit the investment through sales of the investor's now-listed equity stake. The pathway to an IPO typically includes additional funding and assistance by the VC investors in management of the firm.

Return targets for VC are large multiples such as 10- or 20-fold increases in value. VC is a large asset class that is often listed separately from other forms of private equity by investment managers.

\section*{Growth Equity to Initial Public Offering}
Growth equity focuses on companies that have established a reliable base of revenues, an established business model, and have opportunities to expand that require more cash than can be funded by existing revenues. Growth equity is provided as additional working capital and/or to facilitate growth by increasing production capacity and developing markets or products. Typical investments in this stage can be $\$ 25$ million or more, to firms of $\$ 100$ million or more in size (middle market size), and annual revenues of $\$ 25$ million to $\$ 50$ million or more.

Growth equity typically does not involve substantial control by the new investors (as opposed to buyouts). Growth equity is usually the last financing round before an IPO or other exit (e.g., a buyout). Return expectations for equity growth are more modest (e.g., less than 10-fold) than the large multiples targeted in VC.

\section*{Buyout to Initial Public Offering}
The largest forms of buyout, detailed in later lessons in this session, involve buying out a public company and taking the company private. However, buyouts of private companies commonly take full control of a company, typically with the eventual goal of taking the firm public. A private company that is a buyout target is typically founder-owned. The key distinction is control. Buyouts typically involve total control by the new investor.

\section*{Contrasting Venture Capital, Growth Equity, and Buyouts}
The exhibit below summarizes eight major distinctions between VC, growth equity, and buyouts based on assets, revenues, control, time horizon, and so forth. Note how growth equity lies between VC and buyouts with respect to most of the distinctions.

\begin{center}
\begin{tabular}{llll}
\multicolumn{3}{c}{Major Distinctions between VC, Growth Equity, and Buyouts} &  \\
 & Venture Capital & Growth Equity & Buyouts \\
\hline
Asset Size & \$10 million+ & \$100 million+ & \$100 million+ \\
Annual Revenue Size & \$0 to $\$ 10$ million & \$25 million+ & \$25 million+ \\
Control by Investor & A Team Approach & No Control Change & Buyer in Control \\
Use of Capital & Establish Product & Revenue Expansion & Earnings Growth \\
Time Horizon & 5-10 Years & 3-7 Years & 3-5 Years \\
Potential Upside & 5- to 20-fold & 3- to 8-fold & 2- to 5-fold \\
Target IRR & $30 \%-60 \%$ & $25 \%-40 \%$ & 20\%-35\% \\
Investment Risk & Very High & Moderately High & Moderate \\
\hline
\end{tabular}
\end{center}


\end{document}