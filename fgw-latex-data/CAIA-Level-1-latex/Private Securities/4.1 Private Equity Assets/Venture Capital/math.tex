\documentclass[11pt]{article}
\usepackage[utf8]{inputenc}
\usepackage[T1]{fontenc}
\usepackage{amsmath}
\usepackage{amsfonts}
\usepackage{amssymb}
\usepackage[version=4]{mhchem}
\usepackage{stmaryrd}

\begin{document}
\section*{APPLICATION A}
Question : A potential VC investment has a projected EBITDA of $\$ 25$ million (if successful) and an EBITDA multiple of 8 , if the project can be exited in 7 years. Ignoring the percentage of the firm that the investor will not own, the costs of providing oversight and managerial assistance, and any other existing claims to the firm such as indebtedness, what is the estimated enterprise value of the investment if its required IRR is $65 \%$ ?

\section*{Answer and explanation}
To solve this problem, we should use Equation 1.

$$
\text { Value of Venture }=\frac{E B I T D A \times E B T I D A \text { Multiple }}{(1+I R R)^{T}}
$$

By multiplying the projected EBITDA by a multiple, the numerator of this equation represents the investment value in seven years, or $\$ 200$ million. We then have to discount this investment value back to the present value using the investment's internal rate of return (IRR), which is $65 \%$.

$$
\text { Value of Venture }=\frac{\$ 25 \text { million } \times 8.0}{1.65^{7}}=\$ 6.01 \text { million }
$$

\section*{APPLICATION B}
Question : A startup is valued at $\$ 10$ million prior to a $\$ 1$ million VC investment of $\$ 1$ million.

Calculate the post-money valuation and the VC investor's proportional ownership.

\section*{Answer and explanation}
Using Equation (2), the post-money valuation is:

POST $=$ PRE + INVESTMENT

POST $=\$ 10$ million $+\$ 1$ million $=\$ 11$ million

Using Equation (3), the ownership proportion of the VC investor is:

Ownership $=$ INVESTMENT $/$ POST

Ownership $=\$ 1 / \$ 11=9.0 \%$


\end{document}