\documentclass[11pt]{article}
\usepackage[utf8]{inputenc}
\usepackage[T1]{fontenc}
\usepackage{amsmath}
\usepackage{amsfonts}
\usepackage{amssymb}
\usepackage[version=4]{mhchem}
\usepackage{stmaryrd}

\begin{document}
Introduction to Private Equity Terms and Background

Private equity is defined broadly in the CAIA curriculum, to such an extent that some private equity securities that are not equity and some that are publicly traded are included in the category. There are no universal categories of private equity. The CAIA curriculum divides private equity into three major categories: (1) venture capital (VC), the financing of start-up companies, (2) growth equity, noncontrolling interests in successfully emerging enterprises, and (3) buyouts, where established companies are purchased for full control.

At the most general level, private securities can be divided into private equity and private credit, although the term private equity is sometimes used as an umbrella term that includes some private credit.

\section*{Private Equity Securities}
Private equity is as old as commerce itself. Virtually every major enterprise began as a small, unlisted firm. Private equity is a long-term investment process that requires patience, due diligence, and hands-on monitoring. Private equity provides the capital investment and working capital that are used to help private companies grow and succeed. The payouts to most private equity investments resemble the payouts to long positions in out-of-the-money calls: The risks are great, but the potential rewards are even greater. This call option view of private equity from the perspective of the investor reflects the frequent total losses and occasional huge gains of private equity investments, especially venture capital.

Consider the three types of assets underlying private equity: $\mathrm{VC}$, growth equity, and buyouts. Venture capital and buyouts focus on opposite ends of the life cycle of a company. Whereas VC represents nascent, start-up companies, buyouts represent established and mature companies. Growth equity tends to lie in the middle between VC and buyouts with underlying firms that are too large and established to be considered VC but too small to be publicly traded firms subject to buyouts.

Several private securities are often termed private equity that are legally or traditionally considered to be fixed-income securities, including mezzanine financing and distressed debt, when the risk characteristics of the securities resemble the risk exposure of equity positions. These securities are introduced briefly in the next three subsections and are covered in detail in the Private Credit and Distressed Debt session.

\section*{Introduction to Mezzanine Debt}
Mezzanine debt blurs the line between equity and debt because it contains both equity-like and debt-like features. It is referred to as mezzanine because it is inserted into a company's capital structure between the floor of equity and the ceiling of senior secured debt. Mezzanine debt is often viewed as a form of private equity because of its high risk and because it often comes with potential equity participation, although it appears as debt on an issuer's balance sheet. More often than not, mezzanine debt represents a hybrid, meaning a combination of debt and equity.

Typically, mezzanine financing is constructed as an intermediate-term bond, with some form of equity kicker thrown in as an additional enticement to the investor. An equity kicker is an option for some type of equity participation in the firm (e.g., options to buy shares of common stock) that is packaged with a debt financing transaction. The equity kicker portion provides the investor with an interest in the upside of the company, whereas the debt component provides a steady payment stream. The gap that mezzanine finance fills can be quite large and include several tranches of junior debt or preferred equity.

\section*{Introduction to Distressed Debt}
Distressed debt investing is the practice of purchasing the debt of troubled companies, requiring special expertise and subjecting the investor to substantial risk. These troubled companies may have already defaulted on their debt, may be on the brink of default, or may be seeking bankruptcy protection. Like the other forms of private equity, this form of investing requires a longer-term horizon and the ability to accept the lack of liquidity for a security for which often no trading market exists.

Similar to the mezzanine debt just discussed, the returns to distressed debt tend to depend little on the overall performance of the stock market. This is because the value of the debt of a distressed or bankrupt company is more likely to rise and fall with the fortunes of the individual company, which in turn are driven mostly by idiosyncratic factors. In particular, the company's negotiations with its creditors have a much greater impact on the value of the company's debt than does the performance of the general economy.

A key to understanding distressed debt investing is to recognize that the term distressed has two meanings. First, it means that the issuer of the debt is troubled; the face value of its liabilities may exceed the value of its assets, or it may be unable to meet its debt service and interest payments as they come due. Therefore, distressed debt investing almost always means that some workout, turnaround, or bankruptcy solution must be implemented for the bonds to appreciate in value. Second, distressed refers to the price of the bonds. Distressed debt often trades for a small percentage of face value. This affords a savvy investor the opportunity to earn extraordinary returns by identifying a company with a viable business plan but a short-term cash flow problem.

Distressed debt investors are often referred to as vulture investors or just vultures because they are alleged to feast on the remains of underperforming companies. They buy the debt of troubled companies, including subordinated debt, junk bonds, bank loans, and obligations to suppliers. Their investment plan is to buy the distressed debt at a fraction of its face value and then seek improvement of their position through major changes in the assets, capital structure, or management of the company.

Both hedge funds and private equity funds invest in distressed debt. The goal of hedge funds in the distressed debt space is mainly to earn short-term trading profits from their event-driven strategy, typically waiting for a catalyst from the resolution of issues in the bankruptcy court. Private equity investors in distressed debt typically have a longer time horizon. In fact, many private equity investors may take control of a company's equity through their distressed debt position, or even hold publicly traded equity that may be distributed through the bankruptcy process.

\section*{Introduction to Leveraged Loans}
Another asset class of fixed-income securities that private equity firms have moved into is leveraged loans. Leveraged loans are the more senior debt security in the capital structure of a firm relative to subordinated debt such as mezzanine financing. They are issued by firms with substantial debt (hence the name leveraged loans) or poor credit. Leveraged loans typically have high interest rates and low credit ratings (i.e., below investment grade). The loans are often created in order to provide the borrower with capital to finance an acquisition, as part of a refinancing, or to provide working capital.

Mezzanine debt, distressed debts, and leveraged loans are detailed in the session Private Credit and Distressed Debt on private credit.


\end{document}