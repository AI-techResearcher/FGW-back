\documentclass[11pt]{article}
\usepackage[utf8]{inputenc}
\usepackage[T1]{fontenc}
\usepackage{amsmath}
\usepackage{amsfonts}
\usepackage{amssymb}
\usepackage[version=4]{mhchem}
\usepackage{stmaryrd}

\begin{document}
Buyouts and Leveraged Buyouts

Buyout is a generic term that denotes a change of ownership. A buyout occurs when capital, often as a mix of debt and equity, is used to acquire an entire existing company (private or publicly traded) from its current shareholders and to operate the company as an independent organization-as opposed to an acquisition, in which the acquired company is folded into the buyer's existing company. Distinctions between buyouts tend to focus on the purpose of the buyout and the management team that will operate the target firm. The largest type of buyout is a leveraged buyout, which is discussed in detail in the lesson entitled Leveraged Buyouts (LBOs).

\section*{Overview of Buyout Types}
A buyout of a private company is a form of private equity that is often executed in lieu of an IPO exit, from the perspective of the shareholders who are selling the company. Buyouts focus on enterprises that are candidates to be transformed into being more profitable and to grow in potential value prior to an IPO. Buyouts of private companies are detailed in the lesson Buyouts of Private Companies.

\section*{Overview of Leveraged Buyouts}
A leveraged buyout (LBO) is distinguished from a traditional investment by three primary aspects: (1) an LBO buys out control of the assets or the firm, (2) an LBO uses substantial leverage, and (3) the resulting leveraged firm is not immediately publicly traded. The target firm of an LBO is typically a publicly traded firm, but the term may also be used to describe buyouts of private firms. Thus, most LBOs transform the target company from being publicly traded to being highly leveraged private equity. LBOs are distinguished from mergers and acquisitions that typically fold the structure and operations of the target firm into the acquiring firm.

When a public company is bought entirely and delisted from the stock exchange, the transaction is referred to as public-to-private (P2P). Target companies for a buyout are established enterprises with tangible assets and are normally beyond the cash-burning stage, which allows the use of debt to finance part of the transaction. Thus these buyouts are referred to as leveraged buyouts.

Control of the new company is concentrated in the hands of the buyout firm and the target company's management, and there are usually no public shares left outstanding. The goal of the buyout is to increase the value of a corporation by unlocking hidden value, maximizing the borrowing capacity of a company's balance sheet, taking advantage of the tax benefits of using debt financing, and/or exploiting existing but underfunded opportunities. Private companies often state that it is easier to make long-term investments without the oversight of investors in public companies, who may focus on short-term results and quarterly earnings.

LBOs are detailed in the lesson, Leveraged Buyouts (LBOs).

\section*{Types of Private Equity Buyouts and Resulting Management}
A management buy-in (MBI) is a type of LBO in which the buyout is led by an outside management team. Control of the new company is taken over by the new (outside) management team, and the old (incumbent) management team leaves. The compensation package, if any, offered to or negotiated by the incumbent managers can be a critical issue, which is discussed in a subsequent lesson.

A buyout that is termed an LBO often involves bringing in a new management team to replace the firm's existing management. An LBO led by the firm's existing managers that retains most top members of the management team is usually referred to as a management buyout. A management buyout (MBO) occurs when the current management acquires the company.

A buy-in management buyout is a hybrid between an MBI and an MBO in which the new management team is a combination of new managers and incumbent managers.

A secondary buyout is an increasingly important sector of buyouts. In most large buyouts, a public company is being taken private. In a secondary buyout (SBO), one private equity firm typically sells a private company to another private equity firm. In effect, a secondary buyout is typically an ownership change among private equity firms. Secondary buyouts provide a secondary-market-like opportunity for private equity firms to exit a buyout. Phalippou notes that the performance of secondary buyouts in a PE fund depends on when the deal is closed relative to the fund's investment period. A buyout fund that purchases a portfolio company through an SBO in the first half of the fund's investment period sees that portfolio company perform inline with other investments in the fund. SBOs purchased late in the fund's investment period, when the GP might be racing to close on deals simply to increase invested capital and subsequent management fees, tend to underperform SBOs purchased earlier in the life of the fund as well as underperforming investments directly in portfolio companies. ${ }^{1}$ Ludovic Phalippou. Private Equity Laid Bare. (CreateSpace Independent Publishing Platform, 2017).

\section*{Private Equity Strategies Based on Their Purpose}
Other private equity strategies that do not fit neatly into the above categories include:

Rescue capital (or turnaround capital) refers to a strategy in which capital is provided to help established companies recover profitability after experiencing trading, financial, operational, or other difficulties.

Replacement capital (also called secondary purchase capital) refers to a strategy in which capital is provided to acquire existing shares in a company from another PE investment organization.


\end{document}