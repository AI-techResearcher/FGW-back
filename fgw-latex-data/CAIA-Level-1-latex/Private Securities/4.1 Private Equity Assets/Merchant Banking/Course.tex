\documentclass[11pt]{article}
\usepackage[utf8]{inputenc}
\usepackage[T1]{fontenc}

\begin{document}
Merchant Banking

Merchant banking is so closely related to buyouts that it is sometimes difficult to distinguish between the two. Merchant banking is the practice whereby financial institutions purchase nonfinancial companies as opposed to merging with or acquiring other financial institutions. Most major banks have merchant banking units. These units buy and sell nonfinancial companies for the profits that they can generate, much as in the case of buyouts. In some cases, the merchant banking units establish limited partnerships, similar to buyout funds. At that point, there is very little distinction between a merchant banking fund and the buyout funds discussed earlier, other than that the general partner is a financial institution.

Merchant banking started as a way for investment banks and money center banks to establish an equity participation in the enterprises they helped fund. If a bank lent money to a buyout group to purchase a company, its merchant banking unit also invested some capital as equity capital and received an equity participation in the deal. Soon, the merchant banking units of investment banks established their own buyout funds and created their own deals.

Whereas merchant banking is designed to earn profits for the bank, it also allows the bank to expand its relationship with the buyout company into other moneygenerating businesses, such as underwriting, loan origination, merger advice, and balance sheet recapitalization. All of this ancillary business translates into fee generation for the investment bank.


\end{document}