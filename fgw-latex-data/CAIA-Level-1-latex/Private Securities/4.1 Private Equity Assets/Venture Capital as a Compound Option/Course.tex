\documentclass[11pt]{article}
\usepackage[utf8]{inputenc}
\usepackage[T1]{fontenc}
\usepackage{amsmath}
\usepackage{amsfonts}
\usepackage{amssymb}
\usepackage[version=4]{mhchem}
\usepackage{stmaryrd}

\begin{document}
Venture Capital as a Compound Option

Previous lessons in this session describe the call-option-like payoff and nature of venture capital. Further valuable insight can be derived from viewing VC as a compound option. A compound option is an option on an option. In other words, a compound option allows its owner the right but not the obligation to pay additional money at some point in the future to obtain an option.

For example, consider a project requiring $\$ 100,000$ of angel capital and expected to last one year to explore a business idea potentially capable of receiving $\$ 2$ million of seed capital. If successfully deployed, the seed capital may lead to early-stage financing of $\$ 5$ million, which in turn could lead to later stages with even higher capital requirements, ultimately leading to the possibility of an IPO.

Money invested in each of these stages of a venture can be viewed as the purchase of a call option on investing in the next stage of the venture, which in turn is a call option. In the very first investment of $\$ 100,000$, the $\$ 100,000$ is the price or premium of the first option on the project, which has an expiration date of one year and a strike price of $\$ 2$ million. If that option is exercised, the venture capitalist acquires another option costing $\$ 2$ million, with a strike price of $\$ 5$ million.

The compound option view of VC is synonymous with the analysis of real estate development as a string of real options in the session, Real Estate Equity. In both cases, the key to the process is that the option's owner delays committing further capital until new information has arrived. Entrepreneurs may be charged with reaching milestones. A milestone is a set of goals that must be met to complete a phase and usually denotes when the entrepreneur will be eligible for the next round of financing. That is, the venture capitalist may explicitly state the specified operating goals of the firm that must be met before more funds are invested in the venture. Milestones may include patents on a product, revenue from a product, improvements in EBITDA, and so on. It is the ability to defer investment decisions until uncertainty has diminished that gives these options their primary value, not the time value of money.

When viewing each VC investment as a compound call option, the option's expiration date is the point in time at which either additional capital has to be invested or the project is abandoned or sold. Options are exercised when the option holder perceives that the value of the next option being acquired exceeds the strike price of the current option. If all options are successfully exercised, even the option to exit via an IPO, the resulting equity in the leveraged public company itself can be viewed and analyzed as a call option. In a VC project, each call option is purchased far out-of-the-money and typically has modest chances of being exercised.

The compound option view of VC facilitates an understanding of the high value to a venture capitalist of being able to make relatively small investments in projects that generate high profits if successful and can be abandoned if unsuccessful. The compound option view also clarifies the two keys to successful VC investing: (1) identifying underpriced options by locating potentially valuable projects for which substantial information regarding likely profitability can be obtained prior to commitment of substantial capital, and (2) abandoning worthless out-of-the-money options when they are expiring by ignoring sunk costs and judiciously assessing likely outcomes of success based on the objective analysis of new information.


\end{document}