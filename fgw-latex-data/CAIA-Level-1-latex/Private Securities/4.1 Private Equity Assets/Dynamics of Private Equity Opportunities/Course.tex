\documentclass[11pt]{article}
\usepackage[utf8]{inputenc}
\usepackage[T1]{fontenc}
\usepackage{amsmath}
\usepackage{amsfonts}
\usepackage{amssymb}
\usepackage[version=4]{mhchem}
\usepackage{stmaryrd}
\usepackage{hyperref}
\hypersetup{colorlinks=true, linkcolor=blue, filecolor=magenta, urlcolor=cyan,}
\urlstyle{same}

\begin{document}
Dynamics of Private Equity Opportunities

This lesson concludes the session with a discussion of dynamics that have implications for private equity investing, especially VC and growth equity.

\section*{Implications of Winner-Take-All Markets}
The process of bringing a venture from a nascent company to a huge enterprise increasingly resembles a winner-take-all market. A winner-take-all market refers to a market with a tendency to generate massive rewards for a few market participants that apparently provide products or services that are only marginally better than their competitors. Perhaps the most successful venture was a little quicker to market or possessed a charismatic founder. The implication of winner-take-all markets for venture capitalists is that being the second- or third-best entrant in a market may result not only in forgone profits but also in losses and bankruptcy.

VC performance is increasingly driven by participation in successful ventures known as unicorns. A unicorn is a VC-backed firm that soars to $\$ 1$ billion or more in private market capitalization over a relatively short period of time. While in previous centuries, growth of firms to huge sizes required decades of successful operations, in recent years the unicorns have been able to explode to enormous market capitalizations with little or no history of profitable operations.

The implications to VC investors of winner-take-all markets and the preeminence of unicorns in generating attractive returns include the importance of diversification and superior management. Private equity investors may increasingly: (1) seek smaller investments in more enterprises, and (2) seek investments only with the most talented private equity managers.

\section*{Implications of Longer Time Horizons to Exits}
VC investments are taking increased time to exit via an IPO relative to the time taken in previous decades. Mäkiaho (2016) finds "strong evidence that the private equity holding periods have significantly lengthened from the pre-crisis average of 4.7 years to 5.8 years after the financial crisis, despite the exit route. ${ }^{1}$ Juho Mäkiaho, "Prolonged Private Equity Holding Periods: European Evidence" (Master’s thesis, Department of Finance, Aalto University, 2016), \href{https://pdfs.semanticscholar.org/6d0f/}{https://pdfs.semanticscholar.org/6d0f/} 3368844b7caf3fca9b25d33d258adbbeaca2. pdf. Additionally, only 42\% of the post-crisis exits were made in less than five years, compared to $61 \%$ for the pre-crisis period."

With the average time for a start-up firm to exit via an IPO increasing, VC investors and private equity funds (discussed in detail in the session, Private Equity Funds) must be prepared for longer-term investment horizons. The lengthening time horizons for venture capital add to the illiquidity of private equity and the importance of growth equity.

\section*{Three Potential Reasons for the Declining Number of Public Firms in the United States}
VC investing and growth equity bridge the gap between nascent private firms and established public firms. The economic environment has undergone major changes in the last few decades. For example, the number of listed firms in the United States has declined markedly from well over 7,000 in the mid-1990s to less than 5,000 in 2021.

The reasons for the decline in the number of publicly listed equities (but not the total value of those equities) are not entirely clear. An article by Schumpeter in the Economist (2017) notes: (1) concerns over increasing regulations, (2) pressure from shareholders of public companies regarding short-term stock price performance, and (3) the decline in IPOs "from 300 a year on average in the two decades to 2000 to about 100 a year since." 2 Schumpeter, "Why the Decline in the Number of Listed American Firms Matters," The Economist, April 22, 2017.

The decline in IPOs provides fewer entrants to public listings. Further, many publicly traded firms are delisting due to: consolidation of public companies (e.g., through mergers), companies going from public to private (e.g., Equity Office Properties Trust), and bankruptcies. Public listing comes with listing fees, pressures such as disgruntled shareholders regarding recent market performance or headline news, and inefficiencies in governance such as nuisance shareholder proposals. Events such as the $\$ 20$ million fine levied by the SEC against Tesla in 2018 (for tweets by its founder, Elon Musk), and the subsequent resignation of Musk as chairman, have increased concern over the efficacy of being listed in markets that are increasingly driven by rapid technological change and young entrepreneurs.

\section*{Competition between Private and Public Ownership Structures}
A key issue in equity markets is the relative strengths of the private equity governance structure and the public equity governance structure. While both structures are sure to persist, the key question is, which structure is better for a particular enterprise at a particular point in time?

Clearly, nascent firms lack the size to be public and can benefit greatly from the managerial expertise and connections that venture capitalists bring to the organization. As discussed in the session, Private Equity Funds, private equity fund managers can bring enormous skill, experience, and contacts to ventures. But under what conditions does it make sense for very large and/or established firms to go private or remain private for the reason that the private equity governance structure better serves the investors?

The key takeaway is to view the choice between being public or private as depending upon relative strengths (e.g., the efficiency and effectiveness) of the private equity and public equity governance structures. The session, Private Equity Funds delves deeply into the role of private equity funds in generating advantages emanating from governance structures.


\end{document}