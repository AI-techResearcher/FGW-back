\documentclass[11pt]{article}
\usepackage[utf8]{inputenc}
\usepackage[T1]{fontenc}

\begin{document}
Buyouts of Private Companies

This lesson details buyouts of private companies by an entity that has a private ownership structure. Buyouts are distinguished from mergers by the extent to which the firm that is bought out is intended to function as a stand-alone business rather than to be folded into the organization of the purchaser. In a buyout of a private company, all of the equity is typically acquired and control is absolute. The terms for purchasing control of a company are not universal or clearly delineated. Since this session is on private equity, the focus is on buyout transactions that generate privately owned claims with equity or equity-like exposures. The most broad and generic term for these transactions is buyouts.

\section*{Buyout Objectives}
Investors in buyouts target lower internal rates of return (IRRs) than in venture capital, although both are quite high. The reason the VC-required IRRs are higher is simple: There is more risk funding a nascent company with brand-new technology than an established company with regular and predictable cash flows.

A buyout of a private company is an alternative exit from an IPO, from the perspective of an early investor such as a venture capitalist. Typically the target company has an established product. The management of the company going forward is driven not by idea generation but often by improving efficiency. Revenues are established, recurring, and fairly predictable. With a buyout, capital is necessary not for product or service development (as in VC) but to optimize the company's efficiencies with the likely ultimate goal of an IPO exit.

Ultimately buyouts must have a value-creating purpose. The next two sections discuss two important potential objectives.

\section*{Buyouts and Capital Structure Optimization}
Large capital requirements and lower risk levels relative to VC result in most buyout managers making a smaller number of investments compared to venture capitalists. Buyout transactions typically use both equity and debt financing to acquire companies. Assets of the acquired company are used as collateral for the debt, and the company's cash flow is used to pay off the debt. Buyout managers conduct intensive financial due diligence and occasionally rely on sophisticated financial engineering. Financial engineering, in this context, refers to the process of creating an optimal capital structure for a company.

In private equity, the capital structure is often made up of different types of financial instruments, such as multiple layers of debt, mezzanine, and equity, each carrying a different risk-reward profile. Buyouts typically use debt financing, either through bank loans or with newly issued debt to purchase the outstanding equity of the target company. Typically, these loans and bonds are secured by the underlying assets of the company being bought.

Mezzanine financing relates to capital provided through the issuance of subordinated debt, with warrants or conversion rights to finance the expansion or transition capital for established companies (usually privately held or below investment grade, or both). While mezzanine financing gives a more predictable cash-flow profile, it is unlikely to provide capital returns comparable to other PE financing forms.

\section*{Buyouts and Operational Efficiency Optimization}
A multitude of approaches to improving operational efficiency can be combined in a transaction, such as divestment of unrelated businesses, vertical or horizontal integration through acquisition, and company turnaround. Buyout managers need to give extensive advice on strategic and business planning, and they tend to focus on consistent rather than outsized returns. Because they target established enterprises, buyout firms experience fewer outright failures but have upside potential that is more limited.

The ability to analyze a company's operations and increase efficiencies, as opposed to the implementation of capital structure changes, is the primary driver of a successful transaction.


\end{document}