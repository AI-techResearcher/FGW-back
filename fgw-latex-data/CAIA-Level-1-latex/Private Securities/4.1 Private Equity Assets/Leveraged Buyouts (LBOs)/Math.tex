\documentclass[11pt]{article}
\usepackage[utf8]{inputenc}
\usepackage[T1]{fontenc}
\usepackage{amsmath}
\usepackage{amsfonts}
\usepackage{amssymb}
\usepackage[version=4]{mhchem}
\usepackage{stmaryrd}

\begin{document}
\section*{APPLICATION A}
Question : Consider a publicly traded firm that is viewed by a privateequity firm as a potential target, since it is failing to use its potential to generate earnings. The company has equity with a market value of \$500 million and debt witha face value of \$100 million. The company is currently generating earnings before interest, taxes, depreciation, and amortization (EBITDA) of \$80 million, whichrepresents the free cash flow from operations that is available for the owners and debtors of the company. This equates to a 13.3\% before-tax return on assets for thecompany’s shareholders and debt holders.\\
An LBO fund uses \$700 million to purchase the equity of the company and pay off the outstanding debt. The debt is paid off at a face value of \$100 million, while theremaining \$600 million is offered to the equity holders to entice them to tender their shares to the LBO fund (i.e., a 20\% premium is offered over the current marketvalue). The \$700 million LBO is financed by the LBO fund with \$600 million in debt at a 10\% coupon rate and \$100 million in equity. Thus, the company must pay \$60million in annual debt service to meet its interest payment obligations. (The debt load in this example is not realistic, but used for illustration only, as buyouts in 2017had average equity contributions of five times equity and six times debt for a debt-to-enterprise value ratio of 54.5\%.)\\
After the LBO, the management of the company improves As a result, assume that the cash flow from operations of the company improves from \$80million to \$120 million per year. By forgoing dividends and using the free cash flow to pay down the remaining debt, the LBO fund can own the target company free and clear of the debt used to finance the acquisition in about seven years. This means that after seven years and ignoring potential growth in cash flows, the LBO firmas the sole equity owner can claim the annual cash flow of \$120 million completely for itself. After the seven-year point, assume a forward-looking long-term growth rate of 2\% per year and a discount rate of 12\%.

Suppose that all other facts remain the same except that the discount rate used at the end of seven years is $15 \%$.

\section*{Answer and Explanation}
The projected value of the company becomes $\$ 120$ million/ $(0.15-0.02)=\$ 923$ million and the seven-year rate of return becomes $(\$ 923 \text { million } / \$ 100 \text { million })^{1 / 7}-1=37.4 \%$.\\
The key issues in this application are that the value of the company at the exit horizon is being changed by the change in the discount factor used in the valuation model. The new exit value generates a new rate of return:

The old valuation was:

$\$ 120$ million/(0.12-0.02) $=\$ 1.2$ billion

The new valuation is:

$\$ 120$ million/(0.15 - 0.02) $=\$ 0.923$ billion

The old rate of return was:

$(\$ 1.2 \text { billion } / \$ 100 \text { million })^{1 / 7}-1=42.6 \%$

The new rate of return is:

$(\$ 0.923 \text { billion } / \$ 100 \text { million })^{1 / 7}-1=37.4 \%$

\section*{APPLICATION B}
Question : Consider a publicly traded firm that is viewed by a privateequity firm as a potential target, since it is failing to use its potential to generate earnings. The company has equity with a market value of \$500 million and debt witha face value of \$100 million. The company is currently generating earnings before interest, taxes, depreciation, and amortization (EBITDA) of \$80 million, whichrepresents the free cash flow from operations that is available for the owners and debtors of the company. This equates to a 13.3\% before-tax return on assets for thecompany’s shareholders and debt holders.\\
An LBO fund uses \$700 million to purchase the equity of the company and pay off the outstanding debt. The debt is paid off at a face value of \$100 million, while theremaining \$600 million is offered to the equity holders to entice them to tender their shares to the LBO fund (i.e., a 20\% premium is offered over the current marketvalue). The \$700 million LBO is financed by the LBO fund with \$600 million in debt at a 10\% coupon rate and \$100 million in equity. Thus, the company must pay \$60million in annual debt service to meet its interest payment obligations. (The debt load in this example is not realistic, but used for illustration only, as buyouts in 2017had average equity contributions of five times equity and six times debt for a debt-to-enterprise value ratio of 54.5\%.)\\
After the LBO, the management of the company improves As a result, assume that the cash flow from operations of the company improves from \$80million to \$120 million per year. By forgoing dividends and using the free cash flow to pay down the remaining debt, the LBO fund can own the target company free and clear of the debt used to finance the acquisition in about seven years. This means that after seven years and ignoring potential growth in cash flows, the LBO firmas the sole equity owner can claim the annual cash flow of \$120 million completely for itself. After the seven-year point, assume a forward-looking long-term growth rate of 2\% per year and a discount rate of 12\%. Supposed that all other facts remain the same except that the growth rate used at the end of the seven years is $5 \%$.

\section*{Answer and Explanation}
The projected value of the company becomes $\$ 120$ million $/(0.12-0.05)=\$ 1.714$ billion, and the seven-year rate of return becomes $(\$ 1.714 \text { billion } / \$ 100 \mathrm{million})^{1 / 7}-1=$ $50.1 \%$.

The key issues in this application are that the value of the company at the exit horizon is being changed by the change in the growth rate used in the valuation model. The new exit value generates a new rate of return.

The old valuation was:

$\$ 120$ million/(0.12-0.02) $=\$ 1.2$ billion

The new valuation is:

$\$ 120$ million/(0.12-0.05) $=\$ 1.714$ billion

The old rate of return was:

$(\$ 1.2 \text { billion } / \$ 100 \text { million })^{1 / 7}-1=42.6 \%$

The new rate of return is:

$(\$ 1.714 \text { billion } / \$ 100 \text { million })^{1 / 7}-1=50.1 \%$

\section*{APPLICATION C}
Question : Consider a publicly traded firm that is viewed by a privateequity firm as a potential target, since it is failing to use its potential to generate earnings. The company has equity with a market value of \$500 million and debt witha face value of \$100 million. The company is currently generating earnings before interest, taxes, depreciation, and amortization (EBITDA) of \$80 million, whichrepresents the free cash flow from operations that is available for the owners and debtors of the company. This equates to a 13.3\% before-tax return on assets for thecompany’s shareholders and debt holders.\\
An LBO fund uses \$700 million to purchase the equity of the company and pay off the outstanding debt. The debt is paid off at a face value of \$100 million, while theremaining \$600 million is offered to the equity holders to entice them to tender their shares to the LBO fund (i.e., a 20\% premium is offered over the current marketvalue). The \$700 million LBO is financed by the LBO fund with \$600 million in debt at a 10\% coupon rate and \$100 million in equity. Thus, the company must pay \$60million in annual debt service to meet its interest payment obligations. (The debt load in this example is not realistic, but used for illustration only, as buyouts in 2017had average equity contributions of five times equity and six times debt for a debt-to-enterprise value ratio of 54.5\%.)\\
After the LBO, the management of the company improves As a result, assume that the cash flow from operations of the company improves from \$80million to \$120 million per year. By forgoing dividends and using the free cash flow to pay down the remaining debt, the LBO fund can own the target company free and clear of the debt used to finance the acquisition in about seven years. This means that after seven years and ignoring potential growth in cash flows, the LBO firmas the sole equity owner can claim the annual cash flow of \$120 million completely for itself. After the seven-year point, assume a forward-looking long-term growth rate of 2\% per year and a discount rate of 12\%., suppose that all other facts remain the same except that the investment requires eight years to exit.

\section*{Answer and explanation}
The projected value of the company becomes $\$ 120$ million / $(0.12-0.02)=\$ 1.2$ billion, and the eight-year rate of return becomes $(\$ 1.2 \text { billion } / \$ 100 \text { million })^{1 / 8}-1=36.4 \%$.\\
The key issues in this application is that the value of the company at the exit horizon is the same, but the time interval changes (from 7 years to 8 years) and the rate of return therefore changes:

The old valuation is still:

$\$ 120$ million/(0.12-0.02) $=\$ 1.2$ billion

The new rate of return is:

$(\$ 1.714 \text { billion } / \$ 100 \text { million })^{1 / 8}-1=36.4 \%$

\section*{APPLICATION D}
Question : Consider a publicly traded firm that is viewed by a privateequity firm as a potential target, since it is failing to use its potential to generate earnings. The company has equity with a market value of \$500 million and debt witha face value of \$100 million. The company is currently generating earnings before interest, taxes, depreciation, and amortization (EBITDA) of \$80 million, whichrepresents the free cash flow from operations that is available for the owners and debtors of the company. This equates to a 13.3\% before-tax return on assets for thecompany’s shareholders and debt holders.\\
An LBO fund uses \$700 million to purchase the equity of the company and pay off the outstanding debt. The debt is paid off at a face value of \$100 million, while theremaining \$600 million is offered to the equity holders to entice them to tender their shares to the LBO fund (i.e., a 20\% premium is offered over the current marketvalue). The \$700 million LBO is financed by the LBO fund with \$600 million in debt at a 10\% coupon rate and \$100 million in equity. Thus, the company must pay \$60million in annual debt service to meet its interest payment obligations. (The debt load in this example is not realistic, but used for illustration only, as buyouts in 2017had average equity contributions of five times equity and six times debt for a debt-to-enterprise value ratio of 54.5\%.)\\
After the LBO, the management of the company improves As a result, assume that the cash flow from operations of the company improves from \$80million to \$120 million per year. By forgoing dividends and using the free cash flow to pay down the remaining debt, the LBO fund can own the target company free and clear of the debt used to finance the acquisition in about seven years. This means that after seven years and ignoring potential growth in cash flows, the LBO firmas the sole equity owner can claim the annual cash flow of \$120 million completely for itself. After the seven-year point, assume a forward-looking long-term growth rate of 2\% per year and a discount rate of 12\%., suppose that all other facts remain the same except that the $\$ 120$ million cash flow estimate given 7 cashflow that anticipated to grow by year 8 .

\section*{Answer and explanation}
The projected value of the company becomes $\$ 122.4$ million / $(0.12-0.02)=\$ 1.224$ billion, and the seven-year rate of return becomes $(1.224$ billion $/ 100$ million) ${ }^{1 / 7}-1=43.0 \%$\\
The key issue is that the cash flow being earned at exit is higher. The application attempts to emphasize that when using the perpetual growth model the proper cash flow to use in the numerator is the cash flow at the end of the year following the valuation date.

The old valuation was:

$\$ 120$ million/(0.12 - 0.02) $=\$ 1.2$ billion

The new valuation is:

$\$ 120$ million * $(1.02) /(0.12-0.02)=\$ 1.224$ billion

The old rate of return was:

$(\$ 1.2 \text { billion } \$ \$ 100 \text { million })^{1 / 7}-1=42.6 \%$

The new rate of return is:

$(\$ 1.224 \text { billion/ } \$ 100 \text { million })^{1 / 7}-1=43.0 \%$


\end{document}