\documentclass[11pt]{article}
\usepackage[utf8]{inputenc}
\usepackage[T1]{fontenc}
\usepackage{amsmath}
\usepackage{amsfonts}
\usepackage{amssymb}
\usepackage[version=4]{mhchem}
\usepackage{stmaryrd}
\usepackage{hyperref}
\hypersetup{colorlinks=true, linkcolor=blue, filecolor=magenta, urlcolor=cyan,}
\urlstyle{same}

\begin{document}
Growth Equity

There is no clear demarcation between financing for the later stages of venture capital and growth equity. The later stages of venture capital are often described as the expansion stage (second or third stage), borrowing stage, late stage, mezzanine stage, and so forth. These stages occur after the VC firm has established the technology and market for its new product or services. The stages can overlap with growth equity.

Growth equity (also called growth capital or expansion capital) covers investment in firms after the VC stage and prior to a buyout, such as becoming publicly traded (exited by IPO). Firms seeking financing in the form of growth equity have been successful in establishing a strong base of revenues, as opposed to VC companies that are striving to be recognized as a leading provider of a good or service. Growth equity financing is the final financing prior to exit via an IPO.

A primary demarcation between a project being in the later part of VC or in the growth equity stage involves control: If the financing includes a "team approach" to managing the nascent enterprise, it is likely late-stage/expansion venture capital. To the extent that the financing comes with limited ongoing managerial involvement by the investor, then it is more like growth equity than VC. Growth equity is used to tap into distribution channels, establish call centers, expand manufacturing facilities, and attract the additional management and operational talent necessary to ramp up operations and transform the company into a longerterm success.

\section*{Describing Growth Equity Investments}
In the Overview of Three Forms of Pre-IPO Private Equity Investing lesson, the exhibit, Major Distinctions between VC, Growth Equity, and Buyouts summarizes and compares key characteristics of growth equity alongside venture capital and buyouts. The exhibit provides approximate ranges for quantitative characteristics such as asset size, annual revenue, potential upside ratios, and target IRRs. This section focuses on qualitative characteristics.

Securities: Growth equity securities are newly originated securities that have a minority position in terms of control but a relatively high position in terms of liquidation priority, such as convertible preferred equities or debt.

Source of Growth: Large increases in the value of a VC company come from increased probability of the successful launch of a product or service. Increases in the value of growth equity primarily come from growth in the firm's revenues and profitability. That growth is enabled or accelerated by financing from growth equity that enables enhanced production and injects working capital to meet the expanded capital requirements of increased production costs and increased accounts receivable.

Cashflow: VC companies have little or no net cash flow. Growth equity companies have little or no interest payments (little or no debt), moderate cash inflow from revenues, but little or no free cash flow.

\section*{Protective Provisions as a Key Deal Characteristic}
Matt Stewart (2012) describes the importance of protective provisions in growth equity investing. ${ }^{1}$ Matt Stewart, "Growth Equity: The Intersection of Venture Capital and Control Buyouts," PE Hub Network, November 9, 2012, \href{http://www.pehub.com/2012/11/growth-equity-the-intersection-venture-capital-control-buyouts/}{http://www.pehub.com/2012/11/growth-equity-the-intersection-venture-capital-control-buyouts/}. Protective provisions in growth equity provide operational control such as investor consent rights on key transactions, with key growth transactions including changes in capital structure, major assets, tax or accounting policies, key employees, and significant operational activities. Also, growth equity investors should require notification and consent on substantial deviations from the budget or changes in the business plan. These provisions help facilitate the monitoring and, if necessary, the control of the enterprise by its growth equity investors, which lies between total control and no control.

\section*{Redemption Rights as a Key Deal Characteristic}
Stewart describes the importance of redemption rights and details their characteristics. ${ }^{2}$ Ibid. Redemption rights grant powers to investors to redeem their position in the company by specifying the triggers and actions that demark the remedies available to the investors.

Four Principal Considerations in Redemption Rights: Four principal considerations in redemption rights are (1) redemption triggers, (2) redemption value, (3) sources of funds, and (4) remedies for defaulted redemption.

Redemption Triggers: The three common redemption triggers are time, performance, and violations of covenants. (1) Time: the time-based trigger for redemption trigger is typically 60-66 months after the original issuance date; (2) performance: performance-based triggers are milestones typically benchmarked to revenue or profitability, and may be tested multiple times during the life of the investment; and (3) covenant-based: covenant-based triggers are similar to default triggers in credit and are typically based on failure to satisfy financial and other covenants.

Redemption Value: If redemption is triggered, the investor has a right to the prenegotiated redemption value. A growth equity redemption value is typically set as the maximum of one of the following or the maximum of two or more of the following: (1) the original issuance price plus a preferred return, (2) a multiple of the original issuance price, and (3) the fair market value of the equity interest.

Sources of Funds: Growth equity investors often obtain contractual requirements of the efforts that the issuer must use to meet requested redemptions. Growth equity redemption sources can be required to include: (1) all "legally available funds," (2) undertaking a "forced sale" or other capital raising transaction, (3) issuing a promissory note for the redemption value, and/or (4) using all other available means in order to effect a required redemption. Default remedies: Growth equity contracts may include default remedies. Growth equity default remedies include springing board remedies and forced sales. A springing board remedy occurs when the investor designates a majority of the defaulting issuer's board of directors. A forced sale remedy occurs when an investor compels a liquidating transaction, such as sale of the entire company or other transactions, to generate cash to meet the redemption obligation.

\section*{The Valuation of Growth Equity Based on Revenue}
The valuation of growth equity is similar to the valuation of venture capital in the sense that most of the value emanates from the uncertain potential for large growth and a favorable exit. As in the case of venture capital, sophisticated techniques for valuation, such as discounted annual cash flow analysis, may not be viable given a high level of uncertainty. This valuation example focuses on a company within a service industry. A popular method of modeling the value of a service\\
company such as an asset manager or consulting firm is based on a multiple of revenues. The times revenue method values an enterprise as the product of its projected annual revenue and a multiple derived from analysis of the value of similar firms.

An enterprise-value-to-revenue (EV/R) multiple value may be estimated based on observation of publicly traded firms or previous deals. The EV/R multiple is often adjusted to reflect deal-specific attributes such as anticipated levels of operating costs, compared to industry norms, and anticipated levels of cash or other assets.

Equation 1 discounts the estimate of the enterprise value on exit.


\begin{equation*}
\text { Value of Venture }=(\text { Annual Revenue } \times \text { Revenue Multiple }) /(1+\mathrm{IRR})^{T} \tag{1}
\end{equation*}


where annual revenue is the estimated sales revenues in $T$ years and revenue multiple is the EV/R anticipated at time $T$ (the exit time).

Consider a growth equity investor requiring a 40\% IRR on a growth equity investment in a firm that has substantial uncertainty with regard to its likely growth. The investor projects potential annual revenues of $\$ 50$ million in five years (if successful), at which point an IPO is anticipated. Analysis of similar firms that are publicly traded indicates a typical ratio of enterprise value to annual revenues (EV/R) of 2.0. Inserting these values into Equation 1 generates:

$$
\text { Value of Venture }=(\$ 50 \text { million } \times 2.0) /\left(1.40^{5}\right)=\$ 18.59 \text { million }
$$

Taking into consideration the percentage of the firm that the investor will own, the growth equity investor must decide whether the $\$ 18.59$ million discounted enterprise value for the entire firm warrants any proposed investment in growth equity for a share of the firm.

Note from inspection of Equation 1 that the model focuses on three especially crucial determinants: (1) the potential revenue of the mature firm, (2) the time that it is expected to get to maturity and exit ( $T$ ), and (3) the level of required rate of return necessary, taking into account the substantial uncertainties of reaching the revenue potential.


\end{document}