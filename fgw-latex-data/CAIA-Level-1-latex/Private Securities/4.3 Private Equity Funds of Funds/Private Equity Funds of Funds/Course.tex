\documentclass[11pt]{article}
\usepackage[utf8]{inputenc}
\usepackage[T1]{fontenc}
\usepackage{amsmath}
\usepackage{amsfonts}
\usepackage{amssymb}
\usepackage[version=4]{mhchem}
\usepackage{stmaryrd}

\begin{document}
Private Equity Funds of Funds

This lesson discusses PE funds of funds: private fund structures with underlying assets composed of private PE funds.

\section*{Private Equity Funds of Funds and Fees}
Private equity funds of funds (PE FoFs) are investment vehicles, similar in legal structure to a typical PE fund, which make investments as limited partners in numerous private equity funds, as opposed to making direct investments into portfolio companies. PE FoFs are often seen as less efficient than single GP PE funds (i.e., direct fund investments) because of the additional layer of management fees and incentive fees. This double layer of fees is perceived to be one of the main disadvantages of this structure. However, given the resources required for an institutional investor to select, monitor, and manage a portfolio of PE funds internally, investing through a fund-of-funds structure might well prove more cost-efficient than a direct fund investing approach.

The cost of outsourcing investment management, including the carried interest of the funds of funds, needs to be compared to performance-related incentives, if any, paid in a program managed internally. Whether an in-house program can work effectively without investment-performance-related incentives is debatable.

According to Otterlei and Barrington (2003), the annual costs of an in-house team can be substantial compared to that of a typical fund of funds. Even with a 5\% carried interest charged by the fund-of-funds manager, these authors find that the fees have an insignificant impact on the net returns of the investor.

\section*{Private Equity Funds of Funds and the Value of Information and Control}
Information can be a valuable asset in an opaque environment such as PE. Funds of funds can provide the necessary resources and address the information gap for inexperienced PE investors through their expertise in due diligence, monitoring, and restructuring. Successful investing in PE funds requires: (1) a wide-reaching network of contacts in order to gain access to high-quality funds, (2) well-trained investment judgment, and (3) the ability to assemble balanced portfolios.

An institutional investor taking the fund-of-funds route (as opposed to taking a direct investment route) can lose access to information and control, essentially a cost in itself. Because PE programs follow a learning curve, inexperienced institutions may initially have little option other than to go through a fund-of-funds vehicle. Experience with the funds-of-funds approach may allow institutional investors new to PE to build knowledge and sophistication, eventually leading to building their own portfolios of funds.

Funds of funds can therefore be used as a first step into PE and may be worth the additional layer of fees in exchange for avoiding expensive learning-curve mistakes and for providing access to a broader selection of funds.

\section*{Private Equity Funds of Funds, Diversification, and Intermediation}
PE FoFs can add value in several respects, especially in the case of investments in new technologies, new teams, or emerging markets. A fund of funds approach allows for reasonable downside protection through greater diversification than many investors can achieve through a direct program. It is important to diversify a PE investment program across vintage years, GPs, industry, stage of investment, and geography. Not surprisingly, various studies have shown that diversification in PE is important. A direct investment has a $30 \%$ probability of total loss. A fund has a very small probability of total loss. A fund of funds has a small probability of any loss. ${ }^{1}$ Tom Weidig and Pierre-Yves Mathonet, The Risk Profiles of Private Equity (Brussels: EVCA, 2004); Pierre-Yves Mathonet and Thomas Meyer, J-Curve Exposure: Managing a Portfolio of Venture Capital and Private Equity Funds (Chichester, UK: John Wiley \& Sons, 2007) For larger institutions, investments in PE funds and especially VC funds may be too cost-intensive when the size of such investments is small compared to administrative expenses. A fund of funds can mediate these potential size issues by either scaling up through pooling of commitments of smaller investors and providing each of them with sufficient diversification, or scaling down through sharing administrative expenses and making such investments less cost-intensive by allowing larger commitment to the fund of funds.

\section*{Private Equity Funds of Funds, Access, Selection Skills, and Expertise}
Some limited partners, especially those new to private equity, may not be able to access the most popular and successful PE funds. Depending on the overall market situation, access to quality funds can be highly competitive, and being a newcomer to the market can pose a significant barrier. Funds of funds are continuously involved in the PE space, speak the language, and understand the trade-offs in the industry. PE FoFs may offer those investors the advantage of being able to invest in top-performing funds, either by having access to successful invitation-only funds or by identifying the future stars among the young and lesser-known funds. PE FoFs may offer better access to top fund managers because those top managers may welcome funds-of-funds investors as a more stable and experienced source of pooled capital.

Liquidity management can also be quite challenging, because it demands a full-time team with insight and an industry network, adequate resources, and access to research databases and models, as well as skills and experience in due diligence, negotiation, and contract structuring. The key points in this section are summarized in the following exhibit.

Why consider PE FoFs over investing in PE funds directly? The response may depend on the type of investor and four key considerations.

Comparing Private Equity Direct Investments vs. Funds of Funds

\begin{center}
\begin{tabular}{|l|l|l|}
\hline
 & Direct Investment in PE Funds & Fund of Funds (FoFs) \\
\hline
Diversification & \begin{tabular}{l}
Requires a large capital commitment to build a \\
diversified portfolio of PE funds \\
\end{tabular} & \begin{tabular}{l}
Pooled capital from multiple investors provide scale to diversify across managers, vintage \\
years, geographies, strategies, and sectors. \\
\end{tabular} \\
\hline
Cost & \begin{tabular}{l}
Cost of internal staff and any external \\
consultants to help build a portfolio of PE funds, \\
manage the cashflow, and the contracts. \\
\end{tabular} & \begin{tabular}{l}
FoFs charge management fees and carried interest in addition to the fees paid to \\
underlying PE funds. The scale of FoFs may help to negotiate lower fees in the underlying \\
PE funds. In addition, FoF may have the ability to earn fee discounts as an anchor investor. \\
\end{tabular} \\
\hline
\end{tabular}
\end{center}

\begin{center}
\begin{tabular}{|l|l|l|}
\hline
 & Direct Investment in PE Funds & Fund of Funds (FoFs) \\
\hline
\begin{tabular}{l}
Selection and \\
monitoring skill \\
\end{tabular} & \begin{tabular}{l}
Investors will need expertise to conduct high- \\
quality investment and operational due \\
diligence, to build and manage a portfolio of PE \\
funds. \\
\end{tabular} & \begin{tabular}{l}
Experienced FoF managers may serve on fund advisory committees of the underlying PE \\
funds. On these committees, they can participate in decision on potential conflicts of \\
interest transactions and waiver of certain restrictions or thresholds set forth in the fund's \\
legal documents. \\
\end{tabular} \\
\hline
Access & \begin{tabular}{l}
Requires the reputation and the capital to access \\
top-performing GPs. \\
\end{tabular} & \begin{tabular}{l}
Some FoFs may have access to higher-quality GPs with oversubscribed funds through their \\
earlier investments with the GP or their reputation as a valuable LP. \\
\end{tabular} \\
\hline
\end{tabular}
\end{center}

Source: CAIA Association


\end{document}