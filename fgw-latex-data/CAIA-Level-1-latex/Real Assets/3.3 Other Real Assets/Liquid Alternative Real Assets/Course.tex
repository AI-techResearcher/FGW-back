\documentclass[11pt]{article}
\usepackage[utf8]{inputenc}
\usepackage[T1]{fontenc}
\usepackage{amsmath}
\usepackage{amsfonts}
\usepackage{amssymb}
\usepackage[version=4]{mhchem}
\usepackage{stmaryrd}

\DeclareUnicodeCharacter{2021}{$\ddagger$}

\begin{document}
Liquid Alternative Real Assets

One of the largest and most rapidly growing alternative investment areas in the United States has been the use of master limited partnerships (MLPs) to provide liquid investment access to operationally intensive real assets.

\section*{Structure of MLPs and the MLP Sector}
MLPs are simply limited partnerships in which the limited partnership ownership units are listed (publicly traded). Limited partners of MLPs are unit holders. MLPs receive tax treatment predicated on adhering to regulations, including that at least $90 \%$ of the entities' revenues come from specified businesses, such as energy.

Although MLPs have existed in the United States since 1981, they have thrived more recently. There are now more than 100 MLPs in the United States with an aggregate market value of over roughly $\$ 500$ billion.

MLPs are typically traded on major exchanges, such as the NYSE, in the same manner as are corporate operating firms. MLPs are not shares in the equity of taxable corporations; they are limited partnership units representing direct ownership of a firm. Many publicly traded securities are described as investments in natural resources, but a closer look indicates that the investment is subjected to substantial development, extraction, and processing operational risks.

Most MLPs are involved in the energy sector, although some MLPs invest in real estate, timber, or other assets as permitted by regulations. The oil and gas sector is divided into upstream, midstream, and downstream operations. Upstream operations focus on exploration and production; midstream operations focus on storing and transporting the oil and gas; and downstream operations focus on refining, distributing, and marketing the oil and gas. Midstream operations and midstream MLPs-the largest of the three segments-process, store, and transport energy and tend to have little or no commodity price risk. For example, a gas pipeline is paid a transportation fee for the quantity of oil or gas transported without regard for the value of the product being transported. Similar to infrastructure investments, midstream MLPs have been called a toll road for energy.

\section*{Tax Characteristics of MLPs}
MLPs have an ownership structure distinct from most traditional investments. The next exhibit, Summary of Three Forms of Ownership highlights three major types of major U.S. business entities: taxable corporations (C corporations), tax-exempt corporations (investment companies), and limited partnerships.

The exhibit below highlights the critical issue of how income is taxed. Investors in the equity of traditional operating corporations in the United States experience double taxation. Double taxation is the application of income taxes twice: taxation of profits at the corporate income tax level and taxation of distributions at the individual income tax level. Most investment companies in the United States, including mutual funds and REITs, can avoid paying corporate income taxes if they distribute almost all of their profits to the corporation's shareholders-a practice generally followed. The distributions are taxed at the individual income tax level.

Summary of Three Forms of Ownership

\begin{center}
\begin{tabular}{|lll|}
\hline
 & Subject to U.S. Corporate & Distributions Subject to \\
 & Income Tax? & U.S. Individual Income Tax? \\
\hline
C corporation & Yes & Yes \\
Investment company & $\mathrm{No}^{*}$ & Yes \\
Limited partnership & $\mathrm{No}^{\dagger}$ & $\mathrm{No}^{\ddagger}$ \\
\hline
\end{tabular}
\end{center}

*Investment companies distributing almost all income to shareholders are not taxed at the corporate level. Examples include mutual funds.

TThe revenues and expenses of limited partnerships pass through the partnership directly into the tax forms of the partners.

‡Investors in limited partnerships are subject to taxes on net income, whether or not that income was distributed.

Limited partnerships in general and MLPs in particular are not directly subject to income taxes at the partnership level. The revenues, expenses, and profits of the partnerships flow directly through the partnerships and into the tax forms of the partners. The limited partners are subject to tax on profits that flow from the partnership, whether or not the profits are distributed to them. Thus, the Summary of Three Forms of Ownership exhibit indicates that partnership distributions, per se, are not taxed at the individual level.

Energy development enterprises in the United States tend to have opportunities to enjoy substantial tax benefits, including credits and accelerated expensing. MLP structures allow tax benefits to pass through the firm level directly to the tax forms of the limited partners. These benefits manifest themselves in the ability of limited investors to enjoy large tax-free distributions, because it is income that is taxed, not distributions. Many of the large distributions from MLPs are sheltered in the short run as return of capital due to generous rules regarding the expensing of costs. Return of capital distributions are tax-free when received. Distributions that represent return of capital serve to lower the tax basis of the MLP investment to the investor. Upon the sale of the MLP, the lowered tax basis tends to cause more of the sales proceeds to be taxable. The recaptured gains attributable to the distributions tend to be taxed at full rates rather than preferred capital gain rates. Thus, the tax-free distributions of MLPs are likely to serve as tax deferrals.

The potential tax benefits of MLPs to U.S. investors need to be weighed against three potential drawbacks. First, MLPs report income on K-1 forms rather than 1099s, which may add substantial complexities and delays to federal tax filing. Second, MLP income is usually subject to income taxation in the states in which the MLPS operate, which means that limited partners with moderate to large holdings may be required to file numerous state income tax returns. Finally, MLPS can cause unrelated business income tax (UBIT) liability for some pension plans and not-for-profit corporations in the United States, as the income generated through businesses unrelated to the tax-exempt purpose of the organization can be taxable. That is, although the income generated from educational activities is tax-exempt for a university, some investment activities may be taxable if determined to be outside of the charitable scope of the organization.

\section*{MLP Valuations and Distribution Rates}
There has been controversy regarding the prospective risks and returns of the MLP sector in general and MLPs with high distribution rates in particular. As noted earlier, MLP investors are taxed on income, not distributions, from MLPs. The MLP structures themselves are not required to pay income taxes. Whereas mutual funds and REITs generally set distributions to be approximately equal to their income, MLPs are free to make distributions as high as their cash flows allow. As previously noted, these distributions are tax-free and are attractive to investors focused on cash income.

Some MLPs are alleged to make distributions at rates that are not sustainable based on the MLP's current and prospective income. It is further alleged that the market prices of these MLPs are inflated by high demand from brokers and investors who are drawn to the high tax-free distributions and who overestimate the sustainability of the distribution rates.

Proponents of the high valuations of MLPs with high distribution rates argue that the high rates are reasonable and sustainable due to the highly profitable transactions and operations underlying the MLPs. New acquisitions are often financed by issuing new partnership units at high market valuations, which enables attractive future cash flow projections. Exceptional prospects for success can be captured as the present value of exceptional growth opportunities.

Analysts who say that MLPs are overpriced often argue that the high cash flows are being driven by proceeds from the secondary offerings of the MLP units and that eventually the distributions will have to be cut when new financings and acquisitions end. Can the proceeds from secondary offerings buy and develop new capacity that will lead to growing cash flows that can support higher levels of cash distributions? While required by law to be factually correct, prospectuses often indicate intricate and complicated arrangements between affiliated entities that make analysis exceedingly complex relative to many traditional investments.

Perhaps one thing is certain: MLP investing, like many other forms of alternative investing, is skill based. Even a broad indexation strategy in which an MLP ETF, an MLP exchange-traded note, or a representative basket of individual MLPs is purchased requires the skillful evaluation of whether the entire MLP sector is fairly valued.


\end{document}