\documentclass[11pt]{article}
\usepackage[utf8]{inputenc}
\usepackage[T1]{fontenc}
\usepackage{amsmath}
\usepackage{amsfonts}
\usepackage{amssymb}
\usepackage[version=4]{mhchem}
\usepackage{stmaryrd}
\usepackage{hyperref}
\hypersetup{colorlinks=true, linkcolor=blue, filecolor=magenta, urlcolor=cyan,}
\urlstyle{same}

\begin{document}
Cash Flows of Intellectual Property

Film production and distribution comprise a subset of IP that often has relatively substantial accounting data availability and thus provides a good example of the methods for estimating and modeling expected future cash flows and accounting profitability.

\section*{Film Production and Distribution Revenues}
Film production and distribution fall into the IP category of artwork. Total global box office revenues were estimated to be over $\$ 38$ billion in 2016 . Film revenues are generated almost exclusively by exhibition, which has a generally stable set and sequence of stages, though not all films will be licensed for exhibition in all forms (See next exhibit, Schedule of Film Exhibition Venues). The revenues from home media and broadcast are larger than global box office revenue, but include programming made both as films and directly for television.

\begin{center}
\begin{tabular}{|lcc|}
\hline
\multicolumn{3}{c}{Schedule of Film Exhibition Venues} \\
\hline
Exhibition Form & Window & Time after Release \\
\hline
Theatrical & 6 months & 0 \\
Home Video & 10 years + & 4 months \\
Pay-per-View & 2 months & 8 months \\
Pay TV & 18 months & 12 months \\
Network & 30 months & 30 months \\
Pay TV Second Window & 12 months & 60 months \\
Basic Cable & 60 months & 72 months \\
Television Syndication & 60 months & 132 months \\
\hline
\end{tabular}
\end{center}

Exhibition forms include theatrical, cable-based and web-based home video, television networks, and DVDs. The expected size of the revenues, the starting time of the revenue streams, and the projected length of the revenue streams are aggregated to project the total revenues through time. While total revenues from film have demonstrated relative stability, the mix of revenue sources has been changing relatively quickly, due in part to technology but also to other financial imperatives, such as the availability (or lack) of capital for new film production. Examples of changing revenue sources include the rise and subsequent relative decline in revenues associated with DVD and similar exhibition technologies, as well as the increasing importance of online and non-U.S. revenues to overall revenue.

Translating revenue numbers into profits is typically impossible without direct knowledge of and participation in the production of particular film assets. However, there are regularities that arise in contracting that can be exploited to conduct an analysis and forecast cash flows. For example, empirical evidence indicates that sequels tend to generate more revenue at lower risk, and different film genres have different risk-return properties. To maximize the potential returns and portfolio benefits from investments in IP, investors should develop or retain analysts with expertise in the underlying assets of the IP.

\section*{Film Production and Distribution Expenses}
Film production itself has several stages. First, the costs of producing the film are collectively called negative costs. Negative costs refer not to the sign of the values but to the fact that these are costs required to produce what was, in the predigital era, the film's negative image. These costs include story rights acquisition; preproduction (script development, set design, casting, crew selection, costume design, location scouting); principal photography and production (compensation of actors, producers, directors, writers, sound stage, wardrobe, set construction); and postproduction (film editing, scoring, titles and credits, dubbing, special effects). These costs are coupled with the substantial cost of prints and advertising, which is the cost of the film prints to be used in theaters, whether digital or physical, and a film's advertising and marketing costs.

\section*{Film Financing}
Financing is achieved through equity or debt financing, or a combination of both. Equity financing structures include slate equity financing, corporate equity, coproduction, and miscellaneous third-party equity financing:

\begin{itemize}
  \item Slate equity financing. In slate equity financing, an outside investor (e.g., hedge fund or investment bank) funds a set of films to be produced by a studio. These slates typically reflect a set of parameters regarding diversification, risk, the number of films to be released, minimum and maximum budgets for film production and P\&A (prints and advertising), and genre diversification requirements. Slate deals emerged to spread financial risk across a series of films, thus limiting the impact of one film's losses on an overall financial investment. Slate financings may have further provisions to ensure against moral hazards; such provisions would, for example, deter a studio from assigning films with lower expected returns or greater risk to slate financings.
  \item Corporate equity: This is equity fund-raising (private placement or public offering) to fund the activities of a production company.
  \item Coproduction: In coproduction, two or more studios partner on a film, sharing the equity costs and, correspondingly, the risks and returns.
  \item Miscellaneous third-party equity. Some combination of high-net-worth individuals, institutional investors, and other third-party investors fund costs not covered by other types of financing; this is particularly common for smaller independent films.
\end{itemize}

Debt financing structures include senior secured debt, gap financing, and super gap financing/junior debt.

\begin{itemize}
  \item Senior secured debt. A bank or another financial institution lends funds to a movie studio or producer to finance the production and/or P\&A of a film. This loan can come in various structures and forms, backed by specific collateral, such as the following:
  \item Negative pickup deal. A negative pickup deal occurs when a film distributor agrees to purchase a film from a producer for a fixed sum upon delivery of the completed film.
  \item Foreign presales. A foreign presale occurs before the film is made, when the producer sells distribution rights for specific foreign territories for a fixed price; all, or nearly all, of this payment is due upon delivery of the completed film.
  \item Tax credits/grants. The producer receives tax credits (which are salable) or grants (paid in cash) for filming in a specific state or country.
  \item Gap financing. Gap financing covers the difference between the production budget and the senior secured debt, which can be collateralized by sales of unsold territories to distributors.
  \item Super gap financing/junior debt. Super gap financing is a second level of gap financing, often syndicated, representing the final gap that the senior lender or gap financier does not want to risk.
\end{itemize}

Further, financing may be supported directly or indirectly with royalty participations. These may, in the case of talent, be in lieu of salary or other noncontingent compensation; or, in the case of financial investors, they may be used to lower the cost of up-front financing. These participations are usually assignable (i.e., transferable to third parties) after a film has been produced.

\section*{Film Profitability}
Films offer profit potential with a right skew. The ratio of estimated worldwide revenues to investment can be $1,000 \%$ or more. According to "The Numbers,"1 \href{https://www.the-numbers.com/movie/Mars-Needs-Moms#tab=summary}{https://www.the-numbers.com/movie/Mars-Needs-Moms\#tab=summary} there have been 17 movies with profits of over $\$ 500$ million, with Avatar earning well over $\$ 1$ billion. Only eight movies generated losses in excess of $\$ 100$ million, with Mars Needs Moms having the largest estimated loss of $\$ 143.5$ million.

Wood Creek Capital Management. 2011. "Film Industry Overview."


\end{document}