\documentclass[11pt]{article}
\usepackage[utf8]{inputenc}
\usepackage[T1]{fontenc}
\usepackage{amsmath}
\usepackage{amsfonts}
\usepackage{amssymb}
\usepackage[version=4]{mhchem}
\usepackage{stmaryrd}

\begin{document}
R\&D and Patents as Unbundled Intellectual Property

Research and development (R\&D) and patents provide important insights into intellectual property (IP) in the context of unbundled IP and in the establishment and preservation of property rights for intangible assets. Unlike tangible assets, for which property rights are typically indicated by possession and usually clearly established, IP often raises challenges regarding its potential nonexcludability. Although this section focuses on R\&D and patents, much of the discussion is applicable to other unbundled IP, such as royalties on music, books, and other copyrights.

\section*{Accessing R\&D through Patents}
Investors have historically accessed the returns to R\&D through private or public equity investments in operating entities. However, to the extent that patents or other protected IP represent the crystallization of prior R\&D, ownership of patents may represent a mechanism for accessing the benefits of R\&D without bearing the operational risk associated with broader investments in companies that own such IP. Investments in patents can take multiple forms, such as direct acquisition or indirect acquisition through firms or funds that specialize in the acquisition and monetization of IP.

Five key strategies for acquisition of and exit from (monetizing) patent-related IP are:

\begin{enumerate}
  \item Acquisition and licensing

  \item Enforcement and litigation

  \item Sale license-back

  \item Lending strategies

  \item Sales and pooling

\end{enumerate}

\section*{Patent Acquisition and Licensing Strategies}
Acquisition and licensing strategies are generally built around agreements regarding royalty streams. Examples of key terms between the licensor/grantor and the licensee include:

\begin{itemize}
  \item Minimum royalty provision: If the royalties do not hit the contracted amount within a specified commercialization period, the licensor may either terminate the license or make the license nonexclusive.
  \item Field-of-use provision: A licensor may grant an exclusive license for a geographical region or a particular market.
  \item Reservation of rights provision: The grantor may make use of the patent, most often for noncommercial research uses.
  \item Improvement provisions: These are provisions dealing with improvements to the patent whereby a more efficient method is created (but the new method would arguably infringe on the claims of the patent); improvements are a difficult part of the license negotiations, because either the licensor or the licensee may be the originator of the improvement.
  \item Audit/reporting/payment due date obligations: Licensors may want to monitor the licensee's royalty payments.
  \item Exclusivity responsibilities: Generally, the licensor has (sometimes limited) duties to enforce exclusivity, whereas the licensee has to report infringement cases to the licensor. This varies a great deal from license to license.
\end{itemize}

In general, license rates are typically specified as a function of revenues associated with products built on the licensed technology.

\section*{Patent Enforcement and Litigation Strategies}
Ownership of patents may require patent enforcement and litigation to protect the value of the IP, meaning that the owner of the IP monitors the use of the patent and takes legal action against those who make uncompensated or unauthorized use of the patent. In fact, an IP investment strategy can be to acquire patents or other protected intellectual property that the potential purchaser believes is being infringed on in the marketplace. This strategy has received increasing scrutiny and public debate, as "non-practicing" holders of patents seek to monetize their intellectual property portfolios.

Typically, an investor who believes that his patent is the subject of infringement will approach users of the technology and seek to negotiate a license agreement with them. This is usually far more cost-effective than litigation. However, should agreement and licensing not be achieved, the owner of the patent may seek litigation against the infringers.

While subject to risks and requiring substantial expertise, in addition to the time and costs of the litigation, actual patent litigation tends to proceed in a relatively orderly fashion, with most patent cases being resolved through settlement. For example, Janicke (2007) finds that most patent litigation ( 80\%) is resolved through settlement rather than trial. Evidence indicates that settlement rates have been relatively stable through time.

The difficulty with settlements, however, is that their terms are not generally reported, so it is difficult to evaluate from public data the extent to which settlementbased outcomes generate sufficient risk-adjusted returns. However, although these outcomes are difficult to evaluate, it is known that cases resolved through trial generate median awards of $\$ 10$ million-a figure generally confirmed by Mazzeo, Hillel, and Zyontz (2013).

In evaluating the returns to litigation, a key factor is the amount of time it takes to resolve a case, in part because length of time is positively correlated with costs; it takes longer to redeploy capital in new cases as old cases drag on; and, of course, there is the time value of money. The timing of resolutions can be summarized with the following stylized facts:

\begin{itemize}
  \item Defaults have the shortest time to resolution. Summary judgments range from 5 months to 35 months in duration.
  \item Trials generally take between 35 and 50 months to resolve.
  \item Late dispositions take the most time to resolve: upwards of 50 months.
\end{itemize}

\section*{Patent Sale License-Back Strategies}
In a strategy that parallels the sale leaseback transactions of the corporate and real estate worlds, the patent sale license-back (SLB) strategy is in use when the patent holder sells one or more patents to a buyer, who then licenses those patents back to the original holder. In doing so, a patent seller is benefiting from the ability to monetize a portion of the intangible assets. The patent buyer then places the patent in a pool of similar technologies for out-licensing to other parties. Often, the patent buyer will participate in the licensing revenue from new licensees. By allowing the patent to be pooled with other patents, the patent owner can benefit from revenue participation generated from the potential synergies of the pooled patents.

There is also a potentially substantial tax benefit if a company lends a patent to an IP holding company in a jurisdiction with a lower tax regime than that of the previous patent holder. However, it is important to note that SLBs can incur structural problems. A borrower who has transferred title of a patent may have difficulty bringing infringement actions.

\section*{Patent Lending Strategies}
Lending strategies backed by patents are typically separated into two classes of transactions, depending on the quality of the underlying IP:

\begin{enumerate}
  \item Securitization: Lending backed by IP collateral allows separation of the IP owner's credit risk from the risk of holding the IP through the bankruptcy process.

  \item Mezzanine IP lending: Lending secured by IP collateral usually includes warrants or other upside. Fischer and Ringler (2014) discuss the use of patents as collateral in debt financings and find that actual collateralizations are driven primarily by the direct economic value of the patent rather than by strategic considerations, such as the ability to potentially exclude other parties from using technology in the case of liquidations.

\end{enumerate}

\section*{Patent Sales and Pooling}
Patent owners seeking to divest patents must find buyers. Traditionally, patent buyers have entered the market for one of three reasons:

\begin{enumerate}
  \item To purchase patents for operational use

  \item To purchase patents to use as "trading cards"

  \item To purchase a patent for strategic use; in this scenario, the purchaser may use the patent for defensive protection in negotiating with patent dealers.

\end{enumerate}

A fourth (and emerging) class of patent buyers is made up of IP asset managers looking to buy patents for monetary exploitation. Patent pooling, in which multiple owners of related patents agree to jointly license a number of patents to external users, is more complex than in-house licensing because of the need to divide royalty income based on revenue-sharing formulas. This can be a practical solution in industries with set standards and large quantities of patented technologies.

Two fairly recent patent pools that were highly effective at setting industry standards were the Moving Picture Experts Group (MPEG) patent pools and the DVD patent pools. Even though multiple pools had to be formed (for different MPEG formats and different DVD formats), it meant that licensees dealt with only one of a couple of pools rather than a myriad of individual companies. This simplification led to the success of both technologies.

\section*{Risks to Investment in Patents}
While there are many strategies involving patent assets, there are also many risks:

\begin{itemize}
  \item Illiquidity: IP assets are highly illiquid assets, which often cannot be easily monetized.
  \item Technology/operational risk: For investors buying cash flow streams generated by IP or purchasing debt collateralized by IP, technological risk and operational risk (which may limit the investors' ability to capitalize on the IP) are major concerns; cash flows depend on successful operation of the asset, particularly when the asset is prone to heavy competitive pressure (e.g., brands or technology in a fast-moving space).
  \item Obsolescence: If new technology displaces current IP, the asset may be rendered worthless.
  \item Macroeconomic/sector risk: If macroeconomic or sector-specific factors drive down an industry, this can have significant effects on the value of a patent or a company's ability to produce cash flows from the patent.
  \item Regulatory risk: IP represents government-issued rights; at any point, the government could change the structure of IP authority or impose regulation on licensing/sales activities.
  \item Legal risk: IP transactions require a thorough understanding of IP law; failure to account for all legal implications of a transaction could result in a loss of IP value.
  \item Expiration risk: A patent's life is 20 years (with some exceptions for extensions, primarily in the pharmaceutical space).
\end{itemize}

This session has reviewed three primary forms of IP. In the case of film production and distribution, revenue and profitability forecasts are difficult. Generally, film production can be viewed as offering return distributions skewed to the right, similar to venture capital returns. Art provides a long and somewhat plentiful history of transaction data from which estimation of historical risk and return is possible. Art has offered relatively low returns with moderate levels of risk and is subject to high transaction costs. R\&D and patents are emerging as stand-alone investments, potentially of offering high returns but requiring expertise in evaluation of the underlying assets.

The case for intellectual property as the bedrock of future long-term economic growth is persuasive. The case for substantial allocation of institutional portfolios to IP is less clear. However, stand-alone, institutional-quality IP appears to be likely to be an important investable sector, and may eventually offer superior returns to first-movers.


\end{document}