\documentclass[11pt]{article}
\usepackage[utf8]{inputenc}
\usepackage[T1]{fontenc}
\usepackage{amsmath}
\usepackage{amsfonts}
\usepackage{amssymb}
\usepackage[version=4]{mhchem}
\usepackage{stmaryrd}

\begin{document}
Commodity Producers

This session focuses on other real assets, including commodity producers, publicly traded partnerships of natural resources, infrastructure, and intellectual property.

Session 3.1, Natural Resources and Land, discussed natural resources as real assets that have experienced little or no alteration by humans. Investments in natural resources attract investor interest based on their perceived ability to serve as diversifiers against general economic fluctuations and the risk of unexpected inflation. However, direct and liquid institutional investment opportunities in natural resources are somewhat limited by the large extent to which global natural resources are owned by the public. Investments in firms with operations involved in developing natural resources are much more accessible. This session discusses investment opportunities of firms that transform natural resources into commodities and other goods and services available for consumption.

Each investment opportunity related to a natural resource may be viewed as lying on a spectrum, ranging from the purest plays on the value of a natural resource to those that are driven more by their operational focus than by the value of the natural resource related to their operations. For example, the rights to the mineral reserves of land containing copper ore are highly driven by the price of copper. The market price of an operating firm that mines and smelts the copper ore is presumably driven by a mixture of the effects of copper prices and other factors. Finally, ownership of the firms that provide products and services to the copper mine operators represents another potential avenue of diversifying into exposures to natural resources.

\section*{Natural Resource Prices as a Driver of Operating Firm Performance}
A key issue is the extent to which investments in firms that process natural resources provide reasonably similar risk and return characteristics to direct investments in the underlying natural resources. For example, are the returns of firms that explore, mine, or refine gold driven by the prices of refined gold?

In many industries, there would seem to be unclear links between the performance of an operating company and the price of the good that underlies the company's production. Thus, the price changes of the equities of manufacturers, technology firms, communications firms, and health-care firms tend to be only moderately correlated with the price changes of their products. For example, when airline ticket prices soar due to rising fuel costs, the stocks of airlines usually decline. In this example, the higher prices for tickets are driven by higher costs to the airline companies, not higher profits. In other cases, operating firms may hedge the exposure of their revenues to commodities, such as in the case of large oil companies that use derivatives to hedge their revenue from oil sales to smooth their profits.

In theory, the correlations between the returns of firms and price changes for their associated goods are driven by three primary factors: the price elasticity of the demand for the good, the price elasticity of the supply of the good, and the extent to which an operating firm is exposed to or has hedged changes in its profits.

There are sound economic reasons to believe that the market prices of firms that provide goods and services related to the extraction and processing of natural resources should be substantially correlated with the prices of the natural resources themselves or the commodities that emanate from the processing. The reasoning is that a dramatic rise in the price of a commodity, such as a metal or an agricultural product, indicates that demand vastly exceeded supply at the previous price. The relatively high demand for a commodity should generally coincide with increased demand for the services of firms that process those commodities. Thus, for example, when a commodity price such as oil soars, the firms that explore for oil, drill for oil, extract oil, and transport oil should generally expect that their services will have much higher demand than during a period following a large decrease in price. Accordingly, absent hedging strategies, large price increases in a commodity should tend to drive anticipation of higher profits in the firms that provide goods and services in the production of that commodity. For example, soaring oil prices have clearly been a boon to the oil and gas development industry.

\section*{Evidence on Commodity Prices and the Equity Prices of Operating Firms}
Empirical evidence can also provide insight into the relationship between commodity prices and the equity prices of operating firms. Let's examine an extreme 10year price move in a major commodity. The price of gold in US dollars soared roughly sixfold, from about $\$ 300$ per ounce in 2002 to a peak of $\$ 1,800$ per ounce in 2012. Did investors in the shares of gold mining firms realize similar profits? No. Roughly, the price of gold mining shares (as represented by the Dow Jones US Gold Mining Index) experienced only a threefold increase. It would appear likely that much of the gain from rising gold prices went to the owners of gold bullion and gold reserves rather than to the firms that explore, develop, extract, and process the resource. But gold mining stocks outperformed the overall market, which rose about $50 \%$ from 2002 to 2012.

Now let's turn to a shorter-term example of gold price changes. In the turbulent economic times of the global financial crisis (October 2008), overall equity prices varied widely. The price of gold in US dollars fluctuated roughly between $\$ 700$ and $\$ 900$ per ounce from early September 2008 to the end of November 2008 . At the end of October, gold was down only about $10 \%$ from its value in early September. Over the entire three-month period, the price of gold was slightly up. Gold therefore provided protection to investors from the panic that devastated equity markets.

On the other hand, US gold mining firms did not fare so well over the same period. The average price of these firms was quite volatile and generally moved downward. As represented by the VanEck Vectors Gold Miners ETF (which tracks the NYSE Arca Gold Miners Index), shares of gold mining firms dropped on average by almost half from early September to their low in October and recovered only partially by the end of November to a net decline of about one-third. Thus, in the short run, it appeared that the publicly traded firms related to gold production were driven more by the volatility of the equity markets than by the volatility of gold prices.

Gold provides evidence that firms related to a commodity have short- and long-term performance that differs substantially from the price performance of the related commodity. The empirical evidence cited in this section substantiates the intuition that the share prices of firms related to a commodity depend only partially on the value of the commodity. If the firms are publicly traded, their performance tends to be substantially correlated with the overall performance of public equity markets.

\section*{Commodity Prices and Operating-Firm Equity Return Correlations}
Let's turn to another commodity for a more formal analysis of correlations based on returns. From an economic perspective, energy is the largest sector of natural resources, and oil is the largest underlying resource within the energy sector. Oil prices vary between quality and location. Based on the private ownership of natural resources in the United States and data availability, we examined data on US oil and equity prices. This analysis uses monthly calendar returns related to US oil and ETF prices from August 2006 through December 2018.

The exhibit Return Correlations of Oil Operating Firms to Oil (USO) and Equities (SPY) lists the correlation coefficients between the returns of four investments: (1) the price of West Texas Intermediate light, sweet crude oil, as represented by United States Oil ETF (ticker USO); (2) the value of the SPDR S\&P Oil \& Gas Equipment \& Services ETF (ticker XES); (3) the value of the SPDR S\&P Oil \& Gas Exploration \& Production ETF (ticker XOP); and (4) the value of the SPDR S\&P 500 (ticker SPY).

Are the returns of oil-industry-related equities related more to oil prices or to stock prices? In the next exhibit the first column depicts correlations of two oil industry ETFs (XES and XOP) with oil prices (USO). The second column depicts correlations of the same ETFs with general US equity prices (SPY, which proxies the S\&P 500).

The results indicate relatively high and positive return correlations that are rather uniform. The ETFs of firms related to oil production (XES and XOP) had reasonably high correlations with oil prices but also had reasonably high correlations with US equity prices. XES focuses on publicly traded oil equipment and services firms, such as Schlumberger and Halliburton. XOP focuses on publicly traded oil exploration and production firms, such as Goodrich Petroleum Corporation.

The correlation between the monthly returns of USO and SPY over the same period was 0.45 (not shown). Thus, the relatively high return correlations between the ETFs of the oil firms and the US equity market indicate that much of the return variation in oil-related industries is driven by overall equity valuations and general economic conditions rather than as a pure play on the price of oil.

\begin{center}
\begin{tabular}{|lll|}
\multicolumn{2}{|c|}{Return Correlations of Oil Operating Firms to Oil (USO) and Equities (SPY)} &  \\
\hline
 & USO & SPY \\
\hline
XES & 0.66 & 0.70 \\
XOP & 0.49 & 0.66 \\
\hline
\end{tabular}
\end{center}

The empirical analysis summarized in the above exhibit reinforces the intuition that investments in firms are not pure plays on the returns of the real assets related to the firm's industry. Rather than the returns of these firms being driven entirely by the contemporaneous prices of related commodities, they are presumably also driven by the market's anticipation of dynamic supply and demand factors more related to the long-term profitability of the goods and services directly offered by those firms, which in turn tend to be correlated with overall equity markets. To the extent that the returns of firms within the ETFs in the above exhibit are driven substantially by operational issues, the investments will serve more as traditional equity investments rather than as diversifiers or any other type of alternative investment vehicle.


\end{document}