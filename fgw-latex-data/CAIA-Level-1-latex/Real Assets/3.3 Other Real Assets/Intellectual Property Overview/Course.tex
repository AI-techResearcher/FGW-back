\documentclass[11pt]{article}
\usepackage[utf8]{inputenc}
\usepackage[T1]{fontenc}
\usepackage{amsmath}
\usepackage{amsfonts}
\usepackage{amssymb}
\usepackage[version=4]{mhchem}
\usepackage{stmaryrd}

\begin{document}
Intellectual Property Overview

A substantial portion of gross domestic product (GDP) is now generated by and composed of intangible assets, namely intellectual property (IP). For an asset to be owned, it must be excludable. An excludable good is a good others can be prevented from enjoying. Exclusivity distinguishes private goods and private property (e.g., houses) from public goods (e.g., air). Many intangible assets are nonexcludable goods, especially in the long run. For example, everyone benefits from ancient inventions such as the wheel without having to pay its inventor. But some intangible assets are naturally excludable (e.g., reputation) or are protected by law (e.g., patents, trademarks, and copyrights).

Intangible assets are excludable, do not have a physical form, are real assets (not financial assets), and can include ideas, technologies, reputations, artistic creations, and so forth. Intellectual property (IP) is an intangible asset that is a creation of the human mind and that is excludable, such as copyrighted artwork. Like tangible assets, IP assets are important components in the production of goods, and have many of the same characteristics as tangible assets. This section provides an overview of IP and then discusses three categories as examples: (1) film production and distribution, (2) visual works of art, and (3) patents and research and development (R\&D).

\section*{Characteristics of IP}
Intangible assets, including IP, are necessary inputs to economic productivity, along with labor, capital, and raw materials. Intangible assets such as technology are the primary source of productivity that determines the relative level of the wealth of societies, both through time and across societies. Much of the value of the stocks and bonds of a modern corporation, such as a pharmaceutical firm or a computer technology firm, is composed of IP.

Historically, most intellectual property was bundled with tangible corporate assets and was available for investment through traditional means, such as an equity investment in a software (or other intellectual property-oriented) company. However, in recent years, there has been an increased interest in unbundling and isolating intangible assets, IP in particular, for stand-alone investment purposes. Examples of such assets include patent portfolios, film copyrights, art, music or other media, research and development (R\&D), and brands. Unbundled intellectual property is IP that may be owned or traded on a standalone basis.

Unbundled IP may be acquired or financed at various stages in its development and exploitation. Ex ante, newly created IP may have widely varying value and use. The value of property such as exploratory research, new film production, new music production, or pending patents will typically be widely uncertain prior to production or implementation. Similar to venture capital investments, many of these types of IP may fail to recapture initial investment or costs, whereas a proportionately small number of cases will capture a large asymmetric return on investment. For example, most movies lose money; yet films, on average, are still profitable. De Vany and Walls (2004) report that, for a sample of more than 2,000 films, $6.3 \%$ of the films generated $80 \%$ of the total profits. Mature intellectual property is IP that has developed and established a reliable usefulness and will have a more certain valuation and a more clear ability to generate licensing, royalty, or other income associated with its use.

The duration of IP varies by type. Most forms of IP, such as patents and copyrights, are wasting assets-that is, assets with relatively large immediate benefits but with value that is expected to diminish through time. Intellectual property generally diminishes in value through time, as its productive advantages are displaced by new creativity or its excludability wanes (e.g., patents expire). However, some IP offers substantial capital accumulation through time. Clearly, many instances of artwork and brand names have exhibited substantial long-term growth in value.

\section*{Intellectual Properties and Six Characteristics of Real Assets}
Financial economists use the term real assets as the counterpart to financial assets, and in this context IP is clearly a real asset despite its intangible nature. A number of investment managers have advanced the idea that certain intellectual property assets have properties that characterize them as real assets in the context of organizing alternative investment portfolio allocations. Recent changes to U.S. gross domestic product (GDP) accounting, which allow certain intellectual property assets to be treated as fixed investments, are consistent with this idea.

Six characteristics recognized as belonging to real assets have been analyzed as generally being common among typical intellectual property assets (Martin 2014). The six general characteristics of real assets that follow are each used to ascertain the extent to which IP fits the investment classification of being a real alternative asset.

\section*{1. Low operating risk}
Investing in proven and established intellectual property (such as established pharmaceutical technology), owning existing media assets rather than funding the creation of new assets (such as movie slates), and using established patents or technological IP that is already in use are lower-risk ways to invest in patents. As with traditional tangible assets, these assets may derive their value from their use in established economic processes and have value that is largely transferable from one owner to another, with less emphasis on strategic complementarities and hence niche value. The preservation of value under transfer of ownership or control is particularly emphasized in IP assets that are subject to license and sale (such as technology patents or pharmaceutical IP). Ownership rights in IP are long-lasting and used repeatedly in the production process.

\section*{2. Positive correlation with inflation}
Positive correlation between rates of inflation and investment returns provides investors with protection from inflation risk. However, Martin (2010) notes that there is no definitive evidence on the correlation of intellectual property assets and inflation. Nevertheless, this lack of certainty also pertains to certain other assets that have been classified as real, such as real estate. Casual observation and economic reasoning indicates that it is likely that intellectual property assets have low correlation with inflation and, like other real assets, represent a diversifier of inflation risk, unlike other traditional assets in institutional portfolios (e.g., equities and nominal bonds).

\section*{3. Preserving value in periods of macroeconomic instability}
In general, excluding some industries such as technology and biotech, intellectual property products have low beta to the overall market. For example, fiveyear rolling betas to the Fama-French market factor indicates that the pharmaceutical sector has a beta in the lower half of all sectors, and that it is generally in the lowest quartile of sector betas. Elevated intermediate cash flows serve to reduce risk generally while also reducing the asset's sensitivity to exit risk in a temporarily unfavorable market environment.

\begin{enumerate}
  \setcounter{enumi}{3}
  \item Benefits from the scarcity of inputs in sectors like energy, manufacturing, and agriculture
\end{enumerate}

Most real assets are viewed as benefiting from scarcity of key inputs. However, IP assets have been established as not possessing this characteristic and, therefore, in this sense do not conform to this traditional characteristic of real assets (in the context of institutional asset classifications).

\section*{5. Are essential parts of economic infrastructure}
Intellectual property, and in particular "intellectual property products" as defined by the Bureau of Economic Analysis in its definition of GDP, is of growing importance. As indicated by the accounts behind the calculation of GDP, long-lived intellectual property products are considered a significant part of the U.S. GDP and therefore are likely an essential part of an economy's infrastructure.

\section*{6. Offers long-term risk and return properties suitable for supporting funding with long-term liabilities}
The focus on intellectual property products with low operating risk, ready transferability or license, and long lives provides a basis for the generation of relatively stable cash flows which in turn supports the proposition that IP assets may be suitable for funding long-term liabilities.

In summary, IP tends to exhibit five of the six characteristics commonly ascribed to the category of real assets in institutional portfolios.

\section*{A Simplified Model of Intellectual Property}
Based on the generalized behavior of IP, a very simplified model of IP values may be constructed as the present value of expected future cash flows. Assume that the value of IP at its creation is the sum of the discounted expected cash flows generated by the property. Further assume that the property has two possible outcomes: the probability $(p)$ of generating large positive cash flows and the probability $(1-p)$ of generating no positive cash flows. Denote the first-year cash flows of the project, if positive, as $C F_{1}$. Finally, assume that the cash flows in years 2 and beyond are equal to $C F_{1}$ adjusted annually by the rate $g$. $C F_{t}=C F_{1}(1+g)^{t-1}$. If the model is being used to value a wasting asset, then the rate $g$ is a negative number that indicates the rate at which the cash flows are decaying through time as a result of obsolescence or other causes of diminished value. Using the perpetual growth model commonly used to value common stock (where $g$ is typically positive), the value of the IP at time zero, $V_{i p, 0}$, discounted at the rate $r$ can be expressed as depicted in Equation 1:


\begin{equation*}
V_{i p, 0}=p \times C F_{1} /(r-g) \tag{1}
\end{equation*}


Equation 1 is identical to the perpetual growth model used for common stocks except for the use of $p \times C F_{1}$ to denote the expected cash flow in the first year and the idea that $g$ is likely to be negative. Note that the present value of the future cash flows, $v_{i p, 0}$, is positively related to $p$ and $g$. Given estimates of $p, C F_{1}, r$, and $g$, a value may be estimated for $V_{i p, 0}$.

The previous computation reflects the idea that investment in new IP may have risk similar to that of an out-of-the-money call option. To illustrate another perspective, Equation 1 can be rearranged to solve for the total annual rate of return, $r$, as shown in Equation 2:


\begin{equation*}
r=p \times\left(C F_{1} / V_{i p, 0}\right)+g \tag{2}
\end{equation*}


The total rate of return is the expected cash flow in the first year expressed as a percentage of the value of the IP minus the rate of decay. The intuition of Equation 2 is that an investment in undeveloped IP is a chance $(p)$ at a potential stream of income $\left(C F_{1}\right)$ that is likely to diminish $(g<0)$ through time as the productivity of the IP wanes.

Financial analysis of IP requires specialized skills, including legal knowledge that should be accessed to reap potential benefits through return and diversification. The reason that legal knowledge is especially important in the case of IP is the tendency of property rights to be complex and dynamic for intangible assets.


\end{document}