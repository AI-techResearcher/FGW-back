\documentclass[11pt]{article}
\usepackage[utf8]{inputenc}
\usepackage[T1]{fontenc}
\usepackage{amsmath}
\usepackage{amsfonts}
\usepackage{amssymb}
\usepackage[version=4]{mhchem}
\usepackage{stmaryrd}

\begin{document}
\section*{APPLICATION A}
Question : Loosely following some of the values indicated earlier in this section for films, assume that the probability of substantial success for an investment in IP $(p)$ is $6 \%$, the rate at which expected cash flows diminish each year after their initial potential $(g)$ is $5 \%$, and the required rate of return $(r)$ is $12 \%$. How much would this investment in IP be worth per dollar of projected possible first-year cash flow $\left(C F_{1}\right)$ ?

\section*{Answer and Explanation}
Essentially, this formula is the perpetuity growth model (albeit with a negative growth rate) multiplied by the probability of success for the investment as shown in Equation 1:

$$
V_{i p, 0}=p \times \frac{C F_{1}}{(r-g)}
$$

To solve this equation to find out how much this investment in IP will be worth per dollar of first-year cash flows, we need to plug in the appropriate value, $p=6 \%$, $\mathrm{g}=-5 \%, \mathrm{r}=12 \%$, and $\mathrm{CF}_{1}=\$ 1$.

$$
\begin{gathered}
V_{i p, 0}=p \times \frac{C F_{1}}{(r-g)} \\
V_{i p, 0}=0.06 \times \frac{1}{(0.12-(-0.05))} \\
V_{i p, 0}=0.06 \times \frac{1}{(0.17)} \\
V_{i p, 0}=0.06 \times 5.88 \\
V_{i p, 0}=0.35
\end{gathered}
$$

\section*{APPLICATION B}
Question : Assume that Equation 2 is an appropriate valuation model and that $C F_{1} / V_{i p, 0}$ is $3.0, p$ is 0.06 , and $g$ is -0.05 . What is the investment's annual rate of return?

\section*{Answer and Explanation}
To solve this application we need to use Equation 2:

$$
r=p \times\left(\frac{C F_{1}}{V_{i p, 0}}\right)+g
$$

With Equation 2 at hand, we need to plug in the given figures and solve for $\mathrm{r}$. In this application $\mathrm{p}=0.06, \mathrm{~g}=-0.05$, and $\frac{C F_{1}}{V_{i p, 0}}=3.0$.

$$
\begin{gathered}
r=0.06 \times 3.0+(-0.05) \\
r=0.18+(-0.05) \\
r=0.13
\end{gathered}
$$

The answer is $\mathrm{r}=13 \%$.


\end{document}