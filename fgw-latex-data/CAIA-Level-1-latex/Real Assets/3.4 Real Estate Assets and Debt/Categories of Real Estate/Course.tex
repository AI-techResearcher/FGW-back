\documentclass[11pt]{article}
\usepackage[utf8]{inputenc}
\usepackage[T1]{fontenc}

\begin{document}
Categories of Real Estate

This is the first of two sessions on real estate. This session provides an overview of real estate assets, followed by a detailed discussion of fixed-income investments backed by real estate. It also discusses liquid alternatives that provide exposure to real estate.

Real estate has been a very large and important portion of wealth for thousands of years. Even as recently as a century ago, real estate dominated institutional portfolios and was classified as property. During recent decades, the preeminence of real estate has yielded to the growing importance of intangible assets, yet real estate remains a valuable part of any well-diversified portfolio. The transition of private real estate from dominating traditional institutional-quality investments to being an alternative investment raises important issues in terms of how to evaluate real estate on a forward-looking basis.

This section describes the main characteristics of various real estate assets, beginning with five especially common categories that can be used to differentiate real estate:

\begin{enumerate}
  \item Equity versus debt

  \item Domestic versus international

  \item Residential versus commercial

  \item Private versus public

  \item Market size of geographic location

\end{enumerate}

Each of these categories is briefly discussed in the following five sections.

\section*{Equity versus Debt}
The traditional method of distinguishing between equity claims and debt claims is to use the legal distinction between a residual claim and a fixed claim. A mortgage is a debt instrument collateralized by real estate, with a value that is more closely associated with the value of the real estate than the profitability of the borrower. Mortgages with substantial credit risk can behave more like real estate equity, and equity ownership of properties with very-long-term leases can behave like debt. Real estate equity investments will be discussed in the session, Real Estate Equity.

\section*{Domestic versus International}
One of the primary motivations of real estate investing is diversification. International investing (i.e., cross-border investing) in general and international real estate investing in particular are regarded as offering substantially improved diversification. However, the heterogeneity of most real estate and the unique nature of many real estate investments make international real estate investing more problematic than international investing in traditional assets. Seven challenges to international real estate investing include: (1) a lack of knowledge and experience regarding foreign real estate markets, (2) a lack of relationships with foreign real estate managers, (3) the time and expense of travel for due diligence, (4) liquidity concerns, (5) political risk (particularly in emerging markets), (6) risk management of foreign currency exposures, and (7) taxation differences. For these reasons, a large share of international real estate investing is done through shares of listed property companies in foreign countries. The continuing emergence of derivative products related to real estate investments in particular nations or regions is an important potential opportunity for exploiting the benefits of international diversification without the challenges of direct international investment.

These challenges are addressed in detail in Section 9.5 of the Level II CAIA curriculum, in the session entitled "Complexity and the Case of Cross-Border Real Estate Investing." That session includes extensive discussion of the potential risks and solutions to currency differentials. Briefly, all of these challenges introduce complexities to cross-border real estate investment that require substantial time, expertise, and resources.

The extent of appropriate international investing depends on the locale of the asset allocator. An asset allocator in a very large economy may be able to achieve moderate levels of diversification without foreign real estate investing. However, an asset allocator in a nation with a small or emerging economy may experience high levels of idiosyncratic risk in the absence of foreign investments.

\section*{Residential versus Commercial}
One of the most important drivers of the characteristics of a real estate investment is the nature of the real estate assets underlying the investment. A broad distinction, especially in mortgages, is residential real estate versus commercial real estate. Residential real estate or housing real estate includes many property types, such as single-family homes, townhouses, condominiums, and manufactured housing. The housing or residential real estate sector is traditionally defined as including owner-occupied housing rather than large apartment complexes.

Within residential real estate, institutional investors are primarily concerned with investing in mortgages backed by housing and residential real estate. Institutional ownership in these instruments is usually established through pools of mortgages.

Commercial real estate properties include the following property sectors: office buildings, industrial centers, data centers, retail (malls and shopping centers, also referred to as "strips"), multi-family, health-care facilities (medical office buildings and assisted-living centers), self-storage facilities, and hotels. Small properties may be directly and solely owned by a single investor. Alternatively, collections of numerous smaller properties and large commercial properties may be managed by a real estate company or through private equity real estate funds, which, in turn, are owned by several institutional investors as limited partners.

The volume of transactions fluctuates significantly depending on the stage of the business cycle, but it is generally high enough to support large investments by institutional investors. For the most part, residential and commercial real estate require very distinct methods of financial analysis. For example, the credit risk of mortgages on residential real estate is typically analyzed with a focus on the creditworthiness of the borrower. Mortgages on commercial real estate tend to focus on the analysis of the net cash flows from the property.

\section*{Private versus Public}
Exposure to the real estate market, especially the equity side, can be achieved via private and public ownership. Private real estate equity investment involves the direct or indirect acquisition and management of actual physical properties that are not traded on an exchange. Public real estate investment entails the buying of shares of real estate investment companies and investing in other indirect exchange-traded forms of real estate (including futures and options on real estate indices and exchange-traded funds linked to real estate). Private real estate is also known as physical, direct, or non-exchange-traded real estate, and may take the form of equity through direct ownership of the property or debt via mortgage claims on the property.

The private real estate market comprises several segments: housing or residential real estate properties, commercial real estate properties, farmland, and timberland. Farmland and timberland, which were discussed in the session, Natural Resources and Land, are often discussed as real assets rather than as real estate. The relative advantages of investing in the private side of real estate equity are that investors or investment managers have the ability to choose specific properties, exert direct control of their investments, and enjoy the potential for tax-timing benefits.

\section*{Real Estate Categorization by Market Size}
Institutional investors often categorize private commercial real estate equity investments by the size of the real estate market in which the property is located. Real estate assets are said to trade in a primary real estate market if the geographic location of the real estate is in a major metropolitan area of the world, with numerous large real estate properties or a healthy growth rate in real estate projects with easily recognizable names. Using the United States for illustration, examples range from cities such as Orlando, Florida, to very large metropolitan areas within huge cities, such as Manhattan in New York City. Large institutional investors focus on investments in these primary markets. Secondary real estate markets include moderately sized communities as well as suburban areas of primary markets. Tertiary real estate markets tend to have less recognizable names, smaller populations, and smaller real estate projects.


\end{document}