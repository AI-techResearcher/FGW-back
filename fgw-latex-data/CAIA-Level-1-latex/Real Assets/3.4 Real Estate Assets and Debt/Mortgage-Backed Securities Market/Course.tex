\documentclass[11pt]{article}
\usepackage[utf8]{inputenc}
\usepackage[T1]{fontenc}
\usepackage{amsmath}
\usepackage{amsfonts}
\usepackage{amssymb}
\usepackage[version=4]{mhchem}
\usepackage{stmaryrd}

\begin{document}
Mortgage-Backed Securities Market

This lesson discusses the mortgage-backed securities market. Mortgage-backed securities (MBS) are a type of asset-backed security that is secured by a mortgage or pool of mortgages. In recent decades, MBS have facilitated cost-efficient real estate financing but have also been blamed for facilitating destabilizing speculation. Although most attention has been focused on the residential mortgage-backed securities (RMBS) market, which is backed by residential mortgage loans, there was substantial growth in the commercial mortgage-backed securities market in the years leading up to the real estate and financial crisis that began in 2007.

There are two basic types of MBS differing by the extent, if any, to which they partition risk within different classes of securities. A pass-through MBS is perhaps the simplest MBS and consists of the issuance of a homogeneous class of securities with pro rata rights to the cash flows of the underlying pool of mortgage loans. Collateralized mortgage obligations (CMOs) extend this MBS mechanism to create different security classes, called tranches, which have different priorities to receiving cash flows and therefore different risks. CMOs are discussed in Topic 6 on structured products.

\section*{Residential Mortgage Prepayment Options}
Residential mortgage markets have been dominated in size by insured mortgages for which there is little or no risk of default to the lender. Most mortgages have scheduled principal repayments that amortize the mortgage's principal value from the initial mortgage amount to zero over the mortgage's scheduled lifetime. Most mortgages also allow the borrower the option to make additional and unscheduled principal payments without penalty. Future unscheduled prepayments are the key unknown variable in determining the values of insured mortgages and mortgage pools.

Residential mortgages are callable bonds. The lender is short a call option on the value of the loan, which may also be viewed as being short a put option on mortgage rates. Borrowers may exercise this option by refinancing if interest rates decline. Exercise of this prepayment option when interest rates fall acts to the detriment of lenders, which presumably must reinvest the prepaid principal at the lower rates.

Unscheduled mortgage principal payments include full mortgage prepayments (e.g., when a loan is refinanced or when it is repaid because a homeowner is moving) and partial repayments, when borrowers decide to make one or more mortgage payments that exceed the minimum required payment (e.g., when the mortgage rate is higher than the interest rate that the borrower can earn on excess cash).

The main problem with unscheduled principal repayments is that the mortgage investor cannot predict the size of the prepayments or the rate at which the unscheduled principal repayments will be received and can be reinvested. Unscheduled repayments on a mortgage issued at an interest rate of $6 \%$ cease earning $6 \%$ to the mortgage investor and presumably begin earning current interest rates, which may be higher or lower than $6 \%$.

The option to make or not make unscheduled principal repayments rests with the borrower. Mortgage borrowers have an incentive to make unscheduled mortgage payments when interest rates are low, for several reasons. First, borrowers are more likely to refinance when rates are low. Second, borrowers are more likely to move and fully prepay mortgages when rates are low. Finally, borrowers are more likely to use excess cash to prepay mortgages when the interest rate on their mortgages substantially exceeds the rate at which the excess cash can be invested. The same incentives reverse when interest rates are high, making borrowers less likely to prepay mortgages. Simply put, borrowers have a prepayment option, and they tend to exercise that option in their favor based on interest rates. However, there are also idiosyncratic factors related to the borrower, such as the ability to make prepayments and the decision to sell a house, that affect prepayment decisions. These factors are not fully driven by interest rates, and they may cause or prevent otherwise optimal exercise of the prepayment option based purely on interest rates. Thus, mortgage prepayments are difficult to predict even under specific interest rate scenarios.

Mortgage lenders write the prepayment options at the initiation of the mortgage, and therefore the lenders, and any subsequent mortgage investors, are short those options until the mortgage is fully repaid. Mortgage investors suffer losses when the borrower's prepayment option moves into-the-money relative to an investor in a similar fixed-income security without the prepayment option. The borrower harvests gains by exercising the prepayment option. Specifically, rational exercise of the prepayment option by borrowers tends to generate higher unscheduled prepayments to lenders when interest rates are low and reinvestment opportunities are least desirable.

Unscheduled principal payments to lenders when interest rates are high and reinvestment opportunities are most desirable would be made only due to the borrowers' idiosyncratic factors. Although they would typically work to the advantage of the mortgage investor, such unscheduled principal payments would be relatively less likely.

Each long-term mortgage has hundreds of scheduled future payments and hundreds of future potential prepayment options. The cash flows, or payments, of individual mortgages are aggregated and form the available cash flows of the mortgage pools underlying the RMBS. These cash flows include the unscheduled principal payments that are passed from the mortgage pool to the RMBS investors. Thus, the main risks of RMBS with insured underlying mortgages involve the prepayment behavior of the underlying pool and its relationship with reinvestment opportunities. These unscheduled payments create uncertainty on the part of investors regarding both the timing of the principal repayments they will receive and the longevity of the interest payments they will receive.

\section*{Measuring Unscheduled Prepayment Rates}
Mortgage returns that are not driven by default risk are primarily driven by the interest rate risk inherent in prepayment risk. Mortgage investors therefore focus on the unscheduled principal payments and the forecasted speed of prepayments.

In essence, the market value of each mortgage or pool of mortgages is a function of its anticipated rate of prepayment. Attempts to earn superior rates of return are generally exercises in predicting prepayment rates and investing in those mortgages or pools of mortgages that will experience more desirable rates of prepayment than are reflected in the current price. With stable interest rates, high rates of prepayment are usually beneficial to the mortgage investor because they reduce the expected longevity of the cash flow stream. However, when mortgages have interest rates higher than prevailing market interest rates, slower prepayment rates may be desirable to the mortgage investor.

More sophisticated insured mortgage analysis focuses on models that combine interest rate behavior with unscheduled principal payment rates. The secondary mortgage market has developed models for deriving interest rate scenarios, correlating those interest rate scenarios with prepayment scenarios, and using the framework to price MBS. This section describes the major metric by which unscheduled principal payments are expressed.

The annualized percentage of a mortgage's remaining principal value that is prepaid in a particular month is known as the conditional prepayment rate (CPR). The exact computation of the CPR involves principal balances and specifies such details as the use of monthly compounding. But the CPR for a particular month is clearly intuitive: It roughly reflects the annual reduction in the mortgage principal that would be anticipated if the same percentage of principal were repaid each month for 12 consecutive months. For example, if $1 \%$ of a mortgage's remaining principal payment is prepaid in a particular month, the CPR for that month would be $11.4 \%$ (which is less than $12 \%$ due to compounding with a declining balance).

The Public Securities Association (PSA) established the PSA benchmark, a benchmark of prepayment speed that is based on the CPR and that has become the standard approach used by market participants. The PSA prepayment benchmark is shown in the next exhibit, PSA Benchmark Pattern.

As indicated in the exhibit below, PSA Benchmark Pattern, the benchmark assumes that for a 30 -year mortgage, a CPR of $0.2 \%$ will apply for the first month of the security. The monthly benchmark CPR then increases by $0.2 \%$ per month for the next 30 months until it reaches a level of $6 \%$. The benchmark CPR is then assumed constant at this rate of $6 \%$ for the rest of the life of the mortgage. The reason behind the initially increasing CPR rate is that only a few borrowers will be expected to prepay in the early years of their loans (e.g., due to moving or refinancing), as their circumstances and market interest rates have had little time to change since they made the decision to take out the loan. However, as time passes, prepayments are assumed to pick up until they level off at a CPR of 6\%.

\begin{center}
\begin{tabular}{|cc|}
\hline
PSA Benchmark Pattern &  \\
\hline
Month & CPR \\
\hline
1 & $0.2 \%$ \\
2 & $0.4 \%$ \\
3 & $0.6 \%$ \\
$\vdots$ & $\vdots$ \\
29 & $5.8 \%$ \\
30 & $6.0 \%$ \\
31 & $6.0 \%$ \\
32 & $6.0 \%$ \\
$\vdots$ & $\vdots$ \\
\hline
\end{tabular}
\end{center}

The key to the benchmark is that it is used as a standard against which each mortgage or mortgage pool is indexed. If a mortgage experiences the same CPR for a particular month, as is listed in the exhibit above, then it is described as prepaying at $100 \%$ PSA. For example, if Mortgage A has a steady CPR of $1 \%$ for every month, in month 2 it would be referred to as $250 \%$ PSA because the actual CPR (1\%) is 2.5 times the PSA standard rate (0.4\%) for the second month of a mortgage's life. In month 30 or beyond in the mortgage's life, a CPR of $1 \%$ would be referred to as 16.7\% PSA because the actual CPR (1\%) is one-sixth the PSA standard rate for those months (6\%).

\section*{Pricing RMBS with PSA Rates}
The cash flows of insured residential mortgage pools can be projected, assuming a given PSA speed. Those cash flows can then be discounted to form an estimated present value or price to the pool. However, the selection of an appropriate discount rate is complicated by the interest-rate-related options of mortgages. Expected cash flows cannot simply be discounted at expected interest rates, since larger cash flows (i.e., higher unscheduled principal repayments) tend to occur when interest rates are lowest. Thus, RMBS pricing models should be based on option pricing technology.

However, mortgage prepayment options are not exercised based purely on interest rates. Some mortgage borrowers prepay mortgages during high-interest rate environments due to personal circumstances (e.g., moving due to a change in employment), and some mortgage borrowers fail to prepay mortgages even when interest rates are low and refinancing appears beneficial. Factors affecting prepayment decisions other than interest rates or other systematic factors are known as idiosyncratic prepayment factors. Idiosyncratic prepayment factors prevent the specification of a precise relationship between unscheduled prepayments and interest rate levels, and option pricing models that include this behavior should be used.

Mortgage prepayment rates can also vary due to systematic prepayment factors other than interest rates. For example, a rise in economic activity or higher housing prices can generate widespread prepayments as borrowers change residences to move into larger houses or accept new jobs. Changes in prepayment rates from systematic factors can also be due to interest-rate-related factors other than current interest rate levels, including the path that mortgage rates have followed to arrive at the current level. For instance, when mortgage rates drop further after having declined substantially in the recent past, refinancing may not occur at a rapid\\
rate, since those who ascertain a benefit from refinancing at lower interest rates will probably have done so when the mortgage rate first dropped. Reduced refinancing speeds due to high levels of previous refinancing activity is known as refinancing burnout.

The prepayment rates experienced by mortgage pools will vary based on such factors as the characteristics of the underlying mortgage pool. These factors include the maturities of the mortgages, the rates of the fixed-rate mortgages, and the terms of any variable-rate mortgages. Another factor is the geographic location of the pool. There are regional prepayment tendencies, regional economic performance levels, and regional impacts on prepayment speeds even within the same country. Geography also comes into play in relation to factors such as the risk of destruction of properties. For example, if a large number of properties in the pool are located closer to major storm risks or earthquake risks, there can be substantial effects on the potential speeds of prepayments of insured and uninsured mortgages.

Analysts build fundamental models of prepayment speeds based on these characteristics and include analysis of past prepayment rates to predict future prepayment rates.

Ownership of mortgage pools is often divided or structured into investment products that have widely varying exposures to prepayment risks. These structured products are discussed in detail in Topic 6.

\section*{Commercial Mortgage-Backed Securities}
Commercial mortgage-backed securities (CMBS) are mortgage-backed securities with underlying collateral pools of commercial property loans. CMBS provide liquidity to commercial lenders and to real estate investors. Commercial lenders can sell commercial loans that they have issued into the CMBS pools. Real estate investors may purchase CMBS and enjoy higher liquidity and diversification than they would through direct ownership of commercial loans.

The emergence of the CMBS market in the United States in the early 1990s can be explained, at least partially, by a large market correction in the U.S. real estate market at that time, which caused a severe lack of liquidity in the sector. The correction damaged many traditional commercial lenders and decreased the level of activity of many others. At that point, CMBS facilitated investment by mortgage investors other than traditional commercial lenders. The use of CMBS rose over the years, along with real estate prices, leading to the financial crisis that began in 2007.

The global financial crisis of 2007-2009 can be partially attributed to real estate markets, especially residential loans to subprime borrowers. In the years leading up until 2007, home ownership substantially increased in the United States, with much of the increase coming from borrowers with lower income or creditworthiness purchasing homes. With lower levels of due diligence, down payments, and creditworthiness, lenders made it easier to purchase homes than in the past. The increase in demand for homes, combined with weakened lending standards, led to increases in home prices, and, by extension, increases in prices in commercial properties. When it became clear that these marginal borrowers could not service their loans, their properties went into foreclosure, moving the strong increase in demand into an increased supply of properties that drove prices quickly lower. With low to no down payments providing equity in the property, borrowers were forced out or abandoned the properties, leaving banks and mortgage lenders to suffer nearly all of the losses from declining property values. When bank and lender capital was found to be insufficient to cover the losses, numerous banks failed both in the United States and in Europe, leading to bailouts from government and taxpayer funds.

Compared to an insured RMBS, a CMBS provides a lower degree of prepayment risk because commercial mortgages are most often set for a shorter term. Fixed-rate commercial mortgages typically charge a prepayment penalty, which makes commercial borrowers substantially less likely to refinance than residential borrowers. However, CMBS are more subject to credit risk. Because they are not standardized, there are lots of details associated with CMBS that make default risks difficult to ascertain and thus make these instruments difficult to value. Many of these differences relate to the more heterogeneous nature of CMBS issues relative to RMBS issues and to their underlying real estate properties. In particular, default risks are complex and heterogeneous due to the unique risks of commercial real estate assets. Factors that may affect CMBS default probabilities include property type, location, borrower quality, tenant quality, lease terms, property management, property seasoning, and year of origination. Further, given the large size and indivisible nature of properties, CMBS issues tend to contain fewer loans. This means that investors in the CMBS market have concentrated risk to a relatively small number of potential defaults.

LTV ratios and debt yields (cash flow divided by the amount of the loan) play a big role in the analysis of CMBS issues, as they do for the underlying commercial mortgages. Most U.S. CMBS issues have had historical average LTV ratios in the $65 \%$ to $80 \%$ region, and CMBS issues with average LTV ratios greater than $75 \%$ would be viewed as risky. However, what is perhaps more important to consider is the percentage of the individual loans in a CMBS with LTV ratios above $75 \%$. In many cases, rating agencies allow a maximum of $15 \%$ of loans with LTV ratios in excess of $75 \%$. The risk of CMBS is also driven by the level of diversification in the pools' mortgages. For example, rating agencies often discourage issues (refuse to assign high ratings) when an individual loan is more than $5 \%$ of a specific CMBS issue.


\end{document}