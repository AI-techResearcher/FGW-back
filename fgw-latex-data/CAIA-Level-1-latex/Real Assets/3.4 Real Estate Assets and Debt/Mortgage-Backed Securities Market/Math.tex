\documentclass[11pt]{article}
\usepackage[utf8]{inputenc}
\usepackage[T1]{fontenc}
\usepackage{amsmath}
\usepackage{amsfonts}
\usepackage{amssymb}
\usepackage[version=4]{mhchem}
\usepackage{stmaryrd}

\begin{document}
\section*{APPLICATION A}
Mortgage B experiences a CPR of $2 \%$ in its 20 th month. How would this prepayment rate be expressed using the PSA benchmark?

\begin{center}
\begin{tabular}{|cc|}
\hline
PSA Benchmark Pattern &  \\
\hline
Month & CPR \\
\hline
1 & $0.2 \%$ \\
2 & $0.4 \%$ \\
3 & $0.6 \%$ \\
$\vdots$ & $\vdots$ \\
29 & $5.8 \%$ \\
30 & $6.0 \%$ \\
31 & $6.0 \%$ \\
32 & $6.0 \%$ \\
$\vdots$ & $\vdots$ \\
\hline
\end{tabular}
\end{center}

\section*{Answer and Explanation}
Mortgage B has a PSA prepayment speed of $50 \%$ in month 20 . Mortgage B's prepayment rate of $2 \%$ is $50 \%$ of the $4 \%$ benchmark. The $4 \%$ benchmark is $0.2 \% \times 20$ months, since the month number is less than 30 .

\section*{APPLICATION B}
Mortgage C experiences a PSA rate of 200\% in each month and is now five years old. What is its CPR?

\begin{center}
\begin{tabular}{|cc|}
\hline
PSA Benchmark Pattern &  \\
\hline
Month & CPR \\
\hline
1 & $0.2 \%$ \\
2 & $0.4 \%$ \\
3 & $0.6 \%$ \\
$\vdots$ & $\vdots$ \\
29 & $5.8 \%$ \\
30 & $6.0 \%$ \\
31 & $6.0 \%$ \\
32 & $6.0 \%$ \\
$\vdots$ & $\vdots$ \\
\hline
\end{tabular}
\end{center}

\section*{Answer and Explanation}
The PSA standard is $6 \%$ at 30 months and beyond, and $200 \%$ of $6 \%$ is $12 \%$. Since the mortgage is already at or beyond month 60 , the CPR for the mortgage is now $12 \%$.


\end{document}