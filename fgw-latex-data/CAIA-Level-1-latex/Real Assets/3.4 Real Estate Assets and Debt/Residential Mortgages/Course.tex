\documentclass[11pt]{article}
\usepackage[utf8]{inputenc}
\usepackage[T1]{fontenc}
\usepackage{amsmath}
\usepackage{amsfonts}
\usepackage{amssymb}
\usepackage[version=4]{mhchem}
\usepackage{stmaryrd}

\begin{document}
Residential Mortgages

This lesson examines residential mortgages from the perspective of the investor. The primary issues regarding residential mortgage investments are the timing and safety of the payments.

A mortgage loan can be simply defined as a loan secured by property. The property serves as collateral against the amount borrowed. If the borrower defaults on the loan, then the lender can take possession of the property. The borrower can usually partially or fully prepay the mortgage before the contractual due date. These partial prepayments may be made by borrowers to save on future interest payments. However, lenders may add prepayment penalties to mortgages to discourage borrowers from refinancing prior to maturity.

A major distinction between mortgages is whether the interest rate used to determine mortgage payments is fixed or variable. A fixed-rate mortgage has interest charges and interest payments based on a single rate established at the initiation of the mortgage. A variable-rate mortgage has interest charges and interest payments based on a rate that is allowed to vary over the life of the mortgage based on terms established at the initiation of the mortgage.

Another major distinction between mortgages is residential versus commercial. Residential and commercial mortgages and their markets differ in a number of ways, such as in the structure of the actual loans and with regard to the characteristics of the securitized markets. Residential mortgage loans are typically taken out by individual households on properties that generate no explicit rental income, since the houses are usually owner occupied. Therefore, the credit risk of residential mortgages depends on the borrower's income and financial position, in addition to the characteristics of the property. In contrast, commercial mortgage loans are largely taken out by corporations or other legal entities. The risk of mortgages on commercial properties often focuses on the rental income generated by the property, which can be used to make the mortgage payments. Another feature of residential mortgage loans is their tendency to be more homogeneous in terms of their price behavior than commercial loans.

\section*{Fixed-Rate Mortgages}
A fixed-rate, constant payment, fully amortized loan has equal monthly payments throughout the life of the loan. These loans give the residential mortgage market some of its unique characteristics, as discussed later in the session. The fixed-rate and constant payment nature of these loans make the value of the loans subject to interest rate risk and inflation risk. The monthly payments of a fixed-rate loan can be calculated using the formula for the present value of a constant annuity, with the payment amount factored into the left-hand side of Equation 1:


\begin{equation*}
M P=M B \times\left\{i /\left[1-(1+i)^{-n}\right]\right\} \tag{1}
\end{equation*}


where $M P$ is the constant monthly payment, $M B$ is the mortgage balance or total amount borrowed, $i$ is the monthly interest rate (defined as the stated annual rate divided by 12 ), and $n$ is the number of months in the term of the loan.

An important feature of the fixed-rate mortgage is that the proportion of the monthly payments that is applied against the principal and the proportion that consists of interest charges change over the lifetime of the loan, as the outstanding principal balance declines. In the early years of the mortgage, the largest portions of the payments represent interest payments rather than principal repayments. The interest component is equal to the monthly interest rate multiplied by the outstanding loan amount from the beginning of the current month or the end of the previous month.

The principal repayment component of the monthly mortgage payment is the residual between the total payment and the interest portion. Reduction in principal due to payments is known as amortization. The next exhibit, illustrates the amortization schedule for the example just presented: a $\$ 100,000$ mortgage with a fixedrate ( $0.5 \%$ a month) constant payment ( $\$ 644.30$ per month) that is fully amortized. An asset is fully amortized when its principal is reduced to zero.

As can be seen in the next exhibit, the first interest payment is equal to $\$ 100,000 \times 0.5 \%=\$ 500.00$. Given that the fixed monthly mortgage payment is $\$ 644.30$, the principal repayment in the first month will be $\$ 644.30-\$ 500.00=\$ 144.30$, and the end-of-month mortgage balance will decline from $\$ 100,000$ to $\$ 99,855.70$.

Amortization Schedule for a Fixed-Rate (6\% per year), Constant Payment ( $\$ 644.30$ per month), Fully Amortized 25 -Year Mortgage of $\$ 100,000$, Assuming No Unscheduled Principal Payments

\begin{center}
\begin{tabular}{|c|c|c|c|c|c|}
\hline
Month & Beginning-of-Month Mortgage Balance & Mortgage Payment & Interest Payment & Principal Payment & End-of-Month Mortgage Balance \\
\hline
1 & $\$ 100,000.00$ & $\$ 644.30$ & $\$ 500.00$ & $\$ 144.30$ & $\$ 99,855.70$ \\
\hline
2 & $\$ 99,855.70$ & $\$ 644.30$ & $\$ 499.28$ & $\$ 145.02$ & $\$ 99,710.68$ \\
\hline
3 & $\$ 99,710.68$ & $\$ 644.30$ & $\$ 498.55$ & $\$ 145.75$ & $\$ 99,564.93$ \\
\hline
$\vdots$ & $\vdots$ & $\vdots$ & $\vdots$ & $\vdots$ & $\vdots$ \\
\hline
59 & $\$ 90,318.56$ & $\$ 644.30$ & $\$ 451.59$ & $\$ 192.71$ & $\$ 90,125.86$ \\
\hline
60 & $\$ 90,125.86$ & $\$ 644.30$ & $\$ 450.63$ & $\$ 193.67$ & $\$ 89,932.18$ \\
\hline
61 & $\$ 89,932.18$ & $\$ 644.30$ & $\$ 449.66$ & $\$ 194.64$ & $\$ 89,737.55$ \\
\hline
$\vdots$ & $\vdots$ & $\vdots$ & $\vdots$ & $\vdots$ & $\vdots$ \\
\hline
299 & $\$ 1,279.96$ & $\$ 644.30$ & $\$ 6.40$ & $\$ 637.90$ & $\$ 642.06$ \\
\hline
300 & $\$ 642.06$ & $\$ 644.30$ & $\$ 3.21$ & $\$ 641.09$ & $\$ 1$ (rounded) \\
\hline
\end{tabular}
\end{center}

Fixed-rate residential mortgages are valued similarly to bonds. As the market level of interest rates increases, the present value of the future payments declines. If the appropriate market interest rate remains at $6 \%$ per year, the market value of the mortgage would be equal to the outstanding principal balance. However, at a new and higher market interest rate, the value of the mortgage would drop below the principal balance.

The above exhibit, illustrates the amortization of a fixed-rate mortgage in the absence of unscheduled principal repayments. If the borrower makes unscheduled principal payments, which are payments above and beyond the scheduled mortgage payments, the mortgage's balance will decline more quickly than illustrated in the Residential Mortgages lesson in the Amoritzation Schedule for a Fixed-Rate Mortgage exhibit, and the mortgage will terminate early. In traditional mortgages, payments that exceed the required payment reduce the principal payment but do not lower required subsequent payments until the mortgage is paid off.

Unscheduled principal payments cause a wealth transfer between the borrower and the lender, depending on the relationship between the mortgage's interest rate and current market interest rates. When market rates are lower than the mortgage rate, unscheduled principal payments generally benefit the borrower and harm the lender. The lender receives additional cash flows that, if reinvested at prevailing interest rates, will earn less return than the mortgage offers. Borrowers can make unscheduled prepayments to reduce the total interest costs of their mortgage by an amount greater than the amount that they could earn from interest income in the market. Thus, borrowers have an incentive to make prepayments on mortgages when interest rates decline below the mortgage's rate.

When market rates are higher than the mortgage rate, unscheduled principal payments generally benefit the lender and harm the borrower. The lender receives additional cash flows that can be reinvested at prevailing interest rates that will earn more return than the mortgage offers. Borrowers are harmed by prepaying a low-rate mortgage when they could earn more by investing in the market at the new and higher rates. Borrowers may make such payments due to idiosyncratic reasons, such as selling the property, refinancing due to liquidity problems, or other personal reasons.

The ability of the borrower to make or not make unscheduled principal payments is an option to the borrower: the borrower's prepayment option. The option is a call option in which the mortgage borrower, much like a corporation with a callable bond, can repurchase its debt at a fixed strike price. Therefore, a mortgage borrower benefits from increased interest rate volatility. The lender, on the other hand, has written the call option and suffers from increased interest rate volatility. The key point is that fixed-rate mortgage investing has interest rate risk that includes the interest rate risk of the borrower's prepayment option. While the prepayment option may be viewed as a call option on the value of the debt, the option may also be viewed as a put option on interest rates. Just like a call option on a price, a put option on a rate rises in value when rates fall and prices rise. Both option views illustrate that during times of declining interest rates and rising fixed-income prices, it may be to the borrower's advantage to refinance the loan, replacing the current high-interest-rate, high-priced debt with a new loan at a lower interest rate.

It must be remembered, however, that options are not free goods. The lender demands compensation for writing the prepayment call option to the borrower.

Although the option may not be explicitly priced as part of the loan, it is implicitly priced in the form of a higher interest rate on the mortgage loan or in up-front points, or fees, charged to the borrower.

\section*{Interest-Only Mortgages}
Some fixed-rate mortgages are interest-only mortgages, which means that the monthly payments consist entirely of interest payments for some initial period. The two most widely used interest-only loans are both 30 -year mortgages. The first begins with a 10 -year interest-only period, followed by a 20 -year fully amortizing period; this type of loan is known as 10/20. The second begins with a 15-year interest-only period, followed by a 15-year fully amortizing period; this type of loan is known as 15/15. A 25 -year mortgage with a 10 -year interest-only period would be referred to as a 10/15 interest-only mortgage. In each case, the interest-only payments are equal to the product of the principal balance and the monthly rate. When the mortgage commences amortization, the payments are computed like fixed-payment mortgages except that they are based on the remaining and shorter period of the mortgage's life.

Interest-only mortgages have the potential advantage that the monthly payments during the interest-only period are lower than those in the case of a fully amortized loan ( $\$ 500$ versus $\$ 644.30$ ). However, during the amortization period the monthly payments are higher ( $\$ 843.86$ versus $\$ 644.30$ ), as the borrower has fewer years to amortize the loan (15 years versus 25 or 30 years).

\section*{Variable-Rate Mortgages}
Particularly during the period from 2004 to 2006, mortgage markets shifted toward increased use of variable-rate or adjustable-rate mortgages (ARMs) and away from fixed-rate mortgages. Although the initial payments in the case of ARMs are calculated in the same manner as conventional fixed-rate loans, the payments are not necessarily constant during the life of the loan, as the interest rate is periodically adjusted by the lender, generally to reflect changes in underlying short-term market interest rates, as prescribed in the mortgage agreement.

Consider a hypothetical variable-rate mortgage in which the interest rate changes each month and is set equal to the one-month interest rate prevailing at the time.

In this extreme and hypothetical example, the mortgage lender would receive the same interest as if the lender had placed funds in a series of short-term (onemonth) interest-bearing accounts. Therefore, investors in this hypothetical variable-rate mortgage would have the same interest rate risk that investors face in the short-term money market. In effect, and ignoring default and reset limits, a variable-rate mortgage that fully resets every $X$ months behaves to the lender or mortgage investor like a series of investments in short-term fixed-income accounts, each with a maturity of $X$ months.

The market value of a variable-rate mortgage (absent default) behaves like a money market account to the extent that the mortgage's rate adjusts quickly and without limits. Therefore, an obvious advantage of a variable-rate type of mortgage to a lender is that it protects the lender from the valuation fluctuations due to interest rate changes experienced with fixed-rate mortgages. An obvious disadvantage of variable-rate loans is the risk to the borrower that interest rates will increase. A variable-rate loan provides the advantage to the borrower of substantially lower initial interest rates.

The next exhibit demonstrates the payment changes for a variable-rate mortgage.

Suppose that a $\$ 100,000,25$-year mortgage is taken out. The initial interest rate that will apply for the first full year is $7 \%$, compounded monthly. This implies that the monthly mortgage payment during the first year is $\$ 706.78(n=12 \times 25=300, i=7 \% / 12, P V=+/-\$ 100,000, F V=\$ 0$, solve for $P M T)$ and that at the end of the first year, the mortgage balance will be $\$ 98,470.30$ ( $n=12 \times 24=288, i=7 \% / 12, P M T=\$ 706.78, F V=\$ 0$, solve for $P V$ ). The variable-rate mortgage begins the same as a fixed-rate mortgage in terms of computational methods, although it usually has a lower initial rate. The payments change when the variable rate changes, as illustrated in the next exhibit.

The monthly payments of the variable-rate mortgage in the exhibit below are based on an adjustable rate that can vary from the $7 \%$ initial rate beginning in month 13 (end of year 1). This variable rate, which applies for the whole next year, is based on an index rate. An index rate is a variable interest rate used in the determination of the mortgage's stated interest rate. Index rates fluctuate freely in the money markets and can be based, for example, on the yield of one-year Treasury securities. Variable rates typically include a margin rate. A margin rate is the spread by which the stated mortgage rate is set above the index rate. (This should not be confused with the same term used to describe a rate associated with margin debt in a brokerage account.)

Amortization Schedule for a Variable-Rate, Variable Payment, Fully Amortized 25-Year

Mortgage of $\$ 100,000$, Assuming No Unscheduled Principal Payments

\begin{center}
\begin{tabular}{|c|c|c|c|c|c|c|}
\hline
Year & Index Rate + & Margin Rate $=$ & Interest Rate & Beginning of Year & Monthly & End of Year \\
\hline
1 &  &  & $7.0 \%$ & $\$ 100,000.00$ & $\$ 706.78$ & $\$ 98,470.30$ \\
\hline
2 & $8.5 \%$ & $1.5 \%$ & $10.0 \%$ & $\$ 98,470.30$ & $\$ 903.36$ & $\$ 97,430.75$ \\
\hline
3 & $10.0 \%$ & $1.5 \%$ & $11.5 \%$ & $\$ 97,430.75$ & $\$ 1,006.05$ & $\$ 96,515.25$ \\
\hline
4 & $8.0 \%$ & $1.5 \%$ & $9.5 \%$ & $\$ 96,515.25$ & $\$ 872.94$ & $\$ 95,150.13$ \\
\hline
\end{tabular}
\end{center}

This example uses a margin rate of 1.5\%. This margin rate is determined as part of the original terms of the mortgage and is added to compensate for the expected or assessed degree of risk, including interest rate risk and the riskiness of the borrower.

The total interest rate is the sum of the index rate and the margin rate.

The process of determining payments continues into the third year, with the interest rate and longevity of the mortgage being adjusted at each reset in order to determine the new payment. The mortgage balance at the end of the second year is equal to $\$ 97,430.75$, which is determined from the amortization, assuming no unscheduled principal payments (using a financial calculator: $n=12 \times 23=276, i=10 \% / 12, P M T=\$ 903.36, F V=\$ 0$, solve for $P V$ ). The mortgage balance at the end of the second year, $\$ 97,430.75$, is then used, along with the third-year mortgage rate, $11.5 \%$, to compute the payments for the third year. This process of computing the remaining mortgage balance and using that balance to compute the new monthly payments, considering the new interest rate that applies each year, continues over the life of the variable-rate portion of the mortgage.

It is also common for interest rates on ARMs to be capped. An interest rate cap is a limit on interest rate adjustments used in mortgages and derivatives with variable interest rates. In the previous example, suppose that the increase in interest rates was capped to $2 \%$ during any one year and to a total increase of $4 \%$ during the life of the mortgage. The effect of these interest rate caps on the mortgage balance and on the monthly payments would be to prevent the mortgage's rate from rising above the annual or lifetime caps. Thus, with the given $2 \%$ cap, the mortgage rate for the second year would be capped at $9 \%$ and would be used in place of $10 \%$ for the second-year calculations. Further, the mortgage rate for the third year would be capped at $11 \%$ and would be used in place of $11.5 \%$ for the third-year calculations due to the limitation of lifetime interest rate increases to $4 \%$ over the mortgage's lifetime $(7 \%+4 \%=11 \%)$ as well as the $2 \%$ per year limitation. Obviously, the borrower must pay for these caps in the form of a higher initial mortgage rate or index rate to compensate the lender for the potential negative effects that the cap rates may have on the lender's future income from the mortgage if future uncapped interest rates were to rise above the mortgage's cap.

\section*{Other Types of Mortgages}
Fixed-rate and variable-rate mortgage loans have other variations as well. For example, it is common, particularly with variable-rate mortgages, for the initial interest rate to be low when compared to short-term market rates and for that low rate to be fixed for an initial period. After this period, the mortgage rate is calculated based on the lender's standard variable interest rate. Another type of loan with relatively low initial payments is a graduated payment loan. This loan is made at an initially fixed interest rate that is relatively low but scheduled to increase slowly over the first few years. Both of these variations are designed to help borrowers qualify for the loan and be able to make the initial payments on the loan. Historically, defaults on mortgage loans tend to be concentrated in the first few years of a loan. Therefore, by offering a reduced rate for the initial years, the lender is not only using the lower rate as a tool for attracting business but also attempting to mitigate the default risk in the early years of the mortgage. Note that in an environment of steadily increasing housing prices, if a mortgage defaults several years after being initiated, the losses to the lender should be minimal, since the collateral would most likely exceed the loan amount.

Another variation in variable-rate mortgages provides payment flexibility. An option adjustable-rate mortgage (option ARM) is an adjustable-rate mortgage that provides borrowers with the flexibility to make one of several possible payments on their mortgage every month. The payment alternatives from which borrowers may select each month typically include an interest-only payment, one or more payments based on given amortization periods, or a prespecified minimum payment amount. Thus, borrowers are granted flexibility to make lower payments than would be required in a traditional mortgage. Option ARMs typically offer low introductory rates and may allow borrowers to defer some interest payments until later years.

One feature of option ARMs that can exacerbate default risk is that they may not be fully amortizing. In fact, when an option ARM allows payments that are below the interest charged on the loan, the loan has negative amortization. Negative amortization occurs when the interest owed is greater than the payments being made such that the deficit is added to the principal balance on the loan, causing the principal balance to increase through time. This negative amortization can generate higher probabilities of default from borrowers taking on too much debt or failing to prepare for future payments.

A further mortgage variation is a loan that includes some form of balloon payment. A balloon payment is a large scheduled future payment. Rather than amortizing a mortgage to $\$ 0$ over its lifetime (e.g., 25 years), the mortgage is amortized to the balloon payment. In other words, at the end of the loan, there is an outstanding principal amount due that is equal to the balloon payment. The balloon payment allows for a lower monthly payment, given the same mortgage rate, since the mortgage is not fully amortized to $\$ 0$. Balloon payments due in a relatively short time period (compared to traditional mortgage maturities of 15 to 30 years) may lower the interest rate risk to the lender and permit a lower mortgage rate.

An extreme example of a balloon payment mortgage is when the loan payments are only interest, which means that no regular principal repayments are required. Therefore, at the end of the loan, all the capital is due. In the previous example, the mortgage's initial value of $\$ 100,000$ would be inserted as the balloon payment, or FV. The remaining payment would simply be the interest on $\$ 100,000$ at $6 \%$ per year, or $\$ 500$ per month. This interest-only form reduces monthly payments.

In an ideal scenario, the capital appreciation of the actual property's value will be substantial, and the borrower will gain substantial equity in the property even though the principal amount of the mortgage remains constant.

\section*{Residential Mortgages and Default Risk}
Default risk is dispersion in economic outcomes due to the actual or potential failure of a borrower to make scheduled payments. For most residential mortgages, the full repayment of the mortgage is backed by a public or private guarantee, such that mortgage investors are focused on interest rate risk rather than default riskthat is, mortgages may be insured by a governmental entity or a commercial insurance company that specializes in backing mortgage loans. These firms that insure the performance of the borrower on mortgage loans will reimburse the lender if the borrower does not pay principal and interest as scheduled and the proceeds from the property sale do not fully repay the balance on the mortgage loan.

Insured mortgage loans are generally extended based on an analysis of the underlying property and the creditworthiness of the borrower. However, especially in the years prior to the 2007 global credit crisis, increasing percentages of newly issued mortgages were uninsured and had borrowers with relatively high credit risk. Uninsured mortgages with borrowers of relatively high credit risk are generally known as subprime mortgages. Prime mortgages are offered to borrowers with lower levels of credit risk and higher levels of creditworthiness.

Analysis of the creditworthiness of the borrower and the protection provided to the lender by the underlying real estate asset is fundamental analysis that generally relies substantially on ratio analysis. Ratios regarding the creditworthiness of the borrower often focus on the ratio of some measure of the borrower's housing expenses to some measure of the borrower's income. For example, a debt-to-income ratio is computed as the total housing expenses (including principal, interest, taxes, and insurance) divided by the monthly income of the borrower, and it might be required to be below a specified percentage for the borrower to qualify for mortgage insurance.

The front-end ratio, including only housing costs, may be limited to $28 \%$ of gross income; the back-end ratio, including both housing costs and other debts, such as credit cards and automobile loans, may be limited to $36 \%$ of gross income. The exact definitions of these types of ratios vary and are part of a larger fundamental analysis that includes indicators of creditworthiness, such as credit scores and credit history.

Fundamental analysis of the real estate property underlying the mortgage typically includes an appraisal and analysis of factors regarding the property, such as availability of services and structural integrity. Ratio analysis is also important in the analysis of the property. Specifically, the loan-to-value ratio (LTV ratio) is the ratio of the amount of the loan to the value (either market or appraised) of the property. Residential mortgages with LTV ratios of $80 \%$ are often viewed as being very well collateralized. LTV ratios of up to $95 \%$ are commonly allowed for insured residential mortgages.


\end{document}