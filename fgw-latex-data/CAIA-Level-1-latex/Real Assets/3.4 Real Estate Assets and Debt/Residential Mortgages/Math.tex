\documentclass[11pt]{article}
\usepackage[utf8]{inputenc}
\usepackage[T1]{fontenc}
\usepackage{amsmath}
\usepackage{amsfonts}
\usepackage{amssymb}
\usepackage[version=4]{mhchem}
\usepackage{stmaryrd}

\begin{document}
\section*{APPLICATION A}
Question : Assume that a borrower takes out a $\$ 100,000,25$-year mortgage (300 months), at a $6 \%$ annual nominal interest rate (a monthly interest rate of $6 \% / 12$, or $0.5 \%$ ). What is the mortgage's monthly payment?

\section*{Answer and explanation}
The monthly payments (principal plus interest) can be calculated using Equation 1 directly, as follows:

$$
M P=\$ 100,000 \times\left\{0.005 /\left[1-(1.005)^{-300}\right]\right\}=\$ 644.30
$$

Using a financial calculator, the monthly mortgage payment is calculated by inputting the following values: $n$ (number of periods) $=12 \times 25=300$ months, $i$ (interest rate per period) $=6 \% / 12=0.5 \%, P V$ (present value) $=+/-\$ 100,000, F V$ (future value) $=\$ 0$, and solving for (compute) $P M T$ (payment).

The $P V$ is entered as either a positive or a negative number, depending on the calculator that is used. Note that some financial calculators require that the interest rate of $0.5 \%$ be entered as .005 and some as . 5 .

\section*{APPLICATION B}
Question : What would be the outstanding mortgage balance at the start of month 61 in terms of remaining principal of a $\$ 100,000,25$-year mortgage (300 months), at a $6 \%$ annual nominal interest rate?

\section*{Answer and explanation}
As shown in the above exhibit, the outstanding mortgage balance at the start of month 61 in terms of remaining principal is $\$ 89,932.18$, five years after the loan has been taken out. This amount does not correspond exactly to a present value computation of the balance using the exact payment amount of $\$ 644.30$ (using a financial calculator: $n=12 \times 20=240, i=6.0 \% / 12=0.5 \%, P M T=\$ 644.30, F V=\$ 0$, solve for $P V$. The reason is that mortgage payments are values that in practice are rounded to the nearest cent, and mortgage amortization computations (such as the above exhibit) are based on this rounded payment amount (\$644.30) rather than a more exact payment amount (\$644.3014).

\section*{APPLICATION C}
Question :Suppose that the market interest rate for the mortgage in the previous Amortization Schedule exhibit rises to $7.5 \%$. What is the market value of the mortgage, assuming it is the start of month 61 ?

\section*{Answer and explanation}
The market value is equal to $\$ 79,978.33$ (using a financial calculator: $n=12 \times 20=240, i=7.5 \% / 12=0.625 \%, P M T=\$ 644.30, F V$ $=\$ 0$, solve for $P$ ). At a new and lower market interest rate of $4.5 \%$, the market value of the mortgage is equal to $\$ 101,841.56$ (found as before except that $i=$ $4.5 \% / 12=0.375 \%)$. These values illustrate that the market value of fixed-rate mortgages, as fixed-income securities, varies inversely with market interest rates.

\section*{APPLICATION D}
Question :Consider a $\$ 100,000$, 25 -year mortgage that is structured as a $10 / 15$ interest only mortgage, with an annual rate of $6 \%$. What would the payments be for the first 10 and the last 15 years?

\section*{Answer and explanation}
For the first 10 years, the monthly payments, which are interest only, would be $\$ 500(\$ 100,000 \times 6.0 \% / 12)$. Between years 11 and 25 , the monthly fixed payment necessary to fully amortize the mortgage for the remaining 15 years would be $\$ 843.86$ (using a financial calculator: $n=12 \times 15=180, i=6 \% / 12=0.5 \%, P V=+/-$ $\$ 100,000, F V=\$ 0$, solve for $P M T$ ).

\section*{APPLICATION E}
Question :What would the monthly payment be for the mortgage in the above exhibit in the second year, when the mortgage's rate climbs to $10.0 \%$ ? Note that it is necessary to decrease the mortgage's original principal to reflect amortization and decrease the months remaining by 12 , to 288 .

\section*{Answer and explanation}
From the above exhibit, the monthly mortgage payment that the borrower would have to make during the second year, for which a higher index rate of $8.5 \%$ applies, is equal to $\$ 903.36(n=12 \times 24=288, i=10 \% / 12, P V=+/-\$ 98,470.30, F V=\$ 0$, solve for $P M T)$. Notice that the increase in interest rates between the first year and the second year has caused a substantial increase $(27.81 \%)$ in the monthly payment that the borrower is obligated to make.

\section*{APPLICATION F}
Question :To illustrate balloon payments, assume that the borrower and the lender in the original example decide that the $\$ 100,000$ loan made at the fixed rate of $6 \%$ per year compounded monthly for 25 years will amortize to a $\$ 70,000$ balance on the 25 -year maturity date rather than being fully amortized to $\$ 0$. This amount of $\$ 70,000$ is known as a balloon payment and will be due at the end of 25 years.

\section*{Answer and explanation}
In this case, the monthly payment would be equal to $\$ 543.29$ (using a financial calculator: $n=12 \times 25=300, i=6 \% / 12=0.5 \%, P V=+/-\$ 100,000, F V=\$ 70,000$, solve for $P M T$ ). Notice that the $\$ 543.29$ monthly payment is less than the $\$ 644.30$ payment that was computed for the case of the fully amortizing loan, even though the interest rates in both mortgages are equal to $6 \%$.


\end{document}