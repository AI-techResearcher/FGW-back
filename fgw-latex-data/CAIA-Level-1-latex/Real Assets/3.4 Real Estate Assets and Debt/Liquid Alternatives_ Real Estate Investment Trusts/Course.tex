\documentclass[11pt]{article}
\usepackage[utf8]{inputenc}
\usepackage[T1]{fontenc}
\usepackage{amsmath}
\usepackage{amsfonts}
\usepackage{amssymb}
\usepackage[version=4]{mhchem}
\usepackage{stmaryrd}

\begin{document}
Liquid Alternatives: Real Estate Investment Trusts

This lesson introduces the concept of REITs (real estate investment trusts). The final lesson of the Real Estate Assets and Debt session reports historical risks and returns of mortgage REITs. Although REITs are not popular in all countries, they are central to illustrating and understanding central points with regard to real estate, liquidity, and liquid alternatives.

Legislation facilitating REITs dates back to 1960 in the United States. Perhaps due to their long-term popularity, REITs are not usually included in lists of liquid alternatives. But REITs fit the definition of a liquid alternative very well; they are publicly traded vehicles that allow retail access to an asset class (real estate) that is often considered to be an alternative asset class.

A real estate investment trust (REIT) is an entity structured much like a traditional operating corporation, except that the assets of the entity are almost entirely real estate. Because most major REITs are listed on major stock exchanges, they are a simple and liquid way to bring real estate exposure into an investor's portfolio. They operate in much the same fashion as mutual funds, especially closed-end mutual funds. They pool investment capital from many small investors and invest the larger collective pool in real estate properties that would not be available to the small investor.

Equity REITs invest predominantly in equity ownership within the private real estate market. Mortgage REITs invest predominantly in real estate-based debt. REITs that invest substantially in both markets have been termed hybrid REITs - a category that has shrunk into very limited use. There are three key advantages of REITs as vehicles to real estate investment. First, REITs provide management services in the selection and operation of properties. Second, REITs provide liquid access to an illiquid asset class. Investors can add to or trim their exposure to real estate quickly and easily through purchase and sale of shares in REITs. Finally, REITs avoid double taxation of income that comes with paying taxes at both corporate and individual levels. REITs avoid corporate income taxation to the extent that they distribute their income and capital gains to their shareholders. Distributions from REITs tend to be subject to income taxation at the individual level.

These potential advantages to REITs may be offset-especially to large, sophisticated real estate investors-by disadvantages, including management fees and lack of influence over management. Also, some analysts argue that exchange-traded real estate investments (i.e., REITs) have greater price risk than private real estate investments because the market prices of REITs take on the volatility of financial markets. Others argue that market prices of REITs reflect the true price risk of real estate, which is masked by other valuation methods, such as appraisals.

An investor can use REITs to form asset allocations to real estate as an asset class. The diverse nature of REITs allows investors to refine their asset allocation within real estate by tilting their real property exposure to particular parts of the real estate market. For example, an investor can choose different categories of REITs, such as mortgage-based versus equity-based REITs, and various subcategories of real estate, such as office buildings, health-care facilities, shopping centers, and apartment complexes.

REITs offer professional asset management of real estate properties to passive investors. These real estate professionals know how to acquire, finance, develop, renovate, and negotiate lease agreements with respect to real estate properties to get the most return for their shareholders. REITs are also overseen by independent boards of directors, which are charged with seeing to the best interests of the shareholders. This provides a level of corporate governance protection similar to that employed for other public companies. REITs strive to provide a consistent dividend yield for their shareholders.

To enjoy the freedom from corporate income taxation in the United States, REITs are subject to the following two main restrictions: $75 \%$ of the income they receive must be derived from real estate activities, and they must pay out $90 \%$ or more of their taxable income in the form of dividends. Other restrictions relate to the ownership structure of the REIT, such as restrictions on the percentage of the shares that can be held directly or indirectly by a small group of investors. As long as a REIT is in compliance with the relevant restrictions, it may deduct dividends from its income in determining its corporate tax liability, which means it pays corporate income taxes only on the retained income. The returns of mortgage REITs are used in the next lesson to indicate the general risks and returns to mortgage investments.


\end{document}