\documentclass[11pt]{article}
\usepackage[utf8]{inputenc}
\usepackage[T1]{fontenc}
\usepackage{amsmath}
\usepackage{amsfonts}
\usepackage{amssymb}
\usepackage[version=4]{mhchem}
\usepackage{stmaryrd}

\begin{document}
Commercial Mortgages

Commercial mortgage loans are loans backed by commercial real estate (multifamily apartments, hotels, offices, retail and industrial properties) rather than owneroccupied residential properties. In contrast to the relative standardization of residential mortgage loans, there is far greater variety when it comes to mortgages in the commercial sector, a fact that has hindered trading of commercial mortgages in secondary markets.

\section*{Commercial Mortgage Characteristics}
Mortgage loans on commercial real estate differ in a number of respects from those in the residential market. Almost all commercial loans involve some form of balloon payment on maturity, since the loan term is almost always shorter than the time required to fully amortize the loan at the required payment. Furthermore, due to the large size of commercial real estate projects, few individuals participate in this market as borrowers or lenders. Most of the borrowers are commercial or financial firms that possess greater financial sophistication than the average homeowner.

An important distinction when examining commercial mortgages is the nature of the loan and, in particular, whether it is for completed projects or for development purposes. Most development loans are shorter-term and phased, wherein the developer draws down funds only as required during the construction phase. This is in contrast to loans for existing properties, which tend to have a longer horizon, usually in the region of 5 to 10 years, and for which the full amount of the loan is drawn immediately.

\section*{Commercial Mortgage Default Risk}
Whereas residential mortgage investors are primarily concerned about interest rate risk and prepayment rates, commercial mortgage investors typically face substantial default risk related to the credit risk of the borrower as well as the price risk of the underlying collateral (i.e., property). Default risk is related to covenants and recourse.

In general, the covenants in a commercial mortgage are more detailed than those in a corresponding residential loan document. Covenants are promises made by the borrower to the lender, such as requirements that the borrower maintain the property in good repair and continue to meet specified financial conditions. Failure to meet the covenants can trigger default and make the full loan amount due immediately. The view that covenants benefit lenders at the expense of borrowers is naïve. Although covenants lower the credit risk to the lender, they are presumably offered by the borrower in exchange for better terms on the loan (e.g., a lower interest rate). The severity and details of covenants required by lenders vary across firms. To some extent, borrowers choose to offer particular covenants by selecting lenders that demand those covenants, because they prefer the lower rates of loans attached to those covenants.

Commercial loans tend to contain far more detail concerning such issues as the seniority of the loan. As with all debts, particularly at the corporate level, lenders need to know their position with respect to seniority in the event of default or financial difficulty. For instance, it may be the case that if the loan is senior or is the original debt (also called the first lien or first mortgage) on the property, the lender has to provide permission before subsequent debts (such as second liens or second mortgages) can be incurred. Another key element in any commercial debt deal is the recourse that the lender has to the borrowing entity. Recourse is the set of rights or means that an entity such as a lender has in order to protect its investment. Recourse may include how the loan is secured, such as the potential ability of the lender to take possession of the property in the event of a default and the potential ability of the lender to pursue recovery from the borrower's other assets. Another type of covenant included in many commercial mortgages but not included in residential mortgages is restriction on the distribution of the rental income from the property, with perhaps a specified proportion being redirected to a reserve account rather than paid straight to the owner. A lender may also insist on a minimum deposit or balance to be maintained in an account with the lender.

In addition to explicit covenants with regard to the debt, commercial mortgages may come attached with a proviso (i.e., condition or limitation) relating to the management and operation of the property. Lenders may insist that minimum levels of cash flow, net operating income, and earnings before interest and taxes need to be achieved or that rental levels may not fall below a previously specified level. Such provisions are designed to ensure that the property is able to generate sufficient income on an ongoing basis for the borrower to service the loan. Lenders may even insist on having some form of either control or consultation with regard to leasing policies, such as examination of new lease terms or credit checks on potential tenants.

Finally, in order to mitigate the risk to which they are exposed, lenders commonly use a cross-collateral provision, wherein the collateral for one loan is used as collateral for another loan. For example, say a corporation has borrowed twice, securing each loan with a property; with a cross-collateral provision, both properties would be used as collateral for both loans. If the corporation fully pays off one of the loans and wishes to sell the related property, the lender may prevent the sale because the property is still serving as collateral to the other loan.

\section*{Financial Ratios for Commercial Mortgages and Default Risk}
Whereas a large number of residential mortgages are insured against default risk, commercial mortgages are generally exposed to default risk. Therefore, commercial mortgage investing usually involves fundamental analysis of default risk. Further, while fundamental analysis of residential mortgage default risk focuses on the credit risk of the borrower, fundamental analysis of commercial mortgage default risk focuses primarily on the role of rental income from the property in covering the mortgage payments.

As with residential loans, the LTV ratio, both at the origination of the loan and on an ongoing basis, is a key measure used by lenders. The LTV ratio at which a lender will issue a loan varies depending on the lender, the property sector, and the geographic market in which the property is located, as well as the stage of the real estate cycle and other circumstances, such as the borrower's creditworthiness. Financial institutions tend to lend at lower LTV ratios in the commercial sector than in the residential sector. It would be rare for senior debt in commercial properties to be lent at an LTV ratio in excess of $75 \%$. Commercial borrowers, then, typically need a larger down payment or equity contribution than do borrowers purchasing residential real estate.

Given that commercial real estate generates rental income, lenders also examine a variety of income-based measures, in addition to the LTV ratio, when assessing the credit risk of a loan. For instance, lenders typically examine the interest coverage ratio, which can be defined as the property's net operating income divided by the loan's interest payments. The interest coverage ratio allows lenders to analyze the level of protection they have in terms of a borrower's ability to service a debt from the property's operating income. Senior secured debt lenders usually require that borrowers meet a minimum coverage ratio of 1.2 to 1.3. This means that the\\
projected net income must be at least $20 \%$ to $30 \%$ greater than the projected interest payments. A related measure is the debt service coverage ratio (DSCR), which is the ratio of the property's net operating income to all loan payments, including the amortization of the loan. A final typically used key ratio with an even broader definition of expenses is the fixed charges ratio. The fixed charges ratio is the ratio of the property's net operating income to all fixed charges that the borrower pays annually. The risk of default needs to be constantly monitored by mortgage investors. Research by Esaki notes that default rates of commercial mortgages are highly cyclical and tend to be explained by both market conditions and lender policies. ${ }^{1} \mathrm{H}$. Esaki, "Commercial Mortgage Defaults: 1972-2000," Real Estate Finance (Winter 2002): 43-52. Loans taken out, for example, during the real estate booms of the late 1980s and mid-2000s-periods that witnessed not only a booming real estate market but also liberal lending policies (including LTV ratios greater than $100 \%$, along with fewer or weaker covenants)-eventually recorded high default rates. In contrast, loans issued during the 1990s experienced much lower default rates, due in part to more conservative lending policies during that period. A major difference between residential and commercial lending is that it is far more likely that defaulting commercial loans will be restructured rather than moved directly to foreclosure, due in part to the size of the individual loans. Esaki, for instance, finds that $40 \%$ of defaulting commercial loans were restructured. ${ }^{2}$ lbid.


\end{document}