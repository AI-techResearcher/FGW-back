\documentclass[11pt]{article}
\usepackage[utf8]{inputenc}
\usepackage[T1]{fontenc}

\begin{document}
Advantages, Disadvantages, and Styles of Real Estate Investments

This lesson introduces potential determinants of allocations to institutional-quality real estate assets.

\section*{Five Potential Advantages of Real Estate}
There are five common attributes of real estate that can encourage its inclusion in an investment portfolio:

\begin{enumerate}
  \item Potential to offer absolute returns

  \item Potential to hedge against unexpected inflation

  \item Potential to provide diversification with stocks and bonds

  \item Potential to provide cash inflows

  \item Potential to provide income tax advantages

\end{enumerate}

These potential advantages, the first three of which are related to portfolio risk, do not necessarily come without costs. In particular, to the extent that markets are competitive and efficient, market prices of real estate will tend to adjust, such that any risk-reducing advantages will be offset by lower expected returns. However, some of the disadvantages of private real estate ownership may lead to higher expected returns in the form of premiums for bearing risks, such as liquidity.

This list of potential advantages to real estate investment is not comprehensive. For example, another motivation could be to own all or part of a trophy property that offers name recognition, prestige, and enhanced reputation to the owner, such as a large, high-quality office property in a prominent location.

\section*{Three Potential Disadvantages of Real Estate}
There are also aspects of real estate that can discourage its inclusion in an investment portfolio (unless the investor receives appropriate compensation in the form of higher expected returns). Included are these three potential disadvantages:

\begin{enumerate}
  \item Heterogeneity

  \item Lumpiness

  \item Illiquidity

\end{enumerate}

Real estate is a highly heterogeneous asset. This heterogeneity may be particularly burdensome in the initial and ongoing due diligence processes. As indicated in the previous lesson, Categories of Real Estate, real estate differs, as evidenced by the numerous categories of real estate. However, real estate can also be highly heterogeneous within its subcategories due to instances where there may be tremendous differences in their economic nature.

For example, consider two office buildings that are similar in size, construction, and location. The first office building has a 20-year noncancellable lease with a large well-capitalized and well-hedged corporation. The lease essentially locks in the rental revenues for the entire property for the next two decades. In this case, the annual income of the property will be similar to that of a corporate bond, and the value of the property to the investor will tend to fluctuate in response to the same factors affecting the value of a corporate bond issued by the tenant (i.e., riskless interest rate changes and changes in the credit spread on the debt of the tenant).

The second office building in the example is vacant. Both buildings are located in a geographic area with an economy strongly linked to oil prices. The value of this empty real estate asset will be especially sensitive to the supply of and demand for office space in the local real estate market. Thus, the value of this property will be driven by the forces that affect the region's economy-in this case, oil prices. The vacant property's value may behave more like equity prices in general and like oil stock prices in particular.

This example shows that assets within a specific type of real estate (e.g., private commercial real estate) may behave like debt or equity securities depending on the characteristics of the individual properties. A particular property may experience dramatic changes in its investment characteristics due to a specific event, such as the signing or termination of a very-long-term, noncancellable lease.

The second potential disadvantage to private real estate is lumpiness. Lumpiness describes when assets cannot be easily and inexpensively bought and sold in sizes or quantities that meet the preferences of the buyers and sellers. Listed equities of large companies are not lumpy, because purchases and sales can easily be made in the desired size by altering the number of shares in the transaction. Direct real estate ownership may be difficult to trade in sizes or quantities desired by a market participant. The indivisible nature of private real estate assets leads to problems with respect to high unit costs (i.e., large investment sizes) and relatively high transaction costs.

The final major disadvantage relates to the liquidity of private real estate. As a non-exchange-traded asset with a high unit cost, private real estate can be highly illiquid, especially when compared to stocks and bonds. An important implication of illiquidity is its effect on reported returns as well as its added risk challenges.

\section*{Real Estate Styles Overview}
The premier approach to organizing private commercial real estate is through styles of real estate investing. Styles of real estate investing refers to the categorization of real estate property characteristics into core, value added, and opportunistic. In 2003, the National Council of Real Estate Investment Fiduciaries (NCREIF) defined these three styles as a way to classify real estate equity investment or real estate managers. Real estate investment styles assist an asset allocator in organizing and evaluating real estate opportunities, facilitate benchmarking and performance attribution, and help investment managers monitor style drift.

The three NCREIF styles divide real estate opportunities from least risky (core) to most risky (opportunistic), with value added in the middle. In terms of risk, core properties are most bond-like, and opportunistic properties are most equity-like. Core properties tend to offer reliable cash flows each year from rents and lease payments, whereas opportunistic properties offer potential capital appreciation and typically have little or no reliable income. Each of the three styles is more fully described in the following three sections.

\section*{Core Real Estate Style}
Core real estate includes assets that achieve a relatively high percentage of their returns from income, are expected to have low volatility, are the most liquid, most developed, least leveraged, and most recognizable properties in a real estate portfolio, and include five specific categories: office, retail, industrial, multifamily, and hotels. Although these properties have the greatest liquidity, they are not traded quickly relative to traditional investments. Core properties tend to be held for a long time to take full advantage of the lease and rental cash flows that they provide. The majority of their returns comes from cash flows rather than from value appreciation, and very little leverage is applied. Core properties are somewhat bond-like in the reliability of their income.

\section*{Value-Added Real Estate Style}
Value-added real estate includes assets that exhibit one or more of the following characteristics: (1) achieving a substantial portion of their anticipated returns from appreciation in value, (2) exhibiting moderate volatility, and (3) not having the financial reliability of core properties. Value-added properties begin to stray from the more common and lower-risk real estate investments included in the core real estate style. The value-added real estate style includes hotels, resorts, assisted-care living facilities, low-income housing, outlet malls, hospitals, and the like. These properties tend to require a subspecialty within the real estate market to be managed well and can involve repositioning, renovation, and redevelopment of existing properties.

Relative to core properties, value-added properties are anticipated to produce less current income and to rely more on property appreciation to generate total return. However, property appreciation is subject to substantial uncertainty, and value-added properties as a whole have experienced prolonged periods of poor realized appreciation. Value-added properties can also include new properties that would otherwise be core properties except that they are not fully leased. A valueadded property can also be an existing property that needs a new strategy, such as a major renovation, new tenants, or a new marketing campaign. These properties tend to use more leverage and generate a total return from both capital appreciation and income.

Pennsylvania's Public School Employees' Retirement System (PSERS) identifies value-added real estate as follows:

Value-added real estate investing typically focuses on both income and growth appreciation potential, where opportunities created by dislocation and inefficiencies between and within segments of the real estate capital markets are capitalized upon to enhance returns. Investments can include highyield equity and debt investments and undervalued or impaired properties in need of repositioning, redevelopment, or leasing. Modest leverage is generally applied in value-added portfolios to facilitate the execution of a variety of value creation strategies. (PSERS 2007)

\section*{Opportunistic Real Estate Style}
Opportunistic real estate properties are expected to derive most or all of their returns from property appreciation and may exhibit substantial volatility in value and returns. The higher volatility of opportunistic properties relative to the other two styles may be due to a variety of characteristics, such as exposure to development risk, substantial leasing risk, or high leverage.

Opportunistic real estate moves away from a core/income approach to a capital appreciation approach. The majority of the returns from opportunistic properties comes from value appreciation over a three- to five-year period, at which time the investor exits or refinances the property. The capital appreciation of opportunistic real estate can come from development of raw property, redevelopment of property that is in disrepair, or acquisition of property that experiences substantial improvement in prospects through major changes, such as urban renewal.

\section*{Differentiating Real Estate Styles with Eight Attributes}
The three NCREIF styles can be differentiated using eight major real estate attributes, or characteristics. These attributes were developed by NCREIF to distinguish the three types of real estate asset styles:

\begin{enumerate}
  \item Property type (purpose of structure, e.g., general office versus specialty retail)

  \item Life-cycle phase (e.g., new/developing versus mature/operating)

  \item Occupancy (e.g., fully leased versus vacant)

  \item Rollover concentration (tendency of assets to trade frequently)

  \item Near-term rollover (likelihood that rollover is imminent)

  \item Leverage

  \item Market recognition (extent that properties are known to institutions)

  \item Investment structure/control (extent of control and type of governance)

\end{enumerate}

The styles and their attributes can be used to organize individual properties. The exhibit, The Underlying Eight Attributes of the Three Real Estate Styles provides descriptions of the three NCREIF styles using the eight attributes of individual real estate properties. Real estate style analysis can be applied to real estate managers (i.e., portfolios) in addition to individual properties. The exhibit Real Estate Portfolio Style Definitions provides summary descriptions of the characteristics of real estate portfolios classified into the three NCREIF styles.

The Underlying Eight Attributes of the Three Real Estate Styles

\begin{center}
\begin{tabular}{|c|c|c|c|}
\hline
 & Core Attributes & Value-Added Attributes & Opportunistic Attributes \\
\hline
Property type & \begin{tabular}{l}
Major property types only: office, apartments, \\
retail, and industrial \\
\end{tabular} & \begin{tabular}{l}
Major property types plus specialty retail, \\
hospitality, senior/assisted-care housing, \\
storage, low-income housing \\
\end{tabular} & \begin{tabular}{l}
Nontraditional property types, \\
including speculative development for \\
sale or rent and undeveloped land \\
\end{tabular} \\
\hline
Life-cycle phase & Fully operating & Operating and leasing & Development and newly constructed \\
\hline
Occupancy & High occupancy & \begin{tabular}{l}
Moderate to well-leased and/or substantially \\
preleased development \\
\end{tabular} & Low economic occupancy \\
\hline
\begin{tabular}{l}
Rollover \\
concentration \\
\end{tabular} & \begin{tabular}{l}
Tend to be held for a long period of time, forming \\
the central component of the real estate portfolio, \\
\end{tabular} & \begin{tabular}{l}
Moderate rollover concentration-a higher \\
percentage of the assets are held for a short- \\
\end{tabular} & \begin{tabular}{l}
High rollover concentration risk-most \\
of the assets are held for appreciation \\
and resale \\
\end{tabular} \\
\hline
\end{tabular}
\end{center}

\begin{center}
\begin{tabular}{|c|c|c|c|}
\hline
 & Core Attributes & Value-Added Attributes & Opportunistic Attributes \\
\hline
 & \begin{tabular}{l}
which is geared toward generating income rather \\
than sales appreciation \\
\end{tabular} & \begin{tabular}{l}
to intermediate-term sale and rollover into \\
new assets \\
\end{tabular} &  \\
\hline
Near-term rollover & Low total near-term rollover & Moderate total near-term rollover & High total near-term rollover \\
\hline
Leverage & Low leverage & Moderate leverage & High leverage \\
\hline
Market recognition & \begin{tabular}{l}
Well-recognized institutional properties and \\
locations \\
\end{tabular} & Institutional and emerging real estate markets & \begin{tabular}{l}
Secondary and tertiary markets and \\
international real estate \\
\end{tabular} \\
\hline
\begin{tabular}{l}
Investment \\
structure/control \\
\end{tabular} & \begin{tabular}{l}
Investment structures often have substantial direct \\
control \\
\end{tabular} & \begin{tabular}{l}
Investment structures often have moderate \\
control, but with security or a preferred \\
liquidation position \\
\end{tabular} & \begin{tabular}{l}
Investment structures often have \\
minimal control, usually in a limited \\
partnership vehicle and with unsecured \\
positions \\
\end{tabular} \\
\hline
\end{tabular}
\end{center}

Real Estate Portfolio Style Definitions

\begin{center}
\begin{tabular}{|lll|}
\hline
Core Portfolio & Value-Added Portfolio & Opportunistic Portfolio \\
\hline
A portfolio that includes a preponderance of core & A Portfolio that generally includes a mix of core real & A portfolio predominantly of noncore investments \\
attributes. As a whole, the portfolio will have low & estate with other real estate investments that have a & that is expected to derive most of its return from the \\
lease exposure and low leverage. According to the & less reliable income stream. The portfolio as a whole is & appreciation of real estate property values and that \\
NCREIF Open-End Diversified Core Equity (ODCE) & likely to have moderate lease exposure and moderate & may exhibit substantial volatility in the total return. \\
index the maximum leverage of core funds is 35\%. A & leverage. According to the NCREIF Fund index-CEVA & The increased volatility and appreciation risk may \\
low percentage of noncore assets is acceptable. Such & the maximum leverage of value-added funds is 75\%. & be due to a variety of factors, such as exposure to \\
portfolios should achieve relatively high-income & Such portfolios would achieve a substantial portion of & development risk, substantial leasing risk, high \\
returns and exhibit relatively low volatility. The & the return from the appreciation of real estate & degree of leverage, or a combination of moderate \\
portfolio attributes should reflect the risk and return & property values and should exhibit moderate & risk factors. A risk-and-return profile substantially \\
profile of the NCREIF Property Index (NPI). & volatility. A risk-and-return profile moderately greater & greater than the NPI is expected. \\
 & than the NPI is expected. &  \\
\end{tabular}
\end{center}

\section*{Three Purposes of Real Estate Style Analysis}
Real estate styles are essentially locators. In other words, they are categories designed to help identify the space in which each property resides or a real estate manager operates. There are three main reasons for introducing styles into real estate portfolio analysis:

\begin{enumerate}
  \item Performance measurement. Investors continually look for tools that can provide them with a better understanding of an investment's or a sector's objectives and success in accomplishing those objectives. This includes identifying peer groups, return objectives, range of risks, return or performance attribution, and peer performance. Simply put, styles may be useful in identifying appropriate benchmarks.

  \item Monitoring style drift. Tracking style drift is another benefit of assessing the style of a portfolio. It is a fact of investing that portfolio managers occasionally drift from their stated risk, return, or other objectives. Classifying different styles of real estate investments allows an investor to assess the association between a portfolio and its underlying investment products as the portfolio changes over time. Identifying the concentration of a portfolio in terms of the styles for each property facilitates a better understanding of the portfolio's risk level at any given point in time.

  \item Style diversification: The ability to compare the risk-return profile of a manager relative to the manager's style may allow for a better diversification of the portfolio, since an investor may be able to construct a portfolio that has a more robust risk-return profile if there is a better understanding of each real estate manager's style location. Simply put, style may be useful in understanding risk, diversifying risk, and managing or controlling risk.

\end{enumerate}

It should be noted that the preceding real estate styles are primarily applied to private commercial real estate equity, although the concepts can also be applied to publicly traded real estate.


\end{document}