\documentclass[11pt]{article}
\usepackage[utf8]{inputenc}
\usepackage[T1]{fontenc}
\usepackage{amsmath}
\usepackage{amsfonts}
\usepackage{amssymb}
\usepackage[version=4]{mhchem}
\usepackage{stmaryrd}

\begin{document}
\section*{APPLICATION A}
Question : The spot price of a commodity is $\$ 10$ while its six-month futures price is $\$ 10.12$. Given that the annual financing rate is $3 \%$, the annual spoilage rate is $2 \%$, and the storage cost per month is $\$ 0.02$, what is the implied annual convenience yield?

\section*{Answer and Explanation}
Remember, when calculating the forward price of a futures contract, any "costs" to the owner of the comodity are additive, while any "benefits" to the owner are subtracted. This problem provides you with all variables, except for the convenience yield (a "benefit"). Since this is a six-month contract, any annualized numbers provided will need to be multiplied by $6 / 12$ or 0.5 :

Futures Price $=$ Spot Price + Financing Cost + Spoilage Cost + Storage Cost - Convenience Yield

We can further break down the Convenience Yield:

Convenience Yield $=$ CY (given as a percentage $) \times$ Spot Price $\times$ Time

$$
=C Y(\text { given as a percentage }) \times \$ 10 \times \frac{6}{12}
$$

So, we really need to find the $\mathrm{CY}$ as a percentage. We can rearrange the equation by substituting the components of Convenience Yield into the rest of the problem:

$$
\begin{gathered}
C Y=\frac{\text { Spot Price }+ \text { Financing Cost }+ \text { Spoilage Cost }+ \text { Storage Cost }- \text { Futures Price }}{\$ 10 \times \frac{6}{12}} \\
C Y=\frac{\$ 10+\$ 0.15+\$ 0.10+\$ 0.12-\$ 10.12}{\$ 10 \times \frac{6}{12}}=0.05
\end{gathered}
$$

The convenience yield is $5 \%$

\section*{APPLICATION B}
Question : Consider a calendar spread that is long the two-year forward contract and short the one-year forward contract on a physical commodity with a spot price of $\$ 100$. Assume that the number of contracts in the long position equals the number of contracts in the short position. The trader put the spread on in anticipation that storage costs, $c$, will rise. Assume that the forward prices adhere to Equation 2 in the lesson Forward Contracts on Assets with Benefits and Costs of Carry and that $r=2 \%, c=3 \%$, and $y=5 \%$. Note that these values were chosen for the simplicity that $r+c-y=0 \%$ so that the forward prices equal the spot prices. (a) What would the profit or loss be to the trader if spot prices rose $\$ 1$ ? (b) What would the profit or loss be to the trader if the storage costs rose one percentage point (from $3 \%$ to 4\%)?

\section*{Answer and Explanation}
We have two positions short and long with the same number of contracts in each (let's say 1,000 units of the underlying commodity). In addition, the spot price equals the forward price because $\mathrm{r}+\mathrm{c}-\mathrm{y}=0 \%$. Therefore, if there is a $\$ 1$ spot price increase, the long position would profit by $\$ 1,000$, but the short position would lose $\$ 1,000$, breaking even.

Now using the same situation, if the storage costs increase to $4 \%$ from $3 \% r+c-y=1 \%$. Thus the spot price is not equal to the forward price as it was in the prior situation. Using equation 2 in the lesson Forward Contracts on Assets with Benefits and Costs of Carry,

$$
F(T)=e^{(r+c-y) T} S
$$

The two-year long forward price increases to:

$$
\begin{aligned}
& F(T)=e^{(r+c-y) T} S \\
& F(T)=e^{(0.02+0.04-0.05) 2} \$ 100 \\
& F(T)=e^{(0.01) 2} \$ 100 \\
& F(T)=\$ 102.02
\end{aligned}
$$

The one-year short forward price increases to:

$$
\begin{aligned}
& F(T)=e^{(r+c-y) T} S \\
& F(T)=e^{(0.02+0.04-0.05) 1} \$ 100 \\
& F(T)=e^{(0.01) 1} \$ 100 \\
& F(T)=\$ 101.01
\end{aligned}
$$

This nets the trader a $\$ 1.01(\$ 2.02-\$ 1.01)$ profit because the trader is long the two-year forward contract that increased by $\$ 2.02$ and short the one-year forward contract that increased by $\$ 1.01$. Note that the above numbers are rounded.

A spread can be viewed as being the same as a trade that has established a long position and a short position with the same number of contracts. A trader with long and short positions of equal total size in terms of number of contracts is protected from a parallel shift (the quantity $r+c-y$ stays constant) in the forward curve, but not necessarily from a non-parallel shift. A rise or fall in ( $\mathrm{r}+\mathrm{c}-\mathrm{y}$ ) will disproportionately affect longer term contracts. A calendar spread is exposed to a shift in $r+c-y$.


\end{document}