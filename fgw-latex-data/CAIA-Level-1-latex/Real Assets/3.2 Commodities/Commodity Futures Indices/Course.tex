\documentclass[11pt]{article}
\usepackage[utf8]{inputenc}
\usepackage[T1]{fontenc}
\usepackage{amsmath}
\usepackage{amsfonts}
\usepackage{amssymb}
\usepackage[version=4]{mhchem}
\usepackage{stmaryrd}

\begin{document}
Commodity Futures Indices

In this section, we review several investable commodity futures indices, analyze their construction, and discuss their use as benchmarks. An investable index has returns that an investor can match in practice by maintaining the same positions that constitute the index.

\section*{Construction and Uses of Commodity Futures Indices}
Financial securities are generally traded in centralized spot markets, so most indices related to traditional investments focus on prices (or returns) from cash markets. Returns on physical commodities are generally better measured using prices of futures contracts rather than spot or cash prices. Spot prices of physical commodities are not generally traded in a single centralized market, and therefore the spot prices vary between locations (a difference that cannot be arbitraged to near zero due to transportation costs). Also, while shares of a particular security are homogeneous and trade at the same price, some commodities have different qualities or grades that trade at different prices. For these reasons, commodity price indices are commonly constructed using futures prices on commodities rather than cash commodity prices.

The construction and the application of commodity futures indices raise several complexities relative to indices of traditional assets. As discussed in the session, Quantitative Foundations, returns on derivative positions such as futures can be based on fully collateralized positions or on leveraged positions. Commodity futures indices are generally constructed as being unleveraged. The face value of the futures contracts is fully supported (collateralized) either by cash or by riskless bonds (e.g., Treasury bills). The commodity indices discussed here are long-only (i.e., hypothetical long positions are established in derivatives to provide economic exposure to commodities equal to the amount of cash dollars being invested in the index).

An investment manager can use commodity futures indices in several ways. First, a commodity futures index can be used as a benchmark for investment performance analysis and return attribution. Second, an investable commodity futures index can be used to implement an active tactical bet by the investment manager that the underlying commodities will generate superior expected or average returns. Finally, an investable commodity futures index can be used in a passive strategy as a strategic long-term exposure, often for the purpose of reducing risk through portfolio diversification.

\section*{Commodity Futures Indices}
The construction or selection of a long-only commodity index involves numerous decisions, including:

\begin{enumerate}
  \item The roll strategy (when is a position liquidated relative to its settlement date and what time-to-settlement is selected for the new exposure)

  \item Which commodities are represented

  \item How the commodities are weighted

\end{enumerate}

There are literally hundreds of regularly published commodity indices in the world. The following three sections discuss three index approaches that are widely used in academia and industry: production-weighted indices, market-liquidity-weighted indices, and tier-weighted indices.

\section*{Production-Weighted Long-Only Commodity Indexes}
A production-weighted index weights each underlying commodity exposure using estimates of the quantity of each commodity produced. A production-weighted index is designed to reflect the relative importance of each of the constituent commodities to the world economy in terms of production levels. The Standard \& Poor's GSCI is a very popular production-weighted long-only index of physical commodity futures. The S\&P GSCI (formerly the Goldman Sachs Commodity Index) is composed of the first nearby futures contract in each commodity. Perhaps the most distinctive feature of the S\&P GSCI is that a futures contract trades on the index itself (on the Chicago Mercantile Exchange [CME]). In other words, investors can purchase a futures contract tied to the spot value of the S\&P GSCI. The weights in the S\&P GSCl are based on five years of data and are heavily dominated by energy commodities (at times well over 70\%) due to their dominant role in global production. The S\&P GSCI is constructed with 24 physical commodities across five main groups of real assets: energy, precious metals, industrial metals, livestock, and agriculture.

\section*{Market-Liquidity-Weighted Long-Only Commodity Indexes}
Another approach to commodity weighting in a commodity index is based on market liquidity as a measure of global economic significance. As an example, the Bloomberg Commodity Index (BCOM), formerly the Dow Jones-UBS Commodity Index (and before that the Dow Jones-AIG Commodity Index), is a market-liquidityweighted long-only index composed of futures contracts on 23 physical commodities in six sectors: energy, grains, precious metals, industrial metals, livestock, and softs. These commodities are diversified and include petroleum products, natural gas, precious metals, industrial metals, grains, livestock, soybean oil, coffee, cotton, and sugar. The weights of each commodity in the index rely primarily on liquidity data, such as trading activity. This index considers the relative amount of trading activity associated with a particular commodity to determine its weight in the index.

The Bloomberg Commodity Index bands the weights within upper (15\%) and lower (2\%) limits to smooth the relative roles played by each commodity in the return of the index. While the bands appear arbitrary, they serve the role of preventing the index from being dominated by any one sector (e.g., energy), as no commodity sector can comprise more than $33 \%$ of the index weight.

\section*{Tier-Weighted Long-Only Commodity Indices}
Another approach to commodity index weighting is a tier-based approach. A tier-based commodity-weighting approach groups commodities of similar characteristics into tiers and then assigns weights to each tier. For example, the Thomson Reuters/CoreCommodity CRB Index is the oldest major commodity index and is currently made up of 19 commodities traded on various exchanges. Originally the CRB Index (Commodity Research Bureau Index), the Thomson Reuters CoreCommodity CRB Index uses four tiers or groups to weight the commodities. The system is designed to reflect the importance of each commodity to global economic development. For example, Tier I currently has $33 \%$ of the index weight and includes only petroleum products. The second tier represents agricultural\\
commodities and is weighted $42 \%$. Tiers III and IV represent precious metals and base/industrial metals with weights of $20 \%$ and $5 \%$, respectively. Weights are described as being determined to make the index a representative indicator of global commodity markets.


\end{document}