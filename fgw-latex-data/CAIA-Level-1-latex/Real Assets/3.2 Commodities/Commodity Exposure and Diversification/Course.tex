\documentclass[11pt]{article}
\usepackage[utf8]{inputenc}
\usepackage[T1]{fontenc}
\usepackage{amsmath}
\usepackage{amsfonts}
\usepackage{amssymb}
\usepackage[version=4]{mhchem}
\usepackage{stmaryrd}

\begin{document}
Commodity Exposure and Diversification

A key justification for allocations to commodities is diversification-however, the exact meaning of diversification and the role of commodities as diversifiers varies. We begin with a discussion of commodities and their diversification with traditional assets.

\section*{Four Explanations of Commodities as Diversifiers}
Commodities are often viewed as an asset class that helps diversify a portfolio of traditional assets (stocks and bonds) through a lack of return correlation between commodities and traditional assets. Here we discuss four reasons why commodity returns may have low correlation with stock prices and bond prices.

First, unlike financial securities, commodities have prices that are not directly determined by the discounted value of future cash flows. Accordingly, commodity prices are not as directly related to changes in forecasted cash flows and changes in market discount rates. Instead, commodity prices are evaluated primarily on forecasts of the commodity's supply and demand. Since commodity prices are driven by different economic fundamentals than are stocks and bonds, they should be expected to have little correlation, or even negative correlation, with the prices of financial assets.

Second, nominal commodity prices should be positively correlated with inflation largely because commodity prices form part of the definition and computation of inflation. Inflation is the decline in the value of money relative to the value of a general bundle of goods and services. A nominal price refers to the stated price of an asset measured using the contemporaneous values of a currency. Thus, the 2020 nominal price of a bushel of corn in U.S. dollars is simply the market price observed in 2020. A real price refers to the price of an asset that is adjusted for inflation through being expressed in the value of currency from a different time period. The 2020 real price of a bushel of corn based on 2010 dollars deflates the 2020 nominal price for the inflation that occurred in the dollar between 2010 and 2020 . Since prices of physical commodities such as oil are an important component of the computation of inflation, we should generally expect that the nominal prices of commodities and other real assets would move in tandem with inflation. Thus, real commodity prices would tend to be unaffected by inflation. On the other hand, both the real and the nominal prices of stocks and especially bonds tend to be negatively correlated with inflation because inflation raises the discount rates applied to their valuations. To the extent that changing rates of inflation drive the prices of real assets differently than they drive the real prices of financial assets, there should be low correlation or perhaps even negative correlation between commodity prices and the prices of most stocks and bonds.

A third reason why commodity price changes may be negatively correlated with the returns of stocks and bonds is that they may react very differently at different parts of the business cycle. The value of stocks and bonds is derived from expectations regarding long-term earnings or coupon payments. Commodities are often priced more on the state of current economic conditions and factors regarding short-term supply and demand. For example, in the midst of a severe and prolonged drought, the price of corn and other agricultural products may be extremely high, despite the long-term expectation that the drought will undoubtedly end and the prices will revert toward more normal levels.

A fourth argument for low or negative correlation between commodity prices and financial assets is based on commodities being a major cost of some corporate producers. In the short run, a major increase in commodity prices (e.g., oil) may cause a substantial decline in corporate profits, and a decline in commodity prices may result in an increase in profits. Thus, as commodity prices soar, corporate stocks and bonds may falter (except those of commodity-producing firms). The result is a negative correlation between commodity prices and the prices of financial assets.

All four of these arguments indicate why there tends to be a low or negative correlation of commodity prices and returns to the prices and returns of financial assets.

\section*{Commodities as Diversifiers in a Perfect Market Equilibrium}
A market is in equilibrium when current market prices equate supply and demand such that further transactions cannot benefit market participants. In an ideal equilibrium, all market participants are well diversified, and the differences between market risk and idiosyncratic risk can be precisely delineated. With a perfect capital market in equilibrium, the role of commodities in diversifying a portfolio is clear, as summarized here.

Diversification is the process of eliminating exposure to idiosyncratic risks while constructing a portfolio that matches the risk characteristics of a perfectly diversified portfolio. In the CAPM (capital asset pricing model), the perfectly diversified portfolio is the market portfolio that contains exposure to all assets and contains exposure to each of those assets in proportion to their total market value (i.e., size). The percentage of the total market portfolio attributable to each asset in that portfolio is known as the market weight. In other words, the market weight of an asset is equal to the percentage of the total global value of that asset relative to the total global value of all assets. For example, if crude oil represents $5 \%$ of the total wealth of the world, then a perfectly diversified portfolio of risky assets should have a $5 \%$ weight in oil. In a perfect market in equilibrium, any investor with a portfolio of risky assets that has an asset exposure greater than or less than its market weight is speculating on idiosyncratic risk. In a perfect market equilibrium, bearing idiosyncratic risk does not offer higher expected returns.

Commodities are a substantial part of total wealth; therefore, commodities should be a substantial part of all portfolios, according to classic equilibrium models such as the CAPM. So, based on the CAPM, the only issue in determining appropriate exposure to each commodity is determining the market weight of that commodity. However, ascertaining the total global market value of a commodity is difficult in practice. Returning to the example of oil, even this simplified view of portfolio allocations introduces a number of difficult questions: How much oil is expected to be discovered? How much oil that is already discovered can be extracted? What price should be attached to oil reserves that are years from being extracted? Further, ascertaining the exposure of financial assets to each commodity is difficult. How much of an oil company's stock price is attributable to oil? How have an oil company's hedging activities modified the company's exposure to oil prices?

\section*{Commodities as Diversifiers in the Presence of Market Imperfections}
In practice, markets are in a continuous state of disequilibrium and might remain substantially out of equilibrium for extended periods of time, which may offer the opportunity for an asset class to offer benefits as a diversifier without also offering commensurately lower returns. Often commodity shortages and oversupplies cannot be quickly corrected by price mechanisms. In disequilibrium, participants tend to hold substantially different exposures to various asset classes-especially commodities-than exposures based on market weights.

Some nations have vast holdings of oil and other resources that cannot be quickly developed and divested, and therefore the nation remains poorly diversified (i.e., highly concentrated), with a very high percentage of wealth exposed to oil prices or other commodity prices. If one market participant holds an asset in a higher proportion than its market weight, then some other market participants must hold that asset in a lower proportion than its market weight. In imperfect markets and\\
in disequilibrium, it is no longer clear that all market participants can or should hold oil or any other commodity in their portfolios with a weight equal to the market weight of that commodity. In that case, investors seek to hold commodities in the proportion that provides the highest return-to-risk ratio based on their existing portfolios and such circumstances as the structure of their liabilities.

\section*{Commodities as Diversifiers against Unexpected Inflation}
One of the most often cited virtues of commodity investment is its ability to diversify a portfolio against the risk of unexpected inflation. When rates of inflation are steady, asset prices tend to adjust such that the expected nominal returns of each asset reflect anticipated inflation. Therefore, steady and anticipated inflation is not generally a serious investment risk or a determinant of real returns. However, unexpected inflation can be a serious risk to investors. For example, a fixed-income security tends to underperform in an environment of unexpectedly high inflation because the value of the promised future cash flows is being diminished at an unexpectedly high rate.

Real assets in general and commodities in particular offer protection against inflation risk. Inflation risk is the dispersion in economic outcomes caused by uncertainty regarding the value of a currency and it emanates from the divergence between realized and anticipated rates of inflation (i.e., unanticipated inflation).

There are two intuitive explanations for the protection from inflation risk provided by commodities. First, as mentioned in a previous section, commodity prices are an important component (i.e., determinant) of the price indices that measure inflation. Therefore, general price indices (and realized inflation rates) tend to be positively correlated with commodities. Second, the value of a commodity is its perceived ability to provide consumption. Because of their homogeneity and their ability to be transported, commodity prices tend to be determined by global factors rather than local factors. Therefore, in each country local commodity prices should adjust quickly to changes in the value of the local currency. For example, when a country's currency experiences hyperinflation, the nominal price of oil in that currency rises, but the real price of oil in the currency of other nations is unaffected. The real returns of commodity investments in each country should be unaffected by the inflation rate of the investor's home currency.


\end{document}