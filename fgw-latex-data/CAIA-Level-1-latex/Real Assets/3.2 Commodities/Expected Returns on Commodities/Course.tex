\documentclass[11pt]{article}
\usepackage[utf8]{inputenc}
\usepackage[T1]{fontenc}

\begin{document}
Expected Returns on Commodities

This section compares the expected returns of commodity exposures through both physical inventories and derivatives.

\section*{Empirical Evidence on Long-Run Commodity Price Changes}
Substantial debate exists as to the long-term performance of spot commodity prices. Consider the two major forms of analysis: empirical and theoretical. The preponderance of the empirical very-long-term evidence on the direction of commodity prices is that spot prices do not rise at or above the riskless nominal rate. Although it is true that nominal prices of many commodities have increased in the past 50 or so years, the very-long-term real returns on many commodities have been negative. For example, from the middle of the nineteenth century until recently, the real prices of aluminum, copper, iron, nickel, silver, and zinc have declined. Virtually every commodity is now less expensive in real terms than they were many centuries ago.

\section*{Theoretical Evidence on Expected Commodity Returns}
Consider three points that argue against investment in physical inventories by financial institutions: lack of systematic risk premia, increasing technologies, and wasted convenience yield.

A starting point in the theoretical analysis of expected returns on investments is to consider a general equilibrium model of expected returns, such as the capital asset pricing model. To the extent that commodities have little or no systematic risk (and serve as effective diversifiers relative to equities and bonds), the required and expected returns on commodities would be modest in equilibrium-perhaps near the riskless rate. However, physical commodity returns have generally failed to keep up with riskless rate, as noted in the previous section.

Theory provides an explanation for the long-term history of commodity price declines (due to increasing technologies) and supports the likelihood that the verylong-term downward trends of real commodity prices will continue. Technology is likely to continue to improve efficiency in the extraction (i.e., lower the cost of extraction) and efficiency in the use of commodities (reducing demand). Technology is also likely to improve the rate of reclamation of many commodities and to develop alternatives to utilizing commodities that experience increasing real prices.

Finally, physical commodities often offer convenience yield. However, financial institutions generally seek only financial returns because they do not benefit from the convenience yield of physical inventories of commodities. Therefore, holding of physical inventories of commodities by financial institutions is generally a waste of any convenience yield.

\section*{Irrelevancy of Commodity Price Expectations to Returns on Futures Contracts}
The previous sections have discussed physical (i.e., direct) ownership of commodity inventory and have provided both empirical and theoretical evidence of the unattractiveness of long-term physical ownership of commodities. This section discusses exposure through forward and futures contracts and demonstrates that derivative contracts can offer competitive returns on commodities even when expected spot prices are dramatically lower than current spot prices.

Consider the market for popular memory chips used in electronics. For decades, the long-term direction in the price of memory chips on a basis of cost per bit has been steeply downward in real terms as technology has advanced. It would make little economic sense to speculate on memory chip prices by accumulating a large physical inventory in the hope of long-term price increases. However, the market for memory chips includes contracts that call for future delivery of the chips. The prices for future delivery of particular memory chips are consistently lower than the spot prices (for immediate delivery)-as should be expected for a "commoditylike" item such as memory chips, for which technology is improving so rapidly. The price of contracts on memory chips with future delivery adjusts with both sides of the contract taking into account the likely decline in spot prices over the life of the contract.

The dynamics are similar to the case of a very-high-coupon bond selling at an enormous premium to its face value. Investors know that the bond is expected to decline in price but are still able to enter into appropriately priced forward contracts on that bond by simply adjusting the forward price to reflect the high distribution rate of the bond. The session, Derivatives and Risk-Neutral Valuation detailed the relation between the spot and forward prices of financial assets and discussed the effect of high distributions rates on forward prices relative to spot prices. The relation between forward prices and spot commodity prices similarly adjusts for anticipations of declining spot prices such as in the long run when improved technologies emerge that are expected to drive down real prices (e.g., memory chips) or in the short-run when a huge harvest is anticipated that will drive down grain prices.

The point is this: Commodity price exposure obtained through financial derivatives occurs at prices that adjust for anticipated spot price changes, leaving holders of long positions in commodity futures the ability to earn competitive returns even when long-term commodity prices are anticipated to decline. Financial institutions therefore have theoretical justifications to invest in commodities through derivative contracts based on the pursuit of broader diversification.

In order for the demand for commodities as an investment to equal the supply of available commodity investment opportunities, some market participants need to be incentivized to overrepresent commodities in their investment portfolios relative to market weights when others underrepresent commodities. The mechanism by which this would occur is through price adjustments, until commodities offered both enhanced returns and diversification benefits sufficient to induce more sophisticated and innovative investors to make large allocations to commodities. Simply put, to the extent that commodity investing is ignored or rejected by some investors, prices would adjust so that other investors would find superior investment opportunities through making higher allocations to commodities. The result would be that commodity beta exposure would offer higher expected returns per unit of risk than would other beta exposures. This is consistent with the view that hedging by commodity producers will drive down futures prices to a point where financial investors find commodity futures to be attractive investments.

The bottom line is this: Commodities do not enhance expected returns when they are efficiently priced and when their systematic risk exposures (betas) are low. If markets are perfect and in equilibrium, market participants should hold exposures to commodities and other asset classes, expecting lower returns in exchange for enjoying lower risk. Financial institutions can utilize forward contracts and futures contracts to attain those exposures.


\end{document}