\documentclass[11pt]{article}
\usepackage[utf8]{inputenc}
\usepackage[T1]{fontenc}
\usepackage{amsmath}
\usepackage{amsfonts}
\usepackage{amssymb}
\usepackage[version=4]{mhchem}
\usepackage{stmaryrd}

\begin{document}
Rolling of Forward and Futures Contracts

The session, Derivatives and Risk-Neutral Valuation discussed the process of rolling futures and forwards positions from one contract to a more distant contract for the purpose of maintaining a long-term exposure to a commodity. Contracts may be rolled over from any point in time to any available contract. This section discusses the implications of these timing and longevity decisions. The excess return of a futures contract is the return generated exclusively from changes in futures prices and is not to be confused with the definition of excess return of a cash security, which is its return minus the riskless rate. Thus, if the futures price of a particular contract on gold rises from $\$ 1,000$ per ounce to $\$ 1,050$ per ounce, the contract experiences an excess return of $5 \%$.

\section*{Why Returns on a Futures Contract Can Differ from the Spot Return}
The previous section defined the basis as the difference between the spot price and the futures or forward price. Consider a fully collateralized position in a futures contract. A fully collateralized position is a position in which the cash necessary to settle the contract has been posted in the form of short-term, riskless bonds. The total returns from fully collateralized futures or forward returns differ from returns on spot positions on the same asset primarily due to basis risk. The basis risk that causes realized returns on a fully collateralized commodity futures contract or forward contract to differ from the total return on the underlying spot position may be divided into three primary sources: (1) when the costs of carry to a marginal investor for the spot position are not the same as the costs implied by the basis, (2) when the convenience yield from the spot position differs from its storage costs, and (3) when the basis changes. The first issue would tend to indicate informational market inefficiency in the pricing of the futures contract. The second issue is that, in equilibrium, if a spot position offers a convenience yield that does not exactly offset its storage costs, $c-y$ is positive and the spot price must be lower than when the convenience yield is high. While the inventory holder may not earn the convenience yield, there is an indirect benefit of having sufficient inventories preventing a stock-out that slows their business operations. The third issue is simply a consequence of uncertainty.

\section*{Components of Futures Returns}
There are two especially useful formulas depicting the components of the total return of a collateralized futures position: a two-component formula and a threecomponent formula.

The return on a fully collateralized position, $R_{f c o l l}$, can be expressed as the sum of two components:


\begin{equation*}
R_{f c o l l}=\text { Collateral Yield }+ \text { Excess Return } \tag{1}
\end{equation*}


Equation 1 expresses the total return from an unleveraged, fully collateralized commodity futures position as the sum of the interest earned from the riskless bonds used to collateralize the futures contract (the collateral yield) and the percentage price change in the futures contract (the excess return).

The price of a futures or forward contract may be viewed as equaling the spot price minus the basis. Thus, the excess return in Equation 1 (the change in the futures price) may be broken into the change in the spot price and the change in the basis. By substituting the change in the spot price and the change in the basis into Equation 1 in place of the excess return, the total return from this unleveraged, fully collateralized commodity futures position can be expressed as coming from three primary sources.

The three primary sources are depicted in Equation 2 as: (1) changes in the spot price of the underlying commodity, (2) the interest earned from the riskless bonds used to collateralize the futures contract, and (3) changes in the contract's basis (i.e., roll yield):


\begin{equation*}
R_{\text {fcoll }}=\text { Spot Return }+ \text { Collateral Yield }+ \text { Roll Yield } \tag{2}
\end{equation*}


Each of the three components can be an important part of the return of a commodity futures position. Let's look at each of these three components closely.

The first component in Equation 2, spot return, is the return on the price of the underlying asset in the spot market. The returns of unhedged futures positions are primarily driven by the spot return. Exposure to spot price changes is the primary reason that most market participants enter futures contracts, and is also why market participants wishing to gain exposure to commodity prices establish positions in futures contracts on commodities. It should be noted that there is not a single centralized spot market or a single network of spot markets that provides a single universally recognized spot price for most physical assets.

The second component, collateral yield, is the interest earned from the riskless bonds or other money market assets used to collateralize the futures contract. Positions in futures contracts are often partially collateralized in that they only post collateral that is equal to the margin required by the futures exchanges. Partial collateralization generates leveraged returns, since the value changes of the entire futures position is borne by a smaller collateral amount. Fully collateralized positions are unleveraged, since the cash invested equals the economic exposure of the futures contract. Depending on interest rate levels, the collateral yield can be a substantial part of the total return to a fully collateralized commodity futures position.

The third component of a futures position is changes in its basis, also known as roll yield or roll return. Roll yield or roll return is properly defined as the portion of the return of a futures position from the change in the contract's basis through time. The basis of a futures contract changes for two reasons. First, as time passes, the time to settlement of the futures contract shortens, and the contract's price (and basis) rolls up or down the term structure of forward prices toward the spot price. Second, as components of the cost of carry vary (interest rates, dividend yields for financial futures contracts, storage costs, or convenience yields), the basis will also vary, since the basis depends directly on the four components of cost of carry. This very important concept is detailed in the next several sections.

\section*{Two Interpretations of Rolling Contracts}
One of the sources of futures returns just discussed is the roll yield, or roll return, which is the subject of alternative understandings. Conflicting interpretations of roll return emanate from ambiguity in the concept of rolling a futures position.

Rolling a contract has two common interpretations. Sometimes it is used to describe the transactions involved in switching, or rolling from, a short-term futures contract to a futures contract with a longer term to settlement in the process of maintaining a continuous exposure to the underlying asset. Other times the rolling of a contract describes how its price "rolls up" (or down) the term structure of forward prices as its time to settlement nears.

The two interpretations of rolling a contract lead to two interpretations of roll return or roll yield.

When rolling a contract is viewed as holding a futures position while its time to settlement nears and its price potentially rolls up or down the forward curve, then roll return is viewed as the change in the contract's basis through time. This view of roll return tends to be associated with a financial economics view of risk and return.

When rolling a contract is viewed as a transaction, roll return is viewed as the profit or loss recognized at the time that a position in a futures contract is rolled from one contract to another. This view of roll return is used to adjust excess futures returns in the process of reporting returns of continuous commodity exposures. This view of roll return tends to be associated with an accounting view of returns.

It should be noted that the transactions of closing a position in a short-term contract and opening a position in a longer-term contract do not directly and immediately cause a gain or loss. Rolling between contracts can be viewed as recognizing a gain or loss that was previously accrued. But recognition of accrued gains (and rolling of contracts) does not create wealth or return.

\section*{Roll Yield and the Slope of the Forward Curve}
The lesson, The Term Structure of Forward Prices on Commodities discussed the concepts of contango (an upward-sloping term structure of forward prices) and backwardation (a downward-sloping term structure of forward prices). An important topic in commodity futures is the relation between the slope of the term structure of forward prices (i.e., the forward curve) and the sources of return from holding a futures contract.

It is often claimed that holding a long position in a futures contract when a market is in backwardation tends to be a successful strategy because it earns roll return (or roll yield). The idea is that as time passes and the time to settlement of a futures contract diminishes, the future's price rises as the futures contract "rolls up" the downward-sloping curve of a backwardated market. In other words, it is often argued that market participants can earn consistently superior risk-adjusted returns from the positive roll yield generated from long positions in futures contracts when markets are backwardated.

The argument that roll return generates superior returns in backwardated markets for long futures positions implies that roll return generates superior returns in contango markets for short futures positions. Also, roll return could be similarly argued to lead to inferior returns for long positions in markets that are in contango and inferior returns for short positions in markets that are backwardated. Can alpha be consistently generated by alternating between long and short positions based on the slope of the forward curve?

Previous sections have demonstrated that the slope of the term structure of forward prices depends on the costs of carry. In an informationally efficient market, a nonzero slope of the term structure of forward prices exists to prevent superior risk-adjusted returns. In other words, the term structure takes on a positive or negative slope (contango or backwardated) based on carrying costs, so that the risk-adjusted returns of spot positions and fully collateralized futures positions will be equal. Commercial holders of inventories can benefit from convenience yield, which would drive the futures price to be less than or equal to the spot price adjusted for interest rates and storage costs.

In an informationally efficient market, roll return is simply the change in the basis that allows identical exposures in cash and futures markets to offer identical total returns. However, no market is perfectly efficient, especially those involving real assets, such as commodities.

\section*{Roll Yield, Carrying Costs, the Basis, and Alpha}
There are three ways of expressing the relation between spot and forward prices through time: (1) the basis, (2) carrying costs, including convenience yield, and (3) roll yield. All three of these terms express the same concept.

In an informationally efficient market (and when the carrying costs are expressed as present values rather than as rates or percentages), the absolute value of the carrying costs will equal the absolute value of the basis. The carrying costs and the basis will have different signs, according to the most common definition of the basis being the spot price minus the forward price.

The roll yield is the same as the basis (and the carrying costs) when viewing the return on a futures contract through its settlement. Note that the roll yield is defined as the change in the basis. Since the basis of a futures contact or forward contract is zero at settlement, the roll yield of a futures contract to settlement must equal the contract's starting basis.

Let's examine the relations first in the context of futures on financial assets and then in the context of futures on real assets (e.g., commodities).

In the case of futures contracts on financial assets, the costs of carry for the underlying financial asset are the financing costs expressed as an interest rate $(r)$ and the rate of dividends, coupons, or other distributions $(d)$ that are received and are entered as a negative cost $(-d)$. While $r$ is observable, $d$ is assumed to only be predictable. We assume that all market participants are unanimous with regard to these values and can engage in transactions to receive or pay the same values of $r$ and $d$. Therefore, the actions of arbitrageurs in these markets should force financial futures toward a high degree of informational efficiency in which roll yield equals the cost of carry, which, in turn, determines (and equals) the basis.

In the case of real assets, the carrying costs of holding the real asset include the storage cost, $c$, and the convenience yield, $y$. Storage costs and convenience yields on real assets are heterogeneous between market participants. A heterogeneous value differs across one or more dimensions. In this case, individual market participants may have different costs and benefits ( $c$ and $y$ ) from holding a real asset. Further, these costs and benefits may be unobservable to others.

A clear benefit of futures markets on real assets is the market's ability to facilitate the efficient bearing of storage costs and reaping of convenience yields. For example, an efficient storage operator of natural gas can store natural gas while hedging its price risk in the futures market. A manufacturer that depends on silver as a raw material can enjoy the convenience of large inventories (e.g., protection from supply disruptions) while hedging the price risk of silver in the futures market.

A major source of potentially superior risk-adjusted returns using futures contracts on real assets emanates from the heterogeneous costs of carry across market participants. The key for a market participant to generate alpha through analysis of carrying costs and the basis is to execute trades when the prices of futures contracts imply costs of carry that deviate from the participant's costs of carry. For example, a trader can generate alpha if the trader's storage costs are less than the storage costs implicit in the basis of the futures contract.

\section*{The Strategy of Rolling Contracts Affects Return Expectations}
This section and the next discuss the rolling of futures or forward positions: the closing of a position prior to or at settlement, and the opening of an otherwise identical contract with a later settlement date in order to maintain continuous exposure. What is an appropriate measure of the return for a long-term continuous exposure to a commodity? The issue is how to establish an appropriate return standard or expectation for a long-term continuous futures contracts exposure to a commodity. There are two issues: rollover decisions and collateral investment decisions.

Within futures markets, individual investors roll their futures positions over at different times relative to settlement and may differ in the selection of which settlement date to use for the new positions. The rollover decision has two components: when to exit a position and which longevity to select for the new futures or forward position. Accordingly, it is not possible to identify a pattern of rollovers that is common to all investors and to identify the return of a particular pattern of rollovers as being representative of the returns achieved by all investors. Returns are determined by the particular rollover strategy implemented.

Finally, the return of a fully collateralized strategy depends on the interest rate that is earned on the collateral. Collateral investment positions are often in defaultfree fixed-income securities. These positions can differ by their duration and therefore their returns can differ.

\section*{The Impact of Rolling Contracts on Alpha}
What is the relation between risk-adjusted expected returns and the selection of a particular rolling strategy (e.g., entry and exit longevities)? Can a particular rollover strategy consistently generate attractive risk-adjusted returns (alpha)? These issues are greatly simplified by viewing the difference between two rollover strategies as being equal to a simple calendar spread strategy.

To illustrate, consider two investors with continuous long positions in the same commodity futures. Suppose that Investor A rolls over contracts one month prior to settlement and establishes a new position in the first deferred contract. Investor B rolls over contracts at settlement and establishes a new position in the new nearby contract. As long as the nearby contract has one month or more to settlement, Investors A and B have the same position. However, when the nearby contract has one month or less to settlement, Investor A rolls into the first deferred contract while Investor B remains in the nearby contract. The difference between the returns of the two strategies occurs only during the month prior to settlement and is equal to the returns of a calendar spread.

For purposes of discussion, let's title the strategy followed by Investor B as a classic rollover strategy: Each contract is held to settlement and then is rolled over into the shortest available contract. All other rollover strategies generate a return that is equal to the return of that classic rollover strategy plus, at some or all times, the return of a short calendar spread. If markets are inefficient, it may be possible to earn a consistently superior or inferior risk-adjusted return through the adoption of a particular rollover strategy (which is to say that superior return is possible if calendar spreads are inefficiently priced). In other words, any advantage between two rollover strategies is identical to the advantage of an equivalent strategy using calendar spreads (in a perfect market). Speculating on rollover strategies is tantamount to speculating on calendar spreads.

It should be noted that, in practice, markets are imperfect and have transaction costs. When transaction costs are included, some rollover strategies may be more cost-effective than others.

\section*{Three Propositions Regarding Roll Return}
The more common definition of roll return (or roll yield) is that it is the return accrued in a futures position through time, attributable to changes in the basis of the futures contract. This section distinguishes this definition of roll return from the accounting usage of the term regarding the closing of one futures position and the opening of another. The following three propositions highlight key issues.

Proposition 1: Roll return is not generated when one position is closed and a new position is opened. For example, roll return is not generated by closing the nearby contract at $\$ 95$ and opening the first deferred contract at $\$ 92$, for a $\$ 3$ profit. The lack of logic to that view of rollover is analogous to selling a short-term Treasury bill for $\$ 99$, buying a longer-term Treasury bill for $\$ 98$, and claiming that the transaction generated a profit of $\$ 1$.

Roll return occurs throughout the time that a particular futures or forward contract is held. Roll return can be viewed as the difference between the price at which a particular contract is opened and the price at which that same contract is closed in excess of the return on a spot position in the contract's underlying commodity. The price difference is based on the same contract at two different points in time.

Proposition 2: Roll return is not necessarily positive when markets are backwardated for holding periods shorter than being held to settlement. It is true that roll return is positive in backwardated markets if none of the components of the costs of carry change. However, if the costs of carry change, then even in backwardated\\
markets there is no guarantee that roll return will be positive prior to settlement. In other words, it is reasonably likely for the term structure of forward rates to shift such that roll return will be negative in a backwardated market.

Proposition 3: A position that generated a positive roll return does not indicate that the position's total returns were superior (i.e., that there was alpha). Roll return is a part of the total returns that make futures contracts and cash positions equally attractive. Roll return is usually negative, to punish the forward position (relative to the cash position) for not requiring a cash investment relative to a spot position. But roll return can clearly be positive, when, for example, there is a high dividend or coupon rate on the underlying asset. In the case of a forward contract or futures contract on a financial asset held to settlement, roll return equals carrying costs times -1 : $(d-r)$. Thus, if the dividend yield of a financial asset exceeds the riskless rate, then roll return is positive if the position is held to settlement.


\end{document}