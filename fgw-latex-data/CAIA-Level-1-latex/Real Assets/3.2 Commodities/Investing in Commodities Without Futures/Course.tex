\documentclass[11pt]{article}
\usepackage[utf8]{inputenc}
\usepackage[T1]{fontenc}
\usepackage{amsmath}
\usepackage{amsfonts}
\usepackage{amssymb}
\usepackage[version=4]{mhchem}
\usepackage{stmaryrd}

\begin{document}
Investing in Commodities Without Futures

Commodities are introduced in the session, What Is an Alternative Investment? The use of forward contracts and futures contracts to gain exposure to commodities is introduced in the session, Derivatives and Risk-Neutral Valuation. This session dives more deeply into commodities and the methods institutions use to gain exposure to commodity returns. Throughout this session, the terms futures and forward contracts will be used interchangeably unless an important difference (e.g., marking-to-market) is being analyzed.

One of the most popular methods of obtaining investment exposure to commodity returns is through positions in futures and forward contracts on commodities. This section discusses four other popular methods of obtaining exposure to commodity returns: direct physical commodity ownership, equity-related commodity investments, exchange-traded funds (or notes), and commodity-linked notes.

\section*{Three Disadvantages of Direct Investment in Physical Commodities}
Institutional-quality investment opportunities in commodities focus on those commodities that are used in large quantities. Institutional investors generally obtain exposure to commodities through derivative contracts such as futures contracts rather than through physical inventories for several reasons: (1) to avoid storage costs and other disadvantages of moving, maintaining, and managing inventories; (2) to avoid wasting the convenience yield implicit in physical inventories; and (3) to avoid the opportunity cost of capital or financing costs of purchasing physical inventories. These three factors are detailed next.

Physical ownership of commodities can be problematic. Storage and transportation costs associated with direct investments in commodities make this an unattractive alternative for most institutional investors. Most institutional investors do not have expertise in managing the storage and transportation issues of physical commodities, and are unwilling to bear these costs of ownership associated with possession of physical commodities.

Convenience yield is the marginal economic benefit that an investor obtains for having physical ownership of a commodity rather than synthetic ownership through futures contracts or other financial securities. Some operating firms prefer physical ownership of a commodity because the firms place a high value in possessing physical inventory (and are perhaps able to maintain an inventory of the commodity at low storage costs). An example is a manufacturer with excess storage capacity and with concerns that commodity supply disruptions (e.g., transportation failures) could disrupt vital operations.

While users of commodities typically derive convenience yield from inventories, speculators or investors who hold inventories of commodities that they do not use are wasting the convenience yield of the commodity. To the extent that convenience yield is efficiently priced, firms that perceive no convenience yield from a particular commodity should prefer investing in commodities in a form other than physical ownership.

Lastly, physical ownership of commodity inventories requires outlay of the purchase costs (which creates opportunity costs to the capital) or borrowing the funds (which entails direct financing costs).

Some firms are purely in the business of storing commodities. Natural gas is an example of a commodity held by storage operators that do not consume that commodity in their business. The seasonal nature of natural gas demand causes periods of physical inventory buildup and drawdown throughout the year. Natural gas storage operators possess the option to receive natural gas during low-demand periods (summer) and deliver the gas during high-demand periods (winter). These firms generally do not invest in commodities to seek exposure to general commodity price levels, but rather are seeking implicit or explicit reimbursement of storage costs (and profits) from providing storage services.

The essential point is that physical ownership of commodities offers the benefit of convenience yield but also the costs of storage and transportation. Physical storage of commodities is typically a poor method of obtaining commodity exposure for institutional investors without a competitive advantage to storing the commodity and without a high convenience yield for the commodity (relative to other market participants).

\section*{Hotelling's Theory and Attractive Direct Commodity Returns}
Theories vary with respect to the expected returns from direct investment in physical commodity inventories. This section and the next discuss two views-or hypotheses-regarding the long-run expected returns of holding direct long positions in commodities. The two hypotheses propose distinct views on direct commodity investments: (1) the expected long-run returns of direct commodity investment are attractive, or (2) they are unattractive.

In 1931, Hotelling discussed the long-run investment prospects of investing directly in a commodity that is available in a fixed quantity such as copper. Hotelling's theory states that prices of exhaustible commodities, such as various forms of energy and metals, should increase by the prevailing nominal interest rate-perhaps with a risk premium. Therefore, ignoring storage costs, expected spot prices of commodity $i$ at a horizon point of $T$ years, $E\left(P_{i, T}\right)$, should be equal to the future value of the current spot price compounded at the nominal riskless rate plus a risk premium as indicated in Equation 1.


\begin{equation*}
E\left(P_{i, T}\right)=P_{i, e} e^{r T} \tag{1}
\end{equation*}


where $r=$ the risk-adjusted continuously compounded return required for holding commodity $i$ for $T$ years. In theory it is conceivable that $r$ for a particular commodity could be less than the default-free nominal interest rate if there were inflation risk-reducing or systematic risk-reducing attributes of the particular commodity that were perceived by the marginal investor to be so attractive as to justify investment with a negative risk premium. Generally, however, Equation 1 expresses Hotelling's theory and indicates that spot commodity prices are expected grow at a rate equal to an appropriately risk-adjusted nominal rate.

Hotelling reasoned this relation by noting the perspective of the owners of natural resources containing the commodities (e.g., oil). Consider the decision faced by the owner of an oil field who can leave the oil in the ground indefinitely or extract and sell it right away. In other words, the owner can keep the oil as a physical asset or turn it into a financial asset and begin to earn interest. In a competitive market, the expected long-run equilibrium price of oil in the market must cause owners of oil to be indifferent between the two alternatives. This will happen if the price of oil (net of extraction costs) is expected to increase at the prevailing rate of interest plus a premium to compensate the owner for the risks associated with keeping the oil in the ground. The logic draws from the profit-maximizing behavior of commodity owners and may be extended to include many other resources.

Although Hotelling's argument does not apply to agricultural commodities whose supplies are not exhaustible, it does suggest that the expected long-run return to various forms of direct investment in energy, industrial metals, and precious metals should be equal to the long-term nominal interest rate and perhaps a risk premium.

\section*{Simon and Unattractive Direct Commodity Returns}
Some commentators (e.g., Stanford University biology professor Paul Ehrlich) have argued that exhaustible commodities reach peak extraction rates and that future declines in extraction rates will lead in some cases to massive shortages, price increases, and economic crises. Others (e.g., University of Maryland business professor Julian L. Simon) argued that innovation and technological advances would cause long-term spot commodity prices to tend to decline in real terms.

Simon and Ehrlich entered into a famous 10-year wager in 1980 that allowed Ehrlich to select five commodity metals he believed would rise in price over the ensuing 10 years. Ehrlich, with the aid of the experts he consulted, picked copper, chromium, nickel, tin, and tungsten. Ehrlich lost the bet: The inflation-adjusted prices of all five commodities trended downward during the 10 -year wager period.

As discussed in the session Natural Resources and Land, the decision to extract and sell a natural resource is based on option theory and a benefit-cost analysis. The most easily extracted reserves will tend to be developed first (the low-hanging fruit theory). Technologies typically emerge to make the cost of extraction decline in real terms. Salt, viewed as a precious commodity centuries ago, now sells for less than $\$ 100$ per ton. The net result is that long-run expected spot prices of commodities (in real or inflation-adjusted terms) can be consistently less than predicted by Hotelling's theory, as indicated in Equation 2.


\begin{equation*}
E\left(P_{i, T}\right) \leq P_{i, 0} e^{r T} \tag{2}
\end{equation*}


Equation 2 views Hotelling's theory as ignoring technological changes. Equation 2 uses an inequality to set only an upper bound on expected spot prices. If the current price, $P_{i,}$, is such that the left side of Equation 2 exceeds the right side, arbitrageurs could purchase commodities (continuing to ignore storage costs) and hold them, with the expectation of selling them at time $T$ at a price that offers a superior expected return. However, restrictions and frictions on short-selling do not allow arbitrage in the other direction. Therefore, models that allow technological innovation predict that commodities can have expected spot prices that are lower than those predicted by Hotelling's theory, making direct investment in physical inventories unattractive in the absence of benefits derived from convenience yield.

\section*{The Idiosyncratic Risks and Two Betas of Commodity-Related Equities}
Another way to gain exposure to commodities is to own the securities of firms that derive a substantial part of their revenues from the sale of physical commodities, such as natural resource companies. A major problem with this approach is that most firms have revenues related to a variety of commodities or have operations that extend outside of activities directly related to the ownership and extraction of commodities. As a result, the share price of most firms will often be poorly correlated with the price of a single commodity.

There are several reasons why even a firm focused on a single commodity might not be a good proxy for a direct investment in the firm's underlying commodity. First, a high correlation between the stock price and the commodity price assumes that the firm has not hedged its exposure to the commodity through short positions in forward or futures contracts. Also, the firm must own the underlying commodities (or rights) rather than purchasing the commodities or leasing the rights at market prices.

Next, consider how the price of a common stock can be viewed as the product of the earnings per share (EPS) and a price-to-earnings (P/E) ratio. Although the EPS of a commodity-producing firm may be somewhat highly correlated to the price of the underlying commodity, the P/E ratio may not be. If the stock market declines quickly, $\mathrm{P} / \mathrm{E}$ ratios tend to fall throughout the various sectors. When commodity prices and inflation are increasing, the decline in overall market P/E ratios could arguably lead to a decline in the $\mathrm{P} / \mathrm{E}$ ratio and prices of commodity-producing firms.

Commodity equities therefore may be viewed as having two betas: one to the underlying commodity market and a second to the equity market. Only the first is attractive for investors with a goal of direct exposure to a commodity. If the goal of commodity investment is to diversify the portfolio away from equity market exposure, commodity-related equity investments may retain more equity market risk than is desirable to meet this diversification goal.

Also, commodity-related equities may generate returns uncorrelated with the price of the commodity that the firm produces (e.g., oil) due to idiosyncratic risk. Investments in commodity-producing firms can have substantial idiosyncratic risks (i.e., stock-specific risks) caused by the operating risks associated with an operating company including major accidents, labor problems, and managerial effectiveness. Also, the firm may have other operations with substantial exposures to other risks.

Note that most diversified investors in the stock market already have a substantial exposure to commodity-related equities. For example, in the United States, the Russell 1000 Index (consisting of roughly the largest 1,000 U.S. stocks) has a weight of nearly $10 \%$ in firms that produce energy, metals, and materials.

\section*{Commodity-Linked Exchange-Traded Funds and Notes}
One of the easiest ways to invest in a basket of commodities or, in some cases, individual commodities, is through an exchange-traded fund (ETF). There are several structures through which commodity ETFs can obtain exposure to commodity prices: futures markets, equity markets, and physical ownership. Many ETFs have underlying commodity exposures diversified across energy, metals, and agricultural commodities. Other ETFs focus on a specific commodity sector, such as energy, or can invest in a single commodity, such as gold. The largest gold ETF has held more than $\$ 30$ billion in client assets. Those ETFs based purely on physical commodities typically invest in a single commodity, such as gold or silver. Investors in these ETFs hold a share of a physical stock of bullion held in a secure warehouse.

Most ETFs offer direct, unleveraged exposures. However, some offer leveraged returns and others offer bear exposures (exposures negatively correlated with commodity prices by holding short positions in futures contracts). Most ETFs tend to be cost-effective for retail investors but may not be adequately cost-effective for institutional-sized portfolios.

Exchange-traded notes (ETNs) are similar to ETFs. Whereas investors in ETFs have a direct claim on an underlying pooled portfolio, investors in ETNs purchase a debt security with cash flows that are directly linked to the value of a portfolio. This debt security is typically issued by an investment bank or a commercial bank that\\
agrees to pay interest and principal on the debt at a rate tied to the change in price of a referenced portfolio (or index). Investors need to be aware of a key difference between ETNs and ETFs: ETNs incur the credit risk of the issuing bank (i.e., counterparty risk), whereas ETF investments do not. The risk of ETNs was highlighted during the 2008 bankruptcy of Lehman Brothers, when related ETNs were delisted as exchange-traded products, and investors holding these notes became general creditors of the firm. Both ETNs and ETFs investing directly in physical commodities have become extremely popular in recent years, especially in the metals markets.

Exposure to commodities obtained through ETFs investing in commodity futures can be complicated by a lack of correlation between futures returns and spot returns due to reasons detailed in later sections. Other issues arising from the use of futures and forwards to obtain commodity exposure are also detailed in later sections.

Finally, some commodity ETFs obtain commodity price exposure by investing in the equity securities of commodity-producing firms. These ETFs may be diversified across commodity sectors or focused on the producers in a single sector, such as energy, metals, or agriculture. Similar to investments in commodity-producing equities, these ETFs are correlated to both the equity market and the commodity market. In a falling equity market, equity-based commodity ETFs can decline in value, even if prices of commodities are rising in the spot or futures markets.

\section*{Three Advantages and One Disadvantage of Commodity-Linked Notes}
A commodity-linked note (CLN) is an intermediate-term debt instrument whose value at maturity is a function of the value of an underlying commodity or basket of commodities. CLNs are often structured products created through financial engineering so that the commodity risk exposures are generated through positions in commodity derivatives. CLNs are often issued by large banks to meet the risk and return preferences of investors; however, they can also be issued by firms that produce the commodities as a source of financing. Whether issued as innovative sources of financing for a commodity-producing firm or financially engineered as structured products, CLN returns and prices can be closely linked to commodity prices.

One major advantage to CLNs is that a commodity-producing issuer of a CLN can benefit by better matching the risks of its assets and liabilities. For example, a goldmining firm has assets and revenues highly positively correlated with the price of gold. A CLN offers the firm the opportunity to be financed with debt securities that hedge risk by having the expenses of the CLN's coupon or principal payments directly related to the same commodity price that drives its revenues.

CLNs have two major advantages to investors. First, an investor does not have to execute the rolling of commodity futures contracts to maintain exposure. If the CLN uses futures contracts to obtain its commodity exposure, the mechanics of rolling the positions becomes the problem of the issuer of the note (who must roll futures contracts to hedge the commodity exposure embedded in the note). Second, the note is, in fact, a debt instrument. Although some institutional investors may have investment restrictions on direct positions in futures contracts (due to their implicit leverage and potentially large losses), they may be able to obtain commodity exposure through CLNs because they are debt instruments. They are recorded as a liability on the balance sheet of the issuer and as a bond investment on the balance sheet of the investor, and they can have a stated coupon rate and maturity just like any other debt instrument.

A major potential disadvantage of CLNs is that they contain the idiosyncratic default risk of the issuing firm.

\section*{Commodity-Linked Notes Example}
Suppose that a pension fund is not allowed to trade commodity futures directly (due to restrictions on leverage) but wishes to invest in the commodity markets as a hedge against inflation. To diversify its portfolio, the fund purchases at par value from an investment bank a $\$ 1$ million structured note tied to the value of an index on commodities, such as the S\&P GSCI (discussed later). Assume that the note has a maturity of one year and is principal guaranteed. The principal guarantee means that the pension fund will receive at least the face value of the note at maturity unless the issuer defaults. However, if the S\&P GSCl exceeds a prespecified level at the maturity of the note, the pension fund will receive this appreciation. Thus, principal repayment can be higher than the original principal amount, depending on the settlement price of the S\&P GSCl at the note's maturity. The pension fund therefore has a call option embedded in the note. If the S\&P GSCl exceeds a predetermined level (the strike price) at the maturity date, the pension fund will participate in the price appreciation. However, if the S\&P GSCI declines, the pension fund has a promise of receiving the original principal amount.

The embedded call option on the S\&P GSCI is not free. Thus, an investor such as a pension fund pays for this option by receiving a reduced coupon payment (or no coupon) on the note. When issued, the closer the call option is to being in-the-money (or the further that it is in-the-money), the lower the coupon payment of the CLN. Let's assume that a plain-vanilla note (i.e., a note with no unusual features) with a face value of $\$ 1$ million from the issuer might carry a coupon rate of $6 \%$. Under normal circumstances, a CLN with the embedded call option might carry a coupon of only $2 \%$. In this case, the pension fund is sacrificing $4 \%$ of coupon income as the price of the call option on the S\&P GSCI.

Assume that at the time the note is issued, the S\&P GSCI is at $\$ 1,000$. Further assume that the strike price on the call option embedded in the note is set $10 \%$ out-ofthe-money, at $\$ 1,100$. If at maturity of the note the value of the S\&P GSCI is above $\$ 1,100$, in addition to receiving the original principal the investor receives its $2 \%$ coupon plus the appreciation of the S\&P GSCI above $\$ 1,100$ (assuming no default occurs). If the S\&P GSCI is at or below $\$ 1,100$, the investor is owed only the original principal and the coupon. Therefore, the final payout of the $\$ 1$ million CLN with a one-year maturity can be expressed as follows:

$$
\left\{\left[1+\max \left(0,\left(\mathrm{GSCI}_{T}-\mathrm{GSCI}_{X}\right) / \mathrm{GSCI}_{X}\right)\right] \times \$ 1,000,000\right\}+\$ 20,000
$$

where $\mathrm{GSCl}_{T}$ is the value of the $\mathrm{S} \& \mathrm{P} \mathrm{GSCl}$ at maturity of the note, and $\mathrm{GSCl}_{X}$ is the strike price for the call option embedded in the note. The $\$ 20,000$ is found as the $2 \%$ coupon multiplied by the $\$ 1$ million face value of the note, assuming annual coupon payments.

If the option expires out-of-the-money (the S\&P GSCI is less than or equal to the strike price of $\$ 1,100$ at maturity), then the investor receives the return of its principal plus a $2 \%$ coupon $(\$ 1,020,000)$. If the option expires in-the-money, then the investor is owed the strike price, the $2 \%$ coupon, plus the percentage gain of the index above the strike price applied to the principal.

For example, if the S\&P GSCI is at $\$ 1,155$ at maturity, the CLN returns a principal payment of $\$ 1,050,000$ in addition to the coupon payment of $\$ 20,000$. The $\$ 1,050,000$ principal payment is found as follows:

$$
\$ 1,000,000 \times[1+(\$ 1,155-\$ 1,100) / \$ 1,100]
$$

The investor (the pension fund) shares in the upside of the commodity price but is protected on the downside. The trade-off for the upside potential is a lower coupon payment relative to a note without the embedded call option. The issuer of the note presumably purchases a one-year call option on the S\&P GSCI as a hedge and, in effect, pays for that call option using savings from issuing a note with an otherwise below-market coupon.

The previous examples have the CLN's principal protected from downside commodity exposure and therefore had the payout of a call option. However, not all CLNs are principal protected. Some notes have principal payments that share fully in the change in value of commodity price changes-up or down. Thus, in this case the value of the principal owed at maturity can be either higher or lower than the note's face value. This may be viewed as a CLN linked to a futures contract instead of an option contract. Further, coupons may be linked to the commodity price or commodity index as well as to the principal.


\end{document}