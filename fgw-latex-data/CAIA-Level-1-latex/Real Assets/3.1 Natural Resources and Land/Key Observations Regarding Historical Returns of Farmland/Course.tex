\documentclass[11pt]{article}
\usepackage[utf8]{inputenc}
\usepackage[T1]{fontenc}
\usepackage{amsmath}
\usepackage{amsfonts}
\usepackage{amssymb}
\usepackage[version=4]{mhchem}
\usepackage{stmaryrd}

\begin{document}
Key Observations Regarding Historical Returns of Farmland

Farmland returns are quarterly returns observed from the first quarter of 2000 to the last quarter of 2021. The exhibit, Statistical Summary of Returns located in the previous lesson, Key Observations Regarding Historical Returns of Timberland provides univariate return statistics in the top panel, partial autocorrelations of returns (discussed in the section Higher-Order Autocorrelation and Partial Autocorrelation in the lesson, Covariance, Correlation, Beta, and Autocorrelation) in the middle panel and a histogram of returns in the bottom panel.

Key observations on farmland returns that are consistent with economic reasoning (and are consistent with and driven by the use of appraisals for valuations) are an essential component of knowledge and include the following:

\begin{enumerate}
  \item Farmland returns had low historic volatility relative to world equities.

  \item Farmland returns generated a very attractive Sharpe ratio of 1.3 .

  \item Farmland returns had a markedly positive skew.

  \item Farmland returns had a markedly positive excess kurtosis.

  \item Farmland returns reported an incredibly small drawdown (i.e., a $-0.1 \%$ drawdown).

  \item Farmland returns had a markedly high fourth-order partial autocorrelation, indicating a large one-year lag in recognizing changes in value.

  \item Farmland returns were very tightly clustered around their mean.

\end{enumerate}

In conclusion, note that both timberland and farmland have highly smoothed returns, as noted by the strong autocorrelation results. Analysts need to adjust for this artificially low level of volatility before using these risk estimates in asset allocation models.


\end{document}