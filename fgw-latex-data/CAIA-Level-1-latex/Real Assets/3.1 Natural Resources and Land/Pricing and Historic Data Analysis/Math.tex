\documentclass[11pt]{article}
\usepackage[utf8]{inputenc}
\usepackage[T1]{fontenc}
\usepackage{amsmath}
\usepackage{amsfonts}
\usepackage{amssymb}
\usepackage[version=4]{mhchem}
\usepackage{stmaryrd}

\begin{document}
\section*{APPLICATION A}
Question : An analyst observes a stale return series over a period of 50 weeks and finds a mean weekly return of $0.24 \%$. The analyst notes that the returns of the week prior to the most recent 50 returns (week 0) was $2.50 \%$ and the return of the most recent period (week 50 ) was $5.00 \%$. What is the mean return of the true return for weeks 1 to 50 based on the analyst's assumption that $\alpha=0.60$ ?

\section*{Answer and Explanation}
In order to solve this problem, we need to use Equation 2:

$$
\mu^{*}=\mu+\left(\frac{1}{T}\right)\left[(1-\alpha)\left(r_{i, o},-, a, r_{i, T}\right)\right]
$$

Substituting into Equation 2:\\
$0.24 \%=\mu+(1 / 50)[(1-0.6)(2.50 \%-5.00 \%)]$

$\mu=0.24 \%-(1 / 50)[.4(-2.50 \%)]=.26 \%$

The large period 0 and $T$ returns caused a relatively minor error $(0.02 \%)$ from using a stale mean to estimate a true mean.

\section*{APPLICATION B}
Question : An analyst observes a stale return series for an index based on appraised values and finds an annualized volatility of $16 \%$ over the same time period in which an index based on market values of otherwise identical assets exhibited an annualized volatility of $27.7 \%$. Based on the assumption that the returns from the series using appraisals is based on an equally weighted average of $\mathrm{N}$ data points (including the contemporaneous data point), how many data points are being averaged in order to estimate an appraised value?

\section*{Answer and Explanation}
In order to solve this problem, we need to use Equation 4:

$$
\sigma\left(r_{i, t}^{*}\right)=\frac{\sigma\left(r_{i, t}\right)}{\sqrt{N}}
$$

We are given $\sigma\left(r_{i, t}^{*}\right)=16 \%$ and $\sigma\left(r_{i, t}\right)=27.7 \%$ so we simply need to solve for $\mathrm{N}$.

$$
\begin{gathered}
\frac{\sigma\left(r_{i, t}\right)}{\sigma\left(r_{r, t}^{*}\right)}=\sqrt{N} \\
N=\left(\frac{\sigma\left(r_{i, t}\right)}{\sigma\left(r_{r, t}^{*}\right)}\right)^{2}=\left(\frac{0.277}{0.16}\right)^{2}
\end{gathered}
$$

$$
N=2.997 \approx 3.0
$$


\end{document}