\documentclass[11pt]{article}
\usepackage[utf8]{inputenc}
\usepackage[T1]{fontenc}
\usepackage{amsmath}
\usepackage{amsfonts}
\usepackage{amssymb}
\usepackage[version=4]{mhchem}
\usepackage{stmaryrd}

\begin{document}
Pricing and Historic Data Analysis

Stale prices are indications of value derived from data that no longer represent current market conditions. This section provides a simplified example of the effects of stale data on estimations of risk and return.

\section*{A Model of Stale Prices}
Consider asset $i$, a real asset with true return of $r_{i, t}$ in period $t$. Assume that the true prices of asset $i$ are observed on a delayed basis and are used by an appraiser. Thus, the appraiser's estimates of the value of asset $i$ are based on transaction prices revealed on a delayed basis. To simplify the analysis, assume that the return reported by the appraiser for asset $i$ in period $t, r_{i, t}^{*}$, is a blend of the contemporaneous true return $\left(r_{i, t}\right)$ and prior period's true return $\left(r_{i, t-1}\right)$ of asset $i$, with $\alpha$ proportion of the return based on the contemporaneous true return and $(1-\alpha)$ based on the previous period's return:


\begin{equation*}
r_{i, t}^{*}=\alpha r_{i, t}+(1-\alpha) r_{i, t-1} \tag{1}
\end{equation*}


For example, if the true return of asset $i$ in period 1 was $10 \%$, the true return in period 2 was $-5 \%$, and the value of $\alpha$ is 0.6 , the appraiser reports a return of $1 \%$ for period 2 . This return is found as: $(.6 \times-5 \%)+(.4 \times 10 \%)$.

The next two sections use this model to evaluate the effect of stale pricing on estimated means, volatilities, and correlations.

\section*{The Effect of Stale Pricing on Historic Mean Returns}
Consider a sample of $T+1$ true returns for asset $i$ from period 0 to period $T$, as well as a sample of $T$ stale (e.g., appraisal-based) returns for asset $i$ from period 0 to period $T$, which are calculated as discussed in the previous section. The estimated mean return using the stale return data for the $T$ periods from period 1 to $T$, $\mu^{\star}$, and the estimated mean return using the true return data for the $T$ periods from period 1 to $T$, $\mu$ :

$$
\begin{aligned}
\mu^{*} & =(1 / T) \sum_{t=1}^{T} r_{i, t}^{*} \\
\mu & =(1 / T) \sum_{t=1}^{T} r_{i, t}
\end{aligned}
$$

Equation 2 can be used to create a relation containing the mean of the stale return series, $\mu^{\star}$, based on the mean of the true returies $\mu$.


\begin{equation*}
\mu^{*}=\mu+\left(\frac{1}{T}\right)\left[(1-\alpha)\left(r_{i, 0}-r_{i, T}\right)\right] \tag{2}
\end{equation*}


Note that the term in brackets on the rightmost side of Equation 2 is the error of approximating $\mu$ based on the use of a stale price from period 0 when $r_{i, t}^{\star}$ was used in place of $r_{i, t}$ as well as a correction for using only $\alpha$ as the weight on $r_{i, T}$ rather than the full weight for period $T$. In other words, the only differences between calculating a mean with the stale returns and the true returns occurs as follows: The mean based on stale data overweights the return in period 0 and underweights the return in period $T$.

For example, consider a true series from time 0 to $6: 4 \%,-2 \%, 8 \%, 0 \%, 2 \%, 6 \%$. The true mean based on the last five returns is $2.8 \%$. Using Equation 1 , the series based on stale prices and $\alpha=0.5$ is: $1 \%, 3 \%, 4 \%, 1 \%, 4 \%$ (each calculated as an average of the current and previous period's true value) and has a mean of $2.6 \%$. Equation 2 isolates the source of the difference, which is due to the stale data's improper use of the period 0 true return (4\%) and the underweight of the period $T$ true return of $6 \%$. Substituting the numbers from this example into Equation 2 verifies the relation:

$$
2.6 \%=2.8 \%+(1 / 5)[(.5) \times(4 \%-6 \%)]
$$

Note that the above equation can be used to explain the difference between the mean returns of the true and stale return series without knowing the true returns from periods 1 to $T$ - 1 .

Here is the key point. The error in estimating the true mean of a return series by using stale returns based on stale prices occurs from overweighting the return prior to period 1 (i.e., it is included in the computation in the mean when it should not be included) and underweighting the return in the final period ( $T$ ) (by including only a partial weight). But, as the number of observations in the sample increases, the magnitude of the difference between the averages decreases. Therefore, for large samples there would typically be only a small difference between the mean based on stale returns and the mean based on true returns, so the use of stale valuation tends to have little effect on estimations of long-run returns. This is important information to understand when stale (or smoothed) return series are used.

\section*{The Effect of Stale Pricing on Volatility}
The key issue regarding volatility and stale (or smoothed) data is to infer the true but unobservable underlying return volatility from the volatility of the available return data (the return series with stale pricing).

Consider a smoothed return series that is formed as an equally weighted average of the true returns of the current time period and one or more previous time periods:


\begin{equation*}
\left(r_{i, t}^{*}\right)=(1 / N)\left[r_{i, t}+r_{i, t-1}+\cdots+r_{i, t-(N-1)}\right] \tag{3}
\end{equation*}


Note that the stale return averages the returns of the true series using the $N$ returns from the true return of the same period and the $N-1$ previous true returns (i.e., the stale returns are a simple moving average of the true returns).

The volatility of the left side of Equation 3 must equal the volatility of the right side since the two sides are always equal:

$$
\sigma\left(r_{i, t}^{*}\right)=\sigma\left\{(1 / N)\left[r_{i, t}+r_{i, t-1}+\cdots+r_{i, t-(N-1)}\right]\right\}
$$

If the true return series on the right side of the above equation is homoskedastic (constant volatility) and has no autocorrelation, the equation can be factored as shown in Equation 4:


\begin{equation*}
\sigma\left(r_{i, t}^{*}\right)=\sigma\left(r_{i, t}\right) / \sqrt{N} \tag{4}
\end{equation*}


Simply put, the observed volatility of the stale return series will equal the volatility of the true return series divided by the square root of $N$. For $N=2$, the return volatility of the true series will be higher than the volatility of the observed (stale) series by a factor of $\sqrt{2}$. For $N=4$, the stale price series will exhibit only half the volatility of the true return series.

The key point is that the observed volatility of a return series based partially on current data and partially on old data (i.e., stale valuations) will understate the true return volatility by a factor that can be economically significant.

Another challenge with using historical data based on appraisals is in the attempt to measure true return correlations. Correlations are a vital part of portfolio management because they are an important determinant of diversification. CAIA Level II provides details on adjusting for smoothed prices in measuring correlation and forming portfolios.


\end{document}