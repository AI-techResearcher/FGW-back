\documentclass[11pt]{article}
\usepackage[utf8]{inputenc}
\usepackage[T1]{fontenc}
\usepackage{amsmath}
\usepackage{amsfonts}
\usepackage{amssymb}
\usepackage[version=4]{mhchem}
\usepackage{stmaryrd}

\begin{document}
Farmland

Farmland represents ownership of a real asset (land), yet unlike many real assets, farmland also generates current cash flow, as crop income is a potentially steady and renewable stream of cash. Farmland differs from traditional real estate in that the annual cash flow is more closely linked to commodity prices (i.e., crop prices) rather than rent; therefore, the market price of farmland may be closely linked to commodity prices.

\section*{Farmland Valuation}
The value of farmland, like all assets, is the sum of the discounted future cash flows. As a potentially perpetual asset, farmland value can be modeled with the perpetuity formula.

The next exhibit provides an example of the potential return to farmland before income taxes. Assume that farmland costs $\$ 10,000$ per acre and that the investor purchases 30 acres, for a total investment of $\$ 300,000$. The landowner finances half the farmland with debt at $8 \%$, for a total interest expense of $\$ 12,000$ per year. The landowner receives as rent $\$ 1,000$ per acre, for an annual income of $\$ 30,000$. There are property taxes of $\$ 200$ per acre, for a total property tax expense of $\$ 6,000$. Insurance and other costs are $\$ 2,000$.

\begin{center}
\begin{tabular}{|lr|}
\multicolumn{1}{c}{Farmland} &  \\
\hline
Purchase price & $\$ 300,000$ \\
Financing & $\$ 150,000$ \\
Equity investment & $\$ 150,000$ \\
Annual revenues & $\$ 30,000$ \\
Less real estate taxes & $\$ 6,000$ \\
Less insurance & $\$ 2,000$ \\
Operating income & $\$ 22,000$ \\
Less interest & $\$ 12,000$ \\
Net income & $\$ 10,000$ \\
ROE $=\$ 10,000 / \$ 150,000=6.67 \%$ &  \\
\hline
\end{tabular}
\end{center}

The exhibit above shows that the return on equity (ROE) (net income/equity) is $6.67 \%$. The return on assets (operating income/assets) is $\$ 22,000 / \$ 300,000$, or $7.33 \%$. In real estate, the cap rate (capitalization rate) or yield is a common term for the return on assets ( $7.33 \%$ in this example). The concept is often used to value real estate so that the value of a property might be viewed as equal to the property's expected annual net operating income divided by an estimate of an appropriate cap rate:

Value of Real Estate $=$ Annual Operating Income $/$ Cap Rate

The annual operating income is the income before financing costs. When Equation 1 is used to value real estate, the cap rate (or yield) is a ratio based on observation of comparable real estate and professional judgment.

\section*{The Structure of Farmland Ownership and Management}
Row crops are crops that need to be replanted each year, such as soybeans and grains, including corn and wheat. Row cropland is annual cropland that produces row crops, such as corn, cotton, carrots, or potatoes from annual seeds. Row cropland comprises approximately $55 \%$ of the NCREIF Farmland Index, a major U.S. index of farmland values.

Permanent crops are crops that do not need to be replanted annually, such as tree-based crops (e.g., apples, oranges, nuts). Permanent cropland refers to land with long-term vines or trees that produce crops, such as grapes, cocoa, nuts, or fruit. To provide an indication of relative sizes, permanent cropland comprises approximately $45 \%$ of the NCREIF Farmland Index.

An investor in farmland does not necessarily actively manage the crops. Typically, the owner of the farmland leases the land to a local farmer, a cooperative, or even an agricultural corporation. Since lease payments are made on a calendar basis, the cash rents provide a steady stream of payments that are not tied to a particular growing season. Investment in farmland and other real assets operated by another party introduces agency risk. Agency risk is the economic dispersion resulting from the consequences of having another party (the agent) making decisions contrary to the preferences of the owner (the principal). Agency relationships are discussed in greater detail in subsequent sessions. In the case of farmland, the agency risk is the possibility, and perhaps the likelihood, that a farmer will fail to maximize the net economic benefits to the owner.

Farmland can be contrasted to the prior discussion of timberland. Timberland has great flexibility in terms of its harvest schedule, which can be timed to take advantage of better pricing. Conversely, farm crops must be harvested annually and generally within a window of just a few weeks. Some crops-such as wheat, soybeans, and corn-can be stored for one to two years, but beyond that, the crop begins to deteriorate (rot). Timber has a long growth cycle between seeding and harvesting. Farmland allows the farmer to harvest from seed to crop within one year. Farmland's shorter growth cycle provides annual cash flows and allows for a more valuable multipurpose option than timberland, since farmland's crop selection is a shorter-term decision, and there are numerous potential crops.

Another risk faced in farmland ownership as well as other forms of land ownership is political risk. Political risk is economic uncertainty caused by changes in government policy that may affect returns, perhaps dramatically. Political risk can arise both from government's failure to take beneficial actions and its initiation of harmful actions. For example, political risk of farmland ownership includes the risk that the government will terminate support payments, such as corn ethanol subsidies, and the risk that the government will abrogate ownership rights or expropriate land, as reportedly occurred in recent years in Venezuela.

\section*{Demand for and Supply of Agricultural Products}
The future demand for agricultural products could be driven by: (1) worldwide population growth rates, (2) substantially changed incomes in emerging markets leading to changing diets, including increasing consumption of animal protein and (3) the use of agricultural products in biofuels and other non-food-based end uses.

To the extent that the global population becomes wealthier and disposable incomes rise, dietary habits tend to shift toward agriculturally more intensive food products, such as increased consumption of meat and other animal proteins, as well as higher-value horticultural crops, such as fruit, vegetables, seeds, and nuts. This demand shift, in turn, leads toward increased demand for animal feed grains (corn, soybeans, etc.), as well as the land and infrastructure necessary for the production of horticultural crops. On a calorie basis, the production of feed grains needed for livestock production requires much more land than the production of the same calories were they consumed by humans directly in plant form.

Biofuels typically use agricultural products with food value, especially corn and sugar, to generate usable fuels or fuel additives. Biofuel production has engendered some controversy regarding its impact on food prices, particularly during periods of high commodity prices. Efforts to produce biofuels from nonfood agricultural products, like corn stalks and various high-biomass grasses, have met limited success to date. The growth in biofuels usage has created additional pressure on productivity. In the United States, a significant portion of acreage is devoted to producing corn destined for ethanol plants.

The future supply of agricultural products could be driven by: (1) changing agricultural yields, (2) changing quantities and qualities of agricultural infrastructure (including irrigation, transportation networks, and processing), and (3) the quantity of land under cultivation and/or changing use of aquaculture.

Growth in yields, particularly in the developed world, has occurred largely as a result of four advances: (1) improved technology (including advancement of seed stock through plant breeding and, in certain cases, transgenic modification); (2) improved agronomy; (3) increasing use of inputs, such as fertilizer; and (4) increasing use of capital assets, like machinery and agricultural infrastructure. Agronomy is the science of soil management, cultivation, crop production, and crop utilization. Agricultural infrastructure, like other forms of infrastructure, derives economic return largely from the value of efficiency gains. The key economic function of agricultural infrastructure is to increase productivity of the agricultural value chain.

\section*{Three Key Benefits and Three Key Disadvantages of Farmland Investment}
The key benefits of farmland investment are: (1) farmland as an inflation hedge, since farmland is a real asset linked to food and energy production and prices; (2) farmland as a diversifying source of return being subject to distinct physical and economic dynamics and not, in the short run, directly linked to financial markets; and (3) the supply of farmland may be more constrained than the demand for agricultural products.

Disadvantages of farmland investment include: (1) like most other forms of real estate investing, farmland is illiquid, with potential exposures to natural disasters; (2) the transaction costs of searching for, buying, and selling farmland tend to be high, with sales that are arranged through brokers that can charge fees of 3\% to 5\% for negotiating the sale of the land and with potentially high search costs; and (3) farmland ownership can involve high levels of agency costs.

\section*{Methods of Accessing Exposure to U.S. Farmland Returns}
There are three primary approaches for institutional investors to access agricultural asset returns: (1) direct ownership of farmland to earn lease income, (2) direct ownership of listed equities in agricultural firms or through pooled funds, and (3) long positions in agricultural futures contracts or similar financial derivative instruments. Note that the third use provides exposure to agricultural prices, not directly land prices, and may not be highly correlated with farmland values.

Regarding publicly traded pools related to agriculture and farmland, there are several stock indices that track the agribusiness industry. These industries vary in their exposure to publicly traded companies in four areas of the agribusiness industry: (1) agricultural products, (2) seed and fertilizer, (3) farm machinery, and (4) packaged foods.

The VanEck Vectors Agribusiness ETF (ticker MOO, a creatively descriptive ticker name) began trading in August 2007, and holds a portfolio of globally diversified stocks in the agribusiness industry. Publicly available REITs with farmland include Gladstone Land (LAND) and Farmland Partners (FPI). Returns are previously shown in the exhibit, Returns Based on Market Price in the Timber and Timberland lesson.

\section*{Three Factors of Multiple-Use Option Values}
The agricultural value of farmland is driven by the profitability of its agricultural use, which in turn is related to commodity prices and farming expenses. A prolonged surplus of a commodity, like corn, generates substantially lower commodity prices. Lower commodity prices, such as lower corn prices, can lead to depressed farmland prices, especially for land areas where corn production has traditionally served as the land's best use.

This highlights the value and importance of assets with multiple purposes, such as farmland. The value of the multiple purposes of farmland is driven by three factors related to the multiple uses (other than the moneyness of the current best use): (1) the current closeness of the profitability of each alternative to each other, (2) the volatility of the profitability of each alternative, and (3) the lack of correlation between the alternatives as to profitability.

For example, suppose that a farmer has two main crops that are suitable for the farmer's land and equipment: corn and soybeans. The option to plant either crop has high value if: (1) each crop becomes the best choice at least periodically, (2) both corn and soybeans have profitability that varies substantially through time, and (3) if corn and soybean profitability is only mildly or negatively correlated. In all three cases, the option to switch from one crop to the other has higher value.

Consider a region where planting one particular crop is consistently the best use of farmland. For example, in the United States, there is a major corn-producing region. In this region, other uses of the land often substantially lower profitability. In such cases, the options for alternative use may be viewed as being far out-ofthe-money. Therefore, the multipurpose aspect of the option has little value, and the land behaves more like a single-use option that is in-the-money. However, having several viable crops with volatile and uncorrelated prices is a valuable option.

The possible multipurpose option of farmland often extends well beyond multiple agricultural uses. Land that is currently most profitably deployed as farmland can become more valuable for other uses, such as development (residential, industrial) and mineral rights. Multiple-use options can be especially valuable when they include both agricultural and nonagricultural uses, because the correlation between the profitability of diverse uses tends to be lower than the correlation between the profitability of similar uses. Low correlation of uses generates higher option value, because when underlying assets diverge in profitability or value, the call option holder can benefit from the rise in the value of one use with limited harm from the fall in the value of the alternative use.


\end{document}