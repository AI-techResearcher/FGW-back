\documentclass[11pt]{article}
\usepackage[utf8]{inputenc}
\usepackage[T1]{fontenc}
\usepackage{amsmath}
\usepackage{amsfonts}
\usepackage{amssymb}
\usepackage[version=4]{mhchem}
\usepackage{stmaryrd}

\begin{document}
\section*{APPLICATION A}
\begin{center}
\begin{tabular}{|lr|}
\multicolumn{1}{c}{Farmland} &  \\
\hline
Purchase price & $\$ 300,000$ \\
Financing & $\$ 150,000$ \\
Equity investment & $\$ 150,000$ \\
Annual revenues & $\$ 30,000$ \\
Less real estate taxes & $\$ 6,000$ \\
Less insurance & $\$ 2,000$ \\
Operating income & $\$ 22,000$ \\
Less interest & $\$ 12,000$ \\
Net income & $\$ 10,000$ \\
ROE $=\$ 10,000 / \$ 150,000=6.67 \%$ &  \\
\hline
\end{tabular}
\end{center}

If the annual revenue in the exhibit, Farmland is expected to rise to $\$ 40,000$ and the market cap rate rises to $8 \%$. then with all other values remaining constant, the farmland's price would rise to $\$ 400,000[(\$ 40,000-\$ 6,000-\$ 2,000) / 0.08]$. With a price of $\$ 360,000$ and an annual operating income of $\$ 40,000$, what would the cap rate be?

\section*{Answer and explanation}
There is a lot of extra information provided in this application, but the key aspects are contained in the last sentence. The price or value of the farmland is $\$ 360,000$ and the annual income is $\$ 40,000$, so manipulating Equation 1 to read:

Cap Rate $=$ Annual Operating Income $/$ Value of Real Estate

We can solve for Cap rate:

$$
\begin{gathered}
\text { Cap Rate }=\text { Annual Operating Income } / \text { Value of Real Estate Cap } \\
\text { Rate }=\$ 40,000 / \$ 360,000 \\
\text { Cap Rate }=11.11
\end{gathered}
$$

The cap rate is $11.11 \%$


\end{document}