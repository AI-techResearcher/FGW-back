\documentclass[11pt]{article}
\usepackage[utf8]{inputenc}
\usepackage[T1]{fontenc}
\usepackage{amsmath}
\usepackage{amsfonts}
\usepackage{amssymb}
\usepackage[version=4]{mhchem}
\usepackage{stmaryrd}

\begin{document}
Contagion, Price Indices, and Biases

The previous sections implicitly viewed returns based on market prices as true indications of risk while viewing smoothed returns based on appraisals as flawed indications that underestimate the true risks. However, in some cases, there is considerable debate regarding the reliability of market prices versus appraisals. This section discusses whether the listing of real assets reveals risk or increases risk.

For U.S. real estate, there are reliable data on both appraised prices from unlisted properties and market prices of similar real estate held inside funds that trade in liquid markets. Often the returns computed from appraised values diverge substantially from the returns computed from market prices, even though the underlying real assets are similar. Specifically, the volatility of returns based on market prices is often substantially higher than the volatility of returns based on appraised values. A critical issue is whether the price volatility of listed real assets reflects true changes in the value of the real assets or whether the price changes reflect trading conditions in the equity markets. For example, if the equity market experiences a huge sell-off and the listed prices of real assets similarly decline, do the large losses correctly reflect actual diminished value of real assets or do they overstate the true losses?

Consider the next exhibit. The prices underlying the column based on market data can be observed daily and reflect up-to-the-minute indications from traders with regard to the value of publicly traded real estate held in REITs. According to the market data, the financial crisis began driving down real estate prices in February 2007. The total decline over the next 25 months was $73 \%$. The appraisal-based data is derived from U.S. commercial real estate appraisals, which are reported quarterly. The appraisal-based data did not reflect the start of a major decline in real estate prices until after the end of the third quarter of 2008-more than 1.5 years later-and indicated that the decline in real estate values lasted only six quarters. Further, as can be seen in the next exhibit, the full decline based on quarterly appraisals was only $24 \%$. It should be noted, however, that the market data is based on REITs that tend to be substantially leveraged. The difference in leverage could explain the large difference between the reported magnitudes of the declines during the financial crisis. However, it is the timing of the declines that raises a clear distinction between the information being signaled.

Market Prices and Appraisals Spanning the Financial

Crisis

\begin{center}
\begin{tabular}{|c|c|c|}
\hline
 & Market Data & Appraisal Data \\
\hline
Date of pre-crisis high & $2 / 2007$ & $6 / 30 / 2008$ \\
\hline
Date of subsequent low & 3/2009 & 12/31/2009 \\
\hline
Duration of decline & 25 months & 6 quarters \\
\hline
Size of decline & $-73 \%$ & $-24 \%$ \\
\hline
\end{tabular}
\end{center}

Sources: Market Data based on NAREIT daily closing prices from Bloomberg. Appraisal data from NCREIF Property Index (NPI) quarterly returns.

Did actual U.S. real estate values plunge from February 2007 to March 2009 or from June 2008 to December 2009? Did agreements regarding sales of commercial real estate begin to reflect lower prices in the United States as early as 2007? Did unleveraged commercial real estate in the United States decline only $24 \%$ from the quarterly high to the quarterly low during the financial crisis? Traditional expert-based appraisals and prices from listed equity markets provided entirely different indications. There is no consensus, but, clearly, indices based on traditional appraisals indicated the declines on a delayed basis. However, the market prices of REITs traded in the U.S. equity markets appear in retrospect to be driven at least in part by contagion.

Contagion is the general term used in finance to indicate any tendency of major market movements-especially declines in prices or increases in volatility-to be transmitted from one financial market or sector to other financial markets or sectors. When comparing the high volatility of listed real estate prices relative to appraised real estate prices, it may be argued that the high volatility of listed real estate prices is driven by contagion effects from other listed securities, such as the equities of operating firms that are listed on the same exchange. Within this interpretation, the high volatility of listed real estate prices were driven by potentially temporary contagion effects rather than indicating true volatility in the value of the underlying properties.

The primary question is: Do listed real asset prices overstate underlying asset volatility because they are unduly influenced by liquidity swings or mood swings in financial markets, or do appraised real asset values understate underlying asset volatility because they fail to reflect value changes on a full and timely basis due to smoothing?

One clue to the resolution of this question can be found in the definition of fair market value, as appraisers seek to measure it. A typical definition is "the amount of cash that a property would bring if exposed for sale in the open market under conditions in which neither buyer nor seller could take advantage of the exigencies of the other."1 California State Board of Equalization, Assessors' Handbook (emphasis added). For example, a liquidity crisis that motivated an owner to accept a relatively low price to convert a real asset into cash would be explicitly ignored in the process of appraising the value of that asset. In contrast, asset values and returns that are measured using actual transaction prices incorporate events such as liquidity crises, as the market events of October 2008 through March 2009 showed. Such events indisputably affect the values at which assets can be sold.

This issue of whether market prices or appraised values better reflect risk is central to the analysis of real assets and important to consider in the analysis of their risks and returns. In the Key Observations Regarding Historical Returns of Timberland lesson, both appraised values and market values of real assets are used to estimate historical mean returns and volatility. Clearly, the results need to be viewed in light of the likelihood that the reported volatility of farmland and timberland based on appraisals substantially underestimated the true volatility because of the use of smoothed valuations rather than market prices.


\end{document}