\documentclass[11pt]{article}
\usepackage[utf8]{inputenc}
\usepackage[T1]{fontenc}
\usepackage{amsmath}
\usepackage{amsfonts}
\usepackage{amssymb}
\usepackage[version=4]{mhchem}
\usepackage{stmaryrd}

\begin{document}
Timber and Timberland

Timber is investment in existing forestland for long-term harvesting of wood. Institutional investors have long recognized the benefits of investing in timberland assets.

\section*{The Structure of Timber and Timberland Ownership and Management}
Forests may be owned by the public sector or by firms or individuals in the private sector. Public ownership refers to the situation in which a government body exercises ownership jurisdiction over lands. Private ownership describes the situation in which individuals, firms, businesses, corporations, and even nongovernmental organizations possess ownership rights to forests. Overall, approximately $86 \%$ of the world's 4 billion hectares of forests are under public ownership. Africa, Asia, and Europe have the highest percentage of public forestland by continent at $90 \%$ or higher. The United States is unique among countries with large forest resource endowments because of the dominant role of private forests. In the United States, forests currently occupy about a third of the total land mass. Approximately $50 \%$ of the forests are privately owned.

Timber ownership has changed in recent decades. At one time the forest products industry was integrated, with firms owning all of the components of the process: trees, pulp mills, and sawmills. More recently there has been reduced integration. Reduced integration in the forest products industry refers to the increased separation of ownership of trees, pulp mills, and sawmills and is a key reason for changes in timberland ownership. The reduced integration occurred over the past 30 years, with timberland increasingly viewed not so much as part of an entire system but as an input into a different system. A rise in leveraged buyouts in the 1970s and 1980s helped break up the integrated companies. These buyouts followed the buyout strategy of purchasing companies that have multiple operating divisions and then breaking them up into their component parts and selling them off to the highest bidders. Corporate raiders in the 1980s recognized that timberlands owned by forest product companies were undervalued assets. Some forest product companies, such as International Paper and Boise Cascade, responded with preemptive action by selling off their timberland and establishing long-term wood supply agreements with the new owners.

A second reason for the change in ownership was the rise of timberland investment management organizations. Timberland investment management organizations (TIMOs) provide management services to timberland owned by institutional investors, such as pension plans, endowments, foundations, and insurance companies, and have been a key reason for changes in timberland ownership. The growth of TIMOs facilitated the migration of timber ownership from longtime corporate manufacturers of timber-related products. Most institutional investors rely on TIMOs to advise them about their investments in forestland. Instead of actually owning the timberland, TIMOs arrange for investors to buy the timberland and then manage the timberland on behalf of those investors. TIMOs usually collect a management fee and a share of the profits at harvest.

A key concept in timber management is rotation. Rotation is the length of time from the start of the timber (typically the planting) until the harvest of the timber. Natural stands of pine frequently require a rotation of 45 to 60 years. Hardwoods may need 60 to 80 years to produce high-quality saw-timber products. Even though intensive management of planted pine can shorten the rotation to approximately 25 to 35 years, the investment is still very long-term and subject to risk-such as fire, drought, and other natural disasters - as well as obsolescence due to innovation or government restrictions on ownership rights, such as harvesting.

But timber does offer harvesting flexibility, which is a timing option. A harvest schedule can be accelerated or postponed by several years in most cases, giving the owner the opportunity to time a harvest to coincide with personal income needs or to wait for a more favorable price situation. There can be a substantial value to delaying the harvest of timber for an additional year. Depending on age, weather, and location, Forest Research Group estimates that northern hardwood experiences a biological growth rate of $1 \%$ to $3.33 \%$ per year. ${ }^{1}$ Jack Lutz, "Biological Growth Rates and Rates of Return," Forest Research Notes 2, no. 3 (2005). Delaying harvest during a year of low timber prices earns an additional year of growth while waiting for timber to rise to a more profitable sales price. Also, timber can be used for a variety of purposes (firewood, pulpwood, chip-n-saw, home building), offering the option to put the timber into a variety of products. To the extent that the prices of the associated products are imperfectly correlated, the multipurpose option can add considerable value.

\section*{Four Publicly Traded Ways to Obtain Exposure to U.S. Timber Returns}
Most timberland is directly owned and privately traded by institutional investors. There are four key publicly traded ways to invest in timber in the United States. First, investors can directly own shares in publicly traded timber-related firms. Second, there are two major ETFs (exchange-traded funds in the United States, with ticker symbols WOOD and CUT) that have been developed to track timber firm values. The two ETFs have a combined market value of very roughly $\$ 1$ billion, track different indexes, and have returns that can differ substantially. The most popular way for retail investors to gain exposure to timber is through real estate investment trusts (REITs), which are discussed in sessions, Real Estate Assets and Debt and Real Estate Equity. There are four primary REITs that specifically invest in timberland and have combined values of over $\$ 30$ billion. Finally, there is a futures contract (Random Length Lumber contracts) that trades on the CME and offers exposure to timber prices, which are part of-but not perfectly correlated-with timberland prices. See the next exhibit.

Returns Based on Market Price

\begin{center}
\begin{tabular}{|lrc|}
\hline
June 2008-Jan. 2019 & Mean & Vol \\
\hline
WOOD & $7.07 \%$ & $23.41 \%$ \\
CUT & $7.33 \%$ & $23.52 \%$ \\
MOO & $4.95 \%$ & $20.87 \%$ \\
NRP & $-9.11 \%$ & $52.83 \%$ \\
Russell 3000 & $12.39 \%$ & $16.09 \%$ \\
\hline
\end{tabular}
\end{center}

\section*{Three Key Benefits and Three Key Disadvantages of Timber Investment}
The three key potential benefits of timber investment are: (1) it has the potential for returns that have a low correlation with traditional stocks and bonds (i.e., diversification potential), (2) timber offers flexibility in the timing of its harvesting (i.e., timing options that may lessen the risk exposure to short-term economic\\
fluctuations), and (3) timber may serve as an effective inflation hedge.

The three key disadvantages to timber include: (1) timber values are tied to cyclical industries such as housing that can experience prolonged slumps, such as the housing slump that began in 2007, (2) timber's long growth cycle makes its value subject to risks of changes in technology and other factors affecting demand that may occur during the long rotation periods, and (3) the potential for losses due to natural disasters or adverse changes in legal standards.


\end{document}