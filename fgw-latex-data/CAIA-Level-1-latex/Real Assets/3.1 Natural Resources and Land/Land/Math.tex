\documentclass[11pt]{article}
\usepackage[utf8]{inputenc}
\usepackage[T1]{fontenc}
\usepackage{amsmath}
\usepackage{amsfonts}
\usepackage{amssymb}
\usepackage[version=4]{mhchem}
\usepackage{stmaryrd}

\begin{document}
\section*{APPLICATION A}
Question : Consider a parcel of land that can be improved at a construction cost that depends on the overall health of the economy. If the economy improves (the up state), theland can be improved at a construction cost of \$100,000 and will create an improved property worth \$160,000. If the economy falters (the down state), theconstruction cost drops to \$80,000 and the improved property would be worth \$70,000. Comparable improved properties now sell for \$100,000.\\
The first step in valuing the land is to use the current price of comparable improved properties (\$100,000) and the two possible values of improved properties at theend of the period (\$160,000 and \$70,000) to determine the risk-neutral probability that the economy will improve.Using the same values except that the construction costs are fixed at $\$ 86,667$ (the original expected value), find the value of the land ?

\section*{Answer and Explanation}
Equation 1 is required to solve this problem:

$$
\text { Current Value }=\text { Expected Value }=(\text { UpValue } \times \text { UpProb })+[\text { DownValue } \times(1-\text { UpProb })]
$$

The first point to understand with this application is that it assumes we are under no obligation to develop the land. That is important to note as it impacts the DownValue state. If we were under an obligation to develop, then we would subtract the value of the land if the economy falters $(\$ 70,000.00)$ by the construction cost if the economy falters ( $\$ 86,667.00$, which is the same construction cost if the economy improves).

However, we find that if we subtract $\$ 70,000.00$ from $\$ 86,667.00$ for a difference of $(\$ 16,667.00)$. Therefore, under any probability we would not want to lose money on an investment and since we do not have the obligation to develop the land, the downstate will be $\$ 0$ with a probability of $2 / 3$ (the probability of the economy falters as outlined in the application). Now, we need to address the expected value of the UpValue. The UpValue is the difference between the $\$ 160,000.00$ (the value of land if the economy improves) and $\$ 86,667.00$ (the construction cost if the economy improves) or $\$ 73,333.00$.

To compute the expected value of the upstate we need to multiply the UpValue or up state payoff by $1 / 3$ (the probability that the up state payoff will occur or in this application it is the probability that the economy improves) for an expected value of $\$ 24,444.33$. Lastly, we need to sum the expected value of the up state payoff and the expected value of the down state payoff, $\$ 24,444.33$ plus $\$ 0$ equals an option price of the land of $\$ 24,444.33$.

\section*{APPLICATION B}
Land that remains undeveloped is estimated to generate an expected return of $5 \%$, and land that is developed is estimated to generate an expected single-period return of $25 \%$. If the probability that a parcel of land will be developed is $10 \%$ over the next period, what is its expected return?

\section*{Answer and Explanation}
To solve this, we will need to use Equation 2 (note: this equation is very similar to 1):

$$
\mathrm{E}\left(R_{l}\right)=\left[P_{d} \times E\left(R_{d}\right)\right]+\left[\left(1-P_{d}\right) \times E\left(R_{n d}\right)\right]
$$

Let's begin by computing the undeveloped expected value. We know that the return of the undeveloped parcel of land is $5 \%$ and the probability that the land remains undeveloped is $90 \%$, found my subtracting $10 \%$ (the probability that the land will be developed) from 1 . If we multiply $5 \%$ (the return of the undeveloped parcel of land) by $90 \%$ (the probability that the land will remain undeveloped) the product is the expected value of the undeveloped parcel of land or $4.5 \%$. Now the expected return of the developed land is calculated by multiplying $25 \%$ (the return of the land once developed) by $10 \%$ (the probability that the land will be developed) for a product of $2.5 \%$ (the expected return of the developed land). We are ready to compute the overall expected value of the parcel of land which is equal to the sum of $2.5 \%$ and $4.5 \%$ or $7 \% .7 \%$ is the expected value of the parcel of land.

\section*{APPLICATION C}
Land that remains undeveloped is estimated to generate an expected return of 5\%, and land that is developed is estimated to generate an expected single-period return of $25 \%$. If $20 \%$ of land in a database is developed in a particular year, by how much will an index based on land that remains undeveloped understate the average return on all land? Inserting the realized values into Equation 2 in place of the expected values generates that the mean return of a portfolio (with $20 \%$ development $)$ is $[(0.20 \times 0.25)+(0.80 \times 0.05)]=9 \%$. The historical index of returns based on land that remained undeveloped was $5 \%$. The negative survivorship bias was $4 \%$.

\section*{Answer and EXPLANATION}
We are calculating the expected returns of a land portfolio that contains both developed and undeveloped land. Let's begin by computing the undeveloped land expected returns. We know that the expected return of undeveloped parcels of land is $5.0 \%$ and the proportion of that the land remains undeveloped is $80 \%$ (found by subtracting $20.0 \%$ from 1). If we multiply $5.0 \%$ (the expected return of the undeveloped parcels of land) by $80.0 \%$ (the proportion of land left undeveloped) the product is the expected return of the undeveloped parcels of land or $4.0 \%$. Now the expected return of the developed land is calculated by multiplying $25.0 \%$ (the expected return of the land once developed) by $20.0 \%$ (the proportion of land developed) for a product of $5 \%$ (the expected return of the developed land). We are ready to compute the overall return of the land portfolio which is equal to the sum of $4.0 \%$ (expected return on undeveloped parcels of land) and 5.0\% (expected return on developed parcels of land) or $9.0 \%$. $9 \%$ is the expected value of land portfolio. Since the historical average index return of land that remained undeveloped is $5 \%$, the index will understate the returns of the land portfolio by $4 \%$. The answer is found by subtracting $5 \%$ (historical average index return of land that remained undeveloped) from $9 \%$ (returns of the land portfolio) for a difference of $4 \%$.


\end{document}