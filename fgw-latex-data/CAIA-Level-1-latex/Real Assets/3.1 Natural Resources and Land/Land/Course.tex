\documentclass[11pt]{article}
\usepackage[utf8]{inputenc}
\usepackage[T1]{fontenc}
\usepackage{amsmath}
\usepackage{amsfonts}
\usepackage{amssymb}
\usepackage[version=4]{mhchem}
\usepackage{stmaryrd}
\usepackage{hyperref}
\hypersetup{colorlinks=true, linkcolor=blue, filecolor=magenta, urlcolor=cyan,}
\urlstyle{same}

\begin{document}
Land

Raw, undeveloped, or unimproved land is land that is not currently generating substantial scarce resources, such as food, shelter, or recreation. The value of any such land must be attributable to the possibility or option that the land can be developed, improved, or otherwise transformed into being productive. The vast majority of land, by area, falls into the category of undeveloped and unimproved. In most jurisdictions, rights to minerals and other natural resources under the land are titled separately.

\section*{Land in Anticipation of Development}
A term for investment in and acquisition of undeveloped land or vacant lots is land banking. Land banking is the practice of buying vacant lots for the purpose of development or disposition at a future date. This practice is common in the home-building industry and allows home builders to secure land tracts for eventual use in the fulfillment of housing development pipelines.

Land banking most commonly refers to the acquisition of unimproved or raw land that sits in the anticipated path of residential growth, but the term also references improved vacant lots held by a third-party entity for home builders who have option agreements to use these lots as needed. This has allowed for the more efficient use of capital by home builders. The key investment strategy is to purchase at a relatively low cost land that is vacant, rural, or underutilized and hold it in anticipation of substantial value increases as the location emerges in the path of future development.

The value of land or lots is distinguished not only by location but also by the level of improvement or development. Generally, three types of lots can be purchased for investment:

Paper lots are sites that are vacant and approved for development by the local zoning authority but for which construction on streets, utilities, and other infrastructure has not yet commenced.

Blue top lots are at an interim stage of lot completion in which the owner has completed the rough grading of the property and the lots, including the undercutting of the street section, interim drainage, and erosion control facilities, and has paid all applicable fees required. At this stage, a home builder can obtain a building permit upon payment of the ordinary building permit fee.

Finished lots are fully completed and ready for home construction and occupancy. All entitlements, including infrastructure to the lot, finished grading, streets, common area improvements, and landscaping, have been completed. All development fees, exclusive of the building permit and inspection, have been paid.

In times past, home builders banked land and developed lots for their own accounts. As they have become increasingly sophisticated public companies, they have largely changed this practice, relying on joint ventures or third-party investors to bank land for them. Because of this, there has been an increased disintermediation of investment in raw land development. Institutional investors now provide a substantial share of the paper lots and finished lot inventories to home builders on an as-needed basis.

The attraction of land investment is based on the ability to purchase land at an attractive price relative to its potential value in development. However, this is a longterm investment strategy. The key risks depend on the type of residential land purchased and where it is located. Finished lots near a major metropolitan area are safer investments than is raw, undeveloped land. Lots far from urban areas trade at steep discounts to potential value because their development is longer-term and less likely, which implies higher risk and possibly more expenses from, for example, building paved roads and providing electricity and sewerage in a pioneering effort. Unfinished lots also face steep discounts because of the expenses required to develop lots into finished products. These concepts are best understood when land is viewed as a call option.

\section*{Land as an Option}
Investment in undeveloped land is an option on development much like investment in land with mineral rights. ${ }^{1}$ For previous discussions of underdeveloped land as options, see Sheridan Titman, "Urban Land Prices under Uncertainty," American Economic Review 75, no. 3 (June 1985): 505-14; and Joseph T. L. Ooi, C. F. Sirmans, and Geoffrey K. Turnbull, "The Option Value of Vacant Land," March 2006, \href{http://ssrn.com/abstract=952556}{http://ssrn.com/abstract=952556} or \href{http://dx.doi.org/10.2139/ssrn.952556}{http://dx.doi.org/10.2139/ssrn.952556} The strike price of the option is the cost of developing or improving the land (e.g., constructing an apartment building). The time to expiration of the option is typically unlimited. The receivable asset of the option is the combination of the land and its improvement or development (e.g., a finished apartment building with the land beneath it). The payoff of the option is the spread between the value of the completed project and the cost of constructing the project.

The cost of construction (i.e., the strike price of the option) tends to be correlated with the price of improved real estate. This is because the actions of developers tend to arbitrage the relative prices whenever the price of improved real estate substantially increases relative to the cost of development. The value of land as an option on development is therefore positively related to the excess of the value of completed real estate projects over the costs of construction. The volatility of the underlying asset is the volatility of the spread between the costs of construction and the value of the improved property. As with any option, the value of land is positively related to the anticipated volatility in the underlying asset. But since construction costs and completed real estate values are positively correlated, the value of the option is reduced relative to the value that would be obtained if the exercise price (construction costs) were fixed.

Land that has multiple potential uses is more valuable than land with a single potential use, all other things being equal. As long as the possible values to the various potential uses are imperfectly correlated, multipurpose land will have higher expected payouts and higher values. The reason is that each potential purpose for the land provides possible payouts that, if imperfectly correlated with the payouts of other purposes, generate higher volatility.

While land is generally a perpetual option, it should be exercised (i.e., developed) when the net benefits of development exceed the net value of retaining the option. Therefore, the decision to develop property can be modeled using option theory and depends on the moneyness of the option. The option value also depends on the volatility of the spread; the dividend yield (income) of the completed project; the risk-free rate; and any costs of holding the undeveloped land, such as property taxes, insurance, and maintenance.

\section*{Example of Land as a Binomial Option}
The Financial Economics Foundations session discussed binomial tree models and provided a single-period example of pricing an option when the price of the underlying asset for the downward branch had a price of zero. In this section, the binomial approach is expanded to allow nonzero prices for the underlying asset in both branches of the tree.

For simplicity, this example is single period and assumes that the risk-free interest rate is zero. These assumptions allow the use of a simplified version of a powerful option-modeling technique called binomial option pricing. Binomial option pricing is a technique for pricing options that assumes that the price of the underlying asset can experience only a specified upward movement or downward movement during each period.

Consider a parcel of land that can be improved at a construction cost that depends on the overall health of the economy. If the economy improves (the up state), the land can be improved at a construction cost of $\$ 100,000$ and will create an improved property worth $\$ 160,000$. If the economy falters (the down state), the construction cost drops to $\$ 80,000$ and the improved property would be worth $\$ 70,000$. Comparable improved properties now sell for $\$ 100,000$.

The first step in valuing the land is to use the current price of comparable improved properties $(\$ 100,000)$ and the two possible values of improved properties at the end of the period $(\$ 160,000$ and $\$ 70,000)$ to determine the risk-neutral probability that the economy will improve. A risk-neutral probability is a probability that values assets correctly if, everything else being equal, all market participants were risk neutral. A risk-neutral probability may be viewed as being equal to a statistical probability that has been adjusted for risk so that it can be used to price risky assets in a risk-neutral framework. More details are provided regarding risk-neutral probabilities in Topic 6 . By assuming that the riskless interest rate is zero, we enjoy the simplicity in this example of not needing to discount future cash flows. So the current value of a comparable property must equal its end-of-period expected value based on risk-neutral probabilities, as shown in Equation 1:


\begin{align*}
& \text { Current Value }=\text { Expected Value }=(\text { UpValue } \times \text { UpProb })+[\text { DownValue } \\
& \quad \times(1-\text { UpProb })] \tag{1}
\end{align*}


where UpValue equals value in the up state, DownValue equals value in the down state, UpProb equals the risk-neutral probability of the up state, and (1 - UpProb) equals the risk-neutral probability of the down state (a faltering economy).

Inserting the comparable property's current value and possible property values into Equation 1 generates a solution for the probabilities:

$$
\$ 100,000=(\$ 160,000 \times \text { UpProb })+[\$ 70,000 \times(1-\text { UpProb })]
$$

Solving this equation generates UpProb $=1 / 3$, which means that the risk-neutral probability that the economy will falter is $2 / 3$.

The second step is to insert the probabilities calculated in the first step into Equation 1 to compute the value of the option (the land). The key is to compute the value of the two development outcomes. In the up state, the developer earns $\$ 60,000(\$ 160,000-\$ 100,000)$. In the down state, the developer loses $\$ 10,000(\$ 70,000-$ $\$ 80,000)$ by developing, so let's assume for simplicity that the developer donates the land to a nature conservancy rather than suffering a cash loss. The value of the option (the land) is the weighted average of the outcomes, since the riskless interest rate is zero.

$$
\text { Option Price }=(\$ 60,000 \times 1 / 3)+(\$ 0 \times 2 / 3)=\$ 20,000
$$

Thus, the value of the land is $\$ 20,000$. Simply put, there is a one-third chance that the economy will do well, netting the developer $\$ 60,000$, and a two-thirds probability that the land will be abandoned to charity. The power of binomial option pricing models emanates from setting or calibrating the probabilities of each path based on market-observed values of efficiently priced assets and then using those probabilities to price an option.

While extremely simplified, this binomial option pricing framework can demonstrate important principles, as illustrated in the following examples.

The option model approach may be used for a variety of purposes, such as computing volatility given an option price and computing probabilities given an option price. The application of binomial option pricing, even in this simplified example, demonstrates the ability of option theory to provide insight into risk. In addition to including a nonzero riskless interest rate, an analyst may wish to consider multiple time periods in applying the option approach.

\section*{Risk and Return of Investing in Land}
Investment in land is a departure from the traditional forms of real estate investment by institutional investors, who tend to purchase commercial real estate that is then leased, providing both capital appreciation and an annual cash flow. As a result, land development tends to be riskier and more speculative than other real estate investing, owing primarily to its lack of revenue, its long holding period, and its uncertain prospects. However, raw land does not deteriorate in value the way developed real estate does. Whereas developed properties require constant upkeep to maintain their value, the downside risk of owning undeveloped land can be low.

Land may be viewed as a call option. As with the expected return of a call option on an equity, the expected return of land depends on its systematic risk. The expected return of land is a probability-weighted average of the expected return of the land if it remains undeveloped and the expected return of the land if it is developed:

$$
E\left(R_{l}\right)=\left[P_{d} \times E\left(R_{d}\right)\right]+\left[\left(1-P_{d}\right) \times E\left(R_{n d}\right)\right]
$$

where $E\left(R_{l}\right)$ equals expected return on land, $P_{d}$ equals probability of development, $E\left(R_{d}\right)$ equals expected return conditioned on land being developed, and $E\left(R_{n d}\right)$ equals expected return conditioned on land not being developed.

Undeveloped land is sometimes criticized as an investment with poor returns, based on the observation that values of undeveloped land do not increase substantially through time. However, historical returns of undeveloped land may suffer from a negative survivorship bias. A negative survivorship bias is a downward bias caused by excluding the positive returns of the properties or other assets that successfully left the database. In this case, a return index on properties that remained undeveloped excludes the high returns obtained on the properties that were developed.

Returning to the option view of land, land that does not get developed tends to be land that in retrospect was a bad investment (unexercised options). Land that gets developed tends to have been a successful investment (exercised options). Consequently, price indices of undeveloped land tend to understate the expected returns of all undeveloped land because they ignore the success stories, meaning the land that became developed during the period in which the returns are being observed.

In most investment analyses, survivorship bias is positive. In the cases discussed in subsequent sessions, the problem is that the index ignores the negative returns of investments that fail. In the case of undeveloped land, the properties that remain in the category tend to be the failures. By excluding the favorable outcomes, historical indices of undeveloped property may substantially understate mean returns and falsely portray undeveloped land as a poor investment.


\end{document}