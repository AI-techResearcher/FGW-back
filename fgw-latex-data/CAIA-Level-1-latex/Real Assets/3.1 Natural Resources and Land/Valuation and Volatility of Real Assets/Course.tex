\documentclass[11pt]{article}
\usepackage[utf8]{inputenc}
\usepackage[T1]{fontenc}
\usepackage{amsmath}
\usepackage{amsfonts}
\usepackage{amssymb}
\usepackage[version=4]{mhchem}
\usepackage{stmaryrd}

\begin{document}
Valuation and Volatility of Real Assets

Private real assets and other assets that are not publicly traded do not have observable market values and instead are often valued by appraisals. This section discusses the effects that smoothing from the appraisal process can exert on return and price volatility.

\section*{Smoothing of Reported Values Can Reduce the Perceived Riskiness of an Asset}
Smoothing is reduction in the reported dispersion in a price or return series. Smoothed returns can mask true risk. An example from money markets illustrates this important concept. Consider a one-year U.S.-government-guaranteed certificate of deposit (CD) and a one-year U.S. Treasury bill (T-bill). The two investments offer the same risk-free cash flow in one year. Assuming that the one-year CD is nonnegotiable and has a substantial withdrawal penalty, the CD is riskier than the one-year T-bill because the T-bill offers the investor better liquidity.

However, the methods of reporting the values of the two securities may vary. Most investors receive financial statements of their positions in T-bills indicating that the market prices of the T-bills fluctuate as interest rates fluctuate. In many financial statements, on the other hand, CDs are given a very stable value that accrues slowly at the CD's coupon rate and ignores the impact of interest rate changes on present values. This accounting simplification causes a smoothed reported price series relative to the economic reality.

The smoothing of the CD prices causes the CD returns to be smoothed. When interest rates change, the true value of a fixed-rate CD changes regardless of whether the valuation method used for accounting purposes recognizes the volatility. The owner of a CD observing the smoothed prices might wrongly conclude that the CD is less risky than the T-bill because its reported value is more stable. Of course, the reality is that the T-bill is less risky because it offers better liquidity.

\section*{Smoothing of True Values to Reduce Reported Risk Measures}
The previous section discussed when reported prices are smoothed relative to their true values. This section discusses smoothing the true values of a portfolio, such as when market transactions are executed by an investment manager with the goal of reducing high returns and buttressing low returns. For example, consider an investment manager with a large portfolio of actively managed equities. The manager regularly buys out-of-the-money put options on a market index while simultaneously writing out-of-the-money calls on that same index during each reporting period. The manager obtains enough cash from writing the calls to fund the purchase of the put options each period. The puts eliminate large losses, and the calls eliminate large profits. The net result is a series of returns in which both the extreme upside and downside returns are smoothed. The result is lower actual and reported volatility.

For simplicity, consider an investment that experiences the following six months of returns (not necessarily in this order): $-3 \%,-2 \%,-1 \%,+1 \%,+2 \%$, and $+3 \%$. Since this series has a sample mean of $0 \%$, the sample variance of the series is simply $(1 / 5) \times\left[\left(-0.03^{2}\right)+\left(-0.02^{2}\right)+\left(-0.01^{2}\right)+\left(0.01^{2}\right)+\left(0.02^{2}\right)+\left(0.03^{2}\right)\right]$. The sample volatility (or standard deviation) of the monthly return series is $2.37 \%$ (rounded). Now consider the measured volatility if the returns of the best and worst months are changed to $+2 \%$ and $-2 \%$ from $+3 \%$ and $-3 \%$ using the option strategy discussed in the previous section. The sample variance of this new series is $(1 / 5) \times\left[\left(-0.02^{2}\right)+\left(-0.02^{2}\right)+\right.$ $\left.\left(-0.01^{2}\right)+\left(0.01^{2}\right)+\left(0.02^{2}\right)+\left(0.02^{2}\right)\right]$, and the sample volatility is $1.90 \%$ (rounded). The reduction in volatility can be a legitimate risk-management technique or can be used to "game" reported risk.

If the highest and lowest returns are smoothed, the observed volatility can be substantially reduced. In the above example, the observed volatility of the smoothed series is approximately $80 \%$ of the size of the unsmoothed series. Smoothing also affects the measured correlation between returns on different assets, as is discussed in detail in Level II of CAIA, wherein portfolio issues are emphasized.

\section*{Managed Returns and Volatility}
Managed returns are returns based on values that are reported with an element of managerial discretion. There are four primary ways that values and returns can be managed: favorable marks, selective appraisals, model manipulation, and market manipulation.

A favorable mark is a biased indication of the value of a position that is intentionally provided by a subjective source. For example, a trader may ask a brokerage firm to provide an indication of the value of a thinly traded asset for reporting purposes when the trader has reason to believe that the brokerage firm has an incentive to bias the valuation process in a particular direction to assist its client. Favorable marks may be used to obtain high real estate appraisals that enable larger mortgages.

Selective appraisals refers to the opportunity for investment managers to choose how many, and which, illiquid assets should have their values appraised during a given quarter or some other reporting period. Appraisals are relatively expensive, so the normal practice is to appraise a subset of assets infrequently (e.g., annually or even once every three years) and to quote asset values between appraisals using inexpensive internal updates. This practice enables investment managers to alter the timing of appraisals and the selection of properties to be appraised to manage reported returns.

Model manipulation is the process of altering model assumptions and inputs to generate desired values and returns. Model manipulation can occur in complex unlisted derivative transactions and other unlisted assets that are valued using models. The reported values can be manipulated by altering the parameter values that are inserted into the model. For example, use of higher estimates of asset volatilities can generate higher option prices.

Market manipulation refers to engaging in trading activity designed to cause the markets to produce favorable prices for thinly traded listed securities. As an example of this extreme practice, a buy order may be placed very near the close of trading to generate a higher closing price (or, conversely, a sell order may be placed to generate a lower closing price) in order to report more favorable returns for the current period or to smooth price variations, since valuations are frequently based on closing prices. To the extent that investment managers and fund managers are rewarded for exhibiting stable returns, there is an incentive to reduce observed volatility by managing returns. Smoothing can also be generated inadvertently. In the case of real assets, the appraisal process can introduce smoothing, as discussed in the next section.

\section*{Appraisals and Return Smoothing Due to Behavioral Biases}
The valuation of many real assets is based on appraisals-that is, by an expert's opinion of value. Appraisals are performed with a variety of methods, including comparative sales, analysis of net assets, and discounted cash flows (or income). Real asset appraisals, such as those of land, timberland, farmland, and other real estate, can be especially subjective because of the heterogeneity of the assets and the resulting ambiguities in comparative analyses. A key issue in appraised valuation is the tendency of appraisals to generate a smoothed series of prices that stray from market-based indicators of values and changes in values.

Much has been written about human nature and the potential tendency of appraisers to be overly conservative and reluctant to modify their beliefs regarding valuation levels. Behavioral finance theory cites an anchoring effect in which participants place an inordinate importance on previously accepted beliefs. Appraisers in 2007 had reported virtually continuous quarterly price increases in commercial real estate for 12 years. It is possible that these appraisers were reluctant to conclude that the trend suddenly had reversed until well into the financial crisis when substantial evidence had emerged of a directional change. Note also that transaction prices may be deceptive if real estate sellers are reluctant to sign contracts for sales at a price substantially lower than the previous appraisal.

\section*{Four Causes of Return Smoothing Due to Reliance on Infrequent Transactions or Stale Data}
The raw data behind an index based on appraisals are subjective estimates that often rely heavily on analyses of transactions data. This section discusses five potential causes of return and price smoothing.

First, in highly illiquid markets such as those for natural resources, timberland and farmland, there can be a substantial gap in time between the date at which a deal is struck and the date at which the transaction is consummated. Appraisers often are forced to rely on data observed from the dates on which transactions are consummated because the transactions and prices are not typically revealed publicly until after the deal is completed. The delay between the agreement on a price and its revelation to the public can cause a substantial delay in the recognition of price changes in appraisals.

Second, the transactions that occur in illiquid markets may be biased indications of widespread valuation changes. For example, it is possible that transaction data in the early stages of an economic slowdown might focus on sales of high-quality properties at relatively high prices, whereas the data in the early stages of a recovery might be drawn more from sales of lower quality properties at or near the previously observed lowest prices.

Third, managerial discretion can often be used to time or select appraisals to smooth performance. In some cases, the property manager's decision of when to update particular appraisals are not random but rather are selected carefully to manage apparent returns-delaying bad news and sometimes saving some of the good news for a future time. In addition, returns can be managed through model manipulation defined as inflated or deflated model inputs to generate particular values. One example of this practice is the use of an unrealistically low discount rate that has the effect of elevating the property's value. A favorable mark (i.e., a biased indication of value that is provided by a third party) can be used to inflate the reported value of a portfolio of real assets.

Fourth, appraisals may rely on data regarding revenues (e.g., rental income) and expenses (e.g., maintenance contracts) that themselves exhibit time delays in reflecting the effects of changes in market conditions. For example, actual rental revenue does not change to reflect changes in market conditions until leases are renewed. Note that variable delays in recognition of changes in market conditions tend to dampen the volatility of appraisal-based prices.

In addition to its effect on volatility, smoothing can also reduce, perhaps substantially, the estimates of correlation such as the correlation between a price series of real assets based on appraisals and a price series of financial securities based directly on market prices. However, in a later section of this session smoothing will be shown to have a minimal impact on estimated long-term average rates of return.


\end{document}