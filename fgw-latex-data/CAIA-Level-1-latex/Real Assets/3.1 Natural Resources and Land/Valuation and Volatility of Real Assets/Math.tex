\documentclass[11pt]{article}
\usepackage[utf8]{inputenc}
\usepackage[T1]{fontenc}
\usepackage{amsmath}
\usepackage{amsfonts}
\usepackage{amssymb}
\usepackage[version=4]{mhchem}
\usepackage{stmaryrd}

\begin{document}
\section*{APPLICATION A}
A fund manager follows a strategy that is expected to generate equally likely outcomes of $+7 \%,+3 \%,+2 \%$, or $-4 \%$ per period. The manager enters into financial derivatives at zero initial cost that cap the fund's returns at $3 \%$ while providing a downside protective floor of a $0 \%$ return. What is the reduction in the fund's true volatility from using the financial derivatives?

\section*{Answer and Explanation}
A true mean and volatility can be calculated because the probabilities provided are known rather than estimated based on a sample. The true mean is $2 \%$. The true variance without the derivative strategy is simply the average of the squared deviations $(1 / 4)\left[\left(.05^{2}\right)+\left(.01^{2}\right)+0+\right.$ $\left.\left(-.06^{2}\right)\right]=0.00155$ for a volatility of .03937 . The true variance with the derivative strategy is $(1 / 4)\left[\left(.01^{2}\right)+\left(.01^{2}\right)+0+\left(-02^{2}\right)\right]=0.00015$, for a volatility of .01225 . The derivative-protected strategy has a volatility that is roughly $31 \%$ of the original strategy.


\end{document}