\documentclass[11pt]{article}
\usepackage[utf8]{inputenc}
\usepackage[T1]{fontenc}
\usepackage{amsmath}
\usepackage{amsfonts}
\usepackage{amssymb}
\usepackage[version=4]{mhchem}
\usepackage{stmaryrd}

\begin{document}
\section*{APPLICATION A}
Question : Assume that U.S. Treasury notes with a seven-year maturity are currently yielding $5.8 \%$, that the liquidity premium is $1 \%$ per year, and that the required or anticipated risk premium for the systematic risk of the real estate project is $2.2 \%$ per year. What is the required rate of return for this real estate project ?

\section*{Answer and Explanation}
There are two ways to solve this application.  Let's start by calculating the required rate of return using equation 5 :

$$
\begin{aligned}
r & \approx R_{f}+E\left(R_{L P}\right)+E\left(R_{R P}\right) \\
r & \approx 0.058+0.01+0.022 \\
r & \approx 0.068+0.022 \\
r & \approx 0.09
\end{aligned}
$$

The approximate required rate of return is $9 \%$. Now, let's calculate the exact required rate of return using Equation 4:

$$
\begin{aligned}
r & =\left[1+R_{f}\right]\left[1+E\left(R_{L P}\right)\right]\left[1+E\left(R_{L P}\right)\right]-1 \\
r & =[1+0.058][1+0.022][1+0.01]-1 \\
r & =[1.058][1.022][1.01]-1 \\
r & =[1.081276][1.01]-1 \\
r & =[1.09208876]-1 \\
r & =0.09208876
\end{aligned}
$$

The exact required rate of return is $9.21 \%$.

\section*{APPLICATION B}
Question : Investment A offers $\$ 80$ per year in taxable income and an additional final non-taxable cash flow in five years of $\$ 1,000$. An investor in a $40 \%$ tax bracket requires a pre-tax return of $8 \%$ and an after-tax return of $4.8 \%$ on investments. What is the value of Investment A on both a pre- tax basis and an after-tax basis?

\section*{Answer and Explanation}
On a pre-tax basis, Investment A is worth $\$ 1,000$, found on a financial calculator as $P M T=\$ 80, F V=\$ 1,000, N=5, I=8 \%$, solve for $P V$. On an after-tax basis, the $\$ 80$ annual\\
income is worth $\$ 48[\$ 80 \times(100 \%-40 \%)]$. On an after-tax basis, Investment A is also worth $\$ 1,000$, found on a financial calculator as $P M T=\$ 48, F V=\$ 1,000, N=5, I$ $=4.8 \%$, solve for $P V$.

In both cases the problem is a simple "bond" problem of discounting a combination of an annuity and a lump sum.


\end{document}