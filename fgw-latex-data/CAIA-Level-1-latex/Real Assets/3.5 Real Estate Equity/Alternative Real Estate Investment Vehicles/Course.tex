\documentclass[11pt]{article}
\usepackage[utf8]{inputenc}
\usepackage[T1]{fontenc}
\usepackage{amsmath}
\usepackage{amsfonts}
\usepackage{amssymb}
\usepackage[version=4]{mhchem}
\usepackage{stmaryrd}

\begin{document}
Alternative Real Estate Investment Vehicles

Several alternative real estate investment vehicles are available, some of which have been recently introduced. (New alternative real estate investment vehicles are anticipated to be launched in the coming years.) These alternative investments include both private and exchange-traded products. This lesson begins by discussing the main characteristics of the following private real estate alternative investments: commingled real estate funds, syndications, joint ventures, and limited partnerships. The remainder of the lesson focuses on public real estate investments, including open-end real estate mutual funds, closed-end real estate mutual funds, and equity real estate investment trusts (REITs).

\section*{Private Equity Real Estate Funds}
Private equity real estate funds are privately organized funds that are similar to other alternative investment funds, such as private equity funds and hedge funds, yet have real estate as their underlying asset. Three specific types of private equity real estate funds (commingled real estate funds, syndications, and joint ventures) are discussed in the sections that follow. These funds collect capital from investors with the objective of investing in the equity or the debt side of the private real estate space. The funds follow active management real estate investment strategies, often including property development or redevelopment. Private equity real estate funds usually have a life span of 10 years: a two- to three-year investment period and a subsequent holding period during which the properties are expected to be sold.

The primary advantage to an investor is the access to private real estate, especially useful for smaller institutions that are limited in the size of the real estate portfolios they are able to construct directly. However, even for larger institutions, there are advantages to investments in private equity real estate funds (and also to commingled real estate funds, explained in the next section), as these investment vehicles can provide access to larger properties in which an institution may be reluctant to invest alone because of the unique asset risk it would need to bear and because a single asset could account for a portfolio allocation that might be too high. The use of private equity real estate funds can also provide access to local or specialized management or to specific sectors and markets in which the institution does not feel it has sufficient market knowledge or expertise.

However, investments through private equity funds do not allow investors direct control over the real estate portfolio and its management. In addition to the loss of control, private equity fund investors often lack a sufficiently liquid exit route. Another major issue with private equity funds is the difficulty of reporting the values of the underlying properties. Hence, the reported performance may not be accurate, and there may be considerable time or uncertainty in realizing reported performance. The finite life of this vehicle tends to make the funds a "hold to liquidation" instrument.

\section*{Commingled Real Estate Funds}
Commingled real estate funds (CREFs) are a type of private equity real estate fund that is a pool of investment capital raised from private placements that are commingled to purchase commercial properties. The investors are primarily large financial institutions that receive a negotiable, although non-exchange-traded, ownership certificate that represents a proportionate share of the real estate assets owned by the fund. Generally, CREFs are closed-end in structure (i.e., without additional shares issued or old shares redeemed), with unit values reported through annual or quarterly appraisals of the underlying properties. Other than the negotiability of the ownership certificates, the advantages and disadvantages of CREFs are similar to those of other private equity real estate funds.

\section*{Syndications}
Syndications are private equity real estate funds formed by a group of investors who retain a real estate expert with the intention of undertaking a particular real estate project. A syndicate can be created to develop, acquire, operate, manage, or market real estate investments. Legally, real estate syndications may operate as REITs, as corporations, or as limited or general partnerships. Most real estate syndications are structured as limited partnerships, with the syndicator performing as general partner and the investors performing as limited partners. This structure facilitates the passing through of depreciation deductions, which are normally high, directly to individual investors, and potentially circumvents double taxation.

Syndications are usually initiated by developers who require extra equity capital to raise money to begin a project. Syndications can be a form of financing that offers smaller investors the opportunity to invest in real estate projects that would otherwise be outside their financial and management competencies. Syndicators profit from both the fees they collect for their services and the interest they may preserve in the syndicated property.

\section*{Joint Ventures}
Real estate joint ventures are private equity real estate funds that consist of the combination of two or more parties, typically represented by a small number of individual or institutional investors, embarking on a business enterprise such as the development of real estate properties. An example of a joint venture would be the case of an institutional investor with an interest in investing in real estate, but with no expertise in this area, that agrees to form a joint venture with a developer. A joint venture can be structured as a limited partnership, an important form of real estate investment that is explained in the next section.

\section*{Limited Partnerships}
Private equity real estate funds, including the three types described in the previous sections, are increasingly organized as limited partnerships. Not only have real estate funds increasingly adopted limited partnership structures, but existing limited partnerships-such as private equity and hedge funds-have increasingly entered the real estate market. As with other limited partnership structures, a private real estate equity fund's sponsors act as the general partner and raise capital from institutional investors, such as pension funds, endowments, and high-net-worth individuals, who serve as limited partners. Generally, the initial capital raised is in the form of commitments that are drawn down only when suitable investments have been identified.

Limited partnership funds in real estate have largely adopted a more aggressive investment, reflected by gearing. Gearing is the use of leverage. The degree of gearing can be expressed using a variety of ratios. In real estate funds, a popular gearing ratio is the percentage of a fund's capital that is financed by debt divided by the percentage of all long-term financing (e.g., debt plus equity). This ratio is often called the LTV (loan-to-value) ratio or the debt-to-assets ratio. Many traditional\\
real estate funds have limited, if any, gearing, whereas a large proportion of the new private equity real estate limited partnerships have LTVs as high as $75 \%$. Gearing ratios are also commonly expressed as the ratio of debt to equity.

Limited partnerships have also tended to adopt the fee structures commonly in place in private equity funds. In addition to an annual management fee, commonly in the region of $1 \%$ to $2 \%$ of assets under management, the newer funds have introduced performance-related fees, commonly in the region of $20 \%$ of returns. Generally, the incentive-based performance fees are subject to some form of hurdle rate or preferred return. The fund sponsors (or general partners) usually contribute some capital to the fund (e.g., $8 \%$ to $10 \%$ ), thus potentially benefiting not only from the explicit incentive and management fees but also from their share of the limited partnership's return through their investments.

\section*{Open-End Real Estate Mutual Funds}
Open-end real estate mutual funds are public investments that offer a non-exchange-traded means of obtaining access to the private real estate market. These funds are operated by an investment company that collects money from shareholders and invests in real estate assets following a set of objectives laid out in the fund's prospectus. Open-end funds initially raise money by selling shares of the funds to the public and generally continue to sell shares to the public when requested. Open-end real estate mutual funds allow investors to gain access to real estate investments with relatively small quantities of capital. These funds often allow investors to exit the fund freely by redeeming their shares (potentially subject to fees and limitations) at the fund's net asset value, which is computed on a daily basis.

However, these funds may limit investors' ability to redeem units and exit the fund when, for example, a significant percentage of shareholders wish to redeem their investments and the fund is encountering liquidity problems. These liquidity problems can be exacerbated when the real estate market is either booming or declining. Given that upward and downward phases in real estate prices may last a considerable length of time, some analysts may view real estate valuations used in some net asset value computations as trailing true market prices in a bull market (and trailing declines in a bear market).

The use of prices that lag changes in true market prices is known as stale pricing. Stale pricing of the net asset value of a fund provides an incentive for existing shareholders to exit (sell) during declining markets and new investors to enter (buy) during rising markets. These actions of investors exploiting stale pricing may be viewed as transferring wealth from long-term shareholders in the fund to the investors exploiting the stale prices. The reason that the purchase transactions in a rising market transfer wealth from existing shareholders to new shareholders is that the stale prices are artificially low and permit new shareholders to receive part of the profit when the fund's net asset value catches up to its true value. Conversely, sales transactions during declining markets transfer wealth from remaining shareholders to exiting shareholders because the stale prices are artificially high and permit the exiting shareholders to receive proceeds that do not fully recognize the true losses, leaving the true losses to be disproportionately borne by the remaining shareholders when the fund's net asset value falls to its true value.

During declining markets, an open-end fund may face redemption problems and be forced to sell some of its real estate assets at deep discounts to obtain liquidity. To protect long-term investors and fund assets, many open-end real estate mutual funds increasingly opt to reserve the right to defer redemption by investors to allow sufficient time to liquidate assets in case they need to do so.

In summary, investors in open-end mutual funds are typically offered daily opportunities to redeem their outstanding shares directly from the fund or to purchase additional and newly issued shares in the fund. This attempt to have high liquidity of open-end real estate fund shares contrasts with the illiquidity of the underlying real estate assets held in the fund's portfolio. This liquidity mismatch raises issues about the extent to which investors will receive liquidity when they need it most and whether realized returns of some investors will be affected by the exit and entrance of other investors who are timing or arbitraging stale prices.

\section*{Options and Futures on Real Estate Indices}
Derivative products allow investors to transfer risk exposure related to either the equity side or the debt side of real estate investments without having to actually buy or sell the underlying properties. This is accomplished by linking the payoff of the derivative to the performance of a real estate return index, thus allowing investors to obtain exposures without engaging in real estate property transactions or real estate financing.

Challenges to real estate derivative pricing and trading include difficulties that arise with the highly heterogeneous and illiquid assets comprising the indices that underlie the derivative contracts. The indices underlying the derivatives may not correlate highly to the risk exposures faced by market participants, and therefore use of the derivatives for hedging may introduce basis risk, discussed in Section 3.2, Commodity Risk Attributes. Nevertheless, real estate derivatives may offer the potential for increased transparency and liquidity in the real estate market.

\section*{Exchange-Traded Funds Based on Real Estate Indices}
Exchange-traded funds (ETFs) represent a tradable investment vehicle that tracks a particular index or portfolio by holding its constituent assets or a subsample of them. They trade on exchanges at approximately the same price as the net asset value of the underlying assets due to provisions that allow for the creation and redemption of shares at the ETF's net asset value. The actions of speculators attempting to earn arbitrage profits by creating and selling ETF shares when they appear overpriced in the market or buying and redeeming ETF shares when they appear underpriced in the market tend to keep ETF market prices within a narrow band of the underlying value of the ETF. These funds have the advantage of being a relatively low-trading cost investment vehicle (in the case of those ETFs that have reached a particular size or popularity among investors); they can be tax efficient; and they offer stock-like features, such as liquidity, dividends, the possibility to go short or to use with margin, and, in some cases, the availability of calls and puts. Exchange-traded funds based on real estate indices track a real estate index such as the Dow Jones U.S. Real Estate Index, which raises issues of basis risk to hedgers. Other ETFs, such as the FTSE NAREIT Residential, track a REIT index. Since REITs are publicly traded, the use of ETFs on REITs may offer cost-effective diversification but may not offer substantially distinct hedging or speculation opportunities.

\section*{Closed-End Real Estate Mutual Funds}
A closed-end fund is an exchange-traded mutual fund that has a fixed number of shares outstanding. Closed-end funds issue a fixed number of shares to the general public in an initial public offering, and in contrast to the case of open-end mutual funds, shares in closed-end funds cannot be obtained from or redeemed by the investment company. Instead, shares in closed-end funds are traded on stock exchanges.

A closed-end real estate mutual fund is an investment pool that has real estate as its underlying asset and a relatively fixed number of outstanding shares. Unlike open-end funds, closed-end funds do not need to maintain liquidity to redeem shares, and they do not need to establish a net asset value at which entering and exiting investors can transact with the investment company. Most important, unlike with open-end funds, the closed-end funds themselves and their existing shareholders are not disrupted by shareholders entering and exiting the fund, especially in an attempt to arbitrage stale prices. This is because shareholders buy and sell shares on secondary markets rather than affecting fund liquidity by redeeming shares or subscribing to new shares.

Since closed-end funds are not required to meet shareholder redemption requests, the fund structure is generally more suitable for the use of leverage than that of open-end funds. The closed-end structure is frequently used to hold assets that investors often prefer to hold with leverage, such as municipal bonds. Similarly, the closed-end fund structure has advantages for investment in relatively illiquid assets, and is often used for such assets as real estate and emerging market stocks.

Like other closed-end funds, closed-end real estate mutual funds often trade at premiums or substantial discounts to their net asset values, especially when net asset values are not based on REITs, since REITs have market values. Closed-end real estate mutual funds usually liquidate their real estate portfolios and return capital to shareholders after an investment term (typically 15 years), the length of which is stated at the fund's inception.

\section*{Equity Real Estate Investment Trusts}
This introduction to public equity real estate investment products concludes with a discussion of equity REITs. REITs are a popular form of financial intermediation in the United States. This discussion focuses on equity REITs, which are REITs with a majority of their underlying real estate holdings representing equity claims on real estate rather than mortgage claims.

An equity REIT acquires, renovates, develops, and manages real estate properties. It produces revenue for its investors primarily from the rental and lease payments it receives as the landlord of the properties it owns. An equity REIT also benefits from the appreciation in value of the properties it owns as well as any increase in rents. In fact, one of the benefits of equity REITs is that their rental and lease receipts tend to increase along with inflation, making REITs a potential hedge against inflation.

One of the biggest advantages is that REITs are publicly traded. Most REITs fall into the capitalization range of $\$ 500$ million to $\$ 5$ billion, a range typically associated with small-cap stocks and the smaller half of mid-cap stocks. The market returns on equity REITs have been observed to have a strong correlation with equity market returns, especially the returns of small-cap stocks (and to a slightly lesser extent those of mid-cap stocks).

The strong correlation of equity REIT returns with the returns of similarly sized operating firms raises a very important issue. Are the returns of equity REITs highly correlated with the returns of small stocks because the underlying real estate assets are highly correlated with the underlying assets of small stocks? Or is this correlation due to the similar sizes (total capitalization values) of REITs and small-cap stocks and the fact that they are listed on the same exchanges? The explanation that REIT returns are highly correlated with the returns of similarly sized operating firms due to the similarity of the risks of their underlying assets seems dubious.

Commercial real estate valuations tend to depend on projected rental income, whereas operating firm valuations tend to depend on sales of products and services that are generally unrelated to real estate. However, the idea that the shared size and shared financial markets explain the correlation runs counter to traditional efficient capital market theory. Financial theory implies that market prices reflect underlying economic fundamentals rather than trading location and size or total capitalization. To the extent that REIT prices are substantially influenced by the nature of their trading would mean that observed returns are more indicative of stock market fluctuations and less indicative of changes in underlying real estate valuations. However, due to problems with other approaches, REIT returns form the basis for the empirical analyses presented at the end of this chapter.

There is no consensus on whether the high correlation of REIT returns with equity market returns is unique to public REITs or is simply masked in private REITs and appraisal-based returns. Further, there is no consensus on the implications of the observed strong correlation of public REIT returns and overall equity market returns on the risk premiums offered by public REITs.


\end{document}