\documentclass[11pt]{article}
\usepackage[utf8]{inputenc}
\usepackage[T1]{fontenc}
\usepackage{amsmath}
\usepackage{amsfonts}
\usepackage{amssymb}
\usepackage[version=4]{mhchem}
\usepackage{stmaryrd}

\begin{document}
Commercial Real Estate Valuation

Valuation is central to finance and essential to real estate analysis. The distinction between the terms valuation and appraisal, if any, vary by source and by context (e.g., residential vs. commercial). In the context of commercial real estate analysis, real estate valuation is often used as a general term describing processes of estimating the worth of a property from various perspectives (especially the perspective of a financial analyst) with regard to the price at which informed investors would be willing to both buy and sell a property.

In the context of commercial real estate analysis, a real estate appraisal is generally viewed as a formal opinion of a value provided by an appraiser and often used in financial reports and in decision-making including lending. For institutional investors, appraisals are often performed once a year on real estate properties and are used in reporting.

In the case of private commercial real estate equity, valuation challenges arise because the respective assets are heterogenous and are not exchange-traded like stocks or bonds of public companies. Each real estate asset valuation is unique, dealing with its unique individual characteristics such as location. Real estate assets are also notorious for their illiquidity, as an individual property may not be traded for a considerable number of years. Nevertheless, in spite of the difficulties, and perhaps because of those difficulties, commercial real estate valuation is necessary.

\section*{Three Reasons for the Emphasis on Commercial Real Estate Equity}
Most of the focus in the remainder of this chapter is on the valuation of equity exposures to private commercial (i.e., income producing) real estate rather than on public real estate, residential real estate, or commercial mortgages. There are three major reasons for institutional focus on equity participations in commercial real estate:

\begin{enumerate}
  \item Most commercial real estate throughout the world is privately held rather than publicly traded (whether owned directly or held through limited partnerships).

  \item Most of the equity of residential real estate is held by the occupier of the property rather than by an institutional investor.

  \item The valuation of equity claims to private commercial real estate drives the pricing of the credit risk in the valuation of commercial mortgages. In other words, real estate debt may be viewed through the structural model (detailed in Level II of the CAIA program) as being well explained through an understanding of the risks of the equity in the same property (since assets equal debt plus equity).

\end{enumerate}

\section*{The Comparable Sales Approach as a Major Valuation Approach}
This section summarizes the first of five approaches used for valuing private commercial real estate equity: the comparable sale prices approach. The other four approaches are the profit approach, the cost approach, the income approach, and multifactor transaction-based approaches.

For non-income-producing properties, such as an owner-occupied single-family residence, a discounted cash flow approach is not viable. In these cases, when there are sufficient transactional data available, valuations are often based on the comparable sale prices approach.

The comparable sale prices approach values real estate based on transaction values of similar real estate, with adjustments made for differences in characteristics by a valuation professional such as an appraiser. The comparable sales approach is one of the two primary valuation methods in real estate (the other being the income approach). In the comparable sales approach, the real estate asset is evaluated against the reported prices of comparable (substitute) properties that have recently transacted. Value adjustments may be made for characteristics such as square footage, date of sale, location, and amenities.

In this approach, the real estate appraiser looks at sales of similar properties in the same geographic region (if not city) as the property being appraised. These actual sale prices give the appraiser an estimate of the cost (i.e., price) per square foot of similar real estate properties. The appraiser then adjusts this cost per square foot for the unique characteristics of the property being appraised: better parking or access, better location, newer lobby, longer-term tenants, and so on.

The comparable sales approach has the advantage of being based on actual sales transactions. However, the approach tends to contain substantial subjectivity in the valuation of most characteristics other than the focal point (e.g., price per square foot). Multifactor transaction-based approaches, detailed later, objectively quantify multiple factors.

The accuracy of the comparable sales approach (and other transaction-based approaches) is lower when there is a lack of frequency of property sales. Every property is unique, so it is hard to adjust a square-foot calculation value from one property to form the value of another accurately. When the comparable sale prices approach is not viable because the number of recent and relevant real estate transactions is very limited, alternative approaches may be used based on either or both of two components: (1) the replacement construction costs of the structure and (2) indications of the market value of the site for its most profitable use. These two approaches are briefly discussed in the next section. More sophisticated approaches based on transaction prices are also discussed briefly later in this chapter and are detailed in Level II of the CAIA curriculum.

\section*{The Profit and Cost Approaches}
The profit approach to real estate valuation is typically used only for properties with a value driven by the actual business use of the premises; it is effectively a valuation of the business rather than a valuation of the property itself. Thus the profit approach should only be used when the value of the property is based primarily on the value of the business that occupies the space. The profit approach is a special case of the income approach, detailed later.

The cost approach assumes that a buyer will not pay more for a property than it would cost to build an equivalent one. In this approach, a property's value is initially based on its cost and can be further refined by adding the values of any improvements to the land value of the property and applying economic depreciation as appropriate. This approach is often suggested when valuing newer structures and in markets with substantial new construction.

\section*{Cap Rates and the Perpetuity Valuation Approach}
A somewhat crude but widespread metric in real estate valuation is the capitalization (cap) rate. The cap rate of a real estate investment is the net operating income (NOI) of the investment divided by some measure of the real estate's total value, such as purchase price or appraised value:


\begin{equation*}
\text { Cap Rate }=\text { NOI } / \text { Value } \tag{1}
\end{equation*}


where $\mathrm{NOI}$ is usually viewed on an annualized basis and represents the expected, normalized cash flow available to the owner of the real estate, ignoring financing costs (e.g., cash flows from rent, net of operating expenses). The variable "Value" used in Equation 1 is an estimate of the market value of the real estate on an unlevered basis.

The exact specifications of the NOI for the numerator of the cap rate (recent, current, forecasted) and the value (beginning of period versus end of period, transaction price versus appraised price) for the denominator of the cap rate vary between users and purposes. Note that NOI does not reflect financing costs, and therefore the $\mathrm{NOI}$-based approach for estimating value depicted in Equation 1 is intended for analysis of unleveraged property values.

Cap rates are often viewed as direct estimates or forecasts of expected returns or required returns. Thus, according to this view, a property with a cap rate of $9 \%$ is expected to generate a return of $9 \%$ to the investor on an unleveraged basis. The view of a cap rate as an estimated expected return is at best a crude approximation in that it typically ignores anticipated capital gains or losses as well as anticipated growth or decline in income. Nevertheless, cap rates are a good starting point for an analysis of expected returns.

Cap rates can be viewed as required rates of return and used as risk-adjusted discount rates to perform risk-adjusted present values in the valuation of properties using discounted cash flow approaches. To illustrate the risk adjustment, an investor may search for a core property that offers a cap rate of at least $7 \%$ while demanding a cap rate of at least $9 \%$ on a value-added property because it is perceived as having higher risk.

Observed cap rates are often used to establish values for particular properties. Note that Equation 2 can be formed by rearranging Equation 1 to express property value as depending on $\mathrm{NOI}$ and cap rate:


\begin{equation*}
\text { Value }=\text { NOI } / \text { Cap Rate } \tag{2}
\end{equation*}


A prospective buyer or analyst divides the estimated NOI of a property by an observed cap rate on properties of similar risk or characteristics to obtain an estimate of the property's unleveraged value.

Other metrics that are commonly used to assess real estate investment opportunities mirror the approaches discussed in private equity, including the internal rate of return (IRR) approach and the multiple of the cash in (the cash projected to be received from an investment on a project) to the cash out (the investment in the project). Numerous ratios of operating performance, interest coverage, and leverage are also used, most of which are analogous to financial ratios used throughout corporate finance. Cap rates represent a tool that is widely used in real estate but less so in other areas of finance.

NCREIF and other organizations estimate cap rates for various commercial property types in the United States. The overall weighted average of cap rates for core commercial property in the United States has varied between $5 \%$ and $10 \%$ since the mid-1980s. The higher end of the range was reached in the mid-1990s and again in 2002 in the United States, whereas the lower end of the range preceded the financial crisis that began in 2007 and was observed again in 2021.

\section*{The Income Approach as a Major Valuation Approach}
The income approach, which is comprised of the direct capitalization method and the discounted cash flow method (DCF), is used to appraise the market value of an income producing property. Market value is defined as the present value of an expected cash flow stream. The income approach is one of the two primary valuation methods in real estate. The Details of the Income Approach to Real Estate Valuation and Illustration of the Income Method of Real Estate Valuation lessons discuss this important approach in detail.

In the income approach, several years of net operating income are projected for a specific property (or portfolio of properties) and then discounted using an estimate of an appropriate discount rate. This approach is particularly useful when valuing income-producing real estate assets, such as commercial real estate. The income\\
approach (or discounted cash flow method) has become the most accepted practice by real estate appraisers for commercial properties. The approach has the advantage of valuing the unique characteristics of the property being appraised. However, it requires the estimation of an appropriate discount rate and is subject to forecasting errors of cash flows due to errors in forecasting occupancy rates, lease growth rates, expenses, the holding period for the property, the terminal value of the property, inflation estimates, and the like.

\section*{Transaction-Based Methods (Repeat-Sales and Hedonic)}
Transaction-based real estate valuation methods are based on relatively large data sets of actual transaction prices of properties within a specified time period and include the repeat-sales and hedonic methods. They may be viewed as more systematic, comprehensive and quantitatively sophisticated forms of the comparable sales approach. Quantitative techniques such as multiple regression are used on relatively large data sets (compared to most comparable sales applications) to build a property's valuation based on multiple attributes.

Transaction-based methods can form a reliable basis for real estate valuation when:

\begin{enumerate}
  \item They are performed with adequate data and with rigorous econometric methods.

  \item Differences among the properties are modeled well.

  \item Statistical noise in the data is minimal.

\end{enumerate}

The two main methods used to estimate transaction-based price indices are the repeat-sales method (RSM) and the hedonic pricing method (HPM). Both methods are detailed in Level II of the CAIA curriculum.

\section*{Two Advantages of Appraisal-Based Models over Transaction-Based Models}
All valuation methods discussed earlier except the two purely transaction-based methods (the repeat-sales and hedonic methods) are viewed as appraisal-based methods, even though the comparable sales approach includes transactions prices as input data.

There are two primary advantages of appraisal-based models:

\begin{enumerate}
  \item In general, they do not suffer from a small sample size problem.

  \item All properties can be appraised frequently and by multiple experts, although this is a costly process.

\end{enumerate}

\section*{Four Disadvantages of Appraisal-Based Models over Transaction-Based Models}
There are four primary disadvantages of appraisal-based models:

\begin{enumerate}
  \item Appraisals are inherently subjective and backward-looking, thus introducing potential errors in the resulting valuation.

  \item In the case of real estate price indices based on appraisals, not all properties are reappraised as frequently as the index is reported. For example, an appraisalbased real estate price index published every quarter is likely to be based on appraisals that are fully performed only annually. This may cause a stale appraisal effect (i.e., errors from the use of dated appraisals), which contributes to the lagged recognition of price changes observed in appraisal-based indices.

  \item Appraisal-based indices are smoothed compared with actual changes in real estate market values, meaning that substantial value changes tend to be reflected on a delayed basis. Thus, measures of volatility of the value of commercial real estate assets based on appraisal-based models are underestimated. Fortunately, unsmoothing techniques (detailed in Level II of the CAIA curriculum) help mitigate this problem.

  \item Appraisal-based methods tend to rely on data from comparable properties. Therefore, the quality of the appraisal will depend critically on the relevance and quality of available data. As a result, appraisals may not be accurate in situations in which multiple comparable properties cannot be identified (e.g., the current property is rather unique) or when there is a significant time lag between the time the data have become available and the time the appraisal takes place.

\end{enumerate}

\section*{The NCREIF Property Index as an Appraisal-Based Index}
The National Council of Real Estate Investment Fiduciaries (NCREIF) is a large U.S. not-for-profit institutional real estate investment industry association that collects data from its members, which include, for the most part, institutional real estate investment managers. NCREIF maintains a massive data set of real estate income and pricing data and uses those data to publish a major real estate index and sub-indices of commercial properties as well as several other indices, such as a farmland index and a timberland index. The NCREIF Property Index (NPI) is a large, popular, value-weighted index published quarterly and is based on unleveraged commercial-property appraisals (or leveraged data adjusted to an unleveraged basis). NCREIF sub-indices are available on sub-categories, including property types and geographical differentiation by region, division, state, and zip code.

The NPI is based on financial information from member institutional investors. Members are required to report information on their real estate holdings on a quarterly basis. The NPI started in the fourth quarter of 1977. Recently the NPI consisted of over 7,500 properties (including the five major categories of apartment, industrial, hotel, office, and retail properties) with a gross fair market value of roughly $\$ 600$ billion. Most valuations are appraisal-based. The reason for the use of appraisals is the illiquid nature of real estate: Properties simply do not turn over frequently enough to compute short-term returns using prices from transactions performed in an arm's-length manner.

The change in value of each property in the NPI is calculated every quarter on an "as if" basis: as if the property were purchased at the beginning of the quarter at its appraised value, held for income during that quarter, and sold at the end of the quarter at its end-of-quarter appraised value. If the property was actually acquired or sold during the quarter, the transaction price is used in place of either the beginning or the ending value.

The total return on the index is calculated as the sum of an income return and a capital value return. The income portion of the total return of each property is a fraction, with net operating income (NOI) in the numerator and an estimate of the property value in the denominator. The estimate of the property value is based on the beginning-of-period appraised value, with adjustments for any capital improvements, any partial sales, and reinvestment of NOI.

The numerator of the capital value return is the change in the estimated value of the property from the beginning of the quarter to the end of the quarter, adjusted for capital improvements and partial sales such that increases in the value due to further investments are not included as profits, and declines in value due to partial sales are not deducted as losses. The denominator of the capital value return is the same as the denominator for the income portion.

The NPI is calculated on an unleveraged basis, as if the property being included in the index were purchased with $100 \%$ equity and no debt. As a result, the returns are less volatile, and there are no interest charges deducted. The calculation of the returns to the NPI is performed on a before-tax basis (and therefore do not include income tax expense) and is performed for each individual property and then value-weighted in the index calculation. The turnover of most real estate properties is infrequent (every six or seven years, on average), so the NPI is based primarily on appraised values rather than market transactions.

Although the NPI is a quarterly index, NCREIF properties are not formally appraised every quarter. Most properties are formally valued at least once per year, but many are appraised only every two or even three years. Appraisals cost money; therefore, there is a trade-off between the benefits of having frequent property valuations and the costs of those valuations as a drain on portfolio performance. In fact, many institutional real estate investors value their portfolio properties only when they believe there is a substantial change in value based on new leases, changing economic conditions, or the sale of a similar property close to the portfolio property.

Even when properties have been recently appraised, it is possible that the appraisal process will be driven by old information, such as previous transactions on comparable properties, or by delays in the willingness of appraisers to adopt new valuation standards brought on by changes in market conditions, such as capitalization (cap) rates. Thus, even recent appraisals can cause smoothing due to delays in fully reflecting changes in true value. Note that the NPI is published quarterly but that quarter-end values are published with a time lag. Thus, even ignoring appraisal-based smoothing, a major decline in asset prices that occurs in October would not be reflected in quarterly index figures until the December 31 appraisal, and the December 31 value would not be published until almost a month later. The Investment Property Databank (IPD) Index is another example of an appraisal-based index. It tracks retail, office, and industrial properties in the United Kingdom and includes data on actual property transactions from property companies and institutional investors. The IPD Index is available monthly and annually.

In contrast, real estate market indices, such as real estate indices based on REIT market prices, are continuously updated and may be able to reflect the market's quickly changing perception of real estate values. Transaction-based indices may offer more timely recognition of appraisal-based indices. NCREIF has begun producing the Transaction Based Index (TBI). The TBI is a hedonic index (discussed in a previous section) that uses transaction data from the NCREIF database. The relative merits of market-based indications of real estate values (i.e., REITs), transaction-based indications of real estate values, and appraisal-based indications of real estate values is a very important topic in the estimation of the risks of real estate investments and therefore the appropriate risk premium. This topic is explored in detail in Level II of the CAIA program.


\end{document}