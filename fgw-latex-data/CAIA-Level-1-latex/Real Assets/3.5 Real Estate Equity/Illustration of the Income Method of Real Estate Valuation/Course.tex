\documentclass[11pt]{article}
\usepackage[utf8]{inputenc}
\usepackage[T1]{fontenc}
\usepackage{amsmath}
\usepackage{amsfonts}
\usepackage{amssymb}
\usepackage[version=4]{mhchem}
\usepackage{stmaryrd}

\begin{document}
Illustration of the Income Method of Real Estate Valuation

Valuation of the income approach may be viewed as having three stages: estimating the cash flows, estimating a discount rate, and calculating the value using the DCF method.

To illustrate the mechanics of the income approach (discounted cash flow model) using an example, suppose a German-based real estate company needs to estimate the value of an office building that it just purchased in Munich. First, project the net operating income for the next four years (see the next exhibit, Projection of Net Operating Income over Next Four Years).

Projection of Net Operating Income over Next Four Years

\begin{center}
\begin{tabular}{|lcccc|}
\hline
\multicolumn{4}{c}{Projection of Net Operating Income over Next Four Years} &  \\
\hline
Potential gross income & $€ 10,000,000$ & Year $\mathbf{2}$ & Year 3 & Year 4 \\
Vacancy and collection losses & $€ 1,000,000 € 1,000$ & $€ 11,025,000 € 11,576,250$ &  &  \\
Effective gross income & $€ 9,000,000 € 9,450,000 € 10,143,000 € 10,650,150$ &  &  &  \\
Operating expenses & $€ 3,800,000 € 3,990,000 € 4,189,500 € 4,398,975$ &  &  &  \\
Net operating income & $€ 5,200,000 € 5,460,000 € 5,953,500 € 6,251,175$ &  &  &  \\
\hline
\end{tabular}
\end{center}

The potential gross income and operating expenses of the first year of operations have been estimated at $€ 10$ million and $€ 3.8$ million, respectively. For simplicity, assume that these amounts are received at the end of the year. A 10\% vacancy loss rate is being assumed for this investment for the first two years, decreasing to $8 \%$ for the subsequent two years. Assume that the real estate company expects to maintain the office building for four years, and that rents and operating expenses are estimated to increase by $5 \%$ per year.

The office building is projected to be sold in four years, providing estimated net sales proceeds of $€ 75$ million at that time. Using discounted cash flow analysis, appraise the value of this office building. The real estate company is using a required rate of return of $6 \%$ for comparable investments. Ignore taxes for simplicity. First, project the net operating income for the next four years (see Projection of Net Operating Income over Next Four Years). The present value of the four years of $\mathrm{NOI}$ and the net sales proceeds to be received in four years, discounted at $6 \%$, is $€ 79,122,255$. This is the appraised value of the office building using the income approach.


\end{document}