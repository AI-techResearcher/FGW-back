\documentclass[11pt]{article}
\usepackage[utf8]{inputenc}
\usepackage[T1]{fontenc}

\begin{document}
Equity REIT Returns

Equity REITs can be either publicly traded or organized as a private fund. Investors are concerned about how the risk and return varies across these two structures.

\section*{Private versus Public REITs}
As introduced and briefly described in the session, Real Estate Assets and Debt, REITs are a popular form of financial intermediation in the United States. This discussion focuses on public equity REITs, which are REITs with a majority of their underlying real estate holdings representing equity claims on real estate rather than mortgage claims. Private REITs are also an important form of real estate investment that are offered to investors through private structures that are exempt from SEC registration in the United States. However, throughout this session and the discussion in the session, Real Estate Assets and Debt on REITs, the focus is on publicly traded REITs. Publicly traded REITs incur higher costs in order to be registered as a public security and to trade on an exchange. However, many investors prefer the liquidity of public REITs and the potential safety from the regulatory oversight as well as the price revelation available from public trading.

\section*{Do Public REITs Offer an Illiquidity Premium?}
As in the case of products such as closed-end funds, REITs hold illiquid private investments, which generate returns that are passed through public REIT structures to form ownership units (shares) that are liquid. A fascinating and unresolved issue is whether owners of publicly traded REITs receive an expected return that includes a risk premium for illiquidity. On the one hand, the underlying properties are illiquid and it is possible that they are acquired at prices that are discounted for their illiquidity given that the real estate trades in a market dominated by private investors. On the other hand, competition by investors to receive a positive illiquidity premium while holding a liquid REIT would appear to drive property values up to the point that they no longer offer a risk premium for illiquidity. Dividend yields on private REIT structures tend to be moderately higher than those on public REITs, suggesting that public REITs offer little or no risk premium for holding illiquid real estate.

\section*{Real Estate Indices Based on Financial Market Prices}
Unlike private commercial real estate, the reported returns of REITs are based on observations of frequent market prices. The ability to observe frequent market prices offers a huge potential advantage to measuring risk and return. Publicly traded real estate, especially REITs in the United States, provides regular market prices with which to observe, measure, and report real estate returns. The FTSE NAREIT US Real Estate Index Series is a family of REIT-based performance indices that covers the different sectors of the U.S. commercial real estate space. As with most investment categories of listed securities, there are numerous indices published.


\end{document}